\chapter{Continuité des fonctions}

    \section{Exercices généraux}

        \subsection{Vitesse d'un cycliste}

            Un cycliste parcours $20$km en $1$h. Montrer qu'il existe un intervalle d'une demi-heure durant laquelle il parcours $10$km.


\chapter{Suites}

    \section{Exercices généraux}

        \subsection{Tours de Hanoï}

            Les Tours de Hanoï est un jeu de palais composé de trois tiges verticales auquelles sont accrochés un certain nombre de palais 
            de tailles différentes. Tout les palais sont au début empilés sur la tige de droite.
            \begin{center}
                \includegraphics[width = 0.5\linewidth]{../../fancy_tex/anim_hanoi_debut.png}
            \end{center}
            Le but du jeu est de déplacer tout les palais sur la tige de droite en sachant qu'on ne peut pas placer un palais sur un autre plus petit.

            \paragraph{1/} En combien de coups minimum peut-on déplacer les $4$ palais représentés ci-contre ? Qu'en est-il de $5$ palais ?

            \paragraph{2/} Soit $n\in\mathbb{N}$, on note $u_n$ le nombre de coup minimum requis pour resoudre le jeu des Tours de Hanoï avec $n$ palais. Calculer $u_n$.



\chapter{Polynômes}

    \section{Exercices généraux}

        \subsection{Calcul de formes géométriques}

            \paragraph{1/} On considère un cube d'arrête $a$ tel que si on ajoute $2cm$ à $a$, l'aire est augmentée de $2402cm^2$. Quelle est la longueure de $a$ ?

            \paragraph{2/} Calculer la longueur des arrêtes d'un rectangle dont le périmètre est $P=34cm$ et l'aire $A$ vaut $60cm^2$.

        \subsection{Intersection de courbes}

        On considère la droite $\mathscr{D}$ d'équation $y = \frac{1}{2}x +1$ et la parabole $\mathscr{P}$ d'équation $y = x^2 - \frac{3}{2}x -1$.
        
        Calculer les coordonnées des points d'intersection de $\mathscr{D}$ et $\mathscr{P}$ ?

    \section{La vitesse des trains}

        Deux trains $A$ et $B$ partent en même temps d'une gare, l'un vers le nord, l'autre vers l'est. Le train $A$ se déplace, en moyenne, 
        à $25km/h$ de plus que le train $B$. Après $2h$, les deux trains sont éloignés de $250km$.

        Quelle est la vitesse moyenne de chaque train ?