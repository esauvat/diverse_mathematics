
% Chapitre 6 : Intégrales paramétrées

\minitoc

\section{Théorème de convergence dominée}

	\theorem{thm}{
		Soit $I$ i.r.n.t et $\suite{f} \in (\K^I)^{^\N}$, $f\in \K^I$ on suppose que 
		\begin{itemize}
			\item $\forall n\in \N , ~f_n\in\cpm (I,K)$ 
			\item $f\in \cpm (I,K)$
			\item $(f_n)$ converge simplement sur $I$ vers $f$
			\item $\exists \varphi \in L^1 (I,\K)$ telle que $\forall n\in\N ,~ \forall x\in I ,~ \abs{f_n(x)} \leq \varphi(x)$
		\end{itemize}\medskip
		
		Alors $~~\left\{ \ard 
			${\scriptsize (i)} $\forall n\in \N ,~f_n \in L^1(I,\K)$ et $f\in L^1(I,\K) \\ 
			${\scriptsize (ii)} $\displaystyle{\int_I f_n \ston \int_I f }
		\arf \right. $
	}
	
	\textit{ \small Ce théorème est admis.} \medskip \\
	
	\theorem{cor}{
		Soient $I,J$ des i.r.n.t. et $(f_\lambda)_{\lambda\in J}$ une famille de fonctions. \\ 
		Soit $\lambda_0\in\overline{J}$ ou $\lambda$ est une borne infinie de $J$ avec
		\begin{itemize}
			\item $\forall \lambda\in J ,~f_\lambda \in\cpm (I,\K)$
			\item $\forall t\in I,~f_\lambda (t)\underset{\lambda\to\lambda_0}{\longrightarrow} f(t)$ \vspace{-3pt}
			\item $f\in\cpm (I,\K)$
			\item $\exists \phi \in L^1 (I,\K)$ telle que $\forall \lambda\in J ,~\abs{f_\lambda} \leq \phi$
		\end{itemize} \medskip
		
		Alors $~~\left\{ \ard 
			${\scriptsize (i)} Les $f_\lambda$ et $f$ sont intégrables sur $I \\ 
			${\scriptsize (ii)} $\displaystyle{\int_I f_\lambda \underset{\lambda\to\lambda_0}{\longrightarrow} \int_I f } 
		\arf \right. $
	} \medskip
	
	
\section{Intégration terme à terme}
	
	\subsection{Cas positif}
		
		
		\theorem{thm}{
			Soit $I$ un intervalle réelle non trivial, on suppose 
			\begin{itemize}
				\item $\forall n\in\N , ~g_n \in L^1 (I,\R)$
				\item $\sum g_n$ converge simplement sur $I$
				\item $\suminf g_n \in \cpm (I,\R)$
				\item $\forall n\in\N ,~g_n \geq 0$
			\end{itemize}
			Alors $\displaystyle{\int_I\suminf g_n = \suminf\int_I g_n }$ dans $[0,+\infty[$ \\ 
			En particulier, $\suminf g_n$ est intégrable $\Leftrightarrow$ $\suminf \int_I g_n < + \infty$ 
		}
		
		\textit{ \small Ce théorème est admis.}
		
	\subsection{Cas réel ou complexe}
		
		\theorem{thm}{
			Soit $I$ un intervalle réelle non trivial, on suppose 
			\begin{itemize}
				\item $\forall n\in\N , ~g_n \in L^1 (I,\R)$
				\item $\sum g_n$ converge simplement sur $I$
				\item $\suminf g_n \in \cpm (I,\R)$
				\item La série \highlight{$\sum_{n\geq 0} \int_I \abs{g_n}$} converge $\heartsuit$
			\end{itemize} \medskip 
			Alors $ \left\{ \ard 
				${\scriptsize (i)} $\displaystyle \suminf g_n \in L^1(I,K) \\ 
				${\scriptsize (ii)} $ \displaystyle \sum_{n\geq 0} \int_I g_n $ converge $ \\ 
				${\scriptsize (iii)} $ \displaystyle \int_I\suminf g_n = \suminf \int_I g_n 
			\arf \right. $ 
		}
		
		\textit{ \small Ce théorème est admis.} \medskip \\
		

\section{Continuité et dérivation d'une intégrale paramétrée}
	
	\uline{Position du problème :} on s'intéresse à des fonctions du type $\displaystyle{ \Phi : x\mapsto \int_a^b f(x,t) dt }$ dont on veut étudier la continuité et la dérivabilité.
	
	\namedtheorem{Théorème 1 $\heartsuit$}{
		Soit $I$ un i.r.n.t. avec $a<b$ ses bornes et $A\subset E$ une partie quelconque de $E$ un espace vectoriel normé de dimension finie.\\
		On considère $f ~\appli{A\times I}{(x,t)}{\K}{f(x,t)} ~~$ avec 
		\begin{itemize}
			\item $ \forall x\in A ,~f(x,\cdot ) \in \cpm(I,\K)$
			\item $ \forall t\in I, ~f(\cdot ,x) \in \cont^0 (A ,\K)$
			\item \uline{Hypothèse de domination :} $ \\ 
			\exists \varphi \in L^1 (I,\K)$ telle que $ \forall (x,t) \in I\times A ,~\abs{ f(x,t) } \leq \varphi (t)$
		\end{itemize} \medskip
		
		Alors $ \displaystyle{ \Phi ~\appli{A}{x}{\K}{\int_a^b f(x,t) dt} } ~~ $ est continue. 
	}{ContIntPar}
	
	\begin{proof}
		On a $f(x, \cdot) \in L^1(I,\K)$ donc l'application est bien définie.\\
		Soit $\alpha \in A$, on considère $\suite{x}$ telle que $x_n \ston \alpha$ et on a alors $\Phi(x_n) \ston \Phi(\alpha)$ par le théorème de convergence dominée puis par caractérisation séquentielle de la continuité (\ref{CarSeqCont})\\ 
		$\Phi$ est continue en $\alpha$ d'où \textsc{cqfd} 
	\end{proof} \medskip
	
	\namedtheorem{Théorème 2 $\heartsuit$}{
		Soit $I$ un i.r.n.t. avec $a<b$ ses bornes et $A$ un i.r.n.t. quelconque. \\
		Soit $f : A\times I\to K$, on suppose
		\begin{itemize}
			\item $ \forall x\in A ,~f(x,\cdot ) \in L^1(I,\K)$
			\item $ \frac{\partial f}{\partial x}$ est bien définie sur $A\times I$ et vérifie les hypothèses du Théorème 1.
		\end{itemize} \medskip
		
		Alors $\Phi$ est bien définie sur $A$, $\Phi \in\cont^1(A,K)$ et \\
		$\forall x\in A$, \highlight{$\Phi'(x) = \int_a^b \frac{\partial f}{\partial x} (x,t) dt$} 
	}{DerIntPar}
	
	\textit{Cette égalité est dite "formule de \textsc{Leibniz}"} \medskip \\
	
	\begin{proof}
		On a l'existence de $\Phi :A\to K$ en tant qu'application.\\
		On considère alors $\alpha \in A$ et on note $T(x) = \frac{\Phi(x)-\Phi(\alpha)}{x-\alpha} ,~\forall \alpha\in A\setminus \{\alpha\}$. On applique alors le théorème de convergence dominée à $T(x_n)$ où $\suite{x} \in A^\N$ converge vers $\alpha$.\\
		Par caractérisation séquentielle d'une limite on a $\Phi^\prime(x) = \int_a^b \frac{\partial f}{\partial x}(x,t) dt$ et vu la formule pour $\Phi'(x)$ et le théorème 1 ; $\Phi'\in\cont^1(A,K)$
	\end{proof}
	
	\paragraph{Fonction $\Gamma$ d'\textsc{Euler}}
		On pose $\displaystyle{\Gamma(x) = \int_0^{\infty} t^{x-1}e^{-t} dt ~,~~x\in\R}$ la fonction $\Gamma$ d'\textsc{Euler}. Cette fonction est définie sur $\R_+^*$.\\
		On a $\Gamma(x+1) = \int_a^b t^xe^{-t}dt = [-e^{-t}t^x]_0^{+\infty} - \int_0^{+\infty} -e^{-t}xt^{x-1} dt = x\Gamma(x)$ donc  \highlight{$\displaystyle{\Gamma(n) = (n-1)! ~,~\forall n\in\N}$}. \\ 
		On en déduit $\Gamma \underset{0}{\sim} \frac{1}{x}$ et on peut montrer par le théorème 2 que $\Gamma\in\cont^{\infty}(\R_+^*,\R)$ \vspace*{0.5cm} \\ 
		
		
\fin
