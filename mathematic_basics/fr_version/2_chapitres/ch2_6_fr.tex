
% Chapitre 6 : Intégrales paramétrées

\minitoc
	\section{Théorème de convergence dominée}
		${}$ \\ \thm{ch6th1}{Théorème}{ThConvDom}{Soit $I$ i.r.n.t et $\suite{f} \in (K^I)^{^\N}$, $f\in K^I$ on suppose que \\
		{\tiny (1)} $\forall n\in \N , ~f_n\in\cpm (I,K)$ \hfill {\tiny (2)} $f\in \cpm (I,K)$ \hfill ${}$ \\
		{\tiny (3)} $(f_n)$ converge simplement sur $I$ vers $f$  \\ 
		{\tiny (4)} $\exists \varphi \in L^1 (I,K)$ telle que $\forall n\in\N ,~ \forall x\in I ,~ 
		\mc{f_n(x)} \leq \varphi(x)$  \\ ${}$ \vspace*{-0.2cm} \\
		Alors $~~\left\{ \ard ${\tiny (i)} $\forall n\in \N ,~f_n \in L^1(I,K)$ et $f\in L^1(I,K) \\ ${\tiny (ii)} $\cm{\int_I f_n \ston \int_I f }
		\arf \right. $}
		\\ \textit{ \small Ce théorème est admis.}
		\vspace*{0.5cm} \\ \thm{ch6th1c}{Corollaire}{IntegrFamillFc}{Soient $I,J$ des i.r.n.t. et $(f_\gl)_{_{\gl\in J}}$ une famille de fonctions.
		\\ Soit $\gl_0\in\overline{J}$ ou borne infinie de $J$, avec \\
		{\tiny (1)} $\forall \gl\in J ,~f_\gl \in\cpm (I,K)$ \\ {\tiny (2)} $\forall t\in I,~f_\gl (t)\underset{\gl\to\gl_0}{\longrightarrow} f(t)$
		\\ {\tiny (3)} $f\in\cpm (I,K)$\\
		{\tiny (4)} $\exists \phi \in L^1 (I,K)$ telle que $\forall \gl\in J ,~\mc{f_\gl} \leq \phi$ \\
		Alors $~~\left\{ \ard ${\tiny (i)} Les $f\l$ et $f$ sont intégrables sur $I \\ ${\tiny (ii)} $\cm{\int_I f_\gl 
		\underset{\gl\to\gl_0}{\longrightarrow} \int_I f } \arf \right. $}
	\section{Intégration terme à terme}
	\subsection{Cas positif}
		${}$ \\ \thm{ch6th2}{Théorème}{IntTATpos}{Soit $I$ un intervalle réelle non trivial, on suppose \\
		{\tiny (1)} $\forall n\in\N , ~g_n \in L^1 (I,\R)$  \\ {\tiny (2)} $\sum g_n$ converge simplement sur $I$ \hfill {\tiny (3)} $\suminft g_n \in 
		\cpm (I,\R)$ \\ {\tiny (4)} $\forall n\in\N ,~g_n \geq 0$ \\ Alors \hspace*{0.5cm} $\cm{\int_I\suminf g_n = \suminf\int_I g_n }$ dans 
		$[0,+\infty[$ \\ En particulier, $\suminft g_n$ est intégrable $\Leftrightarrow$ $\suminft \int_I g_n < + \infty$ }
		\\ \textit{ \small Ce théorème est admis.}
	\subsection{Cas réel ou complexe}
		${}$ \\ \thm{ch6th3}{Théorème}{IntTAT}{Soit $I$ un intervalle réelle non trivial, on suppose \\
		{\tiny (1)} $\forall n\in\N , ~g_n \in L^1 (I,\R)$  \\ {\tiny (2)} $\sum g_n$ converge simplement sur $I$ \hfill $\suminft g_n \in 
		\cpm (I,\R)$ \\ {\tiny (4)} La série \highlight{$\sum_{n\geq 0} \int_I \mc{g_n}$} converge $\heartsuit$ \\ 
		Alors \hspace*{0.5cm} $ \ard ${\tiny (i)} $\suminf g_n \in L^1(I,K) \\ ${\tiny (ii)} $ \sum\limits_{n\geq 0} g_n $ converge $ \\ 
		${\tiny (iii)} $ \int_I\suminf g_n = \suminf \int_I g_n \arf $ }
		\\ \textit{ \small Ce théorème est admis.}
	\section{Continuité et dérivation d'une intégrale paramétrée}
		\uline{Position du problème :} on s'intéresse à des fonctions du type $\cm{ \Phi : x\mapsto \int_a^b f(x,t) \dd t }$ dont on veut étudier 
		la continuité et la dérivabilité.
		\vspace*{0.5cm} \\ \thm{ch6th4}{Théorème 1 $\heartsuit$}{ContIntPar}{Soit $I$ un i.r.n.t. avec $a<b$ ses bornes.\\
		$A\subset E$ une partie quelconque de $E$ un espace vectoriel normé de dimension finie.\\
		$f ~\appli{A\times I}{(x,t)}{\K}{f(x,t)} ~~$ avec \\ $ \ard $ {\tiny (1)} $ \forall x\in A ,~f(x,\cdot ) \in \cpm(I,\K) \\ $ {\tiny (2)} $ 
		\forall t\in I, ~f(\cdot ,x) \in \cont^0 (A ,\K) \\ $ {\tiny (3)} \uline{Hypothèse de domination :} $ \\ \hspace*{0.5cm} \exists \varphi \in L^1 (I,\K)$ 
		telle que $ \forall (x,t) \in I\times A ,~\mc{ f(x,t) } \leq \varphi (t) \arf $ \\ ${}$ \\
		Alors $ \cm{ \Phi ~\appli{A}{x}{\K}{\int_a^b f(x,t) \dd t} } ~~ $ est continue. }
		\begin{proof}
		On a $f(x, \cdot) \in L^1(I,\K)$ vu {\tiny(1)} et {\tiny (2)} donc l'application est bien définie.\\
		Soit $\alpha \in A$ et on considère $\suite{x}$ telle que $x_n \ston \alpha$, on a alors $\Phi(x_n) \ston \Phi(\alpha)$ par le théorème de 
		convergence dominée puis par caractérisation séquentielle de la continuité (\ref{CarSeqCont}) \\ $\Phi$ est continue en $\alpha$ d'où 
		\textsc{cqfd} \end{proof}
		${}$ \\ \thm{ch6th5}{Théorème 2 $\heartsuit$}{DerIntPar}{Soit $I$ un i.r.n.t. avec $a<b$ ses bornes et $A$ i.r.n.t. \\
		Soit $f : A\times I\to K$, on suppose\\
		$ \ard $ {\tiny (1)} $ \forall x\in A ,~f(x,\cdot ) \in L^1(I,\K) \\ $ {\tiny (2)} $ \frac{\partial f}{\partial x}$ est bien définie sur $A\times I$ et vérifie les hypothèses du Théorème 1.$ \arf $ \\ ${}$ \\
		Alors $\Phi$ est bien définie sur $A$, $\Phi \in\cont^1(A,K)$ et $\forall x\in A$, \highlight{$\cm{\Phi'(x) = \int_a^b \frac{\partial f}{\partial x} (x,t) \dd t}$} }\\
		\textit{Cette égalité est dite "formule de \textsc{Leibniz}"}
		\begin{proof}
		On a l'existence de $\Phi :A\to K$ en tant qu'application.\\
		On considère alors $\alpha \in A$ et on note $T(x) = \frac{\Phi(x)-\Phi(\alpha)}{x-\alpha} ,~\forall \alpha\in A/backslash \{\alpha\}$. On applique alors le théorème de convergence dominée à $T(x_n)$ où $\suite{x} \in A^\N$ converge vers $\alpha$.\\
		Par caractérisation séquentielle d'une limite on à la dérivabilité de $\Phi$ au point $\alpha$ avec $\Phi'(\alpha) = \int_a^b \frac{\partial f}{\partial x} (x,t) \dd t$ puis sur $A$. 
		Vu la formule pour $\Phi'(x)$ et le théorème 1, on a $\Phi'\in\cont^1(A,K)$
		\end{proof}
	\paragraph{Fonction $\Gamma$ d'\textsc{Euler}}
		On pose $\cm{\Gamma(x) = \int_0^{\infty} t^{x-1}e^{-t} \dd t ~,~~x\in\R}$ la fonction $\Gamma$ d'\textsc{Euler}. Cette fonction est définie sur $\R_+^*$.\\
		On a $\cm{\Gamma(x+1) = \int_a^b t^xe^{-t}\dd t = [-e^{-t}t^x]_0^{+\infty} - \int_0^{+\infty} -e^{-t}xt^{x-1} \dd t = x\Gamma(x) }$\\
		Puis ainsi \highlight{$\cm{\Gamma(n) = (n-1)! ~,~\forall n\in\N}$} , on en déduit $\Gamma \underset{0}{\sim} \frac{1}{x}$\\
		De plus, on peut montrer par le théorème 2 que $\Gamma\in\cont^{\infty}(\R_+^*,\R)$ \vspace*{0.5cm} \\ 
		\begin{center}
		\fin
		\end{center}