
% Chapitre 7 : Séries entières

\minitoc
	\section{Rayon de convergence}
	\subsection{Définition}
		\traitd
		\paragraph{Série entière}
			Soit $\suite{a}\in\C^\N$ , on appelle \uline{série entière associée à cette suite} la série de fonctions \begin{center}
			\highlight{$\cm{\sum_{n\geq 0} a_n\times z^n}$}
			\end{center}
		\trait
		\thm{ch7L1}{Lemme d'\textsc{Abel}}{LemmeAbel}{Soit $\suite{a}\in\C^\N$, soit $z_0\in\C$ tel que $(a_nz_0^n)_{_{n\geq 0}}$ soit bornée,\\
		Alors $\forall z\in\C ~,~~ \mc{z} < \mc{z_0} ~ \Rightarrow ~ \sum\limits_{n\geq 0} a_nz^n$ converge absolument.}
		\traitd
		\paragraph{Rayon de convergence $\heartsuit$} ${}$ \\
			Soit une série entière $\se$ , on note $\mathscr{B}=\{r\in\R^+ ~;~ (a_nr^n)$ est bornée $\}$\\
			$\bullet $ Si $\mathscr{B}$ est \uline{majorée} on pose \highlight{$R=\mathrm{sup}\mathscr{B}$} $~~~~$ 
			$\bullet$ Sinon on pose \highlight{$R=+\infty$}\\
			$R$ est dit \uline{rayon de convergence de la série entière}.\trait
		\thm{ch7th1}{Théorème}{DisqueConv}{Soit une série entière $\se$ et $R$ son rayon de convergence alors $\forall z\in\C$ ,\\
		{\tiny (1)} $\mc{z} < R ~\Rightarrow ~ \se $ converge absolument.\\
		{\tiny (2)} $\mc{z} > R ~\Rightarrow ~ \se $ diverge grossièrement.}
		\begin{proof}
		On suppose $\mc{z} < R$ alors $\mc{z}$ n'est pas un majorant de $\mathscr{B}$ et par le Lemme d'Abel on a le résultat.\\
		Si $\mc{z} > R$ alors $\mc{z} \notin \mathscr{B}$ donc $(a_nz^n)$ n'est pas bornée et a fortiori ne tends pas vers $0$
		\end{proof} ${}$
		\paragraph{Critères de détermination de $R$} Soit $r\in \R^+$ \\
		$\bullet$ 1 : $\sum_{n\geq 0} a_n r^n$ converge $\Rightarrow$ $r<R$ et $\sum_{n\geq 0} a_n r^n$ diverge $\Rightarrow$ $r>R$ \\
		$\bullet$ 2 : $(a_nr^n)$ bornée $\Rightarrow$ $r<R$ et $(a_nr^n)$ non bornée $\Rightarrow$ $r>R$. \\
	\subsection{Continuité de la somme}
		${}$ \\ \thm{ch7th2}{Théorème}{CVNSE}{Soit $\se$ une série entière et $R$ son rayon de convergence, \\ Alors $\se$ converge normalement sur tout disque fermé $D_f(0,r)$ où $r<R$}
		\begin{proof}
		On considère $r<R$ et on pose $\alpha_n = \mc{a_n}r^n ~,~~\forall n\in\N$. On a alors pour tout $z\in\C$ tel que $\mc{z}\leq r$, $\se$ converge normalement.
		\end{proof}
		${}$ \\ \thm{ch7th2c}{Corollaire}{SEContDisc}{$S:z\mapsto \suminf a_nz^n$ est bien définie et continue sur le disque ouvert de convergence $D(0,R)$}
		\vspace*{0.5cm} \\ \thm{ch7L2}{Lemme : Cas de la variable réelle}{VarRe}{$\suite{a}\in\C^\N$, $R$ le rayon de convergence de la série entière \\
		{\tiny (1)} $\sum a_nx^n$ converge normalement sur tout $[a,b] \subset ]-R,R[$ \\
		{\tiny (2)} $x\mapsto \suminft a_nx^n$ est $\cont^0$ sur $]-R,R[$}
	\subsection{Utilisation de la règle de d'\textsc{Alembert}}
		${}$\\ \thm{ch7L3}{Lemme de suppression des termes nuls}{SupprTermNul}{Soit $E$ un espace vectoriel normé quelconque, $\suite{a}\in E^\N$ \\ 
		et $\varphi : \N\to\N$ une injection croissante, on suppose $\forall k\in\N\backslash\varphi(\N)$ on a $a_k=0$ alors \\
		\hspace*{0.5cm} $\ard $ {\scriptsize (1)} $\sum\limits_{k\geq 0} a_k$ converge $\Leftrightarrow$ $\sum\limits_{n\geq 0} a_{\varphi(n)}$ converge $ \\ $ {\scriptsize (2)} Dans ce cas $\suminf a_n = \suminf a_{\varphi(n)} \arf$ }
		\vspace*{0.5cm} \\ \thm{ch7L4}{Lemme}{LemmeTechMinoR}{Soit une série entière $\sum a_nz^n$ et $R$ son rayon de convergence \\
		Soit $\mu \in\R^+$, on suppose $\forall r\in [0,\mu[ ~,~~\sum a_nz^n$ converge \\
		 \hspace*{1.5cm} Alors \fbox{$R\geq \mu$}}
		 \newpage ${}$ \\ \thm{ch7th3}{Théorème : Règle de d'\textsc{Alembert} pour les séries entières}{DAlembSE}{Soit une série entière $\se$ avec $a_n\neq 0 ,~\forall n\geq n_0$ \\ On suppose l'existence de $\lambda\in [0,_\infty]$ tel que $\frac{\mc{a_{n+1}}}{\mc{a_n}} \ston \lambda$ \\ \hspace*{1.5cm} Alors \highlight{$R=\frac{1}{\lambda}$}}
		 \begin{proof}
		 Soit $r\in\R_+^*$ et $u_n = a_nr^n$ alors $u_n$ est positif à partir d'un certain rang et $\frac{\mc{u_{n+1}}}{\mc{u_n}} = r\frac{\mc{a_{n+1}}}{\mc{a_n}} \ston l$ \\
	On utilise ensuite le critère 2 de détermination de $R$ et on a le résultat par disjonction de cas.	 
		\end{proof}
	\subsection{Rayon de convergence d'une somme, d'un produit de \textsc{Cauchy}}
		${}$ \\ \thm{ch7th4}{Théorème}{RcvSomme}{Soit les séries entières $\left\{\ard \se$ de rayon de convergence $R_1 \\ \sum_{n\geq 0} b_n z^n$ de rayon de convergence $R_2 \arf \right.$ \\
		On note $R$ le rayon de convergence de $\sum\limits_{n\geq 0} (a_n+b_n)z^n$\\
		\hspace*{0.5cm} Alors $\ard $ {\scriptsize (i)} $R\geq \min \{R_1,R_2\} \\ $ {\scriptsize (ii)} $R_1\neq R_2 \Rightarrow R=\min \{R_1,R_2\} \arf$}
		\begin{proof}
		On suppose $R_1\geq R_2$ et on note $c_n=a_n+b_n$ 
		Soit $r\in\R_+$, on suppose $r<R_1$, alors $\sum_{n\geq 0} c_nr^n$ converge, donc $R\geq R_1 = \min\{R_1,R_2\}$ \\
		On suppose maintenant $R_1\neq R_2$ et $R>R_1$, on a alors l'existence de $r_0\in ]R_1,\min\{R,R_2\}[ $ tel que $\sum_{n\geq 0} a_n r_0^n$ converge or $r_0>R_1$ donc par l'absurde $R=\min\{R_1,R_2\}$ 
		\end{proof} ${}$\\ \traitd
		\paragraph{Produit de \textsc{Cauchy}}
			Soient $\SE$ et $\sum\limits_{n\geq 0} b_n z^n$ des séries entières quelconques, on appelle \uline{produit de \textsc{Cauchy} de ces deux séries entières} la série entière $\sum\limits_{n\geq 0} c_nz^n$ où $c_n=\sk{0}{n} a_kb_{n-k}$ \trait
		\thm{ch7L5}{Lemme}{7-L5}{Soient les séries entières $\se$ et $\sum_{n\geq 0} b_nz^n$ ; \\$\sum_{n\geq 0} c_nz^n$ leur produit de \textsc{Cauchy} alors $\forall z\in\C$ on a\\
		La \uline{série} $\sum c_n z^n$ est  le produit de \textsc{Cauchy} des \uline{séries} $\sum a_nz^n$ et $\sum b_nz^n$}
		\vspace*{0.5cm} \\ \thm{ch7th5}{Théorème}{RdCPDCSE}{Soient deux séries entières $\se$ et $\sum_{n\geq 0} b_nz^n$ \\ de rayons de convergence respectifs $R_1$ et $R_2$. \\
		Soit $\sum_{n\geq 0} c_nz^n$ leur produit de \textsc{Cauchy}, de rayon de convergence $R$\\
		Alors $\ard ${\scriptsize (1)} $\forall z\in\C ,~ \mc{z}<\min\{R_1,R_2\} ~\Rightarrow ~\left( \suminf a_nz^n \right) \times \left( \suminf b_nz^n\right) = \left( \suminf c_nz^n \right) \\ ${\scriptsize (2)} \highlight{$R\geq \min \{R_1,R_2\}$}$ \arf $ }
		\begin{proof}
		Soit $z\in\C$, on suppose $\mc{z} <\min \{R_1,R_2\}$ alors on a la convergence absolue des deux séries puis par théorème de leur produit de \textsc{Cauchy}. \\La propriété {\scriptsize (2)} se déduit clairement du Lemme de minoration de $R$ (\ref{LemmeTechMinoR})
		\end{proof}
		\newpage \traitd
		\paragraph{Dérivée}
			Soit une série entière $\SE$ on appelle \uline{série entière dérivée de $\SE$} la série entière \[ \sum_{n\geq 0} (n+1)a_{n+1}z^n \] \trait
		\thm{ch7th6}{Théorème}{CarSEDerivee}{Une série entière et sa dérivée on le même rayon de convergence.}
		\begin{proof}
		Soit $\se$ un série entière, $R$ son rayon de convergence et $\sum_{n\geq 0} b_nz^n$ sa dérivée. On a $\forall n\in\N ,~ b_n=(n+1)a_{n+1}$, donc pour $r\in\R_+^*$, $\mc{b_nr^n} = \frac{1}{r} \mc{a_{n_1} r^{n_1} }(n+1) \geq \frac{1}{r} \mc{a_kr^k}$ où $k=n+1$ \\
		$\bullet$ Si $r>R$, alors $(a_nr^n)$ n'est pas bornée et $(b_nr^n)$ non plus.\\
		$\bullet$ Si $r<R$ on fixe $\rho \in ]r,R[$ et $\mc{b_nr^n} = \frac{1}{r}\vert\underbrace{a_{n+1}}_{\mathrm{bornee}}\rho^{n+1}\vert(n+1)\left( \frac{r}{\rho}\right)^{n+1} \ston 0$\\
		On a alors la résultat par le critère 2.
		\end{proof}
		${}$ \\ \thm{ch7L6}{Lemme}{7-L6}{Pour toute série entière $\se$, \\$\se$ et $\sum_{n\geq 0} na_nz^n$ ont le même rayon de convergence.}
	\subsection{Coefficients comparables}
		${}$ \\ \thm{ch7L7}{Lemme}{SECoeffComp}{Soient les séries entière $\left\lbrace \ard ~\se $ de rayon de convergence $R_a \\ ~\sum_{n\geq 0} b_nz^n$ de rayon de convergence $R_b \arf \right. $ \\
		Alors $\ard $ {\scriptsize (1)} $a_n = \bigcirc_{n\to+\infty} (b_n) ~\Rightarrow ~R-a \geq R_b \\ $ {\scriptsize (2)} $a_n \underset{\infty}{\sim} b_n ~\Rightarrow ~ R_a=R_b \arf $}
	\section{Série entière de la variable réelle}
		On considère ici $x (\in\R) \mapsto S(x) = \suminf a_nx^n$ avec $\suite{a}\in\C^\N$
	\subsection{Dérivation et intégration}
		${}$ \\ \thm{ch7th7}{Théorème}{DerSEreelle}{Soit une série entière $\se$ de rayon de convergence $R$ avec $R>0$, \\
		On note $S(x)=\suminf a_nx^n ,~ \forall x\in]-R,R[$	\\
		Alors $S$ est dérivable sur $]-R,R[$ avec $\forall x\in]-R,R[ ,~ S\prime (x) = \sum\limits_{n=1}^{+\infty} na_{n-1}x^{n-1}$ }
		\begin{proof}
		C'est une conséquence du théorème de dérivation des séries de fonctions (\ref{ConvDeriveeSeries})
		\end{proof}
		${}$ \\ \thm{ch7th7c}{Corollaire}{SECinf}{Soit $\se$ une série entière de rayon de convergence $R>0$, on note $I=]-R,R[$\\
		\hspace*{0.5cm} Alors $\ard ~\bullet$ $S\in\cont^{\infty}(I,\R)$ et $\forall k\in\N$, $S^{(k)}$ s'obtient par dérivation terme à terme.$ \\ ~\bullet$ Une primitive de $S$ sur $]-R,R[$ est $T:x\mapsto \suminf \frac{a_n}{n+1}x^{n+1} \arf$}
		\newpage ${}$ \\ \thm{ch7L8}{Lemme}{EcritureAn}{Soit $\delta>0$ et $\suite{a}\in\C^\N$, on suppose que $\forall x\in ]-\delta,\delta[, ~f(x) = \suminf a_nx^n$\\
		\hspace*{0.5cm} Alors $f|_{]-\delta,\delta[}$ est $\cont^{\infty}$ et \highlight{$\forall n\in\N ,~ a_n=\frac{f^{(n)}(0)}{n!}$}}
		\vspace*{0.5cm} \\ \thm{ch7L8c}{Corollaire $\heartsuit \heartsuit$}{UniCoeffSE}{On suppose que $\se= \sum_{n\geq 0} b_nx^n$ pour tout $x$ dans \uline{un voisinage de $0$} \\
		\hspace*{0.5cm} Alors $\forall n\in\N ,~a_n=b_n$}
		\vspace*{0.5cm} \\ \thm{ch7L9}{Lemme}{UniCoeffSEStat}{Soit \fbox{$\delta>0$}, $\suite{a},\suite{b} \in\C^\N$ on suppose que $\forall x\in ]0,\delta[, ~\suminf a_nx^n = \suminf b_nx^n$ \\
		\hspace*{0.5cm} Alors $\forall n\in\N ,~a_n=b_n$}
		\vspace*{0.5cm} \\ \thm{ch7th8}{Théorème d'\textsc{Abel} radial}{ThAbelRadial}{Soit $\se$ une série entière de rayon de convergence $R\in\R_+^*$\\
		On suppose que $\sum a_n R^n$ converge \\
		\hspace*{0.5cm} Alors $\suminf a_nx^n \stox{R} \suminf a_nR^n$}
		\begin{proof} (HP)\\
		On note $S(x) = \suminft a_nx^n ,~\forall x\in ]-R,R[$, on suppose $R=1$ (sans perte de généralité)\\
		Notons $S=\suminft a_n$ (convergente) et $R_n = \sum_{k=n+1}^{+\infty} a_k$\\
		Pour tout $x\in[0,R[$ on a par transformation d'Abel \fbox{$S-S(x) = \sum_{n=0}^{+\infty} R_n (x^n-x^{n+1})$}\vspace*{0.3cm}\\
		Soit $\varepsilon >0$ ; soit $N\in\N$ tel que $\forall m\geq N ,~\mc{R-n}<\frac{\varepsilon}{2}$, alors $\forall x\in[0,R[$ 
		\[ \mc{S-S(x)} \leq \mc{Q(x)} + \sum_{n=N}^{+\infty} \frac{\varepsilon}{2} (x^n - x^{n+1}) = \mc{Q(x)} + \frac{\varepsilon}{2} x^N \] 
		où $Q(x) = \sum_{n=0}^{n-1} R_n(x^n - x^{n+1})$, avec $Q(x) \stox{1} 0$ (somme finie) \vspace*{0.2cm} \\
		Soit alors $\theta \in [0,1[$ tel que $\forall x\in ]\theta,1[ ,~ \mc{Q(x)} <\frac{\varepsilon}{2}$. Alors $\forall x\in ]\theta,1[,~ \mc{S-S(x)} <\varepsilon$ \vspace*{0.2cm} \\
		D'où $S(x) \stox{1} S$
		\end{proof}
	\subsection{Développement en série entière}
		\traitd
		\paragraph{Fonction développable $\heartsuit$}
			Une fonction $f$ de la variable réelle est dite \uline{développable en série entière en $x_0$} ($DSE_0$) s'il existe $\delta>0$ et $\suite{a}\in\C^\N$ tels que \[ \forall x\in ]x_0-\delta , x_0+\delta[ ,~ f(x) = \suminf a_n(x-x_0)^n \]\trait
		\textit{\uline{NB} : Quitte à considérer $t\mapsto f(x_0+t)$, il suffit de traiter le cas $x_0 = 0$}\\
		\textbf{\uline{Rq} :} $f$ est $DSE_0$ $\Rightarrow$ $f$ est $\cont^{\infty}$ sur un voisinage de $0$ \\\textsc{Attention !} La réciproque est fausse !! \newpage \traitd
		\paragraph{Série de \textsc{Taylor}}
			Lorsque $f$ est $\cont^{\infty}$ sur un voisinage de $0$, la série entière $\sum_{n=0}^{+\infty} \frac{f^{(n)}(0)}{n!}$ est dite \uline{série de \textsc{Taylor} en $0$} \trait
		\textsc{Attention !} Elle ne coïncide pas toujours avec $f$ sur un voisinage de $0$ !!\\
		\traitd
		\paragraph{Fonction complexe développable}
			Soit $z_0\in\C$. Une fonction $f$ de la variable complexe est dite développable en série entière en $z_0$ s'il existe $\delta>0$ et $\suite{a}\in\C^\N$ tels que \[ \forall z\in D(z_0 , \delta ) ,~ f(z) = \suminf a_nz^n \] \trait
	\subsection{Développements en série entière de référence}
		On pose d'autorité $\bullet ~\forall z\in\C ,~\exp (z) = \suminf \frac{z^n}{n!}$ avec $R=+\infty$
		\vspace*{0.5cm} \\ \thm{ch7L10}{Lemme}{CNSParité}{On suppose $\delta>0$ et $\forall x\in ]-\delta,\delta[ ,~f(x) = \sum_{n\geq 0} a_nx^n$ \\
		\hspace*{0.5cm} Alors $\ard $ {\scriptsize (1)} $f$ est paire (sur $]-\delta,\delta[$) $\Leftrightarrow ~\forall k\in\N ,~a_{2k+1} = 0 \\ $ {\scriptsize (2)} $f$ est impaire (sur $]-\delta,\delta[$) $\Leftrightarrow ~\forall k\in\N ,~a_{2k} = 0 \arf$ }
		\vspace*{0.5cm} \\ \thm{ch7th9}{Théorème}{DES0cossin}{$\forall x\in\R$, \\
		\hspace*{1cm} $\begin{array}{ll}
		\cos (x) = \suminf \frac{(-1)^n}{(2n)!} x^{2n} ~=~1-\frac{x^2}{2} + \frac{x^4}{24} + \cdots & R=+\infty \\ \sin (x) = \suminf \frac{(-1)^n}{(2n+1)!}x^{2n+1} ~=~ x-\frac{x^3}{6} + \cdots & R=+\infty
		\end{array}$ }
		\begin{proof}
		Soit $x\in\R$ {\footnotesize (c'est en fait vrai sur $\C$)} \\
		\hspace*{0.5cm} $\rightarrow$ $\cos (x) = \frac{1}{2} (e^{ix} + e^{-ix}) = \suminft \frac{1}{2} i^n \big(1+(-1)^n \big) \frac{x^n}{n!} = \suminft (i^2)^{^n} \frac{x^{2n}}{(2n)!}$\\
		On a de même pour le sinus.
		\end{proof}
		${}$ \\ \thm{ch7th10}{Théorème}{DES0chsh}{ $\forall x\in\R$, 
		$\ard \cosh (x) = \suminf \frac{x^{2n}}{(2n)!} \\ \sinh (x) = \suminf \frac{x^{2n+1}}{(2n+1)!} \arf$ avec dans les deux cas $R=+\infty$ }
		\vspace*{0.5cm} \\ \thm{ch7th11}{Théorème}{DSE0ln}{$\forall x\in ]-1,1] ,~ \ln (1+x) \sum\limits_{n=1}^{+\infty} (-1)^n \frac{x^n}{n}$ avec $R=1$ $\heartsuit$}
		\begin{proof}
		On pose $h(x) = \ln(1+x) , ~\forall x>-1$, on a alors $h'(x) = \frac{1}{1+x}$ est bien définie $\forall x>-1$\\
		Par théorème on a en posant $T(x) = \sum_{n=1}^{+\infty} (-1)^n \frac{x^{n+1}}{(n+1)!}$, $(T-h)'=0$ et vu $T(0)=0=h(0)$ \\
		on a $T(x) = h(x)$ d'où le résultat.
		\end{proof}
		${}$ \\ \thm{ch7th12}{Théorème}{DSE0puissalpha}{Soit $\alpha\in\R$ alors \\
		\hspace*{1cm} $\forall x\in ]-1,1[ ,~ (1+x)^{\alpha} = 1 + \sum\limits_{n=1}^{+\infty} \alpha\times(\alpha-1)\times\cdots \times(\alpha-n+1) \frac{x^n}{n!}$\\
		Notons $R$ son rayon de convergence alors :\\
		$\bullet $ si $\alpha\in\N$ on a $R=+\infty$ (binôme de \textsc{Newton}) \\
		$\bullet $ si $\alpha\in\R\setminus\N$ on a $R=1$ }
		\begin{proof}
		
		\end{proof}