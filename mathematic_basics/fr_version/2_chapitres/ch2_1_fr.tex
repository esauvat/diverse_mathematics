
% Chapitre 1 : Suites et séries 

\textit{\small On considèrera comme acquis en sup les cas réel et complexe : Notament:\\ -> Théorème des gendarmes \hfill -> Théorème de la limite monotone} \hfill ${}$

\minitoc

\section{Norme}

	\subsection{Généralités}
	
		\vspace{-15pt}
		\traitd
		\paragraph{Norme} Une \emph{norme} sur $E$ est une application $N: E \rightarrow\mathbf{R}$ vérifiant : 
			\begin{itemize}
				\item $\forall x\in E, ~N(x) = 0_{\mathbf{R}} ~\Leftrightarrow ~x=0_E$
				\item $\forall x\in E, ~\forall \lambda \in \mathbf{K} , ~N(\lambda .x) = \abs{\lambda}N(x)$
				\item $ \forall x,y \in E, ~N(x+y) \leq N(x)+N(y)$ 
			\end{itemize}
		\trait 
		
		\theorem{lem}{Soit $(E,N)$ un espace vectoriel normé, \\
			On a $N\geq 0$ \hspace*{1cm} {\small (i.e. $\forall x\in E,~N(x-y) \geq 0$)}
			}
			
		\traitd
		\paragraph{Distance} Une \emph{distance} sur $X$ est une application $d:X^2\rightarrow\mathbf{R}$ vérifiant:
			\begin{itemize}
				\item $\forall x,y\in E, ~d(x,y) = 0 ~\Leftrightarrow ~x=y$
				\item $\forall x,y\in E, ~d(x,y)~=~d(y,x)$
				\item $\forall x,y,z\in E, ~d(x,z)~\leq~d(x,y)+d(y,z)$
			\end{itemize}
		\trait
		
		\theorem{lem}{
			Soit $(E,N)$ un espace vectoriel normé. \\
			Si $\forall (x,y)\in E^2$, $d(x,y) = N(x-y) $ alors $d$ est une distance sur E.
			}
			
		\traitd
		\paragraph{Boule ouverte et fermée} Soient $a\in E,~r\in\mathbf{R}$ On pose
			\[B(a,r) = \left\{ x\in E ~|~ d(x,a)<r\right\} \hspace*{1.5cm} B_f (a,r) = \left\{ x\in E ~|~ d(x,a) \leq r\right\}\]
			\hspace*{0.6cm} Les \emph{boules ouverte et fermée} de centre $a$ et de rayon $r$. 
		\trait
		
		\newpage
		
		\traitd
		\paragraph{Segment et ensemble convexe} Soit $E$ un $\mathbf{K}$ espace vectoriel quelconque \vspace*{0.2cm}\\
			\hspace*{0.5cm} -> Pour $(a,b)\in E^2$ on défini le \emph{segment} : $[a,b]=\left\{ (1-t)a +tb ~\vert ~t\in [0,1]\right\}$ \\
			\hspace*{0.5cm} -> $\mathcal{C} \subset E$ est dit \emph{convexe} si $\forall (a,b)\in \mathcal{C}^2,~[a,b]\subset \mathcal{C}$
		\trait 
		
		\theorem{lem}{
			Dans $E$ un EVN quelconque les boules sont convexes
		} \medskip
		

	\subsection{Normes euclidiennes}
		Ici $E$ est un $\mathbf{R}$ espace vectoriel muni d'un produit scalaire\footnotemark[1] 
		\[
			\Phi :  \appli{E^2}{(x,y)}{\mathbf{R}}{\scal{x}{y}}
		\]
			
		\footnotetext[1]{Un produit scalaire est une forme bilinéaire symétrique définie positive}
		
		On a alors par théorème\footnote[2]{Voir cours de sup} ,  $x\mapsto \sqrt{\scal{x}{x}}$ est une norme sur $E$.  
		On notera \begin{center} \highlight{ $\norm{x}_2 = N_2(x) = \sqrt{\scal{x}{x}}$ } \end{center} 
		
		\remarque{L'inégalité triangulaire pour $\norm{.}_2$ est dite inégalité de \emph{Minkovsky}} \medskip
			
		\theorem{lem}{Si $E=\mathbf{C}^n ,~z=\left(z_1,\dots ,z_n\right) ~, ~N(z) = \sqrt{\sum\limits_{k=1}^n\abs{z_k}^2}$ est une norme
			} \medskip
		
		\theorem{lem}{$E = \cont^0\left([a,b],\mathbf{C}\right)$ Soit $f\in E$ on pose \\
			$N(f) = \sqrt{\int_a^b \abs{f(x)}^2\mathrm{d}x}$ \hspace*{1cm} alors $N$ est une norme sur $E$
			} \medskip
			
			
	\subsection{Exemple de normes}

		\paragraph{Norme $N_{\infty}$ :} ~
		
			Dans $E=K^n$ soit $x=(x_1,\dots ,x_n) ,~ N_{\infty}(x) = 
			\underset{i\in\ent{1,n}}{\mathrm{max}} \abs{x_i}$
		
			Dans $E=\cont^0([a,b],K)$ soit $f\in E,~N_{\infty} (f) = \underset{x\in [a,b]}{\mathrm{sup}} \abs{f(x)}$
			
		\paragraph{Norme $N_1$ :} ~

			Dans $E=K^n$ soit $x=(x_1,\dots ,x_n) ,~ N_1(x) = \sum_{i=1}^n \abs{x_i} $
			
			Dans $E=\cont^0([a,b],K)$ soit $f\in E,~N_1 (f) = \int_a^b \abs{f(x)} \mathrm{d}x$
			
		\paragraph{Norme $N_2$ :} ~

			Dans $E=K^n$ soit $x=(x_1,\dots ,x_n) ,~ N_2(x) = \sqrt{\sum_{i=1}^n x_i^{~2}} $
			
			Dans $E=\cont^0([a,b],K)$ soit $f\in E,~N_2 (f) = \sqrt{\int_a^b \left( f(x)\right)^2 \mathrm{d}x}$
		
		\medskip


\section{Suites}
		
		\vspace{-15pt}
		\traitd 
		\paragraph{Suite convergente}	
			Soit $u=\left( u_n\right)_{_{n\in\mathbf{N}}} \in E^{\mathbf{N}}$ et $\ell\in E$.
			On dit que $u$ converge vers $\ell$ et on note 
			\[
				u_n \underset{n\rightarrow +\infty}{\longrightarrow} \ell ~\mathrm{ssi}~\forall 
				\varepsilon >0 ,~\exists n_0 \in \mathbf{N} ~:~ \forall n\geq n_0,~d(u_n ,\ell) <\varepsilon
			\vspace{-25pt}
			\] 
		\trait 
			
		\namedtheorem{Lemme 1 : Unicité de la limite}{
			Si \hspace*{0.2cm}
			$\ard u_n \underset{n}{\rightarrow} \ell_1 ~\in E \\u_n \underset{n}{\rightarrow} \ell_2 ~\in E \arf$ \hspace*{0.2cm} Alors $\ell_1 = \ell_2$
			}{UniqLimit}
			
		\begin{proof}
			Par l'absurde, on suppose $\ell_1\neq \ell_2$.\\
			Soit $\varepsilon = \frac{1}{2} d(\ell_1,\ell_2) >0$
			On a alors $\ard n_1 \in \mathbf{N} ~:~\forall n\geq n_1 ,~d(u_n,\ell_1) < \varepsilon \\  
			n_2 \in \mathbf{N} ~:~\forall n\geq n_2 ,~d(u_n,\ell_2) < \varepsilon \arf $  et soit $p=max(n_1,n_2)$ \vspace*{0.1cm}\\
			$d(\ell_1,\ell_2) \leq d(\ell_1,u_p) + d(\ell_2,u_p) < 2\varepsilon = d(\ell_1,\ell_2)$  \emph{impossible}
		\end{proof} \medskip
		
		\theorem{lem}{
			Soit $\left(u_n\right) _{_{n\in \mathbf{N} }} \in E^{\mathbf{N}} ,~\ell\in E $\hspace*{0.2cm} 
			Alors $u_n \underset{n}{\rightarrow} \ell ~\Leftrightarrow ~\norm{u_n - \ell} \underset{n}{\rightarrow} 0$
		}
		
		\begin{proof}
			Notons $v_n = \norm{u_n-\ell}$ et $\lambda = 0$
			Alors $d(u_n,\ell) = \norm{u_n - \ell} = v_n = \norm{v_n - \lambda} = d(v_n,\lambda )$ \\
			or $u_n \underset{n}{\rightarrow} \ell$ \emph{ssi} : $\forall \varepsilon >0 ,~\exists n_0 \in \mathbf{N} ~:~ 
			\forall n\geq n_0,~d(u_n ,\ell) <\varepsilon ~\Rightarrow ~ d(v_n,\lambda)<\varepsilon ~\Rightarrow ~ v_n \underset{n}{\rightarrow} 0$
		\end{proof} \medskip

		\theorem{lem}{
			Soient $u_n ,~v_n \in E^{\mathbf{N}} ~$ et $ \lambda\in K$ si on a 
			$u_n \underset{n}{\rightarrow} \alpha$ et $v_n \underset{n}{\rightarrow} \beta$\\ 
			\hspace*{0.2cm} Alors $\lambda u_n + v_n \underset{n}{\rightarrow} \lambda\alpha + \beta$
		} \medskip

		\namedtheorem{Lemme : Inégalité triangulaire renversée}{
			Soit $x,y \in E$ alors $\abs{N(x) - N(y)} \leq N(x-y)$
		}{InegTriRenv}

		\begin{proof}
			$N(x)\leq N(x-y) + N(y) \Rightarrow ~ \underbrace{N(x) - N(y)}_{t\in \mathbf{R}} \leq N(x-y)$\\
			On conclut alors par agument de symétrie.
		\end{proof} \medskip

			
		\theorem{lem}{
			Soit $u_n \in E^{\mathbf{N}} ,~\alpha\in K$ 
			on a $u_n\underset{n}{\rightarrow} \alpha ~\Rightarrow ~ \norm{u_n}\underset{n}{\rightarrow}\norm{\alpha}$ 
		} \\
		{\small \emph{Attention !} La réciproque est fausse !}

		\newpage
		
		\traitd
		\paragraph{Suite bornée}
			Soit $\left(u_n\right)_{n\in\mathbf{N}} \in E^{\mathbf{N}}$ on dit que $\left(u_n\right)$ est bornée si
			$\exists M \in \mathbf{R} ~:~ \forall n\in\mathbf{N} ,~\norm{u_n}\leq M$. 
		\trait

		\theorem{lem}{
			Toute suite $\left(u_n\right)_{n\geq 0} \in E^{\mathbf{N}}$ convergente est bornée
		} \medskip

		\theorem{lem}{
			On suppose $\left\{\ard \lambda_n \underset{n}{\rightarrow} \mu \in K \\ 
			u_n \underset{n}{\rightarrow} v\in E \arf\right.$ \hspace*{0.2cm} Alors $\lambda_n u_n \underset{n}{\rightarrow} \mu v$
		} 
		
		\traitd
		\paragraph{Suite extraite}
			Soit $u \in E^{\mathbf{N}}$ on appelle \emph{suite extraite} (ou sous-suite) de $u$ toute suite
			$\left(u_{\varphi (n)}\right)_{n\in \mathbf{N}}$ où $\varphi : \mathbf{N} \rightarrow \mathbf{N}$ 
			est une extractrice (injection croissante) \\ \textit{\small NB : en fait $\left(v_n\right)_{n\geq0} = 
			\left(u_{\varphi (n)} \right)_{n\geq 0} ~\Leftrightarrow ~v= u\circ\varphi$} 
		\traitdouble
		\paragraph{Valeur d'adhérence}
			$\ell\in E$ est une valeur d'adhérence de $u$ s'il existe une suite extraite de $u$ qui converge vers $\ell$.
			On notera $\mathcal{V}_u$ l'ensemble des valeurs d'adhérence de $u$. 
		\trait
		
		\theorem{thm}{
			Soit $u\in E^{\mathbf{N}}$ si $u$ \\
			converge vers $\ell\in K$ 
			alors toute suite extraite de $u$ converge vers $\ell$
		}
		
		\begin{proof}
			Soit $\varphi : \mathbf{N} \rightarrow \mathbf{N}$ une extractrice et 
			$\left(v_n\right)_{n\geq 0} = \left(u_{\varphi (n)} \right)_{n\geq 0}$ \\
			Soit $\varepsilon >0$ et $ n_0 \in\mathbf{N} ~:~ \forall n\geq n_0 ,~d(u_n ,\ell)<\varepsilon$ 
			donc $\varphi (n) \geq n_0$ et ainsi $d\left( u_{\varphi (n)},\ell\right)<\varepsilon$ et $v_n \underset{n}{\rightarrow} \ell$
		\end{proof} \medskip

		\theorem{cor}{
			Toute suite admettant au moins $2$ valeurs d'adhérence est \\divergente
		} \medskip 


\section{Normes équivalentes}
	
	\subsection{Définition}
		
	\vspace{-15pt}
		\traitd \vspace{-7pt}~\\
			Soit $E$ un $K$ espace vectoriel, $N$ et $N'$ deux normes sur $E$.
			$N$ et $N'$ sont dites \\équivalentes ($N\sim N'$) si $\exists \alpha ,\beta \in \mathbf{R} ~:~ \alpha N \leq N'\leq \beta N$
		\trait

		\vspace{-10pt}
		\remarque{On peut aussi l'écrire $	N'\leq \beta N$ et $N\leq \frac{1}{\alpha} N'$\\} 
		
		\theorem{lem}{
			Soit $N,N'$ des normes équivalentes sur $E$, $u\in E^{\mathbf{N}} ,~\ell\in E$ alors\\
			1) $u_n \underset{n}{\rightarrow}\ell$ dans $(E,N)$ $\Leftrightarrow$ $u_n \underset{n}{\rightarrow}\ell$ dans $(E,N')$ \\
			2) $u$ est bornée dans $(E,N)$ $ \Leftrightarrow$ $u$ est bornée dans $(E,N')$
		}\label{1.3.1} \medskip
		
		\theorem{lem}{
			Sur $K^n$, $N_1$, $N_2$ et $N_{\infty}$ sont équivalentes et plus précisément
			\\ \hspace*{2cm} $N_{\infty} \leq N_1 \leq \sqrt{n}N_2 \leq nN_{\infty}$
		} \medskip


	\subsection{Cas de espaces de dimension fini}
		
		\rappel{Un espace vectoriel $E$ est de dimension finie s'il existe une famille d'éléments de $E$ 
			libre et génératrice, c'est alors une base de $E$.} \medskip
			
		\theorem{thm}{
			Sur un $K$-ev de dimension finie, 
			toutes les normes sont équivalentes.
		} \\
		\textit{\small Sera démontré ultérieurement.} \\

		\theorem{cor}{
			Dans un $K$ espace vectoriel de dimension finie, 
			la notion de convergence \\
			ne dépend pas de la norme.
		} \\ 
		\emph{Attention !} C'est faux en dimension quelconque ! \\

		\theorem{lem}{
			Soit $E$ de dimension finie et $e = (e_1,\dots ,e_p)$ une base de $E$. \\
			Soit $\left(x_n\right)_{n\geq 0} \in E^{\mathbf{N}}$ et $\alpha \in E$.
			On écrit $\left\{\ard x_n = x_{1,n}e_1 + \cdots + x_{p,n}e_p \\ \alpha = \alpha_1e_1 + \cdots + \alpha_pe_p \arf\right.$\\
			On a alors $x_n \underset{n}{\rightarrow} \alpha ~\Leftrightarrow ~\forall k \in \ent{1,p} ,~x_{k,n} \underset{n}{\rightarrow} \alpha_k$
		} \medskip
		
		\theorem{thm}{
			Soient $p,q,r \in \mathbf{N}^*$
			\hspace*{0.3cm}$\left\{\begin{array}{ll} A_n \underset{n}{\rightarrow} A & dans~\mathcal{M}_{p,q}\left(\mathbf{R}\right) \\ 
			B_n \underset{n}{\rightarrow} B & dans~\mathcal{M}_{q,r}\left(\mathbf{R}\right) \end{array}\right.$ \hspace*{0.3cm}
			Alors $A_nB_n \underset{n}{\rightarrow} AB$
		}
		
		\begin{proof}
			Soit $(i,j)\in\ent{1,p}\times\ent{1,r}$ \\ 
			$\left(A_nB_n\right) _{i,j} = \sum_{k=1}^{q} \underbrace{\left(A_n\right)_{i,k}}_{\rightarrow a_{i,k}}
			\underbrace{\left(B_n\right)_{k,j}}_{\rightarrow b_{k,j}}$ $\underset{n}{\rightarrow} \sum_{k=1}^{q} a_{i,k}b_{k,j} = (AB)_{i,j}$ donc 
			$A_nB_n \underset{n}{\rightarrow} AB$
		\end{proof}


\section{Comparaisons asymptotiques}
	
		Soient $\left(u_n\right)_{n\geq n_0} ,\left(v_n\right)_{n\geq n_0} \in \mathbf{C}^{\mathbf{N}}$ 
		
		\vspace{-10pt}
		\traitd
		\paragraph{Négligeabilité}
			On dit que $u_n$ est négligeable devant $v_n$ quand $n\rightarrow +\infty$ noté $u_n \underset{n\rightarrow +\infty}{=} \circ (v_n)$ s'il existe $n_0 \in \mathbf{N}$ et $\left(\delta_n\right)_{n\geq n_0}$ tel que
			\begin{itemize}
				\item $\forall n\geq n_0 ,~u_n = \delta_nv_n$
				\item $\delta_n \underset{n\rightarrow +\infty}{\longrightarrow} 0$
			\end{itemize}
		\traitdouble
		\paragraph{Domination}
			On dit que $u_n$ est dominée par $v_n$ quand $n\rightarrow +\infty$ noté $u_n \underset{n\rightarrow +\infty}{=} \bigcirc (v_n)$ s'il existe $n_0 \in \mathbf{N}$ et $\left(B_n\right)_{_{n\geq n_0}}$ tel que
			\begin{itemize}
				\item $\forall n\geq n_0 ,~u_n = B_nv_n$ 
				\item $\left( B_n \right)_{_{n\geq n_0}}$ est bornée
			\end{itemize}
		\traitdouble
		\paragraph{Équivalence}
			On dit que $u_n$ est équivalent à $v_n$, noté $u_n \sim v_n$ 
			si : 
			\[
				u_n - v_n \underset{n\rightarrow +\infty}{=} \circ (v_n) 
			\vspace*{-20pt}
			\]
		\trait

		\remarque{$u_n \sim v_n ~\Leftrightarrow ~u_n = v_n + \circ (v_n)$}
		
\section{Séries dans un K espace vectoriel de dimension finie}
		
		\remarque{On note par abus " $dim E < \infty$ "}
		
		Le cas scalaire est abordé en MPSI. \\

		Soit $u = (u_n) \in E^{\mathbf{N}}$ ; pour $n \in \mathbf{N}$ on pose $U_n = \sum_{k=1}^{n} u_k$. 
		
		\newpage
		
		\traitd
		\paragraph{Sommes partielles} La suite $(U_n)$ est dite suite des \emph{sommes partielles} associée à $u$.
		\traitdouble
		\paragraph{Série convergente} On dit que la \emph{série de terme général $u_n$} converge si $\left(U_n\right)$ converge.
			\\\hspace*{2cm} Dans ce cas on pose $~$ \highlight{$\sum_0^{+\infty}~=~\underset{n\rightarrow +\infty}{\lim} U_n~\in E$}
		\trait
		
		\theorem{lem}{
			$\left( \sum u_n \mathrm{converge} \right) ~ \Rightarrow ~ 
			\left( u_n \underset{n}{\rightarrow} 0 \right) $
		} \\ 
		\emph{Attention !} La réciproque est fausse ! (ex : $(H_n)$)
		
		\traitd
		\paragraph{Divergence grossière}
			Lorsque $u_n \underset{n}{\not\rightarrow} 0$, la série $\sum u_n$ est dite \\
			\emph{grossièrement divergente} "$\sum u_n$ DVG" ainsi : $\left(\sum u_n \mathrm{DVG} \Rightarrow \sum u_n \mathrm{DV} \right)$
		\trait

		\namedtheorem{Théorème : Reste d'une série convergente}{
			On suppose $\sum u_n$ converge, on note $S = \suminf u_n$ \\
			la "limite de la somme" et $R_n = \sum_{k=n+1}^{+\infty} u_k$ le "reste d'ordre $n$". \\
			Alors $\left\vert \ard \forall n\in \mathbf{N}$, $S=U_n + R_n \\ R_n \to 0 \arf \right.$
		}{RestSérie}
			
		\begin{proof}
			bien-fondé ?\\Soit $n\in\mathbf{N}$ pour $m\geq n+1$, $\sum_{k=n+1}^{m} u_k = U_m - U_n \underset{m}{\rightarrow} S-U_n$ donc $R_n$ existe 
			avec $R_n = S-U_n$ \\d'où $S=U_n+R_n$ 
			puis $R_n = S-U_n \to S-S=0$
		\end{proof} \medskip
		
		\theorem{lem}{
			Soit $(u_n) ,~ (v_n) ~\in E^{\mathbf{N}}$ et $\lambda\in K$\\
			On suppose que $\sum u_n$ et $\sum v_n$ convergent alors : \\ \hspace*{2cm} -> $\sum \lambda u_n + v_n$ converge \\
			\hspace*{2cm} -> $\suminf \lambda u_n + v_n = \lambda \suminf u_n + \suminf v_n$
		}
		
		\traitd
		\paragraph{Convergence absolue} Soit $(u_n) ~\in E^{\mathbf{N}}$ on dit que $\sum u_n$ \emph{converge absolument} 
			\\ si $\sum \norm{u_n}$ converge.
		\trait 
		
		\newpage
		
		\remarque{Vu $dimE < \infty$, ceci ne dépend pas du choix de la norme} \medskip
		
		\theorem{thm}{
			Dans un $K$ espace vectoriel de \emph{dimension finie}, 
			toute série absolument \\ convergente est convergente {\small " CVA $\Rightarrow$ CV "}
		} \medskip \\ \textit{\small Sera démontré ultérieurement.}\footnote[1]{ 
			TODO : add ref	
		} \\

		\emph{Attention !} Faux dans un EVN quelconque ! \\
			
		\theorem{lem}{
			Soit $(E,N)$ un $K$ espace vectoriel normé de dimension finie \\ On supposons que 
			$\sum u_n$ CVA. Alors $\norm{\suminf u_n} \leq \suminf \norm{u_n}$
		} \medskip
		
\section{Complément sur les séries numériques}

		\rappel{Soit $z\in \mathbf{C}$ alors $\sum z^n$ CV $ \Rightarrow$ $\abs{z} <1$}

		\hspace*{2.5cm} -> Lorsque $\abs{z} < 1$ on a \highlight{$\suminf z^n = \frac{1}{1-z}$}
		
		\hspace*{2.5cm} -> On définie \highlight{ $\mathrm{exp}(z) = \suminf \frac{z^n}{n!}$ } \medskip

	\subsection{Règle de \emph{Dalembert}}
	
		\medskip
		\namedtheorem{Théorème : Règle de \emph{Dalembert}}{
			Soit $(u_n) \in\left(\mathbf{C}^*\right)^{\mathbf{N}}$ \\
			On suppose l'existence de $\ell \in \mathbf{R}\cup \{+\infty \}$ tel que $\abs{\frac{u_{u+1}}{u_n}} \to \ell$ \\
			\emph{Alors} : \hspace*{11pt} {\small 1)} $\ell<1~\Rightarrow~\sum u_n$ CVA \\ 
			\hspace*{50pt} {\small 2)} $\ell>1~\Rightarrow~\sum u_n$ DVG
		}{RglDalemb}

		\begin{proof}
			{\small 1)} On suppose $\ell<1$ et on note $r_n = \abs{\frac{u_{u+1}}{u_n}}$. On pose $\theta \in [\ell,1] $ et $\varepsilon = \theta -\ell$
			On a alors \\ $\exists n_0 \in \mathbf{N} ~:~ \forall n\geq n_0 ,~\abs{r_n -\ell} <\varepsilon$  soit en particulier $r_n < \ell+\varepsilon 
			=\theta$ Ainsi $\forall n\geq n_0 ,~\abs{u_{n+1}} < \theta\abs{u_n}$ \\ et $ \abs{u_n} \leq \theta^{n-n_0} \abs{u_{n_0}}$ {\small (REC) }
			On a alors $\forall n\geq n_0 ,~\abs{u_n} \leq \underbrace{\theta^{-n_0} \abs{u_{n_0}}}_{\mathrm{cte}} \theta^n$ 
			or $\sum\theta^n$ converge car $\theta \in ]0,1[$ \\donc par théorème de comparaison $\sum \abs{u_n} $ converge.
			\vspace*{0.2cm}\\ {\small 2)} On suppose $\ell>1$ et on fixe $\theta \in \mathbf{R}$ tel que $ 1<\theta <\ell$, 
			on a alors $\exists n_0 \in \mathbf{N} ~:~\forall n\geq n_0 ,~r_n > \theta$ (\dots)\\
			on obtient $\abs{u_n} \to +\infty$ donc $u_n \underset{n}{\nrightarrow} 0$ donc $\sum u_n$ DVG
		\end{proof} \medskip 
		

	\subsection{Séries alternées}
		
		\vspace{-15pt}
		\traitd
		\paragraph{Défnition}
			La série réelle $\sum u_n$ est dite \emph{alternée} si $\left\{\ard \forall n\in \mathbf{N} ,~u_n = (-1)^n \abs{u_n} \\
			\forall n\in\mathbf{N},~u_n = (-1)^{n+1} \abs{u_n} \arf\right.$ 
		\trait
	
		\namedtheorem{Théorème : Critère spécial des série alternées}{
			Soit $(u_n)$ une suite, on suppose \\
			{\small 1)} $\sum u_n$ est alternée \\ 
			{\small 2)} $u_n \rightarrow 0 $ \\
			{\small 3)} $\left( \abs{u_n}\right)_{n\geq 0}$ décroit.  \\
			alors \highlight{$\sum u_n$ converge } et de plus, $\forall n\in \mathbf{N}$ \\
			-> $\abs{R_n} \leq \abs{u_{n+1}}$ \\
			-> $R_n$ et $u_{n+1}$ ont le même signe \\ 
			-> $S$ est compris entre $U_n$ et $U_{n+1}$ 
		}{CSSA} \medskip

	\subsection{Sommation des relations de comparaisons}
			
		\namedtheorem{Théorème : Cas convergent}{
			Soit $(u_n) ,~(v_n) \in\mathbf{R}^{\mathbf{N}}$ 
			et $v_n\geq 0,~\forall n\geq n_0$. On suppose que \\
			$\sum u_n$ et $\sum v_n$ converge et on pose $R_n =\sum_{k=n+1}^{+\infty} u_n$ 
			et $R'_n =\sum_{k=n+1}^{+\infty} v_n$ \\ 
			Alors : \\
			{\small 1)} $u_n = o_{n\rightarrow +\infty} (v_n) ~\Rightarrow ~ R_n = o_{n\rightarrow +\infty} (R'_n)$ \\ 
			{\small 2)} $u_n = \bigcirc_{n\rightarrow +\infty} (v_n) ~\Rightarrow ~ R_n= \bigcirc_{n\rightarrow +\infty} (R'_n)$ \\
			{\small 3)} $u_n \underset{n\rightarrow +\infty}{\sim} v_n ~\Rightarrow ~ R_n \underset{n\rightarrow +\infty}{\sim} R'_n$
		}{SumCompCV} \medskip
		
		\namedtheorem{Théorème : Cas divergent}{
			Soit $(u_n) ,~(v_n) ~\in\mathbf{R}^{\mathbf{N}}$ 
			et $v_n\geq 0,~\forall n\geq n_0$. On suppose que \\
			$\sum u_n$ et $\sum v_n$ diverge et on note $U_n = \sum_{k=0}^{n} u_n$ 
			et $V_n = \sum_{k=0}^{n} v_n$ \\
			Alors : \\
			{\small 1)} $u_n = \circ_{n\rightarrow +\infty} (v_n) ~\Rightarrow ~ U_n = \circ_{n\rightarrow +\infty} (V_n)$ \\ 
			{\small 2)} $u_n = \bigcirc_{n\rightarrow +\infty} (v_n) ~\Rightarrow ~ U_n= \bigcirc_{n\rightarrow +\infty} (V_n)$ \\
			{\small 3)} $u_n \underset{n\rightarrow +\infty}{\sim} v_n ~\Rightarrow ~ U_n \underset{n\rightarrow +\infty}{\sim} V_n$
		}{SumCompDV} \medskip
			
		\namedtheorem{Théorème de \emph{Cesàro}}{
			Soit $(u_n) \in\mathbf{R}^{\mathbf{N}}$\\
			{\small 1)} Si $u_n \to \lambda$ avec $\lambda \in\mathbf{R}$, alors $\frac{1}{n+1} \sum_{k=0}^{n} u_k \to \lambda$ \\
			{\small 2)} Si $u_n \to +\infty$ alors $\frac{1}{n+1} \sum_{k=0}^{n} u_k \to +\infty$
		}{THMCesaro}
			
		\begin{proof}
			{\small 1)} Supposons $u_n \to \lambda$ alors $u_n - \lambda = \circ (1)$, on pose ensuite $v_n = 1$ alors $\sum v_n$ diverge 
			et d'après le théorème de sommation en cas divergent \\ $\sum_{k=0}^{n} u_k - \lambda = \circ ( \sum_{k=0}^{n} 1 ) ~
			\Rightarrow ~\frac{1}{n+1} ( \sum_{k=0}^{n} u_k ) - \lambda \to 0$ \\
			{\small 2)} Supposons $u_n \to +\infty$ et posons $a_n = \frac{1}{n+1} \sum_{k=0}^{n} u_k$
			Soit $A\in\mathbf{R}$ et $A'=A+1$ \\ Soit $n_0 \in\mathbf{N} ~:~\forall n\geq n_0 ,~u_n >A'$ , puis pour $n\geq n_0$ avec $C = \sk{0}{n_0-1} u_k$ :\\
			$a_n = \frac{1}{n+1} ( C + \sum_{k=n_0}^{n} u_k)$ 
			donc $a_n > \frac{C}{n+1} + A'\frac{n+1-n_0}{n+1} = A' + \frac{C-n_0A'}{n+1}$ \\ 
			Soit $n_1\geq n_0$ tel que $\forall n\geq n_1 ,~\abs{\frac{C-A'n_0}{n+1}} <1$ alors $\forall n\geq n_1 ,~a_n > A$ d'où $a_n \to +\infty$
		\end{proof} \medskip
		

\section{Produit de deux séries absolument convergentes}

		\vspace{-15pt}
		\traitd
		\paragraph{Produit de \emph{Cauchy}}
			Soient $\sum u_n$ et $\sum v_n$ des séries quelconques (convergentes ou non) de nombres complexes.\\
			On pose $\forall n \in \mathbf{N}$ : $w_n = \sum\limits_{i+j=n} u_i v_j = \sum_{k=0}^{n} u_k v_{n-k}$ (somme finie !) \\
			La série $\sum w_n$ est appelée \emph{produit de \emph{Cauchy}} de $\sum u_n$ et $\sum v_n$.
		\trait
				
		\textbf{\emph{Attention !} \\Lorsque $\sum u_n$ et $\sum v_n$ convergent on a pas forcément 
			$\left(\sum u_n \right) \times \left(\sum v_n \right) = \sum w_n$} \medskip \\
		
		\theorem{thm}{
			Si $\sum u_n$ et $\sum v_n$ \emph{convergent absolument} alors :\\
			{\small 1)} $\sum w_n$ CVA \\
			{\small 2)} $\left( \suminf u_n \right) \times \left( \suminf u_n \right) = \suminf w_n$
		} \medskip

		\textit{\small Signalé :}

		\namedtheorem{Théorème de \emph{Mertens}}{
			Si $\left\{ \ard \sum u_n ~\mathrm{CVA} \\ \sum v_n ~\mathrm{converge}
			\arf \right.$ \\alors $\sum w_n$ converge et $\left( \suminf u_n \right) \times \left( \suminf u_n \right) = \suminf w_n$ 
		}{TH-Mert} \medskip
		

	\section{Dualité série-suite}
		
		\textit{Toute suite peut-être envisagée comme une série}
			
		Ici $(E,N)$ est un EVN de dimension finie.\\${}$ \\On pose $\forall n\in\mathbf{N}^* ~~ \left\{ \ard  b_0 = a_0 \\ b_n = a_n - a_{n-1} \arf
			\right.~~$ On a alors pour $n\in \mathbf{N}$ \\
			\[ \sum_{k=0}^{n} b_k = b_0 + \sum_{k=1}^{n} (a_k - a_{k-1} ) = a_0 +a_n - a_0 = a_n ~~~~\mathrm{soit} ~~~~ a_n = \sum_{k=0}^{n} b_k\]
			On sait ensuite que $(a_n)$ converge si et seulement si $\sum b_k$ converge donc 
			
		\begin{center}
			\highlight{ $ (a_n) ~\mathrm{converge} ~\Leftrightarrow ~ \sum a_n - a_{n-1} ~ \mathrm{converge}$ }
		\end{center} \medskip
		

\fin