
% Chapitre 4 : Suites de fonctions 

\textit{Cadre : $E$, $F$ des espaces vectoriels normés de dimensions finies ; $A\subset E$.
\\\hspace*{2.5cm} $f_n : A\longrightarrow f ~;~f : A\longrightarrow F$}
\minitoc
\section{Convergences}

\traitd
\paragraph{Convergence simple} ${}$\\
    Soit $f\in \left( F^A \right)$ et $\suite{f} \in \left( F^A \right)^{\N}~~$  On dit que $\suite{f}$ \underline{converge simplement vers 
    $f$ sur $A$} si \[ \forall x\in A , ~ f_n(x) \underset{n\rightarrow +\infty}{\longrightarrow} f(x) \] \trait \vspace*{-1.2cm}
\\{\small \textsc{Attention !} La convergence simple ne préserve pas la continuité !} \traitd
\paragraph{Convergence uniforme} ${}$\\
    Soit $f\in \left( F^A \right)$ et $\suite{f} \in \left( F^A \right)^{\N}~~$  On dit que $\suite{f}$ \underline{converge uniformément 
    vers $f$ sur $A$} si \[ \forall \varepsilon >0 ,~\exists n_0 \in\N ~\mathrm{tel} ~\mathrm{que} ~\forall n\geq n_0 ~;~ \forall x\in A~;~
    \norm{f_n(x) - f(x) } \leq \varepsilon \]	 \trait	
\thm{ch4L1}{Lemme}{CVUImplCVS}{La convergence uniforme entraine la convergence simple.}		
\vspace*{0.5cm} \\ \thm{ch4th1}{Théorème}{CVUConta}{On suppose $a\in A ~;~ f:A\rightarrow F ~;~ \forall n\in\N , ~f_n:A\rightarrow F$ et \\
    $\ard \bullet ~~\forall n\in\N ,~ f_n$ est $\cont^0$ au point $a \\ \bullet ~~ \suite{f} $ CVU sur $A$ vers $f \arf$ \hspace*{0.5cm}
    Alors $f$ est $\cont^0$ au point $a$}
    \begin{proof}
    Soit $\varepsilon >0$ \\ Vu la CVU, soit $n_0\in\N$ tel que $\forall x\in A ,~\norm{f_n(x)-f(x)} \leq \frac{\varepsilon}{3}$\\
    Vu $f_{n_0}$ est $\cont^0$ au point $a$, soit $\delta >0$ tel que $\forall x\in A,~\norm{x-a} <\delta \Rightarrow \norm{f_{n_0}(x) - 
    f_{n_0}(a)} <\frac{\varepsilon}{3}$ \\ On a alors $d(f(x),f(a)) \leq \dd (f(x),f_{n_0}(x)) + \dd (f_{n_0}(x),f_{n_0}(a)) + 
    \dd (f_{n_0}(a) , f(a)) <\varepsilon$ \\ Ainsi $f$ est $\cont^0$ au point $a$
    \end{proof}
${}$ \\ \thm{ch4th1c}{Corollaire}{CVUCont}{Toute limite uniforme sur $A$ d'une suite de fonctions continues sur $A$ est continue sur $A$.}
\vspace*{0.5cm} \\ \thm{ch4th1c2}{Corollaire}{CVUVoisinageCont}{Soit $f:A\rightarrow F ~;~ f_n:A\rightarrow F , ~\forall n\in\N$  
    Soit $a\in A$ On suppose que \\ $\ard \bullet ~~\forall n\in\N ,~f_n$ est $\cont^0$ au point $a \\ \bullet ~~ \suite{f} $ converge 
    uniformément vers $f$ sur un voisinage relatif de $a$ dans $A \arf$ \\ Alors \hspace*{0.5cm} $f$ est $\cont^0$ au point $a$}		
\vspace*{0.5cm} \\ \textit{Le Lemme suivant permet d'établir l'\underline{absence} de convergence uniforme}
    \\ \thm{ch4L2}{Lemme}{NonCVU}{Soit $f,f_n  :A\rightarrow F ~;~ B\subset A$ \textsc{alors} \\ $\Big( \exists \suite{x} \in B^{\N}$ tel que 
    $f_n(x) - f(x) \underset{n}{\nrightarrow} 0 \Big) ~\Leftrightarrow  ~\textsc{non} \Big( \suite{f} $CVU/$A$ vers $f \Big)$}		
\vspace*{0.5cm} \\ \thm{ch4th2}{Théorème de la double limite}{DoublLimit}{Soit $f,f_n :A\rightarrow F ~;~ a\in A$ On suppose que \\ 
    $\ard \bullet ~~ \forall n\in\N  ,~\exists b_n \in F$ tel que $f_n(x) \stox{a} b_n \\ \bullet ~~ \suite{f} $ converge uniformément sur $A$ 
    vers $f \arf$ \\ Alors \hspace*{0.5cm} $\ard ${\tiny (1)} $ \exists \beta \in F$ tel que $b_n\ston \beta \\ ${\tiny (2)} $ f(x) 
    \stox{a} \beta \arf $}
    \\\textit{\small En particulier $\limit{x}{a} \big(\limit{n}{+\infty} f_n(x)\big) = \limit{n}{+\infty} \big( \limit{x}{a} f_n(x) \big) $}
    \vspace*{0.2cm}\\\textsc{Attention !} Faux sans la convergence uniforme !
    \begin{proof}
    On suppose que $b_n \ston \beta$ \\ Soit $\varepsilon >0$ ; soit $n_0\in \N$ tel que $\forall n\geq n_0 ,~\dd (f(x),f_n(x) ) <\frac{1}{3}
    \varepsilon$ et $\dd (b_n ,\beta )< \frac{1}{3}\varepsilon$ \\ Soit $\delta >0$ tel que $\forall x\in B(a,\delta ) ,~\dd (f_{n_0}(x) , 
    b_{n_0})<\frac{1}{3}\varepsilon$ \\ Pour un tel $x$ on a $\dd (f(x),\beta ) <\varepsilon$ donc $f(x) \stox{a} \beta$ d'où {\tiny (1)} $
    \Rightarrow$ {\tiny (2)} \vspace*{0.3cm} \\ Par convergence uniforme soit $p\in\N$ tel que $\forall n\geq p ,~\forall x\in A 
    ,~\norm{f_n(x)-f(x)}\leq 1$ \\alors $\norm{f_n(x)-f_p(x)} \leq 2$ et par passage à la limite $b_n\in B(b_p ,2)$ compact vu $F$ de dimension 
    finie. \\Montrons que $\suite{b}$ admet au plus une valeur d'adhérence : 
    \\On suppose $\left\{ \ard b_{\varphi (n)} \ston \beta_1 \\ b_{\psi (n)} \ston \beta_2 \arf \right.$ alors en appliquant le début de la 
    démo on a $\left\{\ard f(x) \stox{a} \beta_1 \\ f(x) \stox{a} \beta_2 \arf \right.$ \\ Donc $\beta_1 = \beta_2$ et par théorème 
    (\ref{ConvCompact}) $b_n \ston \beta$ d'où {\tiny (1)}
    \end{proof}
\traitd
\paragraph{Norme infinie}
    Soit $\varphi :A (\subset E) \rightarrow F ,~a\neq\emptyset$ et $\varphi$ bornée alors on pose 
    \[\norm{\varphi}_{\infty} ~=~ \underset{x\in A}{\mathrm{sup}} \norm{\varphi (x) }\] \trait
\thm{ch4L3}{Lemme}{CVUNormInf}{Soit $f,f_n :A\rightarrow F$ \textsc{alors} \\$ \suite{f}$ CVU sur $A$ vers $f 
    ~\Leftrightarrow ~\left\{\ard \norm{f_n-f}_{\infty}$ est bien définie \textsc{apcr}$ \\ \norm{f_n-f}_{\infty} \stox{+\infty} 0\arf\right.$}		
 \newpage \underline{Cas des fonctions bornées :} \\ 
     Soit $\mathscr{B} (A,F) = \{f :A\rightarrow F ~;~ f$ est bornée $\}$ alors $\ard ${\small 1)} $\mathscr{B} (A,F)$ est un $K$ espace 
     vectoriel $ \\ ${\small 2)} $\norm{.}_{\infty}$ est une norme sur $\mathscr{B} (A,F) \arf $ 
     \\ \textit{ $\norm{.}_{\infty}$ est dite norme de la convergence uniforme.}	 		 	

\section{Série de fonctions}
    Soit $\suite{g} \in \left( F^A \right)^{^{\N}}$ on pose $f_n = g_0 + g_1 + \cdots + g_n  ~(:A\rightarrow F)$ \\
     Etudier la série de fonction $\sum_{n\geq 0} g_n$ revient à étudier la suite de fonction $\suite{f}$. \\
     On dit que $\ard $ {\tiny (1)} $\sum g_n $ converge simplement sur $A$ si $\suite{f} $ converge simplement sur $A \\ $ {\tiny (2)} $\sum 
     g_n$ converge uniformément sur $A$ si $\suite{f}$ converge uniformément $A \arf $\vspace*{0.3cm}\\ 
     \hspace*{1.5cm} \underline{Lorsque $\sum g_n$ converge simplement sur $A$} \\ 
     $\left| \ard \forall x \in A ,~f_n(x) = \sk{k}{n} g_k (x) \ston f(x) $ ainsi $\sum g_n (x)$ converge et $\suminf g_k(x) = f(x) \\ 
     $On pose $\suminf g_k : \ard A\rightarrow F \\ x\mapsto \infty g_k (x) \arf$ et on pose $R_n = \sk{n+1}{+\infty} g_k \arf \right.$	
     \vspace*{0.5cm} \\ \thm{ch4L4}{Lemme}{CarSerieCVU}{$\sum g_n $ converge uniformément sur $A$ \\ $\Leftrightarrow ~\sum g_n$ converge 
     simplement sur $A$ et $\suite{R}$ converge uniformément sur $A$}	\\ 	\traitd
     \paragraph{Convergence normale}
         Soit $g_n :A\rightarrow F , ~\forall n\in\N$ \\ $\sum g_n$ est dite \underline{normalement convergente sur $A$} si \[ \ard ${\small 1)} 
         $\norm{g_n}_{\infty}$ existe à partir d'un certain rang $n_0 \\ ${\small 2)}  \highlight{$\sum \norm{g_n}_{\infty}$ converge}$ \arf \] 	 			\vspace*{-0.45cm}\trait 
     \thm{ch4th3}{Théorème : Caractérisation de la convergence normale}{CarCVN}{${}$ \\ $\sum _n$ converge normalement sur $A$ \\ 
     $ \Leftrightarrow $ Il existe $n_0 \in\N$ et $\suite{\alpha} $ une suite réelle tels que \\ \centering $ \ard ${\tiny (1)} $\forall n\geq 
     n_0 ,~\forall x\in A ,~ \norm{g_n (x) } \leq \alpha_n \\ ${\tiny (2)} $\sum\limits_{n\geq n_0} \alpha_n $ converge $ \arf $}
     \begin{proof}
     \fbox{$\Rightarrow$} Clair en posant $\forall n\geq n_0 ,~\alpha_n = \norm{g_n}_{\infty}$ \\ \fbox{$\Leftarrow$} On suppose que 
     $\suite{\alpha}$ vérifie {\tiny (1)} et {\tiny (2)} ; soit $n\geq n_0 $ alors $\forall x\in A ,~ \norm{g_n(x)} \leq \alpha_n $ 
     \\Donc $0\leq \norm{g_n}_{\infty} \leq \alpha_n$ et vu {\tiny (2)} par comparaison de série à terme général positif 
     $\sum \norm{g_n}_{\infty}$ converge 
     \end{proof} ${}$ \\ 
     \thm{ch4th4}{Théorème}{CVNImplCVUCVA}{La convergence normale entraine la convergence uniforme et la convergence absolue en tout point.}
     \begin{proof}
     Soient $n_0$ et $\suite{a}$ qui vérifient la caractérisation de la convergence normale (\ref{CarCVN}). Soit $x\in A$ \\
     \hspace*{1cm} $\left| \ard \forall n\geq n_0 ,~0\leq \norm{g_n(x)} \leq a_n$ or $\sum a_n$ converge donc $\sum \norm{g_n(x)}$ converge $ \\ 
     $Donc $\sum g_n (x)$ converge absolument et vu la dimension finie de $F$, $\sum g_n(x)$ converge $ \arf \right.$ \\ 
     Ainsi $\sum g_n$ converge simplement. \vspace*{0.3cm} \\ 
     Soit $n\geq n_0$ ; soit $x\in A$, on a $R_n(x) = \sum_{k=n+1}^{+\infty} g_k (x)$ donc $\norm{R_n(x) - 0} \leq \sk{n+1}{+\infty} 
     \norm{g_k(x)} \ leq \underbrace{\sk{n+1}{+\infty} a_k}_{=~\rho_n }$ \\ 
     Or $\rho_n \ston 0$ donc $\norm{R_n - 0}_{\infty} \ston 0$ ainsi par théorème (\ref{CVUNormInf}) $\suite{R}$ converge uniformément sur $A$ 
     vers $0$ \vspace*{0.3cm} \\ On a alors $\sum g_n$ converge uniformément sur $A$ (\ref{CarSerieCVU}). 
     \end{proof} 
     ${}$ \\ \thm{ch4th5}{Théorème}{CVUContSeriea}{Soit $g_n: A\rightarrow F ,~\forall n\in N~;~a\in A$ On suppose que \\ $\ard \bullet ~\forall 
     n\in N ,~g_n$ est $\cont^0$ en $a \\ \bullet ~\sum g_n$ converge uniformément sur $A \arf $ \textsc{alors} $\suminf g_k $ est $\cont^0$ au 
     point $a$}
    \begin{proof}
    $f_n = \sk{0}{n} g_k$ est $\cont^0$ au point $a$ et $f_n \ston f$ uniformément sur $A$ où $f= \suminf g_k$ \\
    Alors par théorème (\ref{CVUConta}), $f$ est $\cont^0$ au point $a$.
    \end{proof} 
    ${}$ \\ \thm{ch4th6}{Théorème de la double limite (Séries)}{DoublLimSerie}{Soit $g_n :A\rightarrow F ,~\forall n\in \N ~;~ a\in 
    \overline{A}$ \\ On suppose que $\forall n\in \N ,~g_n (x) \stox{a} c_n \in F$ et $\sum g_n $ converge uniformément sur $A$ \\
    \hspace*{1cm} \textsc{alors} $\ard $ {\tiny (1)} $\sum c_n$ converge $ \\ $ {\tiny (2)} $\suminf g_n(x) \stox{a} \suminf c_n \arf$ }
    \\\textit{\small En particulier $\limit{x}{a} \suminf g_k(x) ~=~ \suminf \limit{x}{a} f_n(x) $} 
    \begin{proof}
    On pose $f_n=\sk{0}{n} g_k$ et $f=\suminf g_k$ \\
    Ainsi $f_n\ston f$ uniformément sur $A$ et $\forall n\in\N ,~f_n\stox{a}\sk{0}{n}c_k=b_n $\\
    Par théorème de la double limite pour les suites (\ref{DoublLimit}) $\ard $ {\tiny (1)} $\exists \beta \in F ~:~b_n \ston \beta \\ $ 
    {\tiny (2)} $f(x) \stox{a} \beta \arf$ \\$\begin{array}[b]{l} $ {\tiny (1)} donc $\sum c_k $ converge et $\beta = \suminf c_k \\ $ 
    {\tiny (2)} donc $\suminf g_k(x) \stox{a} \suminf c_k \arf $
    \end{proof}	

\section{Intégration et dérivation}
\subsection{Cas général}
    \underline{Question :} Si $\forall t\in [a,b] ,~f_n(t) \ston f(t)$, a-t-on $\int_a^b f_n(t) \dd t \ston \int_a^b f(t) \dd t~$ ?\\
    \textsc{\underline{non !}} \textit{\small Exemple : \\ 
    $f_n : \R^+ \rightarrow \R$ telle que si $0\leq x\leq \frac{1}{n}$, $f_n(x) = n^2x$ ; 
    si $\frac{1}{n} \leq x \leq \frac{2}{n},~f_n(x) = 2n-n^2x$ ; si $x\geq \frac{2}{n} ,~f_n(x) = 0$}
    \vspace*{0.5cm} \\ \thm{ch4th7}{Théorème}{IntCVU}{Soit $a<b$ réels ; dim$F<\infty$. On suppose \\
    $\forall n\in\N^*,~f_n \in \cont^0 ([a,b] ,F)$ et $\suite{f}$ converge uniformément sur $[a,b]$ vers $f$ \\ \hspace*{1cm}
    Alors $\ard $ {\tiny (1)} $f\in\cont^0 ([a,b],F) \\ $ {\tiny (2)} $\int_a^b f_n(x) \dd x \ston \int_a^b f(x) \dd x \arf$}
    \\\textit{\small Formellement $\limit{n}{+\infty} \int_a^b f_n(x) \dd x ~=~ \int_a^b \limit{n}{+\infty} f_n(x) \dd x$}
    \begin{proof}
    {\tiny (1)} Connu.\\
    {\tiny (2)} On note $u_n = \int_a^b f_n \in F$ et $l= \int_a^b f \in F$ \\
    Alors $\norm{u_n-l} \ leq \int_a^b \norm{f_n(x) - f(x) } \dd x \leq \int_a^b \norm{f_n-f}_{\infty} \dd x = (b-a) \norm{f_n-f}_{\infty} 
    \ston 0$ \\Par théorème des gendarmes $\norm{u_n-l} \ston 0$ soit $u_n \ston l$
    \end{proof}
    ${}$ \\ \thm{ch4th7c}{Corollaire}{IntCVUSerie}{Soit $a<b$ réels ; dim$F<\infty$. On suppose \\
    $\forall n\in\N ,~g_n \in \cont^0 ([a,b],F)$ et $\sum g_n$ converge uniformément sur $[a,b]$ \\
    \hspace*{1cm} Alors $\ard $ {\tiny (1)} $\suminf g_k \in \cont^0 ([a,b],F) \\ $ {\tiny (2)} $\sum \int_a^b g_n$ converge et 
    $\suminf \int_a^b g_k(x) \dd x = \int_a^b \suminf g_k(x) \dd x \arf $}
    \vspace*{0.5cm} \\ \thm{ch4L5}{Lemme}{PrimCVUSegment}{Soit $\varphi_n \in\cont^0 (I,F) ,~\forall n\in\N ~;~I$ i.r.n.t. ; dim$F<\infty 
    ~;~a\in I$ \\ On suppose que $\varphi_n \ston \varphi$ \underline{uniformément sur tout segment de $I$} et on pose \\
    $\Phi_n(x) = \int_a^x \varphi_n(t) \dd t ~;~\Phi (x) = \int_a^x \varphi (t) \dd t$ \\ ${}$ \\ \hspace*{1cm}
    Alors $\Phi_n \ston \Phi$ uniformément sur tout segment de $I$.}
    \vspace*{0.5cm} \\ \thm{ch4th8}{Théorème}{ConvDerives}{On suppose $\left| \ard \bullet ~\forall n\in\N ,~f_n \in\cont^1(I,F) \\ \bullet  ~
    (f_n)$ converge simplement sur $I$ vers $f \\ \bullet~(f'_n)$ converge vers $h$ \underline{uniformément sur tout segment de $I$} $
    \arf\right.$ \\ Alors $\ard $ {\tiny (1)} $(f_n)$ converge vers $f$ uniformément sur tout segment de $I \\ $ {\tiny (2)} $f\in\cont^1(I,F)$ 
    et $f'=h \arf$ }
    \begin{proof}
    \underline{Soit $a\in I$}, $\forall x\in I ,~f_n(x)-f_n(a) = \int_a^x f'_n (t) \dd t = \Phi_n(x) ,~\forall n\in\N$ \\
    D'où $f(x)-f(a) = \int_a^x h \dd t = \Phi(x)$ vu la CVU sur $[a,x]$ (éventuellement $[x,a]$) et $\Phi$ est $\cont^1$ avec $\Phi'=h$ \\
    donc $f=f(a)+\Phi$ est $\cont^1$ et $f'=h$ \\
    \underline{Soit $S$ un segment inclu dans $I$}, vu $(\Phi_n)$ converge vers $\Phi$ uniformément sur tout segment de $I$ on a \\
    $\forall n\in\N , ~0\leq \norm{f_n-f}_{\infty ,S} \leq \norm{f_n(a)-f(a)} + \norm{\Phi_n-\Phi}_{\infty,S} \ston 0$
    \end{proof}
    \textsc{Attention} : La convergence uniforme ne conserve pas la dérivabilité !
    \vspace*{0.5cm} \\ \thm{ch4th8c}{Corollaire}{ConvDeriveeSeries}{On suppose $\left| \ard \bullet ~\forall n\in\N ,~g_n \in\cont^1(I,F) \\ 
    \bullet ~\sum g_n$ converge simplement sur $I \\ \bullet~\sum g'_n$ converge \underline{uniformément sur tout segment de $I$} $
    \arf\right.$ \\ Alors $\ard $ {\tiny (1)} $\sum g_n$ converge uniformément sur tout segment de $I \\ $ {\tiny (2)} $\suminf g_n \in 
    \cont^1(I,F)$ et $\big(\suminf g_n\big)'=\suminf g'_n \arf $ } \\
\subsection{Application aux matrices}
    ${}$ \\ \thm{ch4L6}{Lemme}{DerExpMat}{Soit $A\in\M_n(K)$, on pose $\phi ~\appli{\R}{t}{\M_p(K)}{\exp(At)}$\\
    Alors $\phi \in\cont^1 \big(\R,\M_p(K)\big)$ et $\forall t\in\R ,$ \highlight{$\phi'(t) = A.e^{tA} = e^{tA}.A$} }
    \vspace*{0.5cm} \\ \thm{ch4L7}{Lemme}{ProdMSerieMat}{On suppose $\left\{ \ard \sum U_n$ converge dans $\M_p(K)\\ M\in\M_p(K) \arf\right.$\\
    Alors $\sum M.U_n$ converge et $\suminf M.U_n = M.\suminf U_n$} \newpage
    ${}$ \\ \thm{ch4L8}{Lemme}{ExpMatCinf}{$\phi : t\to e^{tA}$ est $\cont^{\infty}$ sur $\R$}
    \vspace*{0.5cm} \\ \thm{ch4L8c}{Corollaire}{DerExpEndo}{Soit $u\in\lin(E)$ avec $\dim(E)<\infty$, on pose $\varphi~
    \appli{\R}{t}{\lin(E)}{\exp(tu)}$ \\ 
    Alors $\varphi\in\cont^1\big(\R,\lin(E)\big)$ et $\forall t\in\R ,~\varphi'(t) = u\circ e^{tu} = e^{tu}\circ u$}
\section{Approximations uniformes}
    ${}$ \\ \thm{ch4th9}{Théorème de \textsc{Weierstrass}}{ThWeierstrass}{Toute fonction $f\in\cont^0([a,b],K)$ \underline{continue} sur un 
    \underline{segment} y est limite uniforme \\ d'une suite de fonctions polynomiales.}
    \begin{proof}
    Sera vu dans le cours de probabilité (\ref{X} <- à compléter)
    \end{proof}
    On note \highlight{$\Esc\big([a,b],F\big)$} l'ensemble des fonctions en escalier sur $[a,b]$.
    \vspace*{0.5cm} \\ \thm{ch4L9}{Lemme}{CarLimUniEsc}{$f$ est limite uniforme d'une suite de fonctions en escalier $\Leftrightarrow$ \\
    $\forall \varepsilon >0 ,~\exists \varphi\in\Esc $ telle que $\norm{\varphi -f}_{\infty} \leq \varepsilon $ }
    \vspace*{0.5cm} \\ \thm{ch4th10}{Théorème}{ApproxEsc}{Toute fonction continue par morceaux sur un segment y est limite uniforme \\ d'une 
    suite de fonctions en escalier.}
    \begin{proof}
    Soit $a<b$ des réels. \\ \underline{1° cas} : Soit $f\in\cont^0\big([a,b],F\big)$ ; soit $\varepsilon>0$, on sait que $f$ est uniformément 
    continue sur $[a,b]$ (\ref{ThHeine}), soit donc $\delta>0$ tel que $\forall (x,y)\in[a,b]^2 ,~\mc{x-y}<\delta \Rightarrow 
    \norm{f(x)-f(y)}<\varepsilon$ \\ Soit $n\in\N$ tel que $\frac{b-a}{n}<\delta$. \\ Posons alors $x_k=a+k\frac{b-a}{n} , ~k\in\ent{0,n-1}$ et 
    $\varphi ~\appli{[a,b]}{x}{F}{\left\{ \begin{array}{ll} f(x_k)$ si $x\in[x_k,x_{k+1}[ \\ f(b) &$si $x=b \end{array}\right.}$ \\
    Ainsi $\varphi\in\Esc$ et $\norm{\varphi - f}_{\infty} \leq \varepsilon$ d'où \textsc{cqfd} \vspace*{0.2cm} \\ 
    \underline{Cas général} : Soit $f\in\cpm\big([a,b],F\big)$ ; soit donc $(a_0,\dots ,a_p)$ une subdivision de $[a,b]$ telle que $\forall 
    i\in\ent{0,p-1} , ~ f|_{]a_i,a_{i+1}[} = f_i|_{]a_i,a_{i+1}[}$ où $f_i\in\cont^0\big([a,b],F\big)$. \\
    On a alors le résultat en itérant le cas précédant.
    \end{proof}
    ${}$\hfill \highlight{$\Esc\big([a,b],F\big)$ est dense dans $\cpm\big([a,b],F\big)$} \hfill ${}$
    \vspace*{0.5cm} \\
    \begin{center}
    \fin
    \end{center}