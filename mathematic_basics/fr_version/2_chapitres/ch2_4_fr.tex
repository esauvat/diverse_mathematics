
% Chapitre 4 : Suites de fonctions 

\textit{Cadre : $E$, $F$ des espaces vectoriels normés de dimensions finies ; $A\subset E$.} \\

\textit{ et $f_n : A\longrightarrow F ~;~f : A\longrightarrow F$} \\

\minitoc

\section{Convergences}

    \traitd
    \paragraph{Convergence simple}
        Soit $f\in \left( F^A \right)$ et $\suite{f} \in \left( F^A \right)^{\N}~~$  On dit que $\suite{f}$ \emph{converge simplement vers 
        $f$ sur $A$} si 
        \[ 
            \forall x\in A , ~ f_n(x) \underset{n\rightarrow +\infty}{\longrightarrow} f(x)
       	\vspace{-25pt}
        \] 
    \traitdouble
    \paragraph{Convergence uniforme} ~\\
        Soit $f\in \left( F^A \right)$ et $\suite{f} \in \left( F^A \right)^{\N}~~$  On dit que $\suite{f}$ \emph{converge uniformément 
        vers $f$ sur $A$} si 
        \[ 
            \forall \varepsilon >0 ,~\exists n_0 \in\N ~\mathrm{tel} ~\mathrm{que} ~\forall n\geqslant n_0 ~;~ \forall x\in A~;~ \norm{f_n(x) - f(x) } \leqslant \varepsilon 
        	\vspace{-25pt}
        \]	 
    \trait	

    \theorem{lem}{
        La convergence uniforme entraine la convergence simple.
    } \medskip
    
    {\small \emph{Attention !} La convergence simple ne préserve pas la continuité !}  \medskip \\

    \theorem{thm}{
        On suppose $a\in A ~;~ f:A\rightarrow F ~;~ \forall n\in\N , ~f_n:A\rightarrow F$ et \\
        $\ard 
            \bullet ~~\forall n\in\N ,~ f_n$ est $\cont^0$ au point $a \\ 
            \bullet ~~ \suite{f} $ CVU sur $A$ vers $f 
        \arf$
        Alors $f$ est $\cont^0$ au point $a$
    }
    
    \begin{proof}
        Soit $\varepsilon >0$ \\ Vu la CVU, soit $n_0\in\N$ tel que $\forall x\in A ,~\norm{f_n(x)-f(x)} \leqslant \frac{\varepsilon}{3}$\\
        Vu $f_{n_0}$ est $\cont^0$ au point $a$, soit $\delta >0$ tel que $\forall x\in A,~\norm{x-a} <\delta \Rightarrow \norm{f_{n_0}(x) - 
        f_{n_0}(a)} <\frac{\varepsilon}{3}$ \\ On a alors $d(f(x),f(a)) \leqslant d(f(x),f_{n_0}(x)) + d(f_{n_0}(x),f_{n_0}(a)) + 
        d(f_{n_0}(a) , f(a)) <\varepsilon$ \\ Ainsi $f$ est $\cont^0$ au point $a$
    \end{proof} \medskip

    \theorem{cor}{
        Toute limite uniforme sur $A$ d'une suite de fonctions continues sur $A$ est continue sur $A$.
    } \medskip

    \theorem{cor}{
        Soit $f:A\rightarrow F ~;~ f_n:A\rightarrow F , ~\forall n\in\N$. Soit $a\in A$ On suppose que \\ 
        $\ard 
            \bullet ~~\forall n\in\N ,~f_n$ est $\cont^0$ au point $a \\ 
            \bullet ~~ \suite{f} $ converge uniformément vers $f$ sur un voisinage relatif de $a$ dans $A 
        \arf$ \\ 
        Alors $f$ est $\cont^0$ au point $a$
    } \medskip \\

    \textit{Le Lemme suivant permet d'établir l'\emph{absence} de convergence uniforme} \medskip
        
    \theorem{lem}{
        Soit $f,f_n  :A\rightarrow F ~;~ B\subset A$ \emph{alors} \\ 
        $\Big( \exists \suite{x} \in B^{\N}$ tel que $f_n(x) - f(x) \underset{n}{\nrightarrow} 0 \Big) ~\Leftrightarrow  ~\emph{non} \Big( \suite{f} $CVU/$A$ vers $f \Big)$
    } \medskip

    \namedtheorem{Théorème de la double limite}{
        Soit $f,f_n :A\rightarrow F ~;~ a\in A$ On suppose que \\ 
        $\ard \bullet ~~ \forall n\in\N  ,~\exists b_n \in F$ tel que $f_n(x) \stox{a} b_n \\ \bullet ~~ \suite{f} $ converge uniformément sur $A$ 
        vers $f \arf$ \\ Alors $ \exists \beta \in F$ tel que $b_n\ston \beta$ avec $ f(x) \stox{a} \beta$
    }{DoublLimit}

    \remarque{En particulier, on a $\limit{x}{a} \big(\limit{n}{+\infty} f_n(x)\big) = \limit{n}{+\infty} \big( \limit{x}{a} f_n(x) \big) $} \medskip
        
    \emph{Attention} ! C'est faux sans la convergence uniforme !
        
    \begin{proof}
        On suppose tout d'abord que $b_n \ston \beta$. Soit alors $\varepsilon >0$ puis $n_0\in \N$ tel que $\forall n\geqslant n_0 ,~d(f(x),f_n(x) ) <\frac{1}{3} \varepsilon$ et $d(b_n ,\beta )< \frac{1}{3}\varepsilon$ \\ 
        On peut alors considérer $\delta >0$ tel que $\forall x\in B(a,\delta )$, on a $d(f_{n_0}(x), b_{n_0})<\frac{1}{3}\varepsilon$. Ainsi, pour un tel $x$ on a $d(f(x),\beta ) <\varepsilon$ donc $f(x) \stox{a} \beta$\\ 
        
        Avec la convergence uniforme on a $p\in\N$ tel que $\forall n\geqslant p ,~\forall x\in A,~\norm{f_n(x)-f(x)}\leqslant 1$ ainsi $\norm{f_n(x)-f_p(x)} \leqslant 2$ et par passage à la limite $b_n\in B(b_p ,2)$ est compact car $F$ est de dimension finie. Il suffit alors de montrer que $\suite{b}$ admet au plus une valeur d'adhérence :
        
        Si $b_{\varphi (n)} \ston \beta_1$ et $b_{\psi (n)} \ston \beta_2$ pour $\varphi$ et $\psi$ deux extractrices, alors en appliquant le début de la 
        démo on a $f(x) \stox{a} \beta_1$ et $f(x) \stox{a} \beta_2$ donc $\beta_1 = \beta_2$ et par théorème 
        %(\ref{ConvCompact}) 
        $b_n \ston \beta$.
    \end{proof} \medskip

    \traitd
    \paragraph{Norme infinie}
        Soit $\varphi :A (\subset E) \rightarrow F ,~a\neq\emptyset$ et $\varphi$ bornée alors on pose 
        \[
			\norm{\varphi}_{\infty} ~=~ \underset{x\in A}{\mathrm{sup}} \norm{\varphi (x) }
		\vspace{-20pt}
		\] 
    \trait
    
    \theorem{lem}{
        Soit $f,f_n :A\rightarrow F$ \emph{alors} \\ 
        $\suite{f}$ CVU sur $A$ vers $f 
        ~\Leftrightarrow ~\left\{\ard \norm{f_n-f}_{\infty}$ est bien définie APCR$ \\ \norm{f_n-f}_{\infty} \stox{+\infty} 0\arf\right.$
    }

    \emph{Cas des fonctions bornées :} \\ 
        Soit $\mathscr{B} (A,F) = \{f :A\rightarrow F ~;~ f$ est bornée $\}$ alors $\ard ${\small 1)} $\mathscr{B} (A,F)$ est un $K$ espace 
        vectoriel $ \\ ${\small 2)} $\norm{.}_{\infty}$ est une norme sur $\mathscr{B} (A,F) \arf $ \\
    
        \remarque{On dit que $\norm{.}_{\infty}$ est la norme de la convergence uniforme.} \medskip


\section{Série de fonctions}
    
    Soit $\suite{g} \in \left( F^A \right)^{^{\N}}$ on pose $f_n = g_0 + g_1 + \cdots + g_n  ~(:A\rightarrow F)$ \\
    
    Étudier la série de fonction $\sum_{n\geqslant 0} g_n$ revient à étudier la suite de fonction $\suite{f}$. \\
    
    On dit alors que $\sum g_n $ converge simplement (resp. uniformément) sur $A$ si $\suite{f} $ converge simplement (resp. uniformément) sur $A$.\\

    \emph{Lorsque $\sum g_n$ converge simplement sur $A$}, on peut considérer les résultats suivants : \\ 
    
    $\left| \ard 
        \forall x \in A ,~f_n(x) = \sk{k}{n} g_k (x) \ston f(x) $ ainsi $\sum g_n (x)$ converge et $\suminf g_k(x) = f(x) \\ 
        $On pose $\suminf g_k : 
        \ard 
            A\rightarrow F \\ 
            x\mapsto \infty g_k (x) 
        \arf$ et on pose $R_n = \sk{n+1}{+\infty} g_k 
    \arf \right.$ \\

    \theorem{lem}{
        $\sum g_n $ converge uniformément sur $A$ \\ $\Leftrightarrow ~\sum g_n$ converge simplement sur $A$ et $\suite{R}$ converge uniformément sur $A$
    }
    
    \traitd
    \paragraph{Convergence normale}
        Soit $g_n :A\rightarrow F , ~\forall n\in\N$ \\ $\sum g_n$ est dite \emph{normalement convergente sur $A$} si 
        \[
            \ard 
                ${\small 1)} $\norm{g_n}_{\infty}$ existe à partir d'un certain rang $n_0 \\ 
                ${\small 2)}  \highlight{$\sum \norm{g_n}_{\infty}$ converge}$ 
            \arf 
        \vspace{-15pt}
        \]
    \trait 
    
    \namedtheorem{Théorème : Caractérisation de la convergence normale}{
        $\sum _n$ converge normalement sur $A$ \\ 
        $ \Leftrightarrow $ Il existe $n_0 \in\N$ et $\suite{\alpha} $ une suite réelle tels que \\
        $ \ard 
            ${\tiny (1)} $\forall n\geqslant n_0 ,~\forall x\in A ,~ \norm{g_n (x) } \leqslant \alpha_n \\ 
            ${\tiny (2)} $\sum\limits_{n\geqslant n_0} \alpha_n $ converge $ 
        \arf $
    }{CarCVN}

    \begin{proof}
    \fbox{$\Rightarrow$} Clair en posant $\forall n\geqslant n_0 ,~\alpha_n = \norm{g_n}_{\infty}$ \\ \fbox{$\Leftarrow$} On suppose que 
    $\suite{\alpha}$ vérifie {\tiny (1)} et {\tiny (2)} ; soit $n\geqslant n_0 $ alors $\forall x\in A ,~ \norm{g_n(x)} \leqslant \alpha_n $ 
    \\Donc $0\leqslant \norm{g_n}_{\infty} \leqslant \alpha_n$ et vu {\tiny (2)} par comparaison de série à terme général positif 
    $\sum \norm{g_n}_{\infty}$ converge 
    \end{proof} \medskip

    \theorem{thm}{
        La convergence normale entraine la convergence uniforme et la convergence absolue en tout point.
    }

    \begin{proof}
    Soient $n_0$ et $\suite{a}$ qui vérifient la caractérisation de la convergence normale %(\ref{CarCVN})
    . Soit $x\in A$ \\
    \hspace*{1cm} $\left| \ard \forall n\geqslant n_0 ,~0\leqslant \norm{g_n(x)} \leqslant a_n$ or $\sum a_n$ converge donc $\sum \norm{g_n(x)}$ converge $ \\ 
    $Donc $\sum g_n (x)$ converge absolument et vu la dimension finie de $F$, $\sum g_n(x)$ converge $ \arf \right.$ \\ 
    Ainsi $\sum g_n$ converge simplement. \vspace*{0.3cm} \\ 
    Soit $n\geqslant n_0$ ; soit $x\in A$, on a $R_n(x) = \sum_{k=n+1}^{+\infty} g_k (x)$ donc $\norm{R_n(x) - 0} \leqslant \sk{n+1}{+\infty} 
    \norm{g_k(x)} \ leq \underbrace{\sk{n+1}{+\infty} a_k}_{=~\rho_n }$ \\ 
    Or $\rho_n \ston 0$ donc $\norm{R_n - 0}_{\infty} \ston 0$ ainsi par théorème %(\ref{CVUNormInf}) 
    $\suite{R}$ converge uniformément sur $A$ 
    vers $0$ \vspace*{0.3cm} \\ On a alors $\sum g_n$ converge uniformément sur $A$. %(\ref{CarSerieCVU}). 
    \end{proof} 
    
    \newpage

    \theorem{thm}{
        Soit $g_n: A\rightarrow F ,~\forall n\in N~;~a\in A$ On suppose que \\ 
        $\ard 
            \bullet ~\forall n\in N ,~g_n$ est $\cont^0$ en $a \\ 
            \bullet ~\sum g_n$ converge uniformément sur $A 
        \arf $ \emph{alors} $\suminf g_k $ est $\cont^0$ au point $a$
    }

    \begin{proof}
    $f_n = \sk{0}{n} g_k$ est $\cont^0$ au point $a$ et $f_n \ston f$ uniformément sur $A$ où $f= \suminf g_k$ \\
    Alors par théorème %(\ref{CVUConta})
    , $f$ est $\cont^0$ au point $a$.
    \end{proof} \medskip

    \namedtheorem{Théorème de la double limite (Séries)}{
        Soit $g_n :A\rightarrow F ,~\forall n\in \N ~;~ a\in \overline{A}$ \\ 
        On suppose que $\forall n\in \N ,~g_n (x) \stox{a} c_n \in F$ et $\sum g_n $ converge uniformément sur $A$ \\
        \emph{alors} 
        $\ard 
            $ {\tiny (1)} $\sum c_n$ converge $ \\ 
            $ {\tiny (2)} $\suminf g_n(x) \stox{a} \suminf c_n 
        \arf$ 
    }{DoublLimSerie} \medskip

    \textit{\small En particulier $\limit{x}{a} \suminf g_k(x) ~=~ \suminf \limit{x}{a} f_n(x) $} \\
    
    \begin{proof}
    On pose $f_n=\sk{0}{n} g_k$ et $f=\suminf g_k$ \\
    Ainsi $f_n\ston f$ uniformément sur $A$ et $\forall n\in\N ,~f_n\stox{a}\sk{0}{n}c_k=b_n $\\
    Par théorème de la double limite pour les suites %(\ref{DoublLimit})
     $\ard 
        $ {\tiny (1)} $\exists \beta \in F ~:~b_n \ston \beta \\ 
        $ {\tiny (2)} $f(x) \stox{a} \beta 
    \arf$ \\ 
    $\begin{array}[b]{l} 
        $ {\tiny (1)} donc $\sum c_k $ converge et $\beta = \suminf c_k \\ 
        $ {\tiny (2)} donc $\suminf g_k(x) \stox{a} \suminf c_k 
    \end{array} $
    \end{proof}	\medskip


\section{Intégration et dérivation}
    \subsection{Cas général}

        \emph{Question :} Si $\forall t\in [a,b] ,~f_n(t) \ston f(t)$, a-t-on $\int_a^b f_n(t) dt \ston \int_a^b f(t) dt~$ ?\\
        \emph{\emph{non !}} \textit{\small Exemple : \\ 
        $f_n : \R^+ \rightarrow \R$ telle que si $0\leqslant x\leqslant \frac{1}{n}$, $f_n(x) = n^2x$ ; 
        si $\frac{1}{n} \leqslant x \leqslant \frac{2}{n},~f_n(x) = 2n-n^2x$ ; si $x\geqslant \frac{2}{n} ,~f_n(x) = 0$} \\

        \theorem{thm}{
            Soit $a<b$ réels ; dim$F<\infty$. On suppose \\
            $\forall n\in\N^*,~f_n \in \cont^0 ([a,b] ,F)$ et $\suite{f}$ converge uniformément sur $[a,b]$ vers $f$ \\ 
            Alors $\ard $ 
                {\tiny (1)} $f\in\cont^0 ([a,b],F) \\ 
                $ {\tiny (2)} $\int_a^b f_n(x) dx \ston \int_a^b f(x) dx 
            \arf$
        } \medskip

        \textit{\small Formellement $\limit{n}{+\infty} \int_a^b f_n(x) dx ~=~ \int_a^b \limit{n}{+\infty} f_n(x) dx$} \\
        
        \begin{proof}
        {\tiny (1)} Connu.\\
        {\tiny (2)} On note $u_n = \int_a^b f_n \in F$ et $l= \int_a^b f \in F$ \\
        Alors $\norm{u_n-l} \ leq \int_a^b \norm{f_n(x) - f(x) } dx \leqslant \int_a^b \norm{f_n-f}_{\infty} dx = (b-a) \norm{f_n-f}_{\infty} 
        \ston 0$ \\Par théorème des gendarmes $\norm{u_n-l} \ston 0$ soit $u_n \ston l$
        \end{proof} \medskip

        \theorem{cor}{
            Soit $a<b$ réels ; dim$F<\infty$. On suppose \\
            $\forall n\in\N ,~g_n \in \cont^0 ([a,b],F)$ et $\sum g_n$ converge uniformément sur $[a,b]$ \\
            Alors $\ard $ 
                {\tiny (1)} $\suminf g_k \in \cont^0 ([a,b],F) \\ 
                $ {\tiny (2)} $\sum \int_a^b g_n$ converge et $\suminf \int_a^b g_k(x) dx = \int_a^b \suminf g_k(x) dx 
            \arf $
        } \medskip

        \theorem{lem}{
            Soit $\varphi_n \in\cont^0 (I,F) ,~\forall n\in\N ~;~I$ i.r.n.t. ; dim$F<\infty ~;~a\in I$ \\ 
            On suppose que $\varphi_n \ston \varphi$ \emph{uniformément sur tout segment de $I$} et on pose \\
            $\Phi_n(x) = \int_a^x \varphi_n(t) dt ~;~\Phi (x) = \int_a^x \varphi (t) dt$ \\ ${}$ \\ 
            Alors $\Phi_n \ston \Phi$ uniformément sur tout segment de $I$.
        } \medskip

        \theorem{thm}{
            On suppose $\left| 
            \ard 
                \bullet ~\forall n\in\N ,~f_n \in\cont^1(I,F) \\ 
                \bullet  ~ (f_n)$ converge simplement sur $I$ vers $f \\ 
                \bullet~(f'_n)$ converge vers $h$ \emph{uniformément sur tout segment de $I$} $
            \arf\right.$ \\ 
            Alors $\ard 
                $ {\tiny (1)} $(f_n)$ converge vers $f$ uniformément sur tout segment de $I \\ 
                $ {\tiny (2)} $f\in\cont^1(I,F)$ et $f'=h 
            \arf$ 
        }

        \begin{proof}
        \emph{Soit $a\in I$}, $\forall x\in I ,~f_n(x)-f_n(a) = \int_a^x f'_n (t) dt = \Phi_n(x) ,~\forall n\in\N$ \\
        D'où $f(x)-f(a) = \int_a^x h dt = \Phi(x)$ vu la CVU sur $[a,x]$ (éventuellement $[x,a]$) et $\Phi$ est $\cont^1$ avec $\Phi'=h$ \\
        donc $f=f(a)+\Phi$ est $\cont^1$ et $f'=h$ \\
        \emph{Soit $S$ un segment inclu dans $I$}, vu $(\Phi_n)$ converge vers $\Phi$ uniformément sur tout segment de $I$ on a \\
        $\forall n\in\N , ~0\leqslant \norm{f_n-f}_{\infty ,S} \leqslant \norm{f_n(a)-f(a)} + \norm{\Phi_n-\Phi}_{\infty,S} \ston 0$
        \end{proof} \medskip

        \emph{Attention} : La convergence uniforme ne conserve pas la dérivabilité ! \\

        \theorem{cor}{
            On suppose $\left| 
            \ard 
                \bullet ~\forall n\in\N ,~g_n \in\cont^1(I,F) \\ 
                \bullet ~\sum g_n$ converge simplement sur $I \\ 
                \bullet~\sum g'_n$ converge \emph{uniformément sur tout segment de $I$} $
            \arf\right.$ \\ 
            Alors $\ard 
                $ {\tiny (1)} $\sum g_n$ converge uniformément sur tout segment de $I \\ 
                $ {\tiny (2)} $\suminf g_n \in \cont^1(I,F)$ et $\big(\suminf g_n\big)'=\suminf g'_n 
            \arf $ 
        }\label{DerSeriesFonctions} \medskip


    \subsection{Application aux matrices}

        \theorem{lem}{
            Soit $A\in\M_n(K)$, on pose $\phi ~\appli{\R}{t}{\M_p(K)}{\exp(At)}$\\
            Alors $\phi \in\cont^1 \big(\R,\M_p(K)\big)$ et $\forall t\in\R ,$ \highlight{$\phi'(t) = A.e^{tA} = e^{tA}.A$} 
        } \medskip

        \theorem{lem}{
            On suppose $\left\{ \ard \sum U_n$ converge dans $\M_p(K)\\ M\in\M_p(K) \arf\right.$\\
            Alors $\sum M.U_n$ converge et $\suminf M.U_n = M.\suminf U_n$
        } \medskip

        \theorem{lem}{
            $\phi : t\to e^{tA}$ est $\cont^{\infty}$ sur $\R$
        } \medskip

        \theorem{cor}{
            Soit $u\in\lin(E)$ avec $\dim(E)<\infty$, on pose $\varphi~\appli{\R}{t}{\lin(E)}{\exp(tu)}$ \\ 
            Alors $\varphi\in\cont^1\big(\R,\lin(E)\big)$ et $\forall t\in\R ,~\varphi'(t) = u\circ e^{tu} = e^{tu}\circ u$
        } \medskip


\section{Approximations uniformes}

    \namedtheorem{Théorème de \emph{Weierstrass}}{
        Toute fonction $f\in\cont^0([a,b],K)$ \emph{continue} sur un \emph{segment} y est limite uniforme \\ 
        d'une suite de fonctions polynomiales.
    }{ThWeierstrass}

    \begin{proof}
    Sera vu dans le cours de probabilité %(\ref{X} <- à compléter)
    \end{proof} \medskip

    On note \highlight{$\Esc\big([a,b],F\big)$} l'ensemble des fonctions en escalier sur $[a,b]$. \\
    
    \theorem{lem}{
        $f$ est limite uniforme d'une suite de fonctions en escalier $\Leftrightarrow$ \\
        $\forall \varepsilon >0 ,~\exists \varphi\in\Esc $ telle que $\norm{\varphi -f}_{\infty} \leqslant \varepsilon $ 
    } \medskip

    \theorem{thm}{
        Toute fonction continue par morceaux sur un segment y est limite uniforme d'une suite de fonctions en escalier.
    }

    \begin{proof}
    Soit $a<b$ des réels. \\ \emph{1° cas} : Soit $f\in\cont^0\big([a,b],F\big)$ ; soit $\varepsilon>0$, on sait que $f$ est uniformément 
    continue sur $[a,b]$ %(\ref{ThHeine})
    , soit donc $\delta>0$ tel que $\forall (x,y)\in[a,b]^2 ,~\abs{x-y}<\delta \Rightarrow 
    \norm{f(x)-f(y)}<\varepsilon$ \\ Soit $n\in\N$ tel que $\frac{b-a}{n}<\delta$. \\ Posons alors $x_k=a+k\frac{b-a}{n} , ~k\in\ent{0,n-1}$ et 
    $\varphi ~\appli{[a,b]}{x}{F}{\left\{ \begin{array}{ll} f(x_k)$ si $x\in[x_k,x_{k+1}[ \\ f(b) &$si $x=b \end{array}\right.}$ \\
    Ainsi $\varphi\in\Esc$ et $\norm{\varphi - f}_{\infty} \leqslant \varepsilon$ d'où \emph{cqfd} \vspace*{0.2cm} \\ 
    \emph{Cas général} : Soit $f\in\cpm\big([a,b],F\big)$ ; soit donc $(a_0,\dots ,a_p)$ une subdivision de $[a,b]$ telle que $\forall 
    i\in\ent{0,p-1} , ~ f|_{]a_i,a_{i+1}[} = f_i|_{]a_i,a_{i+1}[}$ où $f_i\in\cont^0\big([a,b],F\big)$. \\
    On a alors le résultat en itérant le cas précédant.
    \end{proof} \medskip

    \hfill \highlight{$\Esc\big([a,b],F\big)$ est dense dans $\cpm\big([a,b],F\big)$} \hfill ${}$
    \vspace*{0.5cm} \\
    
    \fin