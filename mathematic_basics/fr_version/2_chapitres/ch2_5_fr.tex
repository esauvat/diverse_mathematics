
% Chapitre 5 : Intégrales généralisées

\minitoc
	\section{Intégrale convergente}
		\traitd
		\paragraph{Définition}
			Soit $f\in\cpm\big([a,b[,K\big)$, l'intégrale \underline{$\int_a^b f(t)\dd t$ est dite convergente} s'il existe $\lambda
			\in K$ tel que \[\int_a^x f(t) \dd t \stox{b^-} \lambda\] La définition est analogue pour $f\in\cpm\big(]a,b],K\big)$ \trait
		\vspace*{-1.1cm} \\ \underline{Dans ce cas} on a $\int_a^b f(t) \dd t = \limit{x}{b^-} \int_a^x f(t)\dd t$
		, \underline{dans le cas contraire} on dit que l'intégrale est divergente.
		\paragraph{Exemples de référence}
			\subparagraph{$\bullet$} Soit $\alpha\in \R$, alors \highlight{$\displaystyle{\int_0^1 \frac{\dd t}{t^{\alpha}}}$ converge $ \Leftrightarrow ~\alpha<1$} et \highlight{$\displaystyle{\int_1^{+\infty} \frac{\dd t}{t^{\alpha}}}$ converge $
			\Leftrightarrow ~\alpha>1$} $\heartsuit\heartsuit$
			\subparagraph{$\bullet$} \highlight{$\displaystyle{\int_0^1 \ln t\dd t}$ converge} et $\int_0^1 \ln t\dd t = -1$ $\heartsuit$\\
			\subparagraph{$\bullet$} Soit $\alpha\in\R$, alors \highlight{$\displaystyle{\int_0^{+\infty} e^{-\alpha t}}$ converge $
			\Leftrightarrow~\alpha>0$} $\heartsuit$
		\newpage ${}$ \\ \thm{ch5L1}{Lemme}{5-L1}{Soit $f\in\cpm\big([a,b[,\R^+\big)$ on note $F(x) = \int_a^x f(t)\dd t$\\
		Alors : $\left| \begin{array}{l} $ {\tiny (1)} $F$ est croissante $ \\ $ {\tiny (2)} $\int_a^{\underline{b}} f(t) \dd t$ converge $
		\Leftrightarrow ~F$ est majorée $ \\ $ {\tiny (3)} Lorsque $\int_a^{\underline{b}} f(t) \dd t$ diverge on a $F(x) \stox{b} +\infty$ 
		\footnotemark[1]$\end{array} \right.$ }
		\vspace*{0.5cm} \\ \thm{ch5L2}{Lemme}{5-L2}{Soit $f\in\cpm\big(]a,b],\R^+\big)$ on note $F(x) = \int_x^b f(t)\dd t$\\
		Alors : $\left| \begin{array}{l} $ {\tiny (1)} $F$ est décroissante $ \\ $ {\tiny (2)} $\int_{\underline{a}}^b f(t) \dd t$ converge $
		\Leftrightarrow ~F$ est majorée $ \\ $ {\tiny (3)} Lorsque $\int_{\underline{a}}^b f(t) \dd t$ diverge on a $F(x) \stox{b} +\infty$ 
		\footnotemark[1]$\end{array} \right.$ }
		\footnotetext[1]{On écrira dans ce cas $\int_a^b f(t) \dd t = +\infty$}
		\vspace*{0.5cm} \\ \thm{ch5th1}{Théorème de comparaison des fonctions positives}{ThCompFcPos}{Soient $f,g \in\cpm\big[a,b[,\R^+\big)$, on 
		suppose que $0\leq f\leq g$ \\ Alors $\int_a^{\underline{b}} g$ converge $\Rightarrow ~\int_a^{\underline{b}} f$ converge }
		\begin{proof}
		On note $\forall x\in[a,b[ ,~F(x) = \int_a^x f(t) \dd t$ et $G(x) = \int_a^x g(t) \dd t$ \\
		\hspace*{0.5cm} $\left|\ard$ \underline{Supposons $\int_a^b g$ converge}, soit donc $\mu\in\R$ tel que $G\leq\mu \\
		$ Alors $\forall x\in[a,b[,~F(x) = \int_a^x f(t) \dd t \leq \int_a^x g(t) \dd t = G(x) \leq \mu \arf \right.$ \\
		Ainsi $F$ est majorée et par le Lemme $\int_a^{\underline{b}} f$ converge
		\end{proof}
		${}$ \\ \thm{ch5th2}{Théorème}{CompO}{Soient $f,g\in\cpm\big([a,b[,\R^+\big)$, on suppose $f(x) = \bigcirc_{x\to b} \big(g(x)\big)$ \\
		Alors $\int_a^{\underline{b}} g$ converge $\Rightarrow ~\int_a^{\underline{b}} f$ converge}
		\begin{proof} Exercice \end{proof}
		${}$ \\ \thm{ch5th3}{Théorème}{CompEq}{Soient $f,g\in\cpm\big([a,b[,\R^+\big)$, on suppose $f(x) = \underset{x\to b}{\sim} g(x)$ 
		\\ Alors $\int_a^{\underline{b}} f$ converge $\Leftrightarrow ~\int_a^{\underline{b}} g$ converge}
		\begin{proof}
		Si $f(x) = \underset{x\to b}{\sim} g(x)$ alors $f(x) = \bigcirc_{x\to b} \big(g(x)\big)$ et $g(x) = \bigcirc_{x\to b} \big(f(x)\big)$
		\end{proof}
		${}$ \\ \thm{ch5L3}{Lemme d'indépendance de la borne fixe}{IndepBornFixe}{Soit $f\in\cpm\big([a,b[,K\big)$ et $\alpha \in [a,b[$\\
		Alors $\ard ~ \bullet ~ \int_a^{\underline{b}} f$ converge $\Leftrightarrow ~\int_{\alpha}^{\underline{b}} f$ converge $ \\ ~\bullet ~$Dans 
		ce cas $\int_a^{\underline{b}} f = \int_a^{\alpha} f + \int_{\alpha}^{\underline{b}} f \arf $}
		\vspace*{0.5cm} \\ \thm{ch5L3c}{Corollaire}{5-L3c}{Soit $f\in\cpm\big([a,b[,K\big)$ telle que $\int_a^{\underline{b}} f$ converge \\
		Alors $\displaystyle{\int_x^{\underline{b}} f \stox{b} 0}$} 
		\newpage ${}$ \\ \thm{ch5L4}{Lemme}{IntGLin}{Soient $f,g\in\cpm\big([a,b[,K\big)$ ; $\lambda\in K$ avec $\int_a{\underline{b}} f$ et 
		$\int_a{\underline{b}}^g$ convergentes \\ Alors $\ard ~ \bullet ~ \int_a^{\underline{b}} \lambda f+g$ converge $ \vspace*{0.1cm} \\ 
		~\bullet ~ \int_a^{\underline{b}} \lambda f+g = \lambda \int_a^{\underline{b}} f + \int_a^{\underline{b}} g \arf $}
		\vspace*{0.5cm} \\ \thm{ch5L5}{Lemme}{IntGCompl}{Soit $f\in\cpm\big([a,b[,\C\big)$ \\ Alors $\ard ~ \bullet ~ \int_a^{\underline{b}} f$ 
		converge $\Leftrightarrow ~\int_a^{\underline{b}} \Re(f)$ et $\int_a^{\underline{b}} \Im(f)$ convergent $ \vspace*{0.1cm} \\ ~\bullet ~ 
		\int_a^{\underline{b}} f = \int_a^{\underline{b}} \Re(f) + \imath\int_a^{\underline{b}} \Im(f) \arf $}
		\vspace*{0.5cm} \\ \thm{ch5L6}{Lemme}{PrimIntG}{Soit $f\in\uuline{\cont^0} \big([a,b[,K\big)$ telle que $\int_a^{\underline{b}}$ 
		converge\\ Posons $\forall x\in[a,b[ ,~G(x) = \int_x^{\underline{b}} f$, alors $\ard ~\bullet ~G\in\cont^1\big([a,b[,K\big) \\ ~\bullet ~
		G'=-f \arf$ } \\ \traitd
		\paragraph{Intégrale généralise sur un intervalle ouvert}
			Soit $f\in\cpm\big(]a,b[,K\big) ~;~c\in]a,b[$ \\
			\hspace*{2cm} $\bullet$ On dit que \underline{$\int_{\uline{a}}^{\uline{b}} f$ converge} si $~~ \int_{\uline{a}}^c f$ et $
			\int_c^{\uline{b}} f$ converge. \\
			\hspace*{2cm} $\bullet$ Dans ce cas on pose $\int_{\uline{a}}^{\uline{b}} f = \int_{\uline{a}}^c f + \int_c^{\uline{b}} f$ \trait
		\vspace*{-1.1cm} \\ $\rightarrow$ \uline{ex} : $\int_{-\infty}^{+\infty} e^{-t^2} \dd t$ converge et \highlight{$\int_{-\infty}^{+\infty} 
		e^{-t^2} \dd t =\sqrt{\pi} $} \\${}$
	\section{Convergence absolue}
		\traitd
		\paragraph{Définition} 
			Soit $f\in\cpm\big[a,b[,K\big) ~a<b$ on dit que \\ \hspace*{2cm} $\ard \bullet$ $\int_a^{\uline{b}} f $ \underline{converge absolument} 
			$ \\ \bullet $ \underline{$f$ est intégrablle sur $[a,b[$} $\arf$\hspace*{0.5cm} si $\displaystyle{\int_a^{\uline{b}}\mc{f}}$ converge. 
			\vspace*{0.2cm}\trait
		\underline{Formule importante} : \highlight{$ \max (\alpha ,\beta) = \frac{\alpha+\beta}{2} + \frac{\mc{\alpha - \beta}}{2}$} $\heartsuit$
		\vspace*{0.3cm}\\ \thm{ch5th4}{Théorème}{CVAImplCv}{La convergence absolue implique la convergence.}
		\begin{proof}
		\fbox{Cas réel} Soit $f\in\cpm_big([a,b[,\R\big)$ telle que $\int_a^{\uline{b}} \mc{f}$ converge on a \\ $0\leq f^+\leq\mc{f}$ donc 
		$\int_a^{\uline{b}} f^+$ converge \big(de même $\int_a^{\uline{b}} f^-$ converge \big) \\ 
		Donc $\int_a^{\uline{b}} f^+ - f^- = \int_a^{\uline{b}} f$ converge.\\
		\fbox{Cas complexe} Soit $f\in\cpm\big([a,b],\C\big)$ telle que $\int_a^{\uline{b}} \mc{f}$ converge avec $f=u+\imath v$ alors
		\\ On a $0\leq\mc{u}\leq\mc{f}$ donc $\int_a^{\uline{b}} \mc{u}$ converge et d'après le cas réel $\int_a^{\uline{b}} u$ converge.\\
		De même $\int_a^{\uline{b}} v$ converge et par un lemme (\ref{IntGCompl}) on a $\int_a^{\uline{b}} f$ converge
		\end{proof}
	\section{Espace des fonctions intégrables sur $I$}
		\textit{Soit $I$ i.r.n.t avec $a<b$ ses bornes, on pose $L^1(I,K) = \big\{f\in\cpm(I,K) ~;~f$ est intégrable sur $I \big\}$}
		\vspace*{0.5cm} \\ \thm{ch5th5}{Théorème}{CarL1Delta}{$L^1(I,K)$ est un $K$ espace vectoriel et\\
		$\Delta ~\appli{L^1(I,K)}{f}{K}{\int_I f}$ est \highlight{linéaire}}
		\begin{proof} Soit $I=[a,b[$ \vspace*{0.2cm}\\
		$L^1(I,K)$ est un \textsc{sev} de $\cpm(I,K)$ : \\ \hspace*{0.5cm}\begin{blockarray}{|l}
		$~\bullet ~0\in L^1(I,K)$ \\ $~\bullet$ Soient $f,g\in L^1(I,K)$ et $\lambda\in K$ on pose $h=\lambda f+g$ avec 
		$\mc{h}\leq \mc{\lambda}\mc{f}\mc{g} = w$ \\ Comme $\int_a^{\uline{b}} w$ cv, par théorème de comparaison (\ref{ThCompFcPos}) on a 
		$h\in L^1(I,K)$  \end{blockarray} \vspace*{0.2cm}\\
		La linéarité de $\Delta$ est connue.
		\end{proof}
		${}$ \\ \thm{ch5th6}{Théorème}{CNSIntFcCompl}{Soit $f\in\cpm(I,\C)$ alors $f \in L^1(I,\C) ~ \Leftrightarrow ~ \big(\Re(f),\Im(f)\big) 
		\in\big( L^1(I,\R)\big)^2$}
		\begin{proof}
		\fbox{$\Leftarrow$} $f=\Re(f)+\imath \Im(f)$ $~~~~$ \fbox{$\Rightarrow$} $0\leq\mc{\Re(f)} \leq \mc{f}$ et $0\leq\mc{\Im(f)}\leq\mc{f}$
		\end{proof}
		${}$ \\ \thm{ch5L7}{Lemme}{PetitThPosInt}{Soit $f\in L^1(I,K)$, alors $\displaystyle{f\geq 0 ~\Rightarrow ~\int_I f\geq 0}$}
		\vspace*{0.5cm} \\ \thm{ch5th7}{Théorème : Inégalité triangluaire}{InegTriIntG}{Soit $f\in L^1(I,K)$ alors $\displaystyle{\mc{\int_I f} 
		\leq \int_I \mc{f}}$}
		\begin{proof}
		Soit $I=[a,b[$ ; soient $f\in L^1(I,K)$ et $x\in [a,b[$\\
		On a $\displaystyle{\mc{\int_a^x f} \leq \int_a^x \mc{f}}$ puis en passant à la limite on a le résultat vu l'intégrabilité de $f$.
		\end{proof}
		${}$\\ \thm{ch5th8}{Théorème de positivité amélioré}{PosAmelioIntG}{Soit $I$ un intervalle réel \highlight{non trivial} et 
		$f\in L^1(I,K)\cap \cont^0(I,K)$ avec $f\geq 0$\\ Alors $\displaystyle{\int_I f = 0 ~\Rightarrow ~\forall x\in I ,~ f=0}$}
		\begin{proof}
		Avec $I=[a,b[$ on note $\forall x\in I ,~F(x) = \int_a^x f$ donc $F(x) \stox{b} \int_a^{\uline{b}} f$ et $F$ est croissante par lemme 
		(\ref{5-L1}) donc $F(x) \leq \int_I f = 0$ or $F(x) \geq 0$ donc $\forall x\in [a,b[, F(x) = 0$ donc $F'=f=0$
		\end{proof}
		\paragraph{Notation} Si $f\in\cpm(I,K)$ avec $\displaystyle{\int_I f}$ converge et $\displaystyle{\int_I \mc{f} }$ diverge alors on dit que 
		$f$ est \highlight{semi-convergente}.
	\section{Calculs}
		${}$ \\ \thm{ch5th9}{Théorème : Changement de variable}{ChgtVar}{Soit $]\alpha,\beta[ ~ \overset{\varphi}{\to} ~ ]a,b[ ~ \overset{f}{\to} ~ 
		K$ On suppose de plus $f$ est $\cont^0$ et $\varphi$ est $\cont^1$ \\ avec $\varphi$ bijective de $]\alpha,\beta[$ sur $]a,b[$ et 
		strictement croissante.\\ Alors \hspace*{0.5cm} $\left| \ard $ {\small (1)} $\displaystyle{\int_a^b f(t) \dd t}$ et $\displaystyle{ 
		\int_{\alpha}^{\beta} f\big(\varphi(u)\big) \times \varphi'(u) \dd u }$ ont la même nature.$ \\ 
		$ {\small (2)} Elles sont égales en cas de convergence.$ \arf \right.$ }
		\begin{proof}
		On a $\limit{u}{\alpha} \varphi(u) = a$ et $\limit_{u}{\beta} \varphi(u) = b$. Soit $\gamma \in ]\alpha,\beta[$ et $c= \varphi(\gamma)$\\
		On pose $G(x)~=~\int_{\gamma}^x g(u)\dd u ~$ et $ F(y)=\int_c^y f(t) \dd t$ , alors $\forall x\in]\alpha,\beta[$, on a 
		$G(x) = F\big(\varphi(x)\big)~~(*)~~$ \\ $\bullet$ On suppose $\int_c^b f$ converge, alors par passage à la limite dans $(*)$ on a la 
		convergence et l'égalité. \\ $\bullet$ La réciproque est clair en considérant $\varphi^{-1}$ \\ On a de même $\int_{\alpha}^{\gamma} g$ 
		converge $\Leftrightarrow$ $\int_a^c f$ converge. \\ D'où \textsc{cqfd}
		\end{proof}
		${}$ \\ \thm{ch5th10}{Théorème : \textsc{ipp}}{IPP}{Soit $f,g\in\cont^1\big(]a,b[,K\big)$ alors la formule \\
		\hspace*{2cm} $\displaystyle{\int_a^b f'\times g ~=~ \big[f\times g \big]_a^b - \int_a^b f\times g'}$ \\
		est légitime dès que $~\ard \bullet ~f\times g$ admet des limites finies en $a$ et en $b \\ \bullet $ L'une des deux intégrales est 
		convergente $ \arf $ }
		\begin{proof}
		On suppose $\int_a^b f'\times g$ converge alors \\
		Si on pose $A$ et $B$ les limites respectives du produit en $a$ et en $b$ puis $c$ un point de l'intervalle ouvert alors 
		$\forall x\in ]a,b[$ on a $\int_c^x fg' = \big[fg\big]_c^x - \int_c^x f'g$ puis $\int_c^b fg' = -(fg)(c) + B - \int_c^b f'g$ \\ On a alors 
		alors de même le résultat en $a$ puis $f$ et $g$ jouent un rôle symétrique d'où \textsc{cqfd}
		\end{proof}
	\section{Comparaison série-intégrale}
		${}$\\
		\textit{\footnotesize Le théorème suivant ne figure plus au programme.}
		\\ \thm{ch5th11}{Théorème}{Seriewn}{Soit $f\in\cpm \big(\R^+,\R\big)$ avec $f\geq 0$ décroissante, on pose \\ $\forall n\in\N^* ,~
		w_n = \int_n^{n+1} f(t)\dd t - f(n)$, \hspace*{0.5cm} Alors $\sum\limits_{n\geq 1} w_n$ converge.}
		\begin{proof}
		Soit $n\in\N^*$, vu la décroissance on a $f(n)\leq f(t) \leq f(n-1) $ pour $n-1\leq t\leq n$ \\donc $0\leq w_n \leq f(n-1)-f(n)$ \\
		Ainsi pour tout $N\in\N^*$ on a $\sk{1}{N} w_k \leq \sk{1}{N} f(n-1)-f(n) \leq f(0)$ d'où la convergence vu la positivité. 
		\end{proof}
		${}$ \\ \thm{ch5th11c}{Corollaire}{EquivSerInt}{Soit les mêmes hypothèses,\\ Alors $\sum\limits_{n\geq1} f(n) $ converge $\Leftrightarrow$ 
		$\int_0^{+\infty} f(t) \dd t$ converge }
	\section{Intégration des relations de comparaison}
		${}$ \\ \thm{ch5th12}{Théorème}{CasConv}{Soit $f,g \in \cpm \big([a,b[ , \R \big)$, $g\geq 0$ avec
		$\int_a^b f$ et $\int_a^b g$ convergent. \\Alors :
		\hspace*{1cm} $\ard \bullet$ {\small 1)} $\cm{f(x) = \circ_{x\to b} \big( g(x) \big) ~\Rightarrow ~\int_x^b f = \circ_{x\to b} 
		\Big(\int_x^b g\Big)} \\ \bullet $ {\small 2)} $\cm{f(x) = \bigcirc_{x \to b} \big( g(x) \big) ~\Rightarrow ~\int_x^b f = \bigcirc_{x\to b} 
		\Big(\int_x^b g\Big)} \\ \bullet $ {\small 3)} $ \cm{f(x) \underset{b}{\sim} g(x) ~\Rightarrow ~ \int_x^b f \underset{b}{\sim} \int_x^b g } 
		\arf$ }
		\begin{proof}
		On notera $F(x) = \int_x^b f$ et $G(x) = \int_x^b g$, on considère alors $\varepsilon>0$ et un voisinage de $b$ sur lequel $\mc{f(x)}\leq 
		\varepsilon g(x)$, on a alors $\mc{F(x)} \leq \varepsilon G(x)$.\\
		La domination se démontre selon le même principe.
		\end{proof}
		${}$ \\ \thm{ch5th13}{Théorème}{CasDiv}{Soit $f,g \in \cpm \big([a,b[ , \R \big)$, $g\geq 0$ avec
		$\int_a^b f$ et $\int_a^b g$ divergent. \\Alors :
		\hspace*{1cm} $\ard \bullet$ {\small 1)} $\cm{f(x) = \circ_{x\to b} \big( g(x) \big) ~\Rightarrow ~\int_a^x f = \circ_{x\to b} 
		\Big(\int_a^x g\Big)} \\ \bullet $ {\small 2)} $\cm{f(x) = \bigcirc_{x \to b} \big( g(x) \big) ~\Rightarrow ~\int_a^x f = \bigcirc_{x\to b} 
		\Big(\int_a^x g\Big)} \\ \bullet $ {\small 3)} $ \cm{f(x) \underset{b}{\sim} g(x) ~\Rightarrow ~ \int_a^x f \underset{b}{\sim} \int_a^x g } 
		\arf$ }
		\begin{proof}
		En notant, $F(x) = \int_a^x f$ et $G(x) = \int_a^xg$, le reésultat se démontre en séparant l'intégrale : \\
		\[ \mc{\frac{F(x)}{G(x)}} = \frac{\mc{F(x)}}{G(x)} \leq \frac{C}{G(x)} + \frac{\varepsilon}{2}~~ \stox{b} 0 ~~~~;~~~~
		{\footnotesize C=\mc{\int_a^{\beta} f} } \]
		\end{proof}  ${}$ \\
		\begin{center}
		\fin
		\end{center}
		
		
%		\newpage
%	\section{Exercices}
%		\paragraph{Comparaison}
%			\subparagraph{5C1} Donner les natures de $\cm{\int_0^1 \frac{1+t^3}{\sqrt{1-t^3}} \dd t } $ et de $\cm{\int_0^{\frac{\pi}{2}} \frac{t}
%			{\cos t} \dd t} $.
%		\paragraph{Convergence absolue}
%			\subparagraph{\uline{5C2}} Montrer que $\cm{\int_{\pi}^{+\infty} \frac{\sin t}{t} \dd t } $. CVA ?
%			\subparagraph{\uline{5C3}} Étudier la nature des intégrales de \textsc{Bertrand} $\cm{\Big( \int_2^{+\infty} 
%			\frac{\dd t}{t^{\alpha}(\ln t)^{\beta}} \Big) }$
%			\subparagraph{5C4} Connaissant le \textbf{5C3}, avec $(\alpha,\beta)\in\R^2$, quelle est la nature de $\cm{\int_0^{\frac{1}{2}} 
%			\frac{\dd t}{t^{\alpha}(\ln t)^{\beta}} }$ ?
%		\paragraph{Calculs}
%			\subparagraph{5C5} Calculer $\cm{ \lambda = \int_\R f}$ où $\cm{ f : t\mapsto \frac{1}{1+t+t^2} }$ et $\cm{ I = \int_0^{\frac{\pi}{2}} 
%			\ln (\sin t ) \dd t }$.
%			\subparagraph{\uline{5C6}} Montrer l'existence de $\gamma$ la constante d'\textsc{Euler}.
%			\subparagraph{5C7} Trouver un équivalent de $\cm{ R_n = \sum_{n+1}^{+\infty} \frac{1}{k^2} } $.
%		\paragraph{Intégration des relations de comparaisons}
%			\subparagraph{5C8} Trouver un équivalent de $\cm{ F(x) = \int_2^x \frac{e^{-1/t}}{\sqrt{t(t+\cos t)}} }$.