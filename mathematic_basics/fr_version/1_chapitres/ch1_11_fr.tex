
% Chapitre 11 : Polynômes et fractions rationnelles

\minitoc
	\section{Anneau des polynômes à 1 indéterminée}
		\paragraph{Anneau des polynômes ($\mathbb{K} [X], +, \times$)}
			Soit $\mathbb{K}^{(\mathbb{N} )}$ l'ensemble des suites à \\ valeurs dans 
			$\mathbb{K}$ sationnaires nulles.
			\[\forall u\in\mathbb{K}^{(\mathbb{N} )} , \exists n_{0} \in \mathbb{N} : 
			\forall n\in\mathbb{N} , (n\leq n_{0} \Rightarrow u_{n}=0)\]
			On définit une addition et une multiplication :
			\[ \forall n\in\mathbb{N} , (u+v)(n) = u_{n}+v_{n} \]
			\[\forall n\in\mathbb{N} , (uv)(n) = \sum\limits_{k=0}^{n} u(k)v(n-k) \]
			On considère la suite $X=(0,1,0,0,...)$ et on a alors $X^{n} =(0,...,0,1,0,...)$
			et $X^{0} = 1_{\mathbb{K}} =(1,0,0,...)$ \hspace*{20pt}
			On peut noter $\mathbb{K}^{(\mathbb{N} )})$ comme $\mathbb{K} [X]$.\\
			$P=(a_{0} ,~a_{1} ,...,~a_{n} ,~0,...) = \sum\limits_{k=0}^{n} a_{k} X^{k} 
			~~\in\mathbb{K} [X]$ \\
			($\mathbb{K}^{(\mathbb{N} )} [X] ,~+,~\times$) est l'anneau des polynômes.
			\subparagraph{Utilisation}
			 $(1+X)^{n+m} = (1+X^{n} (1+X)^{m} 
			 $\\
			 $\sum\limits_{l=0}^{n+m} \binom{n+m}{l} X^{l} = \sum\limits_{k=0}^{n} 
			 \binom{n}{k} X^{k} = \sum\limits_{j=0}^{m} \binom{m}{j} X^{j} \hspace*{10pt}$ 
			 Qui donne \hspace*{10pt} 
			 $$\binom{n+m}{l} = \sum\limits_{k=0}^{l} \binom{n}{k} \binom{m}{l-k} $$
		\subsection{Degré d'un polynôme}
			Si $P\in\mathbb{K} [X]~,~P\neq 0$ on appelle degré de P noté deg(P) ou d°(P) :
			\[ deg(P) = max\{ n\in\mathbb{N} \vert a_{n} \neq 0\} ~~
			(P=\sum\limits_{k=0}^{+\infty } a_{k} X^{k} )\]
			avec par convention le polynôme nul de degré $-\infty$.
			\paragraph{Degré de la somme et du produit}
			 Soit $(P,Q) \in (\mathbb{K} [X] )^{2}$\\
			 \[  deg (P+Q) \leq max\{ deg(P),deg(Q)\} \] égalité si $deg(P)\neq deg(Q)$\\
			 \[deg(PQ) = deg(P) + deg(Q)\]
			\paragraph{Ensemble}
			 Si $n\in\mathbb{N}$,
			 $\mathbb{K}_{n} [X]$ est l'ensemble des polynômes de degré au plus n.
			 \[ \mathbb{K}_{n} [X] = \{P\in\mathbb{K} [X] ~\vert ~ deg(P) \leq n\}\]
			 \subparagraph{Remarque}
			  $\mathbb{K}_{n} [X] $ est stable par combinaison linéaire
			  \[\forall (\lambda ,\mu )\in \mathbb{K}^{2} , ~\forall (P,Q)\in 
			  (\mathbb{K}_{n} [X])^{2} \hspace*{20pt} \lambda P+\mu Q \in\mathbb{K}_{n} [X]\]
			\paragraph{Intégrité de l'anneau ($\mathbb{K} [X],+,\times$)}
			 $\forall (P,Q) \in (\mathbb{K} [X] )^{2} $
			 \[ PQ = 0 ~\begin{array}{l}
			 ~\Leftrightarrow ~~deg(PQ) = -\infty \\
			 ~\Leftrightarrow ~~deg(P) + deg(Q) = -\infty \\
			 ~\Leftrightarrow ~~deg(P) = -\infty ~ou~ deg(Q) = -\infty \\
			 ~\Leftrightarrow ~~P=0 ~ou~ Q=0  
			 \end{array} \]	
			
		\subsection{Composition de polynômes}
			Si $(P,Q) \in (\mathbb{K} [X] )^{2}$ avec $P=\sum\limits_{k=0}^{+\infty} a_{k} X^{k}$
			\[P\circ Q ~=~ \sum\limits_{k=0}^{+\infty} a_{k} Q^{k} \]
			\paragraph{Degré du polynôme composé}
			 Si $(P,Q)\in \mathbb{K} [X] \times \mathbb{K} [X]\backslash \mathbb{K}_{0} [X]$
			 \[deg(P\circ Q) = deg(P) \times deg(Q)\]
			\paragraph{Coefficient dominant}
				Si $P=\sum\limits_{k=0}^{+\infty} a_k X^k \in \mathbb{K} [X] \backslash \{0\}$\\
				$a_{degP}$ s'appelle le coefficient dominant de $P$. 
				Si il vaut $1$ $P$ est dit unitaire.
	\section{Divisibilité et Division Euclidienne}
		\subsection{Divisibilité des polynômes}
			Si $(A,B) \in (\mathbb{K} [X])^{2}$\\
			On dit que A	 divise B si il existe $Q \in\mathbb{K} [X]$ tel que $B=AQ$\\
			On note alors $A\vert B$ (sinon $A\not\vert B$)
			\paragraph{Propriété}
				Soit $A\in\mathbb{K} [X] \backslash\{0\}$ et $B \in\mathbb{K} [X]$
				$A\vert B$ $\Rightarrow$ degA $\leq$ degB
				\subparagraph{Preuve}
				 $A = BQ \Rightarrow$ degA = degB + degQ $\geq$ degB
		\subsection{Polynômes associés}
			$(A,B) \in (\KX )^2$ est un couple de polynômes associés si
			\[A\vert B \hspace*{15pt} B\vert A\]
			\paragraph{$\rightarrow$}
				$(A,B)$ est un couple de polynômes associés si et seulement si
				\[\exists\lambda\in\mathbb{K}* ~:~A=\lambda B\]
		\subsection{Division euclidienne polynômiale}
			Si $B \neq 0$ , $B = \sum\limits_{k=0}^m b_k X^k$ avec $b_m\neq 0$ \vspace*{10pt}\\
			\[Si~ A\in\mathbb{K} [X] \backslash \mathbb{K}_{m-1} [X] ~~il~existe~(Q_0,R_0)
			\in (\mathbb{K} [X])^2~:~A=BQ_0+R_0 ~et~degR_0 < degA\]
			(Si $A=\sum\limits_{k=0}^{n} a_{k} X^{k}$ il suffit de considérer 
			$Q_0 =\frac{a_n}{b_m} X^{n-m}$)
			\newtheorem*{th10}{Théorème de la division euclidienne polynômiale}
			\begin{th10}\label{Th DEucl Pol} 
				${}$\\Si $B\in \mathbb{K} [X] \backslash \{0\}$ alors pou tout 
				$A\in\mathbb{K} [X]$
				\[\exists (Q,R) \in (\mathbb{K} [X])^2 ~:~ \left|\begin{array}{l}
				A=BQ+R\\ degR < deg B
				\end{array}\right. \] 
			De plus $Q$ et $R$ sont uniques appelés quotient et reste de la division 
			euclidienne de $A$ par $B$.
			\end{th10}
			\begin{proof}
				\underline{Existence} \medskip Récurence sur degA \\
				\hspace*{20pt} \textit{Initialisation} \hspace{20pt} Si degA < degB
				\[A=B\times 0 + A = BQ + R \]
	
				\hspace{20pt} \textit{Hérédité} \hspace{20pt} On suppose la propriété vraie 
				pour tout polynôme de degré $k<n$ avec $n\geq$ degB. 
				D'après la remarque préliminaire on a :
				\[\exists (Q_0,R_0) \in (\mathbb{K} [X])^2 ~:~A=BQ_0 +R_0 ~~~~ degR_0<degA=n\]
				\begin{center} d'après l'hyspothèse de récurrence $\exists (Q_1,R_1) 
				\in (\mathbb{K} [X])^2$ \\
				\hspace*{10pt} $R_0 =BQ_1+R_1$ avec $degR_1<degB$ soit 
				\[A=B(Q_0 + Q_1) + R_1\] \end{center}
				\hspace*{40pt} \underline{Unicité} \\
				\hspace*{20pt} Supposons $A=BQ_1 +R_1 = BQ_2 +R_2$ avec $degR_1,degR_2<degB$\\
				alors $B(Q_1 - Q_2) = R_1 - R_2$ donc
				\[degB + deg(Q_1 -Q_2) = deg(R_1 - R_2)~\leq~max\{degR_1,degR_2\} < degB\]
				d'où $deg(Q_1 -Q_2) = -\infty$ soit $Q_1=Q_2$ puis $R_1=R_2$
			\end{proof}
	\section{Fonctions polynômiales et racines}
		\subsection{Fonction polynômiale associée}
			À tout polynôme $P = \sum\limits_{k=0}^n a_k X^k \in \mathbb{K} [X]$ on peut associer 
			la fonction polynômiale 
			\[\widetilde{P} \left( \begin{array}{l}
			\hspace*{10pt}\mathbb{K} \longrightarrow \mathbb{K} \\
			x \mapsto \sum\limits_{k=0}^n a_k x^k \end{array} \right)\]
			\paragraph{Calculs}
				$\forall (P,Q)\in (\mathbb{K} [X])^2 ~~ \forall (\lambda ,\mu )\in\mathbb{K}^2$
				\[\widetilde{\lambda P+\mu Q} = \lambda\widetilde{P} +\mu\widetilde{Q}\]
				\[\widetilde{PQ} = \widetilde{P}\widetilde{Q} \hspace*{40pt} \widetilde{P\circ Q} 
				= \widetilde{P}\circ\widetilde{Q}\]
		\subsection{Racines du polynôme}
			$a\in\mathbb{K}$ est une racine de $P\in\mathbb{K} [X]$ si \[\widetilde{P}(a) = 0\]
			On notera ensuite $\mathcal{Z}(P)$ l'ensemble des racines (ou zéros) de $P$.
			\newtheorem{th11}{Divisibilité par $(X-a)$}
			\begin{th11}\label{Div X-a}
				$\forall P\in\mathbb{K}[X] ~~~~ \forall (a_1,~\cdots ~,a_n)\in\mathbb{K}^n$ 
				distincts \[\{a_1,~\cdots ~,a_n\} \subset \mathcal{Z}(P) ~\Leftrightarrow ~
				\prod\limits_{i=1}^n (X-a_i)\vert P\]
			\end{th11}
			\begin{proof}
			Récurrence sur n : P(n)(\ref{Div X-a})\\
			\underline{Initialisation} \hspace*{10pt}
			$(x-a)\vert P$ si et seulement si $\exists Q : P=(X-a)Q$ alors
			\[\widetilde{P}(a) = \widetilde{(X-a)}(a)\widetilde{Q}(a) = 0\]
			Si $a\in\mathcal{Z}(P)$ et $P=\sum\limits_{k=0}^n \alpha_k X^k$
			\[P = P-P(a) = \sum\limits_{k=0}^n \alpha_k X^k - \sum\limits_{k=0}^n \alpha_k a^k 
			= \sum\limits_{k=0}^n \alpha_k (X^k - a^k) \]
			\[= (X-a)\sum\limits_{k=0}^n \alpha_k \sum\limits_{l=0}^{k-1} a^{k-1-l} X^l = 
			(X-a)Q\vspace*{20pt}\]
			\underline{Hérédité} \hspace*{10pt} Supposons P(n) et considérons
			$\{a_1, ~\cdots ~,a_n,a_{n+1}\}\in\mathbb{K}^{n+1}$ distincts \\
			Par l'hypothèse de récurrence on a 
			\[\exists Q\in\mathbb{K}[X] ~:~ P=(\prod\limits_{i=1}^n (X-a_i))Q\]
			\[ a_{n+1} \in\mathcal{Z}(P) ~\Leftrightarrow ~\widetilde{P}(a_{n+1}) = 0
			\Leftrightarrow ~(\prod\limits_{i=1}^n (a_n+1-a_i))\widetilde{Q}(a_{n+1})=0\]
			\[ \Leftrightarrow ~a_{n+1} \in\mathcal{Z}(Q) ~\Leftrightarrow ~X-a_{n+1}\vert Q\]
			\end{proof}
			\paragraph{Nombre de racines}
				Le nombres de racines d'un polynôme \underline{non nul} est majoré par son degré.
				\subparagraph{dem.}
				 Par récurrence si $deg(\prod\limits_{i=1}^n (X-a_i)) = n$ et $P\neq O$
				 \[\prod\limits_{i=1}^n (X-a_i)\vert P ~\Longrightarrow ~n\leq P\]
			\paragraph{Corollaire : Caractérisation du polynôme nul}
				Le seul polynôme admettant une infinité de racines ou $n+1$ racines est le 
				polynôme nul.
				\subparagraph{appli.}
				 Soit $E=\{P\in\mathbb{K}[X]\vert\exists T\in\mathbb{K}* : \forall x\in\mathbb{K}
				 ,~\widetilde{P}(x+T) = \widetilde{P}(x)\}$, déterminons $E$.\\
				 $\mathbb{K}_0[X]\subset E$\\
				 Réciproquement si $\P\in E ~T-p\' eriodique~(T\neq 0)$ et $Q=P-\widetilde{P}(0)$ 
				 on a $T\mathbb{Z} \subset\mathcal{Z}(P)$ d'où $P-\widetilde{P}(0) = 0$ donc
				 $P= \widetilde{P}(0) \in\mathbb{K}_0[X] $\\
				 En conclusion on a $E = \mathbb{K}_0[X]$.
		\subsection{Ordre de multiplicité}
			$\forall P\in\KX $ \\
			Si $a\in\mathbb{K}$, $k\in\mathbb{N}$ et $(X-a)^n\vert P$ on dit que $a$ est une 
			racine de $P$ d'ordre de multiplicité au moins $n$.\\
			Si de plus $(X-a)\not\vert P$ alors $a$ est une racine de $P$ d'ordre de multiplicité 
			exactement $n$.
			\paragraph{$\rightarrow$}
			$a$ est une racine de $P$ d'ordre de multiplicité k si et seulement si 
			$\exists Q\in\KX$ tel que 
			\[P = (X-a)^k Q ~~et~~ a\not\in \mathcal{Z}(P)\]
		\subsection{Méthode de Horner pour l'évaluation polynômiale} 
			${}$\\ Soit $\Psn$ et $x_0\in\mathbb{K}$ on veut déterminer 
			$\widetilde{P}(x_0)$. \\On considère la suite
			$\left\{\ar* u_0 = a_n \\ u_{k+1} = u_kx_0+a_{n+1-k} \ar\right.$
			\[u_k = a_nx_0^k + ~\cdots ~+a_{n_k} ~~et~~u_n = \widetilde{P}x_0\]
		\subsection{Polynôme scindé}
			Un polynôme $P\in\KX$ est \underline{scindé} s'il peut s'écrire comme produit de 
			polynômes de degré 1.\\
			\thm{th12}{Formule de Viete}{fViete}{${}$\\ \underline{Relations entre les coefficients et les racines d'un polynôme scindé}
				\\Soit $P$ un polynôme scindé, $P = \Psn = \lambda\prod\limits_{k=1}^n (X-x_k)$
				\\ $\left.\ar* n\geq 2 \\ a_n \neq 0\ar\right| \Leftrightarrow \left\{\ar* 
				a_n = \lambda \\ a_{n-l} = \lambda\prod\limits_{1\leq i_1\leq 
				\cdots\leq i_l\leq n} \sum\limits_{r=1}^l (-x_{i_r}) \ar\right. $
				\\ $ \Leftrightarrow \left\{\ard a_n = \lambda \\ \prod\limits_{1\leq i_1\leq 
				\cdots\leq i_l\leq n} \sum\limits_{r=1}^l x_{i_r} = \frac{(-1)^l a_{n-l}}{a_n}
				\arf \right. $ }
			
			\begin{proof}
			Faire arbre
			\end{proof}
	\section{Dérivation}
		Si $P=\Ps \in\KX$ on appelle polynôme dérivé de P
		\[P' = \sum\limits_{k=1}^{+\infty}ka_kX^{k-1} = 
		\sum\limits_{k=0}^{+\infty}(k+1)a_{k+1}X^k\]
		puis par récurrence avec $\forall n\in\mathbb{N} ~~P^{(n++1)} = (P^{(n)})'$
		$P^{(n)} = \sum\limits_{k=n}^{+\infty} k(k-1)\cdots (k-n+1)a_kX^{k-n}
		= \sum\limits_{k=0}^{+\infty} (k+n)(k+n-1)\cdots (k+1) a_{k+n}X^k 
		 = \sum\limits_{k=n}^{+\infty}\frac{k!}{(k-n)!} a_kX^{k-n}
		= \sum\limits_{k=0}^{+\infty} \frac{(k+n)!}{k!} a_{k+n}X^k$
			\hspace*{40pt} \underline{\textbf{Calcul :}}
			$\forall (P,Q)\in (\KX )^2$ \\
			\[\rightarrow ~~ \forall (\lambda ,\mu )\in\mathbb{K}^2 ,~(\lambda P+\mu Q)'
			= \lambda P'+\mu Q'\]
			\[\rightarrow ~~ (PQ)' = P'Q + PQ' \hspace*{40pt} \rightarrow ~~(P\circ Q)'
			= P'\circ Q \times Q\]
		\begin{proof}
		$P = \Ps ~~Q =\sum\limits_{k=0}^{+\infty} b_kX^k$ \\
		$\rightarrow ~~ (\lambda P+\mu Q)' = (\sum\limits_{k=0}^{+\infty} 
		(\lambda a_k +\mu b_k)X^k)' = \sum\limits_{k=1}^{+\infty} 
		k(\lambda a_k +\mu b_k)X^{k-1} \\ = \lambda 
		(\sum\limits_{k=1}^{+\infty} ka_k X^{k-1} ) +\mu 
		(\sum\limits_{k=1}^{+\infty} kb_k X^{k-1} ) = \lambda P' +\mu Q'$\\
		$\rightarrow ~~ PQ = \sum\limits_{k=0}^{+\infty} c_k X^k ~~~~c_k = 
		\sum\limits_{l=0}^k a_l b_{k-l} \hspace*{30pt} donc~~~~
		(PQ)' = \sum\limits_{k=0}^{+\infty} (k+1)c_{k+1} X^k \\ avec \hspace*{40pt} P' = 
		\sum\limits_{k=0}^{+\infty}(k+1)a_{k+1} X^k ~~~~et~~~~
		Q' = \sum\limits_{k=0}^{+\infty} (k+1)b_{k+1} X^k \\
		PQ' = \sum\limits_{k=0}^{+\infty} d_k X^k ~~~~ d_k = \sum\limits_{l=0}^k
		a_l (k+1-l)b_{k+1-l} \\ P'Q = \sum\limits_{k=0}^{+\infty} \delta_k X^k  
		\delta_k = \sum\limits_{l=0}^k (l+1)a_{l+1} b_{k-l} \\
		d_k +\delta_k = \sum\limits_{l=0}^k a_l (k+1-l)b_{k+1-l} + \sum\limits_{l=0}^k 
		(l+1)a_{l+1}b_{k-l} \\ 
		= a_0 (k+1)b_{k+1} + (k+1)a_{k+1}b_0 + \sum\limits_{l=1}^k 
		a_l (k+1-l)b_{k+1-l} +\sum\limits_{l=1}^k la_l b_{k+1-l} \\
		= (k+1)\sum\limits_{l=0}^{k+1} a_lb_{k+1} = (k+1)c_{k+1}$
		\end{proof}