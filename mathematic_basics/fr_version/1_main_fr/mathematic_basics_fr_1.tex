
%%%%%%%%%%%%%%%%%%%%%%%%%%%%%%%%%%%%%%%%%%%%%%%%%%%%%%%%%%%%%%%%%%%%
%                               PREAMBLE
%%%%%%%%%%%%%%%%%%%%%%%%%%%%%%%%%%%%%%%%%%%%%%%%%%%%%%%%%%%%%%%%%%%%

\documentclass[a4paper, 10pt, twoside]{report}
\usepackage[top = 2.5cm,left=2.54cm,right=2.54cm,bottom=2.54cm]{geometry}
\usepackage[french]{babel}
\usepackage[utf8]{inputenc} 
\usepackage[T1]{fontenc}
\usepackage{lxfonts}
\usepackage[mathcal, mathbf]{euler}
\usepackage[hidelinks]{hyperref}
\usepackage{ulem}
\usepackage{fancyhdr}
\usepackage{shorttoc}

\usepackage{xcolor}
\usepackage{pagecolor}
\usepackage{graphicx}
\usepackage{tikz}
\usepackage{epigraph}
\usepackage{fancyhdr}
\usepackage{blkarray}
\usepackage{booktabs}

\usepackage{amsfonts}
\usepackage{amsmath}
\usepackage{amsthm}
\usepackage{stmaryrd}
\usepackage{multicol}
\usepackage{bbold}
\usepackage{amssymb}
\usepackage{yhmath}
\usepackage{xfrac}
\usepackage{dsfont}
\usepackage{mathrsfs}
\usepackage{cancel}
\usepackage{soul}

\usepackage{shorttoc}
\usepackage{minitoc}

% \renewcommand{\thefootnote}{\fnsymbol{footnote}}

\pagestyle{fancy}
\setlength{\headheight}{13.59999pt}
\fancyhead[LE]{\thepage}
\fancyhead[RE]{\small \leftmark}
\fancyhead[LO]{\small \rightmark}
\fancyhead[RO]{\thepage}


%%%%%%%%%%%%%%%%%%%%%%%%%%%%%%%%%%%%%%%%%%%%%%%%%%%%%%%%%%%%%%%%%%%%
%                            Cover Page Details
%%%%%%%%%%%%%%%%%%%%%%%%%%%%%%%%%%%%%%%%%%%%%%%%%%%%%%%%%%%%%%%%%%%%

\renewcommand\epigraphflush{flushright}
\renewcommand\epigraphsize{\normalsize}
\setlength\epigraphwidth{0.7\textwidth}

\definecolor{titlepagecolor}{cmyk}{1,.60,0,.40}

\DeclareFixedFont{\titlefont}{T1}{ppl}{b}{it}{0.5in}

\makeatletter                       
\def\printauthor{%                  
    {\large \@author}}              
\makeatother
\author{%
    Mis en forme par \\
    Émile Sauvat \\
    \texttt{emile.sauvat@ens.psl.eu}\vspace{20pt}}

% The following code is borrowed from: https://tex.stackexchange.com/a/86310/10898

\newcommand\titlepagedecoration{%
\begin{tikzpicture}[remember picture,overlay,shorten >= -10pt]

\coordinate (aux1) at ([yshift=-15pt]current page.north east);
\coordinate (aux2) at ([yshift=-410pt]current page.north east);
\coordinate (aux3) at ([xshift=-4.5cm]current page.north east);
\coordinate (aux4) at ([yshift=-150pt]current page.north east);

\begin{scope}[titlepagecolor!40,line width=12pt,rounded corners=12pt]
\draw
  (aux1) -- coordinate (a)
  ++(225:5) --
  ++(-45:5.1) coordinate (b);
\draw[shorten <= -10pt]
  (aux3) --
  (a) --
  (aux1);
\draw[opacity=0.6,titlepagecolor,shorten <= -10pt]
  (b) --
  ++(225:2.2) --
  ++(-45:2.2);
\end{scope}
\draw[titlepagecolor,line width=8pt,rounded corners=8pt,shorten <= -10pt]
  (aux4) --
  ++(225:0.8) --
  ++(-45:0.8);
\begin{scope}[titlepagecolor!70,line width=6pt,rounded corners=8pt]
\draw[shorten <= -10pt]
  (aux2) --
  ++(225:3) coordinate[pos=0.45] (c) --
  ++(-45:3.1);
\draw
  (aux2) --
  (c) --
  ++(135:2.5) --
  ++(45:2.5) --
  ++(-45:2.5) coordinate[pos=0.3] (d);   
\draw 
  (d) -- +(45:1);
\end{scope}
\end{tikzpicture}%
}

\begin{document}


%%%%%%%%%%%%%%%%%%%%%%%%%%%%%%%%%%%%%%%%%%%%%%%%%%%%%%%%%%%%%%%%%%%%
%                            Command Definitions
%%%%%%%%%%%%%%%%%%%%%%%%%%%%%%%%%%%%%%%%%%%%%%%%%%%%%%%%%%%%%%%%%%%%

\newcommand{\dessin}[2]{
    \begin{center}
        \input{dessins/#1}
        \captionof{figure}{#2}
    \end{center}    }


%%%    Some command to create theorem and definition easier      %%%

\theoremstyle{plain}
\newtheorem{thm}{Théorème}[section]
\newtheorem{lem}[thm]{Lemme}
\newtheorem{prop}[thm]{Proposition}
\newtheorem*{cor}{Corollaire}
\newcommand{\theorem}[2]{
    \hspace{15pt} {\vrule width 1pt} \kern2pt
    	\begin{minipage}[t]{0.9\textwidth}
            	\begin{#1} ${}$ \vspace*{0.15cm}\\	
                #2 
     		\end{#1}
     	\end{minipage}
     \medskip
    }
    %commande pour créer un théorème, le premier argument est le type et le second le contenu
        % ex : \theorem{lem}{contenu du lemme}
    
\newcommand{\namedtheorem}[3]{
    \newtheorem*{#3}{#1}
    \hspace{15pt} {\vrule width 1pt} \kern2pt
    	\begin{minipage}[t]{0.9\textwidth} 
            \begin{#3} ${}$ \vspace*{0.15cm}\\      
                #2 
            \end{#3} 
        \end{minipage}
    \medskip
    }

\theoremstyle{remark}
\newtheorem*{nd}{Note}
\newcommand{\remarque}[1]{\begin{nd} #1 \medskip \end{nd}}
    %commande pour créer une remarque, le contenu passe en argument
        % ex : \remarque{contenu de la remarque}
\newtheorem*{rap}{Rappel}
\newcommand{\rappel}[1]{\begin{rap} #1 \medskip \end{rap}}

\newcommand{\traitd}{\begin{center} ~ \medskip  \\ \rule{0.9\textwidth}{0.75pt} \vspace{-10pt}\end{center}}
\newcommand{\trait}{\vspace{-5pt} \begin{center} \rule{0.9\textwidth}{0.75pt} \end{center} ~ }
\newcommand{\traitdouble}{\trait \vspace{-30pt} \traitd}

\makeatletter
\newcommand*{\ar}{\@ifstar\ard\arf}
\makeatother
\newcommand{\ard}{\begin{array}{l}}
\newcommand{\arf}{\end{array}}


%%%     Maths misc      %%%

\newcommand{\abs}[1]{\left\vert #1 \right\vert}

\newcommand{\appli}[4]{
    \left(
    \begin{array}{ccccc} 
        & #1 & \longrightarrow & #3 & \\ 
        & #2 & \longmapsto & #4 &
    \end{array}
    \right)
}


\newcommand{\ent}[1]{{\llbracket #1 \rrbracket}}                    
	% Ensemble d'entiers
\newcommand{\N}{{\mathbf{N}}}                                       
	% Entiers naturels
\newcommand{\Z}{{\mathbf{Z}}}                                       
	% Entiers relatifs
\newcommand{\R}{{\mathbf{R}}}                                       
	% Réels
\newcommand{\C}{{\mathbf{C}}}                                       
	% Complexes
\newcommand{\K}{{\mathbf{K}}}                                       
	% Corps K
\newcommand{\Q}{{\mathbf{Q}}}                                       
	% Rationnels
\newcommand{\KX}{{\mathbf{K}[X]}}                                   
	% Polynômes sur K
\newcommand{\M}{{\mathcal{M}}}                                      
	% Matrices
\newcommand{\GL}{{\mathcal{G}\hspace*{-0.05cm} \mathcal{L}}}        
	% Groupe Linéaire
\newcommand{\cont}{{\mathcal{C}}}                                   
	% Fonctions continues
\newcommand{\Cinf}{\mathscr{C}^{\infty}}
    % Fonctions C infini
\newcommand{\cpm}{{\mathcal{C}_{pm}^0}}                             
	% Fonctions continues par morceaux
\newcommand{\lin}{{\mathcal{L}}}                                    
	% Applications linéaires
\newcommand{\Sym}{{\mathcal{S}}}                                    
	% Matrices symétrique
\newcommand{\Orth}{{\mathcal{O}}}                                   
	% Matrices orthogonales
\newcommand{\bigo}{{\mathcal{O}}}
    % Dommination
\newcommand{\Part}{\mathcal{P}}                                     
	% Parties
\newcommand{\A}{{\mathscr{A}}}                                      
	% Tribu
\newcommand{\Esc}{{\mathcal{E}\big([a,b],\R\big)}}
    % Fonction en escalier
\newcommand{\prehilb}{(E,\scal{.,.})}
    % Espacee préhilbertien


\newcommand{\norm}[1]{\left\Vert #1 \right\Vert}
    % Norme
\newcommand{\normop}[1]{{\left\vert\kern-0.15ex\left\vert\kern-0.15ex\left\vert #1 \right\vert\kern-0.15ex\right\vert\kern-0.15ex\right\vert}} 
    % Norme subordonnée
\newcommand{\scal}[1]{\left< #1 \right>}
    % Produit scalaire


\newcommand{\suminf}{\sum_{n=0}^\infty}
    % Somme infinie en n
\newcommand{\sk}[2]{\sum_{k= #1 }^{#2}}
    % Somme en k
\newcommand{\si}[2]{\sum_{i= #1 }^{#2}}
    % Somme en i
\newcommand{\pk}[2]{\prod\_{k= #1 }^{#2}}
    % Produit en k
\newcommand{\pdi}[2]{\prod_{i= #1 }^{#2}}
    % Produit en i

\newcommand{\Ps}{\sum_{k=0}^{+\infty} a_k X^k}
    % Polynôme infini
\newcommand{\Psn}{\sum_{k=0}^n a_k X^k}
    % Polynôme degré n

\newcommand{\sumi}{\sum\limits_{i\in I}}
    % Somme sur un ensemble I
\newcommand{\prodi}{\prod\limits_{i\in I}}
    % Produit sur un ensemble I


\newcommand{\suite}[1]{(#1_n)}
    % Suite en n
\newcommand{\stox}[1]{\underset{x \to #1}{\longrightarrow}}
    % Convergence selon x
\newcommand{\ston}{\underset{n}{\rightarrow}}
    % Convergence en n


\newcommand{\pfrac}[2]{\frac{\partial #1}{\partial #2}}
    % Dérivée partielle


%%%     Format      %%%

\newcommand{\highlight}[1]{\colorbox{lightgray}{#1}}
    % Surligner en gris

\newcommand{\col}[2]{\begin{multicols}{2} #1 \columnbreak \\ #2 \end{multicols}}
    % Deux colonnes

\newcommand{\fin}{
	~ \vspace{10pt}\\
    \begin{center}
        { \LARGE $\star$ \hspace{0.7cm} $\star$ \hspace{0.7cm} $\star$ }
    \end{center}
    }


%%%%%%%%%%%%%%%%%%%%%%%%%%%%%%%%%%%%%%%%%%%%%%%%%%%%%%%%%%%%%%%%%%%%
%                               COVER
%%%%%%%%%%%%%%%%%%%%%%%%%%%%%%%%%%%%%%%%%%%%%%%%%%%%%%%%%%%%%%%%%%%%

\begin{titlepage}

\noindent
\titlefont ~\vspace{50pt}\\Mathématiques \\ \hspace*{150pt} Préparatoires I\par
\epigraph{Ce document est une synthèse du cours de mathématiques dispensé par M. Jean-François \textsc{Mallordy} 
    en classe préparatoire au lycée Blaise Pascal, Clermont-Ferrand en 2022-2023. Il s'agit d'un complément au cours de Maths Spé 
    et ne saurait en aucun cas y être un quelconque remplacement !} %
{\textit{Paris, 2024}}
\null\vfill
\vspace*{1cm}
\noindent
\hfill
\begin{minipage}{0.45\linewidth}
    \begin{flushright}
        \printauthor
    \end{flushright}
\end{minipage}
%
\begin{minipage}{0.02\linewidth}
    \vspace{-20pt}
    \rule{1pt}{55pt}
\end{minipage}
\titlepagedecoration
\end{titlepage}

\setcounter{tocdepth}{2}
\renewcommand{\thefootnote}{\fnsymbol{footnote}}
\renewcommand{\contentsname}{Table des matières - Première année}
\setcounter{minitocdepth}{4}
\renewcommand{\mtctitle}{Contenu}
\dominitoc

\setcounter{chapter}{-1}

\theoremstyle{plain}

\shorttoc{Chapitres}{0}


%%%%%%%%%%%%%%%%%%%%%%%%%%%%%%%%%%%%%%%%%%%%%%%%%%%%%%%%%%%%%%%%%%%%
%                          DOCUMENT START
%%%%%%%%%%%%%%%%%%%%%%%%%%%%%%%%%%%%%%%%%%%%%%%%%%%%%%%%%%%%%%%%%%%%

\chapter{Introduction}

    
% Chapitre 1 : Introduction

\emph{Tout les éléments mathématiques seront déclarés et définis, les textes seront différenciés des formules mathématiques.}

\minitoc

\section{Règles d'écriture}
    \subsection{Quantificateurs}

        En écriture mathématique, on utilise les quatificateurs suivants :
        \\$\exists : $ Existence$ \hfill \forall : $ Quelque soit$ \hfill ~$
        \\\textit{Exemples :}
        \[\forall ~y \in \mathbb{R},~\exists ~x \in\mathbb{R} ~:~y=x^7-x ~~ \neq ~~\exists ~x /\in\mathbb{R} ~:~y\in\mathbb{R},~y=x^7-x\]
        $$\forall ~\epsilon >0,~\exists ~n_0 \in\mathbb{N} : ~\forall ~n\in\mathbb{N} ~(n_0\ge n \Rightarrow \mid u_{n} - l\mid \ge \epsilon ) ~~~\rightarrow ~~~ (u_n~converge~vers~l) $$
        \newpage
    \subsection{Conditions Nécessaires et Suffisantes}

        \begin{multicols}{2}
        Une condition Q est nécessaire pour avoir P si dès que P est vraie Q est vraie. $P\Rightarrow Q$
        \\\textsl{ABCD est un parrallélogramme est une condition nécessaire pour que ABCD soit un losange.}
        \columnbreak
        \\Une condition Q est suffisante pour avoir P si dès que Q est vraie P est vraie. $Q\Rightarrow P$
        \\\textsl{ABCD à 4 côtés égaux est une condition suffisante pour que ABCD soit un losange.}
        \end{multicols}

        Si P est nécessaire et suffisante pour avoir Q alors P est nécessairement suffisante pour avoir Q. On dit aussi que P et Q sont logiquement équivalentes. $P\Leftrightarrow Q$
        \\\textsl{ABCD est un quadrilatère à 4 côtés égaux et ABCD est un losange sont logiquement équivalente.}
    \subsection{Éléments de logique}

        En mathématiques, pour exprimer un raisonnement où une propriété, on utilise des assertions. Ce sont des formules logique ayant une valeurs de vérité "vraie" ou "fausse".

        \emph{ex} : \textsl{"Un carré a quatre côtés égaux"} est une assertion vraie et \textsl{"Un carré possède 5 angles"} est une assertion fausse. \\
        
        Pour assembler des assertions et les lier entre elles, on utilise les connecteurs logiques suivants :
        \begin{table}[htbp]
            \centering
            \begin{tabular}{|c|c|c|c|c|}
                \toprule
                $\wedge$ & $\vee$ & $\neg$ & $\Rightarrow$ & $\Leftrightarrow$ \\
                \midrule
                "et" & "ou" & "non" & "implique" & "équivalente" \\
                \bottomrule
            \end{tabular}
        \end{table}


\section{Modes de démonstaration}

    \subsection{Modus Ponen}
        Soit P et Q deux assertions. On démontre que P est vraie et que P est une condition suffisante pour avoir Q. On a alors Q.
        \\$$P\wedge (P\Rightarrow Q)~\Rightarrow ~Q$$
        \\\textsl{On peut utiliser la transitivité de l'implication. $P\wedge ((P\Rightarrow Q)\wedge (Q\Rightarrow R))~\Rightarrow ~R$}

    \subsection{Contraposée}
        $$(P\Rightarrow Q)~\Longleftrightarrow ~(\neg Q\Rightarrow\neg P)$$
        Pour montrer que P est une condition suffisante pour avoir Q, on peut montrer que la négation de P est une condition suffisante pour avoir la négation de Q.

    \subsection{Disjonction de cas}
        $$Soient~P,~Q~et~R~trois~assertions.~~(P\vee Q)\wedge (P\Rightarrow R)\wedge (Q\Rightarrow R)~\Rightarrow ~R$$
        Pour montrer qu'un condition A est suffisante pour en avoir une seconde B, on la sépare en plusieurs cas, puis on montre que chaque cas est une condition suffisante pour avoir B.

    \subsection{Absurde}
        $$(\neg P\Rightarrow Q\wedge\neg Q)~\Rightarrow ~P$$
        \textsl{L'ensemble des nombres naturel est infini}

    \subsection{Analyse Synthèse}
        Utilisé pour démontrer l'existence et l'unicité d'un objet mathématique.
        \paragraph{Analyse}
        On détermine un certain nombre de conditions nécessaires.
        \paragraph{Synthèse}
        On détermine une condition suffisante parmis les nécessaires.

    \subsection{Récurrence}
        On définit un prédicat dépendant d'une variable.
        \\On montre alors que le prédicat est vrai pour un certain rang de la valeur.
        \\On montre ensuie que le prédicat vraie à un certain rang (ou sur une série de rangs) est une condition suffisante pour avoir le prédicat vrai à un autre rang.
        $$P(n)\Rightarrow P(n+1)~/~P(n_0)\wedge ...\wedge P(n)\Rightarrow P(n+1)~/~(P(n)\Rightarrow P(2n))\wedge (P(n+1)\Rightarrow P(n))$$
        \thm{ch0th1}{Théorème : Premier principe de récurrence}{Threcurrence}{Soit P(n) un prédicat définit sur $\mathbb{N}$
        \\Si on a $\left\{ \begin{array}{l} P(0)\\ \forall n\in\mathbb{N} ,~P(n)\Rightarrow P(n+1) \end{array} \right.$
        \\alors $\forall n\in\mathbb{N} ,~P(n)$}
        \begin{proof}
        On suppose au contraire $\exists n_0\in\mathbb{N} ^*$ tel que $\neg P(n_0)$.
        \\On considère alors A=$\left\{ 
        k \mid \neg P(k)
        \right\}$ 
        \\On a alors A$\neq\varnothing$ car $n_0\in$A et A$\subset\mathbb{N} ^*$ donc d'après le principe du bon ordre dans $\mathbb{N} ^*$ A admet un plus petit élément noté $k_0$. 
        \\Par suite $k_0-1\in$A soit P($k_0-1$) puis d'après l'hérédité P($k_0$).
        \end{proof} ${}$\\
        \thm{ch1th1c}{Corollaire : Principe de récurrence forte}{RecForte}{Soit P(n) un prédicat défini sur $\mathbb{N}$
        \\ Si $\left\{ \begin{array}{l} P(n_0)\\ \forall n\in\mathbb{N} ,~n\geq n_0,~P(n_0)\wedge ...\wedge P(n)~\Rightarrow ~P(n+1) \end{array} \right.$
        \\Alors $\forall n\in\mathbb{N} ,~n\geq n_0,~P(n)$} 
        \begin{proof}
        On considère le prédicat Q(n) = P($n_0$)$\wedge ...\wedge$P(n)
        \\ On a alors
        $\left\{
        \begin{array}{l}
        Q(n_0)\\
        \forall n\in\mathbb{N} ,~n\geq n_0,~Q(n)\Rightarrow Q(n+1)
        \end{array}
        \right.$
        \\ D'où d'après le premier principe de récurrence on a $\forall n\in\mathbb{N}$, $n\geq n_0, ~ P(n)$
        \end{proof}

    \subsection{Exemples}

        \paragraph{Irrationnalité de $\sqrt{2}$}

        \subparagraph{Preuve 1}

            On suppose $\exists$(p,q)$\in\mathbb{N} ^{*2}$ : $\sqrt{2} =\frac{p}{q}$ avec q minimal.
            \\On considère alors \[\frac{2q-p}{p-q} =\frac{2-\frac{p}{q}}{\frac{p}{q} -1} =\frac{\sqrt{2} (\sqrt{2} -1)}{\sqrt{2} -1} =\sqrt{2}\]
            \\avec p=$\sqrt{2}$q donc p<2q donc p-q<q.

        \subparagraph{Preuve 2}

            On suppose $\exists$(p,q)$\in\mathbb{N} ^{*2}$ : $\sqrt{2} =\frac{p}{q}$, soit 2q²=p².
            \\On a alors, d'après le théorème fondamental de l'arithmétique p² qui possède 2k fois 2 dans sa décomposition en facteurs premiers alors que 2q² le posssède 2k'+1 fois, ce qui est impossible par unicité de la décomposition.

        \subparagraph{Preuve 3}

            Pour i$\in\mathbb{N}$ on considère \[\epsilon _i=(\sqrt{2} -1)^i\]
            On a $\frac{8}{4} <\frac{9}{4}$ donc par strcite croissance de $f:x\mapsto\sqrt{x}$ $\sqrt{2} <\frac{3}{2}$ donc 0<$\sqrt{2}$-1<$\frac{1}{2}$
            \[Donc~\forall i\in\mathbb{N} ^*~\epsilon _i<\frac{1}{2^i}\]
            D'autre part pour tout entier $i$ il existe des entiers $a_i$ et $b_i$ tels que 
            \\$(\sqrt{2} -1)^i=a_i+\sqrt{2} b_i$
            \\Si $\exists (p,q)\in\mathbb{N} ^{*2}$ : $\sqrt{2} =\frac{p}{q}$ alors \[\epsilon _i=a_i+b_i\frac{p}{q} =\frac{a_iq+b_ip}{q} =\frac{A_i}{q} ~~~~A_i\in\mathbb{N} ^*\]
            Soit pour tout entier $i$ $\epsilon _i\geq\frac{1}{q}$ d'où $\frac{1}{q} <\frac{1}{2^i}$

        \paragraph{Infinité de l'ensemble des nombres premiers}

        \subparagraph{Lemme}

            Tout entier supérieur ou égal à $2$ admet un diviseur premier
            \\\textsl{Preuve} : Soit $n$ un entier supérieur à $2$ notons $p$ le plus petit de ses diviseurs.
            \\On a alors $p$ premier car tout diviseur de $p$ divise $n$.

        \subparagraph{Preuve d'Euclide}

            S'il y avait un nombre fini de nombres premiers, leur produit additionné de 1 serait divisible par l'un d'entre eux (\textsl{Lemme}), qui diviserait alors la différence, 1.

        \paragraph{Inégalité arithmético-géométrique}

        \subparagraph{Lemme de Couchy}

            Soit A un partie de $\mathbb{N} ^*$ qui contient 1 et \\telle que
            $\left\{
            \begin{array}{l}
            (1) ~\forall n\in\mathbb{N} ^*,~n\in A~\Rightarrow ~2n\in A \\
            (2) ~\forall n\in\mathbb{N} ^*,~n+1\in A~\Rightarrow ~n\in A
            \end{array}
            \right.$
            alors A=$\mathbb{N} ^*$
            \\\textsl{Preuve} : On veut démontrer Q(p) : $\begin{array}{l}
            2^p\in A \\
            \forall n\in [2^p, 2^{p+1}]\times\mathbb{N} , n\in A\\
            \Leftrightarrow \forall n\in [0, 2^p]\times\mathbb{N} , 2^{p+1}-n\in A
            \end{array}$
            \begin{multicols}{2}
            P(k) : $2^k\in A$  avec P(0)
            \\$2^k\in A \Rightarrow 2\times 2^k=2^{k+1}\in A$
            \\D'après le principe de récurrence on a $\forall k\in\mathbb{N} , 2^k\in A$
            \columnbreak
            H(n) : n>$2^p\vee 2^{p+1}-n\in A$ avec H(0)
            \\Si H(n) et n+1$\leq 2^p$, on a $2^{p+1}$-(n+1)$\in A$ d'apèrs (2)
            \\D'après le principe de récurrence on a $\forall (p,n)\in\mathbb{N} ^2, n>2^p\vee 2^{p+1}-n\in A$
            \end{multicols}

        \subparagraph{Preuve de \textsc{Cauchy}}

            On considère A=
            $\left\{
            \begin{array}{l}
            n\mid\forall (x_1,...,x_n)\in (\mathbb{R} _+^*)^n, \frac{x_1+...+x_n}{n}\geq\sqrt[n]{x_1...x_n}
            \end{array}
            \right\}$
            avec $1\in A$
            Soit le prédicat P(n) : $\forall (x_1,...,x_n)\in (\mathbb{R} _+^*)^n, \frac{x_1+...+x_n}{n}\geq\sqrt[n]{x_1...x_n}$   On a $P(1)\wedge P(2)$
            \\Supposons $n\in A$ et considérons $(x_1,...,x_n,x'_1,...,x'_n)\in (\mathbb{R} _+^*)^{2n}$
            \[\frac{x_1+...+x_n+x'_1+...+x'_n}{2n} =\frac{\frac{x_1+...+x_n}{n} + \frac{x'_1+...+x'_n}{n}}{2}\] \\\[\geq\sqrt{\frac{x_1+...+x_n}{n}\times\frac{x'1+...+x'_n}{n}}\geq\sqrt[2]{\sqrt[n]{x_1...x_n}\times\sqrt[n]{x'_1...x'n}}\] \\\[=\sqrt[2n]{x_1...x_nx'_1...x'_n} ~~~~Soit~P(n)\Rightarrow P(2n)\]
            \\On considère maintenant $\forall (x_1,...,x_n,x_{n+1})\in (\mathbb{R} _+^*)^{n+1}, \frac{x_1+...+x_n+x_{n+1}}{n+1}\geq\sqrt[n]{x_1...x_nx_{+1}}$
            \\Soit $\forall (x_1,...,x_n)\in (\mathbb{R} _+^*)^n$ Posons $x_{n+1}=\frac{x_1+...+x_n}{n}$ on a alors avec $P=x_1\cdots x_n$ et $A = x_{n+1}$ :
            \[\frac{x_1+...+x_n+\frac{x_1+...+x_n}{n}}{n+1}\geq\sqrt[n+1]{x_1...x_nx_{n+1}}
            \Leftrightarrow\frac{(n+1)\frac{x_1}{n} +...+(n+1)\frac{x_n}{n}}{n+1}\geq\sqrt[n+1]{PA}\]
            \[\Leftrightarrow\frac{x_1+...+x_n}{n}\geq\sqrt[n+1]{PA} ~\Leftrightarrow ~A\geq\sqrt[n+1]{PA}\]
            \[\Rightarrow A^{n+1}\geq PA \Rightarrow A^n\geq P\Rightarrow A\geq\sqrt[n]{P} ~~donc~P(n+1)\Rightarrow P(n)\]
        
        \subparagraph{Preuve d'\textsc{Enguel}}
            \textsl{Lemme} : $\forall x\in\mathbb{R} ^*, \ln x\leq x-1$ avec égalité ssi x=1
            \\\textsl{Preuve} : Soit $(x_1,...,x_n)\in (\mathbb{R} _+^*)^n ~~A=\frac{1}{n} \sum\limits_{i=1}^nx_i$
            $\forall i\in [\![i,n]\!] ,~\ln (\frac{x_i}{A})\leq\frac{x_i}{A} -1$
            \\ En sommant on obtient:
            \[\sum\limits_{i=1}^n\ln (\frac{x_i}{A}) = \ln (\frac{x_1...x_n}{A^n})\leq\sum\limits_{i=1}^n (\frac{x_i}{A} -1) = 0~~~
            \Rightarrow x_1...x_n\leq A^n ~~\Rightarrow ~\frac{x_1+...+x_n}{n}\geq\sqrt[n]{x_1...x_n}\] \\ 
    
    \begin{center}
    \fin
    \end{center}

\chapter{Ensembles et applications}

    
% Chapitre 1 : Ensemble et applications

\minitoc
	\section{Opérations sur les Parties}
	\subsection{Notations}
		\traitd
		\paragraph{Complémentaire}
			Le complémentaire de A dans E est $E\backslash A=\overline{A} =A^c$
			\[E\backslash A=\{x \in E \mid x \notin A \}\] \vspace*{-0.7cm} \trait ${}$ \vspace*{-1.4cm} \traitd
		\paragraph{Union}
			L'union de deux ensembles est \[A \cup B = \{ x \in E \mid x \in A \vee x \in B\} \] 
			\vspace*{-0.7cm} \trait \newpage \traitd
		\paragraph{Intersection}
			L'intersection de deux ensembles est \[A \cap B = \{ x \in E \mid x \in A \wedge x \in B\} \]
			\vspace*{-0.7cm} \trait ${}$ \vspace*{-1.4cm} \traitd
		\paragraph{Différence}
			La différence de deux ensemble est \[ A \backslash B = \{ x \in E \mid x \in A \wedge x \notin B \} = A \cap \overline{B} \] 
			\vspace*{-0.7cm} \trait
	\subsection{Propriétés}
		Soit A et B deux parties de E
 		\col{$E \backslash (E \backslash A) \equiv A $ \\$A \cap (B \cap C) \equiv (A \cap B) \cap C$ \\ $A\cap (B\cup C) \equiv (A\cap B)\cup 
 		(B\cap C)$ \\ $E\backslash (A\cap B)\equiv (E\backslash A)\cap (E\backslash B)$}{$A \cup (B \cup C) \equiv (A \cup B) \cup C $\\$ A \cup 
 		(B\cap C) \equiv (A\cup B)\cap (A\cup C)$\\$ E\backslash (A\cup B)\equiv (E\backslash A)\cup (E\backslash B)$}
	\section{Recouvrement disjoint et Partitions}
	Soit $E$ un ensemble et $\big(A_i \big)_{_{i\in I}}$ une famille d'éléments de $E$.
		\traitd
		\paragraph{Famille de parties disjointes de E}
			\underline{$(A_i)$ est une famille de parties disjointes de $E$} si 
			\[\left\{ \begin{array}{l} \forall (i,j)\in I^2,~i\neq j~\Rightarrow A_i\cap A_j=\varnothing \\ \forall i\in I,~A_i \in \mathcal{P}(E)
			\end{array} \right. \] \vspace*{-0.7cm} \trait ${}$ \vspace*{-1.4cm} \traitd
		\paragraph{Recouvrement disjoint de B}
			\underline{$(A_i)$ est un recouvrement disjoint de $B$} si \vspace*{0.3cm}\\ les $A_i$ sont deux à deux disjoints et 
			$B\subset \bigcup\limits_{i\in I} A_i = \{x\in E \mid\exists i\in I : x\in A_i\}$ \trait ${}$ \vspace*{-1.4cm} \traitd
		\paragraph{Partition de E}\label{def partition}
			$(A_i)_{_{i\in I}}$  est une partition de E si
			\[E=\bigcup\limits_{i\in I} A_i  ~~~~\wedge ~~~~
			\left\{ \begin{array}{l} \forall i\in I, A_i \in \mathbb{P}(E)\\
			\forall i\in I, A_i \neq \varnothing\\
			\forall (i,j)\in I^2,~i\neq j~\Rightarrow A_i\cap A_j=\varnothing \end{array} \right.\] \vspace*{-0.7cm} \trait
		\thm{ch2P1}{Propriétés : Lois de Morgan}{LoisMorgan}{
		Soit E un ensemble, $(A_i)_{_i\in I)}$ une famille de parties de E \\ \hspace*{0.5cm}
		Alors $\Big(\bigcup\limits_{i\in I} A_i \Big)^c \equiv \bigcap\limits_{i\in I} A_i^c $ et $ \Big(\bigcap\limits_{i\in I} A_i\Big)^c \equiv 
		\bigcup\limits_{i\in I} A_i^c $ }
	\section{Éléments applicatifs}
	\subsection{Graphe}
		Soit un fonction $f\in\mathcal{F}(E,F)$, son graphe est :\begin{center} \highlight{$ \displaystyle{ \Gamma =\{ (x,f(x))\mid x\in E \}\in 
		\mathcal{P} (E\times F) } $ } \end{center}
	\subsection{Indicatrice}
		\traitd
 		\paragraph{Définition}
 			On définit l'indicatrice de A dans E comme 
 			\[\mathbb{1} _A \left( \begin{array}{l} E\longrightarrow \{0;1\} \\ x\mapsto \left\{ \begin{array}{l} 1~si~x\in A \\ 0~si~x\in A^c 
 			\end{array} \right. \end{array} \right)\] \vspace*{-0.7cm} \trait
 		\hspace*{1.5cm} \newtheorem*{ch2P2}{Propriétés}  \begin{minipage}{12.71cm} \begin{ch2P2} ${}$ \vspace*{0.15cm}\\ \hspace*{0.21cm} 			
 		\begin{blockarray}{|ll}
 		Soit $A$ et $B$ deux parties d'un ensemble & \hspace*{-0.25cm}$E$ on a  \\
 		$A \equiv B ~\Leftrightarrow ~ \mathbb{1}_A = \mathbb{1}_B$ & $\mathbb{1}_{A\cap B } = \mathbb{1}_A \cdot \mathbb{1}_B$ \\ 
 		$\forall x\in E ,~\mathbb{1}_A + \mathbb{1}_{E\backslash A} = \mathbb{1}_E = 1$ & $\mathbb{1}_{A\cup B} = \mathbb{1}_A + \mathbb{1}_B - 
 		\mathbb{1}_{A\cap B}$
 		\end{blockarray} \end{ch2P2} \end{minipage} \vspace*{-0.25cm} ${}$
	\section{Relations binaires}
		\traitd
		\paragraph{Définition}
			Une relation binaire sur $E$ est la donné d'une partie $\Gamma$ de $E\times E$ telle que \\
			\[ \forall (x,y) \in E^2 ~,~~x\mathcal{R} y ~\Leftrightarrow ~(x,y) \in \Gamma \] \vspace*{-0.7cm} \trait
		\vspace*{-1.1cm} \\ $\Gamma$ est appelé graphe de la relation binaire $\mathcal{R}$ 
		\\ -> \underline{ex}: $\Gamma \subset \mathbb{R}^2 ~~~~ (x,y) \in \Gamma ~\Leftrightarrow ~ y \leq x$ 
		\traitd
		\paragraph{Caractéristiques} Soit $\mathcal{R}$ une relation binaire sur un ensemble $E$
			\\ \hspace*{2.5cm} {\small 1)} $\mathcal{R}$ est \underline{réflexive} si $~~\forall x\in E ~,~~ x\mathcal{R} x$
			\\ \hspace*{2.5cm} {\small 2)} $\mathcal{R}$ est \underline{symétrique} si $~~\forall (x,y) \in E^2 ~,~~ x\mathcal{R} y 
			~\Leftrightarrow ~ y\mathcal{R} x$ \\ \hspace*{2.5cm} {\small 3)} $\mathcal{R}$ est \underline{antisymétrique} si $~~\forall (x,y) 
			\in E^2 ~,~~ x\mathcal{R} y ~\wedge ~y\mathcal{R} x ~\Rightarrow ~x=y$ \\ \hspace*{2.5cm} {\small 4)} $\mathcal{R}$ est 
			\underline{transitive} si $~~\forall (x,y,z) \in E^3 ~,~~ x\mathcal{R} y \wedge y\mathcal{R} z ~\Rightarrow ~ x\mathcal{R} z$ \trait
		\traitd
		\paragraph{Relation d'ordre}
			Un relation binaire $\mathcal{R}$ est une relation d'ordre si \\ $\mathcal{R}$ est \underline{réflexive, antisymétrique et transitive}.
			\trait
		->\underline{ex}: $\forall (z,z') \in \mathbb{C}^2 \\ \Big(\Re(z)<\Re(z')\Big) \vee \Big(\Re(z)=\Re(z') \wedge \Im(z)\leq\Im(z')\big) $ 
		est une relation d'ordre sur $\mathbb{C}$
		\subparagraph{Caractère total}
			Une relation d'ordre $\mathcal{R}$ est dite totale si $~\forall (x,y) \in E^2 ,~~ x\mathcal{R} y ~\vee ~y\mathcal{R} x$
		\traitd
		\paragraph{Relation d'équivalence}
			Un relation binaire $\mathcal{R}$ est une relation d'ordre si \\ $\mathcal{R}$ est \underline{réflexive, symétrique et transitive}.
			\trait
		->\underline{ex}: Si $a\in \mathbb{R}$, $\forall (x,y) \in \mathbb{R}^2 ,~~ x\mathcal{R} y ~\Leftrightarrow ~\exists k\in \mathbb{Z} ~:~
		y-x = ka$ \\ $\mathcal{R}$ est appelée relation de congruence modulo $a$ et on note $x \equiv y [a]$
		\traitd
		\paragraph{Classe d'équivalence} 
			Si $x\in E$ l'ensmeble $\{y\in E ~\vert ~x\mathcal{R} y\}$ souvent noté $Cl(x)$ est la classe d'équivalence de $x$ \trait
		\thm{ch2P3}{Propriété}{ClEqPartE}{Si $E\neq \varnothing$, les classes d'équivalence forment une partition de $E$}
		\vspace*{0.5cm} \\ 
		\begin{center}
		\fin
		\end{center}

\chapter{Calculus}

    
% Chapitre 2 : Calculus

\minitoc
	\section{Sommes et Produits}
		On considère une famille $(a_i)_{_{i\in I}}$ de réels.
		\col{$\sumi$ est la \textbf{somme} de ses termes}{$\prodi$ est le \textbf{produit} de ses termes}
 
 \paragraph{Somme et Produit Téléscopique} 
 ${}$ 
 \col{$\sk{1}{n-1} (a_{k+1}-a_k) ~=~ a_n - a_1$}{$\pk{1}{n-1} (\frac{a_{k+1}}{a_k} ) ~=~ \frac{a_n}{a_1}$}
 
 \paragraph{Permutations}
 Soit $\sigma$ une bijection de $I$ sur $I$, 
 $~~\sumi a_{\sigma (i)} ~=~ \sumi a_i$ \\
 ->\underline{ex}: \hspace*{25pt} $\sk{1}{n} a_k ~=~ \sk{1}{n} a_{n+1-k}$
 
 \paragraph{Méthode de perturbation}
 Soit $(a_i)_{_{i\in I}}$ on note $S_n = \sk{1}{n} a_k$\\
 \[\underline{S_{n+1}} = \underline{S_n} + a_{n+1} = a_1 + \underline{\sk{2}{n+1}}\]\\
 ->\underline{ex}:\hspace*{25pt} Soit $S_n = \sk{1}{n} 2^k$ \\
 $S_{n+1} = S_n + 2^{n+1} = 2 + \sk{2}{n+1} 2^k = 2 + 2\times\sk{1}{n} 2^k ~ \Rightarrow ~ S_n + 2^{n+1} = 2S_n + 2 \\\Rightarrow ~S_n = 2^{n+1} - 2$
 
 \paragraph{Sommes doubles}
 
 Soit $(a_i)_{_{i\in I}}$ et $(b_j)_{_{j\in J}}$ des familles de réels 
 \col{$\sum\limits_{(i,j)\in I\times J} a_i b_j ~=~ \left( \sumi a_i \right) \left( \sum\limits_{j\in J} b_j \right)$ \\ $ \sum\limits_{1\leq i<j\leq n} a_ib_j ~=~ \sum\limits_{i=1}^{n-1} a_i \sum\limits_{j=i+1}^{n} b_j$}{${}$\\$\sum\limits_{(i,j)\in I\times J} a_{ij} ~=~ \sumi \sum\limits_{j\in J} a_{ij} ~=~ \sum\limits_{j\in J} \sumi a_{ij}$}
 Si $(a_k)$ et $(b_k)$ on la même monotonie $~~\sum\limits_{1\leq j<k\leq n} (a_k - a_j )( b_k - b_j) ~\geq ~0$
 
 \section{Coefficients binomiaux}
 
 $\forall (n,p) \in \mathbb{N}^2 ,~~ \binom{n}{p} = \pk{1}{p} \frac{n-k+1}{k} ~~~~ =~~~~ \left\{ \ar* 0 ~si~p>n \\\frac{n!}{k!(n-k)!} ~~sinon \ar \right.$
 
 \paragraph{Calculs sur les coefficients binomiaux}
 
 \subparagraph{Relation de Pascal}
 
 Si $1\leq p \leq n$ alors $~~~~ \binom{n}{p} = \binom{n-1}{p} + \binom{n-1}{p-1}$
 
 \subparagraph{Propriété de symétrie}
 
 $\forall (n,p) \in \mathbb{N}^2 ,~~~~p\leq n ~\Rightarrow ~\binom{n}{p} = \binom{n}{n-p}$
 
 \subparagraph{Formule d'absorbtion}
 
 $\forall (n,p)\in\mathbb{N}^2 ,~~ \binom{n}{p} = \frac{n}{p}\binom{n-1}{p-1}~$ ou $~p\binom{n}{p} = n\binom{n-1}{p-1}$
 
 \paragraph{Binôme de \underline{Newton}}
 
 \[\forall (a,b)\in\mathbb{R}^2 ,~~\forall n\in\mathbb{N} ~,~~ (a+b)^n ~=~ \sk{1}{n} \binom{n}{k} a^k b^{n-k}\]
 
 \section{Valeur absolue}
 
 On note $a^+ = \max (a,0)$ et $a^- = \max (-a, 0)$. On a alors\\
 $\forall a\in \mathbb{R} ~,~~ a = a^+ - a^-$ et $ \vert a\vert = a^+ + a^- = \left\{ \ar* a~si ~a\geq 0 \\ -a ~~sinon \ar \right.$
 
 \paragraph{Somme et produit}
 
 $\left\vert \prod\limits_{i=1}^{n} a_i \right\vert = \prod\limits_{i=1}^{n} \vert a_i\vert ~~~~$ et $~~~~\left\vert \sum\limits_{i=1}^n a_i \right\vert \leq \sum\limits_{i=1}^n \vert a_i \vert$
 
 \section{Trigonométrie}
 
 On défini deux fonction \textbf{sin} et \textbf{cos} par la relation :
\\$\mathcal{C} (0;1) = \{(\cos x , \sin x ) ~\vert ~x\in\mathbb{R}\}~~$ ou encore $~~\forall x\in\mathbb{R} ~,~~ \cos^2 x + \sin^2 x = 1$
\col{$\cos x = \cos a ~\Leftrightarrow ~\left\{\ar* x\equiv a[2\pi] \\ x \equiv -a [2\pi] \ar \right.$}{$\sin x = \sin a ~\Leftrightarrow ~\left\{\ar* x\equiv a[2\pi] \\ x \equiv \pi -a [2\pi] \ar \right.$}
 
 \paragraph{Formules majeures}
 
 \subparagraph{Addition}
 $\hspace*{20pt}\left\vert \ar* \cos (\alpha + \beta ) ~=~ \cos\alpha\cos\beta - \sin\alpha\sin\beta \\ \sin (\alpha + \beta ) ~=~ \sin\alpha\cos\beta + \sin\beta\cos\alpha \ar\right\vert$
 
 \subparagraph{Duplication}
 
 $\left\vert \ar* \cos (2\alpha ) ~=~ 2\cos^2 (\alpha ) -1 ~=~ 1-2\sin^2 (\alpha ) \\ \sin (2\alpha ) ~=~ 2\sin\alpha\cos\alpha \ar\right\vert$
 
 \subparagraph{Dérivation}
 
 $ \hspace*{25pt}\left\vert \ar* \cos 'x ~=~ -\sin x ~=~ \cos (x+\frac{\pi}{2} ) \\ \sin 'x ~=~ \cos x ~=~ \sin (x+\frac{\pi}{2} ) \ar \right\vert$
 
 \paragraph{Tangente}
 
 On définit $\tan x ~=~ \frac{\sin x}{\cos x}~~$ avec $~~ \mathcal{D}_{\tan} = \mathbb{R}\backslash \{\frac{\pi}{2} + k\pi ~\vert ~k\in \mathbb{Z} \}$
 
 \col{$\tan (\alpha + \beta ) ~=~ \frac{\tan\alpha + \tan\beta}{1 - \tan\alpha\tan\beta}$}{$\tan (\alpha - \beta ) ~=~ \frac{\tan\alpha - \tan\beta}{1 + \tan\alpha\tan\beta}$}
 
 \col{$\cos x ~=~ \frac{1 - \tan^2 (\frac{x}{2})}{1+\tan^2 (\frac{x}{2} )}$}{$\sin x ~=~ \frac{2\tan (\frac{x}{2} )}{1+ \tan^2 (\frac{x}{2} )}$}

\chapter{Nombres Complexes}

    
% Chapitre 3 : Nombres complexes
 
 On définit $i$ tel que $i^2 = -1$ \hspace*{20pt} \underline{\textsc{Attention}} On ne peut pas écrire $i = \sqrt{-1}$
 \minitoc
 \section{Calcul dans $\C$}
 \paragraph{Puissances de $i$}
 $\forall p\in \mathbb{Z} , ~~\ar* i^{4p} ~=~ 1 \hspace*{40pt} i^{4p+1} ~=~ i \\ i^{4p+2} ~=~ -1 \hspace*{22pt} i^{4p+3} ~=~ -i \ar$
 \paragraph{Identitées remarquables} ${}$\\
 \hspace*{25pt} Si $z\in\mathbb{C} ~,~~ n\in\mathbb{N} ~,~~ \sk{0}{n} z^k ~=~ \left\{ \ar* n+1 ~si~ z=1 \\ \frac{1-z^{n+1}}{1-z} ~~sinon \ar\right.$ \\
 \hspace*{25pt} Si $(a,b)\in\mathbb{C}^2 ~,~~ n\in\mathbb{N}^* ~,~~ a^n -b^n ~=~ (a-b) \sk{0}{n-1} a^k b^{n-1-k} $\\
 \hspace*{25pt} Si $(a,b)\in\mathbb{C}^2 ~,~~ n\in\mathbb{N}^* ~,~~ (a+b)^n ~=~ \sk{0}{n} \binom{n}{k} a^k b^{n-k}$
 \section{Conjugaison et module}
 \subsection{Opération de conjugaison}
 On définit l'opération \textbf{involutive} de \textbf{conjugaison} :\\ $\forall z=a+ib ~\in\mathbb{C} ~~\varphi : a+ib ~\mapsto  ~a-ib~~$ et $~~ \varphi \circ \varphi ~=~ Id_{\mathbb{C}}$\vspace*{10pt}\\
 Avec $~~\forall (z_1, \cdots , z_n) \in\mathbb{C}^n ,~~ \overline{\sk{0}{n} z_k} ~=~ \sk{0}{n} \overline{z_k}~~$ et $~~ \overline{\pk{0}{n} z_k} ~=~ \pk{0}{n} \overline{z_k}$
 \paragraph{Parties réelles et imaginaires} $~~\forall z\in\mathbb{C} ~$ on a $~\Re (z) ~=~ \frac{z+\overline{z}}{2} ~$ et $~\Im (z) ~=~ \frac{z-\overline{z}}{2}$
 \subsection{Module du complexe}
 On définit le \textbf{module} de $z\in\mathbb{C}$ comme le \textbf{réel} positif qui vérifie $\abs{z}^2 ~=~ z\overline{z}$\\On a alors l'égalité $\abs{z} ~=~ \sqrt{a^2 + b^2}$
 \subsection{Inégalité triangulaire}
 \paragraph{Propriété préliminaire} On a $~\forall z\in\mathbb{C} ~,~~ \left\{ \ar* \abs{\Re (z)} ~\leq ~ \abs{z} \\ \abs{\Im (z)} ~\leq ~ \abs{z} \ar\right.$\vspace*{5pt}\\
 \thm{th5.1}{Inégalité Triangulaire}{InegT1}{${}$\\$\forall (z,z') \in\mathbb{C}^2 ~~$ on a $~~\abs{z+z'} ~\leq ~\abs{z} + \abs{z'}$\\ Avec égalité dans l'inégalité si et seuleument si $\exists \lambda \in \mathbb{R}^+ $ tel que $z = \lambda z'$ ou si $z' =0$}
 \begin{proof}
 $\forall (z,z') \in\mathbb{C}^2  ~~~~ \abs{z+z'}^2 = \abs{z}^2 + 2\Re (z\overline{z'}) + \abs{z'}^2$ \\ avec $~\Re (z\overline{z'}) \leq \abs{\Re (z\overline{z'})} \leq \abs{zz'} = \abs{z}\abs{z'}$ d'où $\abs{z+z'}^2 \leq \left(\abs{z} + \abs{z'} \right)^2$\vspace*{5pt}\\
 avec égalité si et seulement si $\Re (z\overline{z'}) = \abs{\Re (z\overline{z'})} = \abs{z\overline{z'}}$ soit $z\overline{z'}\in\mathbb{R}^+$\vspace*{5pt}\\ Si $z\neq 0$ alors $z\overline{z'}\in\mathbb{R}^+ ~\Leftrightarrow ~ z\frac{\overline{z'} z'}{z'}\in\mathbb{R}^+ ~\Leftrightarrow ~ z\frac{\abs{z'}^2}{z'} \in\mathbb{R}^+ \\ \Leftrightarrow ~z = \lambda z'$ avec $ \lambda = \frac{z}{z'} \in\mathbb{R}^+$
 \end{proof} ${}$\\
 \thm{th5.2}{Seconde inégalité triangulaire}{InegT2}{${}$\\  $\forall (z,z') \in \mathbb{C}^2 ~,~~ \left\{ \ar* \abs{z-z'} \geq \abs{z} - \abs{z'} \\ \abs{z'-z} \geq \abs{z'} - \abs{z} \ar\right. ~~\Rightarrow ~ \abs{z-z'} ~\geq ~\abs{\abs{z} - \abs{z'}}$}
 \section{Unimodulaires et trigonométrie}
 Dans le plan complexe le cercle trigonométrique $\mathcal{C} (0,1)$ est l'ensemble des nombres complexes unimodulaires noté $\mathbb{U} ~=~ \{z\in\mathbb{C} ~\vert ~\abs{z} = 1 \}$ \\ \hspace*{25pt} -> $\mathbb{U} ~=~ \{\cos\theta +i\sin\theta ~\vert ~\theta \in [0, 2\pi [ \} ~=~ \{e^{i\theta } ~\vert ~\theta \in [0,1\pi [ \}$
 \paragraph{Calculs} $\mathbb{U} \subset \mathbb{C}^*$ est stable par produit et quotient et $\forall z\in \mathbb{U} ~,~~ \frac{1}{z} = \overline{z}$
 \paragraph{Formules d'Euler} $\forall z\in \mathbb{U} ~,~~ z=e^{i\theta } ~~(\theta \in \mathbb{R} )$
 \col{$\Re (z) = \frac{z+\overline{z}}{2} ~\Leftrightarrow ~\cos\theta ~=~ \frac{e^{i\theta }+e^{-i\theta }}{2}$}{$\Im (z) = \frac{z-\overline{z}}{2i} ~\Leftrightarrow ~ \sin\theta ~=~ \frac{e^{i\theta } - e^{-i\theta }}{2i}$}
 \subsection{Technique de l'angle moitié} 
 ${}$\\
 \thm{th5.3}{Angle moitié 1}{AngleMoitié}{${}$\\ $\forall t \in \mathbb{R} \ar* 1+e^{it} = 2\cos (\frac{t}{2} )e^{i\frac{t}{2}} \\ 1-e^{it} = 2i\sin (-\frac{t}{2} )e^{i\frac{t}{2} } \ar ~\left\vert ~~ \forall (p,q) \in \mathbb{R}^2 \ar* e^{ip} + e^{iq} = 2\cos (\frac{p-q}{2} )e^{i\frac{p+q}{2} } \\ e^{ip} - e^{iq} = 2i\sin (\frac{p-q}{2} e^{i\frac{p-q}{2}}\ar \right.$}
 \\${}$\\
 \thm{th5.4}{Angle moitié 2}{AngleMoitié2}{$~~\forall (p,q) \in \mathbb{R}^2$\\ $\ar* \cos p + \cos q = 2\cos\frac{p-q}{2}\cos\frac{p+q}{2} \vspace*{3pt} \\ \sin p + \sin q = 2\cos\frac{p-q}{2}\sin\frac{p+q}{2} \ar ~\left\vert ~~ \ar* \cos p - \cos q = -2\sin\frac{p-q}{2}\sin\frac{p+q}{2} \vspace*{3pt} \\ \sin p - \sin q = 2\sin\frac{p-q}{2}\cos\frac{p+q}{2} \ar \right.$}
    
\chapter{Fonctions}

    
% Chapitre 4 : Fonctions

\textit{Toute les fonctions considéré sont des fonction d'une variable réelles à valeurs dans $\R$ définies sur $I\subset \R$}
\minitoc
	\section{Généralités sur les fonctions}
		\traitd
		\paragraph{Ensemble de définition}
			Si $f$ est une fonction on défnit \underline{$D_f$ son ensemble de définition} comme la plus grande partie de $\R$ sur laquelle $f$ est 
			définie. \trait ${}$ \vspace*{-1.4cm} \traitd
		\paragraph{Représentation graphique}
			Soit $f$ un fonction la \underline{représentation graphique de $f$} est la partie de $\R^2$ $C_f = \{\big(x,f(x)\big) ~|~
			x\in D_f \}$ \trait
		\thm{ch5P1}{Propriété}{PariteCourbe}{Soit $f$ une fonction à valeurs réelles on a \\ $\bullet$ Si $f$ est paire alors $C_f$ admet 
		$(0x)$ comme axe de symétrie \\ $\bullet$ Si $f$ est impaire alors $C_f$ admet $0$ comme centre de symétrie.} \traitd
		\paragraph{Périodicité}
			On dit que $f$ est périodique s'il existe $T\in\R^*$ tel que $\forall x\in D_f,~ x+T\in D_f$ et $f(x+T)=f(x)$, on dit alors que $f$ est 
			$T$-périodique. \trait
		\vspace*{-1.1cm} \\ \underline{Rq} : la périodicité n'est stable ni par somme, ni par produit.
		\vspace*{0.5cm} \\ \thm{ch5P2}{Propriétés}{5-P2}{Soit $f$ et $g$ deux fonctions on a \\ {\small 1)} Si $f$ et $g$ admettent un parité, 
		alors $f+g$ et $f.g$ admettent la même parité. \\ {\small 2)} Si $f$ et $g$ sont $T$-périodiques, alors $f+g$ et $f.g$ sont $T$-
		périodiques. \\ {\small 3)} $g\circ f$ est paire si $f$ est paire ou si $f$ est impaire et $g$ est paire. \\ {\small 4)} $g\circ f$ est 
		impaire si $f$ et $g$ le sont.} \traitd
		\paragraph{Fonction croissante} On dit que $f$ à valeurs réelles est \underline{croissante sur $I$} (resp. décroissante) si 
		\[\forall (a,b) \in I^2 ,~a\leq b\Rightarrow f(a)\leq f(b) ~~ \big(resp. ~a\leq b\Rightarrow f(a) \geq f(b) \big) \]
		On définie de même les strictes croissance et décroissance avec des inégalités strictes. \trait
		\thm{ch5P3}{Propriété}{CarFCroiss}{$f$ est croissante (resp. strictement) sur $I$ \underline{si et seulement si} \\
		$\forall (a,b) \in I^2 ,~a\neq b\Rightarrow \frac{f(b)-f(a)}{b-a} \geq 0$ \big(resp. $>0$\big) }
	\section{Dérivation}
		\traitd
		\paragraph{Dérivabilité en $a$}
			On dit que \underline{$f$ est dérivable en un point $a$} de $I$ qui n'est pas une extremité de $I$ si $\tau_a(f)$ admet une limite 
			finie en $a$. On note alors $f'(a)$ cette limite.\trait ${}$\vspace*{-1.4cm} \traitd
		\paragraph{Dérivabilité sur $I$}
			On dit que \underline{$f$ est dérivable sur $I$} si $f$ est dérivable en tout point de $I$. On note alors $f'$ la fonction définie sur 
			$I$ qui à chaque point $a$ associe $f'(a)$. \trait
		\thm{ch5P4}{Propriétés}{OpeDeriv}{Si $f$ et $g$ sont deux fonction dérivable en $a$ on a \\ {\small 1)} $\forall \alpha \in \R,~
		\alpha f+g$ est dérivable en $a$ et $\big(\alpha f+g\big)'(a) = \alpha f'(a) + g'(a)$ \\ {\small 2)} $f.g$ est dérivable en $a$ et $
		\big(f.g\big)'(a) = f'(a).g(a) + f(a).g'(a)$ \\ {\small 3)} Si $g(a)\neq 0$ alors $\frac{f}{g}$ est dérivable en $a$ et 
		$\Big(\frac{f}{g}\Big)'(a) = \frac{f'(a).g(a)-f(a)g'(a)}{\big(g(a)\big)^2}$}
		\newpage ${}$ \\ \thm{ch5P5}{Proposition}{DerivCompo}{Si $f$ est dérivable en $a$ et $g$ est dérivable en $f(a)$ \\Alors $g\circ f$ est 
		dérivable en $a$ et $\big(g\circ f \big)'(a) = \big(g'\circ f\big)(a) \times f'(a)$}
		\vspace*{0.5cm} \\ \thm{ch5P6}{Proposition : Caractérisation des fonctions constantes}{CarFcCte}{Une fonction définie sur $I$ à valeurs 
		réelles ou complexes est constante \\\underline{si et seulement si} elle est dérivable sur $I$ et sa dérivée est nulle sur $I$}
		\vspace*{0.5cm} \\ \thm{ch5P7}{Propriété}{StrictCroissCNS}{Si $f$ est dérivable sur $I$ alors $f$ est strictement croissante sur $I$ \\
		$\Leftrightarrow $ $\left\{ \ard f'$ est positive sur $I \\ $il n'existe pas d'intervalle ouvert $I\subset J$ tel que $f'|_J=0\arf\right.$}
		\vspace*{0.5cm} \\ \thm{ch5th1}{Théorème}{DerStrictCroiss}{Soit $f$ dérivable qur un intervalle ouvert $I$ strictement monotone sur $I$\\
		alors $f$ réalise une bijection de $I$ sur $f_d(I)=J$ et $f^{-1}$ est continue et \\dérivable sur $J$ avec $\forall b=f(a)\in J , 
		~\big(f^{-1}\big)'(b) = \frac{1}{f'(a)} = \frac{1}{f'\circ f^{-1}(b)}$}
		\vspace*{0.5cm} \\ \thm{ch5P8}{Propriété}{CfCf-1Sym}{Si $f$ est à valeurs réelles bijectives et $\R^2$ rapporté à un repère orthonormé 
		direct \\ Alors $C_f$ et $C_{f^{-1}}$ sont symétriques par rapport à la première bisectrice.} \traitd
		\paragraph{Classe $\cont^1$}
			On dit que \underline{$f$ est de classe $\cont^1$ sur $I$ à valeurs dans $\R$} si $f$ est dérivable sur $I$ et so $f'$ est continue 
			sur $I$. On dit aussi que $f$ est continuement dérivable sur $I$. \\ On note $\cont^1\big(I,\R\big)$ l'ensemble des fonctions de 
			classe $\cont^1$ sur $I$ à valeurs dans $\R$. \trait
	\section{Fonctions usuelles}
		\traitd
		\paragraph{Logarithme népérien}
			La fonction $\ln$ est l'unique primitive de $\appli{\R_+^*}{x}{\R}{\frac{1}{x}}$ avec $\ln(1)=0$. \trait
		\thm{ch5P9}{Propriétés}{Calcln}{{\small 1)} $\forall (a,b)\in \R_+^* ,~\ln(ab) = \ln(a)+\ln(b)$ \hspace*{0.5cm} 
		{\small 2)} $\forall x \in \R_+^* ,~\ln \big( \frac{1}{x} \big) = -\ln(x) $ }
		\vspace*{0.5cm} \\ \thm{ch5th2}{Théorème}{InegFondAnalyse}{$\forall x\in \R_+^* ,~\ln(x+1)\leq x$} \traitd
		\paragraph{Exponnentielle}
			La fonction $\exp$ est la bijection réciproque de $\ln$, définie sur $\R$ à valeurs dans $\R_+^*$. Elle est dérivable sur $\R$ avec 
			$\exp'=\exp$ et $\forall (x,y)\in\R ,~\exp(x+y) = \exp(x)\times \exp(y)$ \trait 
		\vspace*{-1.1cm} \\ $\forall (a,b) \in \R_+^* \times\R ,~a^b = \exp\big(b\times\ln(a)\big)$ \newpage \traitd
		\paragraph{Logarithme en base $a$}
			Soit $a\in\R_+^*\backslash\{1\}$, on appelle \underline{logarithme en base $a$ noté $\ln_a$} la fonction 
			$\appli{\R_+^*}{x}{\R}{\frac{\ln x}{\ln a} }$ et on note $\exp_a$ sa bijection réciproque. \trait
		\thm{ch5L1}{Lemme}{CCln}{$\forall \alpha \in \R_+^*$, on a \hspace*{1cm} 
		{\small 1)} $\frac{\ln x}{x^\alpha} \stox{+\infty} 0$ \hspace*{1cm} {\small 2)} $x^\alpha \ln x \stox{0^+} 0$}
		\vspace*{0.5cm} \\ \thm{ch5L1c}{Corollaire}{CCexp}{$\forall \alpha \in \R_+^*$, on a $\frac{x^\alpha}{e^x} \stox{+\infty} 0$}
		\vspace*{0.5cm} \\ \thm{ch5P10}{Proposition}{DefExpLim}{Soit $x\in\R$ alors $\big(1+\frac{x}{t} \big)^t 
		\underset{t\to +\infty}{\rightarrow} e^x$}
		\vspace*{0.5cm} \\ \thm{ch5P11}{Proposition}{InegExp}{$\forall x\in \R, ~e^x \geq x+1$ avec égalité $\Leftrightarrow$ $x=0$}
		\traitd
		\paragraph{Arcsinus}
			La restriction de $\sin$ à $\big[-\frac{\pi}{2},\frac{\pi}{2}\big]$ réalise un bijection de $\big[-\frac{\pi}{2},\frac{\pi}{2}\big]$ 
			sur $[-1,1]$. On appelle \underline{arcsinus noté $\arcsin$} cette fonction telle que $\forall (x,y) \in [-1,1]\times 
			\big[-\frac{\pi}{2}, \frac{\pi}{2}\big] ,~y=\arcsin(x) \Leftrightarrow x=\sin(y)$ \trait
		\thm{ch5P12}{Proposition}{ArcsinCroissDer}{La fonction $\arcsin$ est continue strictement croissante sur $[-1,1]$ \\ et dérivable sur 
		$]-1,1[$ avec $\forall x\in ]-1,1[, ~\arcsin'(x) = \frac{1}{\sqrt{1-x^2}}$} \traitd
		\paragraph{Arccosinus}
			La restriction de $\cos$ à $[0,\pi]$ réalise un bijection sur $[-1,1]$. On appelle \underline{arccosinus noté $\arccos$} cette fonction 
			telle que $\forall (x,y) \in [-1,1]\times [0,\pi] ,~y=\arccos(x) \Leftrightarrow x=\cos(y)$ \trait
		\thm{ch5P13}{Propriété}{Arcos+Arcsin}{$\forall x\in [-1,1] , ~\arccos (x) + \arcsin(x) = \frac{\pi}{2}$}
		\traitd \vspace*{0.3cm} \thm{ch5th3}{Théorème - Arctangente}{Arctan}{$\tan$ réalise un bijection de $\big]-\frac{\pi}{2},\frac{\pi}{2}
		\big[$ sur $\R$, on appelle \underline{$\arctan$} cette fonction. \\ $\arctan$ est dérivable sur $\R$ avec $\forall x\in\R , ~\arctan'(x) = 
		\frac{1}{1+x^2}$} \vspace*{0.15cm} \trait
		\vspace*{-1.4cm} \begin{proof}
		$\tan$ est dérivable sur $\big]-\frac{\pi}{2},\frac{\pi}{2}\big[$ donc $\arctan$ est dérivable en tout point $a=\tan(y)$ \\ 
		avec $\arctan'(a) = \frac{1}{\tan'(y)} = \frac{1}{1+\tan^2(y)} = \frac{1}{1+a^2}$
		\end{proof}
		${}$ \\ \thm{ch5P14}{Proposition}{Arctan+inv}{$\forall x\in \R, ~\arctan(x) + \arctan\big(\frac{1}{x}\big) = 
		\frac{x}{\abs{x}}\times\frac{\pi}{2}$} \newpage \traitd
		\paragraph{Cosinus hyperbolique - Sinus hyperbolique}
			On appelle \underline{cosinus hyperbolique (resp. sinus} \underline{hyperbolique) noté $\cosh$ (resp. $\sinh$)} la partie paire (resp. impaire) de 
			$\exp$. \[ \forall x\in\R ,~\left\{ \ard \cosh(x) = \frac{e^x+e^{-x}}{2} \\ \sinh(x) = \frac{e^x - e^{-x}}{2} \arf \right. \]
		\vspace*{-0.7cm} \trait \vspace*{-1.1cm} \\
		\textit{Ces fonctions sont indéfiniment dérivables sur $\R$ avec $\cosh' = \sinh$ et $\sinh'=\cosh$}
		\vspace*{0.5cm} \\ \thm{ch5L2}{Lemme}{ch>1}{$\forall x\in\R,~\cosh(x) \geq 1$ avec égalité \underline{ssi} $x=0$}
		\vspace*{0.5cm} \\ \thm{ch5P15}{Proposition}{Ch+Sh}{$\forall x\in\R ,~\cosh^2(x) - \sinh^2(x) = 1$}
		\vspace*{0.5cm} \\ \thm{ch5P16}{Propriété}{5-P16}{$\forall(a,b)\in\R^2$, on a \\
		$\cosh(a+b) = \cosh(a)\cosh(b) + \sinh(a)\sinh(b)$ \\ $\sinh(a+b) =\sinh(a)\cosh(b) + \sinh(b)\cosh(a) $ \\
		$\cosh(a-b) = \cosh(a)\cosh(b) - \sinh(a)\sinh(b)$ \\ $\sinh(a-b) = \sinh(a)\cosh(b) - \sinh(b)\cosh(a) $} \traitd
		\paragraph{Tangente hyperbolique}
			La fonction \underline{tangente hyperbolique notée $\tanh$} est définie sur $\R$ par $\tanh = \frac{\sinh}{\cosh}$ \trait
		\thm{ch5P17}{Propriété}{PropriTanh}{$\tanh$ est impaire et indéfiniment dérivable sur $\R$ avec \\ $\forall x\in\R , ~\tanh'(x) = 
		\frac{1}{\cosh^2(x)} = 1 - \tanh^2(x)$}
	\section{Dérivation d'une fonction complexe}
		\textit{On étudie ici des fonctions définies sur $I\subset \R$ à valeurs dans $\C$} \traitd
		\paragraph{Dérivabilité en un point}
			On dit que $f:I\to\C$ est dérivable en $x_0\in I$ si $\frac{f(x)-f(x_0)}{x-x_0}$ possède une limite en $x_0$. 
			\big(Si $\forall \varepsilon>0 ,~\exists \delta>0 ~:~ \forall x\in I , ~\abs{x-x_0} \leq\delta \Rightarrow \abs{f(x)-f(x_0)}
			\leq\varepsilon$\big) \\ On note alors $f'(x_0)$ cette limite. \trait
		\thm{ch5P18}{Proposition}{FderCCNS}{$f:I\to\C$ est dérivable en $x_0\in I$ \underline{si et seulement si} $\Re(f)$ et $\Im(f)$ sont 
		dérivable en $x_0$. \\ On a alors $f'(x_0) = \big(\Re(f)\big)'(x_0) + i\big(\Im(f)\big)'(x_0)$}
		\newpage ${}$ \\ \thm{ch5P19}{Proposition}{ThOpFcCompl}{Les théorèmes opératoires sur la somme, le produit, et la fraction 
		%(\ref{OpeDeriv}) 
        \\sont identique pour des fonctions à valeurs complexes (pas la composition !)}
		\vspace*{0.5cm} \\ \thm{ch5P20}{Proposition}{DerivExpCompl}{Si $\varphi$ est une fonction dérivable sur $I$ de $\R$ à valeurs complexes \\
		Alors $\psi ~ \appli{I}{t}{\C}{\exp\big(i\varphi(t)\big)}$ est dérivable sur $I$ et \\ 
		$\forall t\in I,~\psi'(t) = i\varphi'(t)e^{i\varphi(t)}$ }
		\vspace*{0.5cm} \\ 
		\begin{center}
		\fin
		\end{center}
        
\chapter{Primitives et équations différentielles}

    
% Chapitre 5 : Primitives et équations différentielles

\minitoc
	\section{Calcul de primitives}
		\traitd
		\paragraph{Primitive}
			Si $I$ est un intervalle de $\R$ on dit que \underline{$F$ est une primitive de $f$} définie sur $I$ à valeurs complexes si $F$ est 
			dérivable sur $I$ et $\forall x\in I,~F'(x)=f(x)$ \trait
		\thm{ch6P1}{Proposition}{PrimCte}{Si $F$ est une primitive de $f$ sur $I$, alors pour toute primitive $G$ \\ de $f$ il existe $C\in\R$ une 
		constante telle que $G=F+C$}
		\vspace*{0.5cm} \\ \thm{ch6P2}{Proposition}{Prim0ena}{Si $f$ est une fonction continue sur $I$ alors $f$ admet des primitives sur $I$ et 
		\\ $\displaystyle{\forall x_0\in I ,~\int_{x_0}^x f(t) \dd t}$ est l'unique primitive de $f$ qui s'annulle en $x_0$.}
		\paragraph{Exemples de référence} ${}$ \\
		Soit $\lambda\in\C^* ~; ~n\in\Z\backslash\{-1\} ~;~\alpha\in\R\backslash\{-1\} ~; ~a\in\R^*$ et $J\subset\{x\in\R ~|~\cos(ax+b)\neq 0\}$
		\begin{center}\begin{blockarray}{||c|c|c||c||c|c|c||}
		$f$ & $F~(+C)$ & $I$ & \hspace*{1cm} & $f$ & $F~(+C)$ & $I$ \\ 
		$e^{\lambda x}$ & $\frac{1}{\lambda} e^{\lambda x}$ & $\R$ & & $\ln x$ & $x\ln x-x$ & $\R_+^*$ \\
		$\cos(ax+b)$ & $\frac{1}{a}\sin(ax+b)$ & $\R$ & & $\frac{1}{x}$ & $\ln\mc{x}$ & $\R_-^*$ ou $\R_+^*$ \\
		$\sin(ax+b)$ & $-\frac{1}{a}\cos(ax+b)$ & $\R$  & & $x^n$ & $\frac{1}{n+1}x^{n+1}$ & $\R^*$ \\
		$\tan(ax+b))$ & $-\frac{1}{a}\ln\mc{\cos(ax+b)}$ & $J$ & & $x^{\alpha}$ & $\frac{1}{\alpha +1}x^{\alpha +1}$ & $\R_+^*$ \\
		$\sinh(ax+b)$ & $\frac{1}{a}\cosh(ax+b)$ & $\R$ & & $\frac{1}{1+x^2}$ & $\arctan x$ & $\R$ \\
		$\cosh(ax+b)$ & $\frac{1}{a}\sinh(ax+b)$ & $\R$ & & $\frac{1}{\sqrt{1-x^2}}$ & $\arcsin x$ & $\R$ \\
		$\tanh(ax+b)$ & $\frac{1}{a}\ln\big(\cosh(ax+b)\big)$ & $\R$ & & $\frac{-1}{\sqrt{1-x^2}}$ & $\arccos x$ & $\R$
		\end{blockarray}\end{center}
		\paragraph{Notation} On note $\int^x f(t) \dd t$ une primitive de $f$.
		\vspace*{0.5cm} \\ \thm{ch6P3}{Proposition : Intégration par partie}{IPP}{Si $u$ et $v$ sont deux fonctions de classe $\cont^1$ sur $I$, $(a,b)\in I$, \\ $\cm{\int_a^b u'(t)v((t)\dd t = \big[u(t)v(t)\big]_a^b - \int_a^b u(t)v'(t) \dd t}$}
		\vspace*{0.5cm} \\ \thm{ch6P4}{Proposition : Formule du changement de variable}{ChgtVar}{Soit $\varphi$ une fonction de classe $\cont^1$ sur un intervalle $I$ de $\R$, \\$f$ une fonction continue sur $J$ avec $\varphi_d(I)\subset J$ \\
		$\cm{\forall (a,b)\in I^2 ~,~~\int_a^b f\big(\varphi(t)\big)\times\varphi'(t) \dd t = \int_{\varphi(a)}^{\varphi(b)} f(x) \dd x}$}
		\vspace*{0.3cm}\\ \begin{center}
		\begin{blockarray}{[c]}		
		\hspace*{0.2cm}\textbf{\highlight{\underline{Règles de \textsc{Bioche}}}} : \textit{Soit $f$ une fonction rationnelle en $\cos t$ et $\sin t$ et $\psi(t) = f(t)\dd t$}\hspace*{0.2cm} \\ \textit{On effectue les changements de variable suivants :}\\ 
		\textit{$\bullet$ Si $\psi$ est invariante par $t\mapsto \pi - t$ alors on pose $x = \sin t$} \\ 
		\textit{$\bullet$ Si $\psi$ est invariante par $t\mapsto -t$ alors on pose $x=\cos t$} \\ 
		\textit{$\bullet$ Si $\psi$ est invariante par $t\mapsto t+\pi $ alors on pose $x = \tan t$}
		\end{blockarray}
		\end{center}
	\section{Équations différentielles du premier ordre}
		\traitd
		\paragraph{Définition}
			Une équation fonctionnelle de la forme \[y' + a(x)y = b(x)\] Où $a$ et $b$ sont des fonctions réelles ou complexes définies sur un intervalle $I$ de $\R$ s'appelle une \underline{équation différentielle linéaire d'ordre $1$} où les \underline{inconnues $y$} sont des \underline{fonctions dérivables sur $I$ à} \underline{valeurs dans $\R$ ou $\C$} \trait
		\thm{ch6P5}{Proposition}{EquaHomog}{Si $a$ et $b$ sont deux fonctions continues sur $I$, \\ \hspace*{2cm}$(E) ~:~y'+a(x)y=b(x)$,\\
		Alors $(E_0) ~:~y'+ a(x)y = 0$ est l'\underline{équation homogène associée à $(E)$} \\de solution \hspace*{1.5cm}$y = C.e^{-A(x)}$ \\où $A(x)$ est une primitive de $a$ sur $I$ et $C$ est une constante.} 
		\newpage ${}$ \\ \thm{ch6P6}{Proposition}{SoluceE}{Si $a$ et $b$ sont deux fonctions continues de $I$ de $\R$ à valeur dans $\K$, \\$\varphi_0$ une solution particulière de $(E)~:~y'+a(x)y = b(x)$\\
		Alors toute solution de $(E)$ est de la forme $x\mapsto \varphi_0(x)  +\psi(x)$\\ où $\psi$ est solution de $(E_0)$}
		\\ \textit{On notera $\mathscr{S}_{(E)} = \varphi_0 + \mathscr{S}_{(E_0)}$}
		\paragraph{Méthode de variation de la constante :}
		$y' + a(x)y = b(x)$ avec $a$ et $b$ continues sur $I$ à valeurs dans $\K$.\\ $(E_0)$ l'équation homogène associée à $(E)$ admet pour solution générale $\varphi_0(x) = C.e^{-A(x)}$ avec $A$ une primitive de $a $sur $I$ et $C$ une constante de $\K$. \vspace*{0.2cm} \\ 
		On cherche une solution particulière de la forme $x\overset{\psi}{\mapsto} C(x) .e^{-A(x)}$ avec $C$ dérivable sur $I$
		\[ \forall x\in I,~\psi'(x) + a(x)\psi(x) = b(x) ~\Leftrightarrow~ \]
		\[ C'(x) e^{-A(x)} \underbrace{-C(x)a(x)e^{-A(x)} + C(x)a(x)e^{-A(x)}}_{=0} = b(x) \]
		$\Leftrightarrow ~ \psi$ est solution de $(E)$ si et seulement si $\forall x\in I, ~C'(x) = b(x) e^{A(x)}$
		\vspace*{0.5cm} \\ \thm{ch6P7}{Proposition}{SolGE}{Sous les mêmes hypothèses et notations la solution générale de $(E)$ est \\
		\hspace*{2cm} $\cm{ \varphi(x) = \Big( C+\int^x b(t)e^{A(t)}\dd t \Big) .e^{-A(x)} }$ \\où $A$ est une primitive de $a$ sur $I$ et $C$ une constante de $\K$ }
		\vspace*{0.5cm} \\ \thm{ch6P8}{Propriété}{EquaDiffPol}{Si $a\in \K$ et $P$ est une fonction polynômiale à coefficients dans $\K$ \\ 
		Alors l'équation différentielle $~~y'+ay = e^{-\alpha(x)}P(x)$ \\ 
		admet une solution particulière de la forme $\varphi_0 : x\mapsto e^{-\alpha(x)} Q(x)$ \\
		avec $Q(x)$ un polynôme à coefficients dans $\K$ \\
		et $\deg Q = \deg P$ si $\alpha \neq a$ et $\deg Q = \deg P+1$ sinon.}
		\vspace*{0.5cm} \\ \thm{ch6P9}{Proposition : Principe de superposition}{PrincSuperpo}{Si $a,b_1,b_2$ sont des fonctions continues sur $I$ à valeurs dans $\K$\\
		\hspace*{1cm} $\varphi_1$ solution particulière de $y'+a(x)y = b_1(x)$\\
		\hspace*{1cm} $\varphi_2$ solution particulière de $y'+a(x)y = b_2(x)$\\
		Alors pour tout $(\lambda_1,\lambda_2)\in\K$, \\$\lambda_1\varphi_1 + \lambda_2\varphi_2$ est solution particulière de $y'+a(x)y = \lambda_1b_1(x) + \lambda_2b_2(x)$ }
		\vspace*{0.5cm} \\ \thm{ch6P10}{Proposition : Problème de \textsc{Cauchy}}{PbCauchy}{$\forall (x_0,y_0) \in I\times\K$, le problème de \textsc{Cauchy} $\left\{ \ard y' + a(x)y = b(x) \\ y(x_0) = y_0 \arf\right.$ \\
		Admet une unique solution \\ \hspace*{2cm}$\cm{ \varphi_0 : x\mapsto \Big( y_0 + \int_{x_0}^x b(t)e^{\int_{x_0}^x a(s)\dd s} \dd t \Big) e^{-\int_{x_0}^x a(s) \dd s} }$}
	\section{Équations différentielles linéaires d'ordre $2$ à coefficients constants}
		On considère ici $(E) ~:~y''+ay'+by = f(x)$ \\ où $a,b$ sont des constantes de $\K$ et $f$ est définie et continue sur $I$ de $\R$ à valeur dans $\K$\vspace*{0.3cm} \\
		$(E)$ s'appelle une \uline{équation différentielle linéaire d'ordre $2$ à coefficients constants dans $\K$}
		\vspace*{0.5cm} \\ \thm{ch6P11}{Proposition}{EquaCar}{Si $r\in \K$ alors $\varphi_r : x\mapsto e^{rx}$ est solution de $(E_0)$ \\si et seulement si $r^2+ar+b=0$ (équation caractéristique associée à $(E)$ (e.c.))}
		\vspace*{0.5cm} \\ \thm{ch6P12}{Proposition}{SolGE02}{Avec les mêmes notations et en notant $\Delta$ le discriminant \\ de l'équation caractéristique associée à $(E)$\\ ${}$ \\
		$\bullet$ $\Delta >0$ et $r_1,r_2$ les solutions de e.c. alors la solution générale de $(E_0)$ \\ est donnée par
		\hspace*{0.5cm} $\cm{x\mapsto C_1e^{r_1(x)} + C_2e^{r_2(x)} }$ \\${}$\\
		$\bullet ~\Delta=0$ et $r$ la solution double de e.c. alors la solution générale de $(E_0)$ \\ est donnée par 
		\hspace*{0.5cm} $\cm{x\mapsto (C_1 x +C_2) e^{rx} }$ \\ ${}$ \\
		$\bullet ~\Delta<0$ et $r = \rho + \imath.\omega$ ($\omega\neq 0$) une solution de e.c. alors la solution générale de $(E_0)$ \\ est donnée par
		\hspace*{0.5cm} $\cm{x\mapsto \big(C_1\cos(\omega x)+C_2\sin(\omega x) \big) e^{\rho x} }$ }
		\vspace*{0.5cm} \\ \thm{ch6P13}{Proposition}{SolGE2}{Si $f$ est une fonction continue sur $I$, $(a,b)\in\K^2$\\
		Alors la solution générale de $(E) ~:~y''+ay'+by=f(x)$ \\ 
		est la somme d'une solution particulière et de la solution générale \\ de l'équation homogène associée.}
		\vspace*{0.5cm} \\ \thm{ch6P14}{Propriété}{EquaDiff2Pol}{Soit $P$ une fonction polynômiale sur $I$ à valeurs dans $\K$ et $\alpha\in\K$\\
		L'équation $(E) ~:~y''+ay'+b=P(x)e^{\alpha x}$ admet une solution particulière de la forme \\
		$x\mapsto Q(x)e^{\alpha x}$ avec $Q$ une fonction polynômiale à coefficients dans $\K$ et \\
		\hspace*{0.5cm} $\deg Q = \deg P$ si $\alpha$ n'est pas solution de e.c.\\
		\hspace*{0.5cm} $\deg Q = \deg P+1$ si $\alpha$ est racine simple de e.c.\\
		\hspace*{0.5cm} $\deg Q = \deg P+2$ si $\alpha$ est racine double de e.c.}
		\vspace*{0.5cm} \\ \thm{ch6P14c}{Corollaire}{EquaDiff2Trig}{L'équation différentielle $y''+ay'+b=\cos (\omega x)e^{\alpha x}$\\
		(respectivement $y''+ay'+b = \sin (\omega x)e^{\alpha x}$ )\\
		Admet une solution particulière de la forme \\$\cm{x\mapsto x^k\big(C_1\cos(\omega x) + C_2 \sin(\omega x)\big)e^{\alpha x}}$ avec $(C_1,C_2)\in\R^2$\\
		\hspace*{0.5cm} $k=0$ si $\alpha+\imath.\omega$ n'est pas solution de e.c.\\
		\hspace*{0.5cm} $k=1$ si $\alpha+\imath.\omega$ est une racine double de e.c.
		\\ \hspace*{0.5cm} $k=2$ si $\alpha+\imath.\omega$ est une racine simple de e.c. } \newpage
		${}$ \\ \thm{ch6P15}{Propriété : Principe de superposition}{PrincSuperpo2}{Soit $(a,b)\in\K^2$ et $(f_1,f_2)$ deux fonctions continues sur $I$ à valeurs dans $\K$\\
		\hspace*{1cm} $\varphi_1$ une solution particulière de $y''+ay'+by=f_1(x)$\\
		\hspace*{1cm} $\varphi_2$ une solution particulière de $y''+ay'+by=f_2(x)$\\
		Alors pour tout $(\lambda_1,\lambda_2)\in\K^2$, \\
		$\lambda_1 \varphi_1 +\lambda_2 \varphi_2 $ est solution de $y''+ay'+by = (\lambda_1 f_1(x) + \lambda_2 f_2(x) ) $}
		\vspace*{0.5cm} \\ \thm{ch6P16}{Proposition : Problème de \textsc{Cauchy}}{PbCauchy2}{Si $(a,b)\in\K^2$, $f$ une fonction continue sur $I$ à valeurs dans $\K$,
		\\ $x_0\in I,~(y_0,y_0')\in\K^2$ le problème de \textsc{Cauchy} $\left\{ \ard y''+ay'+by=f(x) \\ y(x_0)=y_0 ~;~y'(x_0)=y_0' \arf \right.$\\
		admet une unique solution.}
		\vspace*{0.5cm} \\ 
		\begin{center}
		\fin
		\end{center}

\chapter{Nombres réels et suites numériques}

    
% Chapitre 6 : Nombres réels et suites numériques

\minitoc
	\section{Ensembles de nombres réels}
		\traitd
		\paragraph{Entiers naturels}
			$0,1,2, \dots$ avec $\leqslant$ une relation d'ordre totale \trait
		\thm{ch7P1}{Propriété : Principe de bon ordre}{BonOrdreN}{{\scriptsize (i)} Toute partie non vide de $\N$ admet un plus petit élément.\\
		{\scriptsize (ii)} Tout partie non vide et majorée de $\N$ admet un plus grand élément.}
		\newpage ${}$ \\ \thm{ch7P2}{Proposition : Division euclidienne sur $\N$}{DivEuclN}{$\forall (a,b)\in\N\times\N^*, ~\exists (q,r)\in \N^2$, unique tel que \\
		\hspace*{2cm} $\cm{a=bq+r}$ avec \uline{$0\leqslant r<b$}} \\ \traitd
		\paragraph{Entiers relatifs}
			$\Z = \N \cup (-\N) = \{\dots , -2,-1,0,1,2,\dots\}$ \trait
		\vspace*{-1.1cm} \\ \textit{La division euclidienne reste valable sur $\Z$}
		\traitd
		\paragraph{Nombres rationnels}
			$\Q = \{ \dfrac{p}{q} ~|~ p\in\Z ,~ q\in \N^* \}$ On dit que $\frac{p}{q}$ est irréductible si $p$ et $q$ sont sans diviseurs communs. \trait
		\thm{ch7P3}{Propriété}{QStable}{$\Q$ est stable par somme, différence et produit.}
		\vspace*{0.5cm} \\ \thm{ch7P4}{Proposition}{EncadrDecimal}{$\forall x\in \R^+ ,~ \exists (x_k)_{_{k\in\N}} \in \N^\N$ unique telle que $\forall n\in \N$ on a \\
		\hspace*{2cm} $\cm{\sk{0}{n} x_k.10^{-k} \leqslant x < \sk{0}{n} x_k.10^{-k} + 10^{-n}} $ \\
		On a de plus $(x_k)_{_{k\in\N^*}} \in \ent{0,9}^\N$ non stationnaire à $9$}
		\traitd \paragraph{Approximation décimale propre}
			Soit $x\in \R$, avec les même notations, on appelle \uline{approximation décimale propre de $x$ à $10^{-n}$ près} la somme $\sk{0}{n}x_k.10^{-k}$\vspace*{0.2cm} \\ On appelle \uline{approximation décimale propre de $x$} la \textbf{limite} : 
			\[ \limit{n}{+\infty} \sk{0}{n} x_k.10^{-k} = x_0,x_1x_2\dots x_n\dots \]
			\trait ${}$ \vspace*{-1.2cm} \traitd
		\paragraph{Nombres décimaux}
			On appelle \uline{nombres décimaux} l'ensemble des nombres réels dont l'approximation décimale propre est \textbf{stationnaire à $0$}. Leur ensemble est noté $\mathds{D}$ avec
			\[ \mathds{D} = \{ x\in\R ~|~ \exists n\in\N ~:~ x\times 10^{n} \in\Z\}\subset \Q \] \trait ${}$ \vspace*{-1.2cm} \traitd
		\paragraph{Densité}
			On dit que $X\in\mathcal{P}(\R)$ est \uline{dense dans $\R$} si pour tout $a<b$ de $\R$ on a $]a,b[ \cap X \neq \varnothing$ \trait
		\thm{ch7P5}{Propriété}{RatIrratDenses}{$\Q$ et $\R\setminus\Q$ sont denses dans $\R$.} \newpage \traitd
		\paragraph{Borne supérieure}
			Soit $X$ un ensemble. \textbf{Sous réserve d'existence}, la borne supérieure de $X$, notée $\sup X$ est le plus petit éléments de l'ensemble des majorants de $X$. \trait
		\thm{ch7th1}{Théorème}{SupNonVideMajR}{Toute partie non vide et majorée de $\R$ admet une borne supérieure.}
		\vspace*{0.5cm} \\ \thm{ch7P6}{Proposition : Caractérisation de la borne supérieure}{CarSup}{Soit $A$ une partie de $\R$, $\alpha$ est la borne supérieure de $A$ si et seulement si \\
		$\forall x\in A,~x\leqslant\alpha$ et $\forall\varepsilon>0 ,~ \exists x'\in A$ tel que $x'> \alpha-\varepsilon$} \\ \traitd
		\paragraph{Intervalle}
			On appelle \uline{intervalle} toute partie $X$ de $\R$ vérifiant 
			\[ \forall (a,b)\in X^2 ~avec~ a\leqslant b,~ [a,b] \subset X\] \trait
		\thm{ch7P7}{Propriété}{EcritureSegment}{$\forall (a,b)\in\R^2$ avec $a\leqslant b$ on a $[a,b] = \{\lambda a + (1-\lambda ) b ~|~\lambda\in [0,1] \}$}
		\vspace*{0.5cm} \\ \thm{ch7P8}{Propriété}{IntersecInterv}{L'intersection de deux intervalles est un intervalle.\\
		Toute intersection (même infinie) d'intervalles est un intervalle.}
	\section{Suites réelles}
	\subsection{Généralités}
		\traitd
		\paragraph{Suite stationnaire}
			Une suite réelle $\suite{u}$ est dite \uline{stationnaire} si
			\[\exists n_0\in\N ~:~\forall n\in \N ,~ (n\geqslant n_0 \Rightarrow u_n = u_{n_0})\] \trait
		\thm{ch7P9}{Propriété}{SuiteBornée}{Une suite $\suite{u}$ est bornée si et seulement si $\big(\mc{u_n}\big)_{_{n\in\N}}$ est majorée.} \\ \traitd
		\paragraph{Convergence}
			Si $\ell\in\R$ et $\suite{u}$ est une suite réelle, \\
			$\bullet$ On dit que \uline{$u$ converge vers $l$} si 
			\[ \forall\varepsilon>0 ,~\exists n_0\in\N ~:~\forall n\in \N,~\big( n\geqslant n_0 \Rightarrow \mc{u_n-\ell}\leqslant\varepsilon \big) \] 
			$\bullet$ On dit que \uline{$u$ tend vers $+\infty$} (resp. $-\infty$) si 
			\[ \forall A\in \R ,~\exists n_0\in\N ~:~\forall n\in\N ,~\big( n\geqslant n_0 \Rightarrow u_n \geqslant A \big) ~~(resp.~u_n\leqslant A) \] 
			\trait \newpage \traitd
		\paragraph{Divergence}
			Une suite réelle est dite \uline{divergente} si elle ne converge pas. \trait
		\thm{ch7P10}{Propriété : Unicité de la limite}{UniLim}{Si $\suite{u}$ tend vers $\ell_1$ et vers $\ell_2$ alors $\ell_1=\ell_2$}
		\vspace*{0.5cm} \\ \thm{ch7P11}{Propriété}{ConvImplBorn}{Toute suite réelle convergente est bornée.}
		\vspace*{0.5cm} \\ \thm{ch7P12}{Propriété}{ProdBornLimNulle}{Le produit d'une suite bornée et d'une suite de limite nulle \\ est une suite de limite nulle.}
		\vspace*{0.5cm} \\ \thm{ch7P13}{Propriétés}{OpeLim}{Si $\suite{u}$ et $\suite{v}$ sont deux suites réelles convergentes dans $\overline{\R}$ \\
		Alors pour tout $\lambda,\mu \in\R$, $\big(\lambda u_n + \mu v_n\big)_{_{n\in\N}}$ et $\big( u_nv_n\big)_{_{n\in\N}}$ convergent \footnotemark[1] avec \\
		\hspace*{2cm} $\bullet ~\lim (\lambda u_n + \mu v_n) = \lambda\lim u_n + \mu \lim v_n$
		\\ \hspace*{2cm} $\bullet ~ \lim (u_nv_n) = (\lim u_n )(\lim v_n)$ }
		\footnotetext{Si la somme et/ou le produit ne sont pas des formes indéterminée de $\overline{\R}$}
		\vspace*{0.5cm} \\ \thm{ch7P14}{Propriété}{QuotientLim}{Si $\suite{u}$ tend vers $\ell\in\overline{\R}\setminus\{0\}$ \\
		Alors $\suite{u}$ est non nulle à partir d'un certain rang $n_0$ \\
		et $\big(\frac{1}{u_n}\big)_{_{n\geqslant n_0}}$ converge vers $\frac{1}{\ell} \in \overline{\R}$}
		\vspace*{0.5cm} \\ \thm{ch7L1}{Lemme}{7-L1}{Soit $\suite{u}$ une suite réelle avec $u_n \ston \ell$ alors $\mc{u_n} \ston \mc{\ell}$}
		\vspace*{0.5cm} \\ \thm{ch7th2}{Théorème : Passage à la limite dans les inégalités larges}{PassLimInegLarges}{\textsc{à compléter}}
		\begin{proof}
		On peut noter que si $\suite{u}$ une suite réelle quelconque est à termes positif à partir d'un certain rang et si $(u_n)$ tends vers $\ell\in \overline{\R}$ alors $\ell \geqslant0$, en effet : par unicité de la limite, $\ell = \mc{\ell}\geqslant 0$\vspace*{0.2cm}\\
		Soit donc $\suite{u}$ et $\suite{v}$ deux suites réelles convergente respectivement vers $\ell$ et $\ell'$ avec $u_n\geqslant v_n$ à partir d'un certain rang alors\\
		\hspace*{0.5cm} $\bullet$ Si $\ell=\ell'=+\infty$ ou si $\ell=\ell'=-\infty$ alors $\ell=\ell'$ et on a le résultat\\
		\hspace*{0.5cm} $\bullet$ Sinon on note $w_n = u_n-v_n$ et on a ainsi $w_n\geqslant 0$ à partir d'un certain rang donc vu $(w_n)$ converge vers $\ell''=\ell-\ell'$ alors $\ell\geqslant\ell'$
		\end{proof}
		${}$ \\ \thm{ch7P15}{Propriété}{CarConv}{Soit $\suite{u}$ et $\suite{v}$ deux suites réelles telles que $(v_n)$ converge vers $0$.\\
		On suppose qu'il existe $\ell\in\R$ tel que \textsc{apcr} $\mc{u_n-\ell} \leqslant v_n$\\
		\hspace*{2cm} Alors $(u_n)$ converge vers $\ell$}
		\vspace*{0.5cm} \\ \thm{ch7th3}{Théorème d'encadrement}{ThEncadr}{Soit $(u_n),(v_n),(w_n)$ trois suites réelles telles que \textsc{apcr} $v_n \leqslant u_n\leqslant w_n$.\\
		On suppose que $(v_n)$ et $(w_n)$ converge vers une même limite.\\
		\hspace*{0.5cm} Alors $(u_n)$ converge vers cette limite commune.}
		\begin{proof}
		On a à partir d'un certain rang $0\leqslant u_n-v_n \leqslant w_n-v_n$ et $(w_n-v_n)$ converge vers $0$ donc d'après la propriété précédente $(u_n-v_n)$ converge vers $0$, or pour tout $n\in\N$, $u_n = (u_n-v_n)+v_n$ d'où $\lim u_n = \lim (u_n-v_n) + \lim v_n = \lim v_n$
		\end{proof}
		${}$ \\ \thm{ch7P16}{Proposition}{LimMinorMaj}{Soit $(u_n)$ et $(v_n)$ deux suites réelle, on suppose \textsc{apcr} $u_n\leqslant v_n$\\
		\hspace*{2cm} Alors $\left\{\ard $Si $u_n\ston +\infty$ alors $v_n \ston +\infty \\ $Si $v_n \ston -\infty$ alors $u_n \ston -\infty \arf \right.$ }
		\vspace*{0.5cm} \\ \thm{ch7th4}{\highlight{Théorème de la limite monotone}}{ThLimMonot}{Toute suite croissante et majorée (resp. décroissante et minorée) converge.}
		\begin{proof}
		Soit $\suite{u}$ une suite réelle, on note $X=\{u_n ~|~n\in\N\}$ partie non vide et majorée de $\R$, on note donc $\ell$ sa borne supérieur (qui existe). On a alors par croissance de $(u_n)$ et caractérisation de la borne supérieur $u_n \ston \ell$.
		\end{proof} ${}$ \traitd
		\paragraph{Suites adjacentes}
			Deux suites réelles $\suite{u}$ et $\suite{v}$ sont dites \uline{adjacentes} si elles sont de monotonies contraires et si $\lim (u_n-v_n)=0$\trait
		\thm{ch7L2}{Lemme}{7-L2}{Si $\suite{u}$ et $\suite{v}$ sont adjacentes avec $(u_n)$ croissante et $(v_n)$ décroissante.\\
		\hspace*{2cm} Alors $\forall (p,q)\in \N^2 ,~u_p\leqslant v_q$}
		\vspace*{0.5cm} \\ \thm{ch7th5}{Théorème des suite adjacentes}{ThSuitesAdjacentes}{Deux suites adjacentes convergent vers une même limite.}
		\begin{proof}
		Soit $(u_n)$ et $(v_n)$ deux suites adjacentes. On suppose sans perte de généralité $(v_n)$ décroissante. D'après le lemme on a alors $(u_n)$ croissante majorée par $v_0$ donc d'après le théorème de la limite monotone $(u_n)$ converge vers $\ell\leqslant v_0$. De même $(v_n) $ converge vers $\ell' \geqslant u_0$ puis vu $\lim (u_n-v_n) = 0$ on a $\ell = \ell'$
		\end{proof} ${}$ \traitd
		\paragraph{Extractrice}
			On a appelle extractrice toute application $\sigma : \N\to\N$ strictement croissante. \trait
		\thm{ch7P17}{Propriété}{ExtracSupN}{Si $\sigma :\N\to\N$ est une extractrice alors $\forall n\in\N ,~\sigma(n) \geqslant n$} \newpage \traitd
		\paragraph{Suite extraite}
			Soit $\suite{u}$ et $\suite{v}$ deux suites réelles. On dit que \uline{$(v_n)$ est extraite de $(u_n)$} s'il existe $\sigma :\N\to\N$ un extractrice telle que \[\forall n\in\N, ~v_n = u_{\sigma(n)} \] \trait
		\thm{ch7P18}{Proposition}{LimStableExtrac}{Si une suite possède une limite, toute ses suites extraites \\ possèdent la même limite.} 
		\vspace*{0.5cm} \\ \thm{ch7P19}{Propriété}{SousSuitePairImpair}{Soit $\suite{u}$ une suite réelle\\ On suppose que $(u_{2n})$ et $(u_{2n+1})$ tendent vers une même limite $\ell$. \\
		\hspace*{2cm} Alors $(u_n)$ tend vers $\ell$}
		\vspace*{0.5cm} \\ \thm{ch7th6}{\highlight{Théorème de \textsc{Bolzano-Weierstrass}}}{ThBW}{Tout suite réelle bornée admet une suite extraite qui converge.}
		\begin{proof}
		Soit $\suite{u}$ une suite bornée. \\
		On considère $A=\{p\in\N ~|~ \forall n\in\N,~ n\geqslant p \Rightarrow u_n <u_p \}$\\On construit alors une extractrice $\sigma$ telle que $(u_{\sigma(n)})$ est strictement décroissante :\\
		$\bullet$ Si $A$ est infinie, on pose $\sigma(0) = \min A$ (principe de bon ordre) puis $\forall p\in\N$, on pose \\$\sigma(p+1) = \min \big(A\cap \rrbracket \sigma(p), + \infty \llbracket \big)$\\
		On a alors $(u_{\sigma(n)}$ strictement décroissante et minorée donc convergent d'après le théorème de la limite monotone. \\
		$\bullet$ Si $A$ est fini, on pose $\sigma(0) = \left\{\ard 0$ si $A$ est vide$ \\ \max A+1$ sinon$ \arf \right. $ \\ On a alors vu $\sigma(0)\notin A$, $\exists n>\sigma(0) ~:~u_n\geqslant u_{\sigma(0)}$, ainsi on pose pour tout $p\in\N, \\ \sigma(p+1) = \min \{n>\sigma(p) ~|~ u_n \geqslant u_{\sigma(p)} \}$ (qui existe vu $\sigma(p) \notin A$ \\
		$(u_{\sigma(p)})$ est donc croissante et majorée et par suite convergente.
		\end{proof} ${}$ \traitd
		\paragraph{Convergence (cas complexe)}
			Soit $\suite{u}\in\C^\N$ on dit que \uline{$(u_n)$ converge vers $\ell\in\C$} si \[\forall\varepsilon>0, ~\exists n_0\in\N ~:~\forall n\in\N ,~\big( n\geqslant n_0 \Rightarrow \mc{u_n-\ell}\leqslant\varepsilon\big) \] \trait
		\thm{ch7P20}{Proposition}{CNSConvCompl}{Soit $\suite{u}\in\C^\N$ une suite complexe, alors $(u_n)$ converge ssi \\
		$ \big(\Im (u_n)\big)_{n\in\N}$ et $\big(\Re (u_n)\big)_{n\in\N}$ convergent.\\
		\hspace*{2cm} On a alors $\left\{ \ard \Re(\lim u_n) = \lim \Re (u_n) \\ \Im (\lim u_n) = \lim \Im (u_n) \arf \right. $ }
		\newpage ${}$ \thm{ch7th7}{Théorème de \textsc{Bolzano-Weierstrass : cas complexe}}{ThBWCompl}{De toute suite complexe bornée on peut extraire une suite qui converge.}
		\begin{proof}
		Clair avec le théorème dans le cas réel vu $\forall z\in\C,~ \Re z \leqslant\mc{z}$ et $\Im z \leqslant \mc{z}$
		\end{proof}
		${}$ \\ \thm{ch7P21}{Proposition : Caractérisation séquentielle de la densité}{CarSeqDensite}{Un partie $X$ de $\R$ est dense dans $\R$ si et seulement si tout réel \\ peut s'écrire comme une suite d'éléments de $X$.}
	\subsection{Suites particulières}
		\traitd \paragraph{Suite arithmétique}
			On dit que \uline{$\suite{u}\in\K^\N$ est une suite arithmétique} si la suite $(u_{n+1}-u_n)_{n\in\N}$ est une suite constante appelée \uline{raison de la suite arithmétique}. \trait
		\thm{ch7P22}{Propriété}{7-P22}{Si $u$ est une suite arithmétique de raison $r$ on a\\
		\hspace*{2cm} $\forall (p,q)\in\N^2 ,~ u_p = u_q +  r(p-q)$} \\ 
		\traitd \paragraph{Suite géométrique}
			On dit que \uline{$\suite{u}\in\K^\N$ est une suite géométrique} si $u$ est stationnaire à $0$ où si $u$ est telle que $\big( \frac{u_{n+1}}{u_n}\big)_{n\in\N}$ est une suite bien définie constante appelée \uline{raison de la suite géométrique}. \trait
		\thm{ch7P23}{Propriété}{7-P23}{Si $u$ est une suite géométrique de raison $q$ alors \\ \hspace*{2cm} $\forall (m,n)\in\N^2 ,~ u_m = u_n\times q^{m-n}$ \\
		\hspace*{2cm} $\cm{\sk{n}{m} u_k } = \left\{ \begin{array}{ll} (m-n+1)\times u_n & si ~ q=1 \\ \frac{u_n-u_m+1}{1-q} & sinon \arf \right. $} \\ \traitd
		\paragraph{Suite arithmético-géométrique}
			On dit que \uline{$\suite{u}\in\K^\N$ est une suite arithmético-géométrique} s'il existe $a\in\K\setminus\{1\}$ et $b\in\K$ tels que $u_0\in\K$ et $\forall n\in\N ,~u_{n+1} = au_n + b$ \trait
		\thm{ch7P24}{Propriété}{EcritureArthGeo}{Soit $\suite{u}\in\K^\N$ une suite arithmético-géométrique, alors avec les mêmes notations\\
		\hspace*{2cm} $\forall \in\N ,~ u_n = a^n\big( u_0 - \frac{b}{1-a}\big) \frac{b}{1-a}$ } \newpage \traitd
		\paragraph{Suite récurrente linéaire d'ordre $2$}
			On dit que \uline{$\suite{u}\in\K^\N$ est récurrente linéaire d'ordre $2$} si $\exists (a,b) \in \K^2$ tel que $\forall n\in\N ,~ u_{n+2} = au_{n} +bu_n$ \trait
		\thm{ch7P25}{Propriété}{SuiteRecLin2}{Soit $\suite{u}\in\K^\N$ une suite récurrente linéaire d'ordre $2$. \\
		On considère $(E) ~:~ z^2 = az +b$ l'équation caractéristique associée alors \\
		\hspace*{0.5cm} $\rightarrow$ Si $\Delta \neq 0,~ (z_1,z_2)\in\C^2$ les racines distinctes de $(E)$ alors \\
		\hspace*{2cm} $\exists (\lambda , \mu)\in \C^2$ tel que $\forall n\in\N ,~ u_n = \lambda z_1^n + \mu z_2^n$ \\
		\hspace*{0.5cm} $\rightarrow$ Si $\Delta = 0,~z_0$ la racine double de $(E)$ alors \\ \hspace*{2cm} $\exists (\lambda,\mu)\in\C^2$ tel que $\forall n\in \N ,~u_n = (\lambda n+\mu) z_0^n$ }
		\\ \uline{Rq} : Si $\suite{u}\in\R^\N$ et $\Delta <0$ alors $\lambda$ et $\mu$ sont conjugué et on a en écrivant $z_1=\rho+\imath .\omega$\\
		 $\exists (\lambda_r,\mu_r)\in\R^2$ tel que $\forall n\in\N ,~ u_n = \rho^n \big( \lambda_r \cos (n\omega)+\mu_r \sin(n\omega) \big)$
		 \vspace*{0.5cm} \\
		 \begin{center}
		 	\fin
		 \end{center}

\chapter{Fonctions d'une variable réelle}

    
% Chapitre 7 : Fonctions d'une variable réelle

\textsl{Les fonctions considérées sont définies sur un intervalle I de $\mathbb{R}$ non réduit à un point à valeur dans $\mathbb{R}$ sauf indications contraires.} 
\minitoc ${}$ \traitd
 \paragraph{Voisinage}
    Une propriété portant sur $f$ définie sur I est vraie au voisinage de a si elle est vraie sur $]a+\delta , a-\delta[$ pour un certain $\delta>0$ si $a\in\mathbb{R}$ ; sur $]A, +\infty [$ ou $]-\infty , A[$ sinon. \trait 

\section{Limites et Continuité}
\subsection{Limite d'une fonction en un point}
       \traitd
       \paragraph{Limite d'une fonction}
           Soit f une fonction, f admet une limite $\ell$ en $a\in D_{f}$ notée $\lim \limits_{x \to a} f(x)=\ell$ si: 
           \[\forall \varepsilon , \exists \delta < 0 : \forall x \in I,~ \big(\vert x-a \vert \leq \delta \Rightarrow \vert f(x)-\ell \vert \leq \varepsilon\big)\] \trait
    \thm{ch8P1}{Propriété : Unicité de la limite}{LimUniq}{
    Si la limite de f en a existe alors elle est unique}
%    	\begin{proof}
%    	$a\in\mathbb{R} , (l_{1}, l_{2})\in\mathbb{R}^{2}$ soit $\varepsilon >0$, il existe $(\delta_{1} , \delta_{2} ) \in (\mathbb{R}_{+}^{*} )^{*} : \forall x\in I$, 
%    	$$(\vert x-a \vert \leq \delta_{1} \Rightarrow \vert f(x)-l_{1} \vert \leq \frac{\varepsilon}{2} )$$ 
%    	$$(\vert x-a \vert \leq \delta_{2} \Rightarrow \vert f(x)-l_{2} \vert \leq \frac{\varepsilon}{2} )$$ 
%    	donc si $0< \delta_{0} \leq min(\delta_{1} , \delta_{2} )$ 
%    	$$\forall x\in I ~~ (\vert x-a \vert \leq \delta_{0} \Rightarrow \vert l_{1} - l_{2} \vert \leq \vert f(x) - l_{1} \vert + \vert f(x) - l_{2} \vert = \varepsilon )$$ 
%    	Soit $l_{1} = l_{2} $ \end{proof}
    \vspace*{0.5cm} \\ \thm{ch8P2}{Proposition : Continuité en un point}{ContA}{Si $f$ est définie en $a$ et admet une limite en $a$ alors \\
    \hspace*{2cm} $\cm{\limit{x}{a} = f(a) } $\\
    On dit alors que \uline{$f$ est continue en $a$} }
    \vspace*{0.5cm} \\ \thm{ch8P3}{Propriété}{LimBorneVois}{Si $f$ possède une limite finie en un point $a$ \\alors $f$ est bornée sur un voisinage de $a$ }
       \vspace*{0.5cm} \\ \thm{ch8P4}{Propriété : Signe au voisinage de a}{SignVoisLimit}{
    Si f admet une limite finie non nulle en $a$ alors f est du signe (strict) \\
    de cette limite sur un voisinage de $a$}
%    	\begin{proof}
%    	$\left\{ 
%    	\begin{array}{l}
%    	a\in \mathbb{R}\\
%    	\ell\in\mathbb{R}_{+}^{*}
%    	\end{array} \right.
%    	\hspace{20pt} \exists \delta >0 : \forall x\in I$
%    	$$
%    	\begin{array}{l}
%    	\vert x-a \vert \leq \delta \Rightarrow \vert f(x) -l\vert \leq \frac{\vert l\vert}{2} = \frac{l}{2}\\
%    	\hspace*{50pt}\Rightarrow f(x) - l \geq -\frac{l}{2}\\
%    	\hspace*{50pt}\Rightarrow f(x) \geq \frac{l}{2} >0
%    	\end{array}$$
%    	\end{proof}
       \vspace*{0.5cm} \\ \thm{ch8th1}{Théorème de caractérisation séquentielle de la limite}{CarSeqLim}{
    $f$ admet $\ell$ comme limite en $a\in I$ si, et seulement si \\ pour toute suite $(u_{n})_{_{n\in \mathbb{N} }} \in I^{\mathbb{N}}$ qui tend vers a, $f(u_{n})$ tend vers $\ell$.}
    \begin{proof} Soit $\ell$ la limite de $f$ en $a\in I$ \\
    \fbox{$\Rightarrow$} Soit $\varepsilon>0$ ; soit $\delta>0$ vérifiant la propriété de limite.\\
    On considère $n_0\in\N$ tel que $\forall n\geqslant n_0 ,~\mc{u_n-a}<\delta$ et on a ainsi \[\forall n\in\N ,~n\geqslant n_0 ~\Rightarrow~ \mc{f(u_n) - \ell}<\varepsilon \]
    \fbox{$\Leftarrow$} Par contraposée, on considère $\varepsilon_0>0$ tel que\\
    $\forall n\in\N,~ \exists x_, \in I$ tel que $\mc{x_n-a}\leqslant\frac{1}{n+1}$ et $\mc{f(x_n)-\ell} >\varepsilon_0 $\\
    On a ainsi $\suite{x}\in I^\N$ convergente vers $a$ avec $\big(f(x_n)\big)$ qui ne converge pas vers $\ell$. \vspace*{0.2cm} \\
    Les preuves pour pour les limites infinies et/ou en l'infini sont analogue.
    \end{proof}
    ${}$ \\ \thm{ch8P5}{Proposition : opérations sur les limites}{OpeLim}{L'opérateur "limite" est stable par somme, produit, quotient\footnotemark[1] \\et composition.}
    \footnotetext[1]{Dans ce cas seulement si la limite au dénominateur est non nulle et que le quotient n'est pas une forme indéterminée de $\overline{\R}$}
    \newpage ${}$ \\ \thm{ch8P6}{Proposition}{LimInegLarge}{Soit $a$ un point de $I$\\ On suppose que $f\leqslant g$ sur un voisinage de $a$, $f(x)\stox{a}\ell$ et $g(x)\stox{a} \ell'$\\
    Alors $\ell\leqslant \ell'$}
    \vspace*{0.5cm} \\ \thm{ch8th2}{Théorème d'encadrement}{ThEncadr}{Soit $f,g,h$ trois fonctions telles que sur un voisinage de $a\in I$ \\
    on a $h\leqslant f\leqslant g$. On suppose que $h$ et $g$ converge vers \\ une même limite $\ell$ en $a$, alors $f$ converge vers $\ell$ en $a$}
    \begin{proof}
    Clair avec la définition et en considérant le plus petit $\delta$
    \end{proof}
    ${}$ \\ \thm{ch8th3}{\highlight{Théorème de la limite monotone}}{ThLimMonot}{Soit $(a,b) \in \R^2$ avec $a<b$ et $f$ une fonction croissante sur $]a,b[$. \\ 
    Alors $f$ admet une limite à gauche et une limite à droite \\ en tout point $x_0\in ]a,b[$ avec \\
    \hspace*{2cm} $\cm{ \limit{x}{x_0^-} f(x) \leqslant f(x_0)\leqslant \limit{x}{x_0^+} f(x) }$ \\
    Si de plus $f$ est majorée (resp. minorée) sur $]a,b[$ alors elle admet \\ une limite à gauche en $b$ (resp. à droite en $a$)}
    \begin{proof}
    Soit $x_0 \in ]a,b[$, on considère $f_d\big(]a,x_0[\big)$ et $\ell$ sa borne supérieure (existe). On peut ensuite montrer que $f(x)\stox{x_0^-} \ell$ puis on fait de même avec $f_d\big(]x_0,b[\big)$
    \end{proof}
\subsection{Continuité en un point}
    \traitd
    \paragraph{Continuité}
        Soit $f$ définie sur $I$ à valeur réelles, on dit que \uline{$f$ est continue au point $a\in I$} si $f(x)\stox{a} f(a)$ \trait ${}$ \vspace*{-1.5cm} \\ \traitd 
    \paragraph{Prolongement par continuité}
        Si $f$ admet une limite finie $l$ en un point $a$ de $\R$ et si $f$ n'est pas définie en $a$, on appelle \uline{prolongement par continuité de $f$ en $a$} la fonction égale à $f$ sur son domaine de définition et à $l$ en $a$. \trait
    \thm{ch8P7}{Proposition : caractérisation séquentielle de la continuité}{CarSeqCont}{$f$ est continue en $a\in I$ si et seulement si pour toute suite $\suite{u}\in I^\N$ \\qui converge vers $a$, $\big( f(u_n)\big)_{_{n\geqslant 0}}$ converge vers $f(a)$}
    \vspace*{0.5cm} \\ \thm{ch8P8}{Propriété}{OpeFCont}{Si $f$ et $g$ sont deux fonction continues en un point $a$ de $I$, alors $f+g$ et $fg$ \\sont continues en $a$. Si de plus $g(a)\neq 0$ alors $\frac{f}{g}$ est continue en $a$. \\ 
    Si $h$ est continue en $f(a)$ alors $h\circ f$ est continue en $a$} \newpage
\subsection{Continuité sur un intervalle}
    \traitd \paragraph{Définition}
        On dit que \uline{$f$ est continue sur $I$} si elle est continue en tout point de $I$.\\
        On note $\cont^0 (I,\R)$ l'ensemble des fonctions continue sur $I$ à valeur dans $\R$ \trait
    \thm{ch8th4}{\highlight{Théorème des valeurs intermédiaires}}{TVI}{Si $f\in \CO (I,\R)$ et $(a,b) \in I^2$ \\ Alors $f$ prend sur $I$ toute les valeurs comprises entre $f(a)$ et $f(b)$.}
    \begin{proof}
    On considère $\alpha = \sup \{ x\in[a,b] ~|~ f(x)<c\}$ et on a alors par continuité de $f$ $\neg (f(\alpha)<c \vee f(\alpha)>c)~\Leftrightarrow ~f(\alpha)=c$
    \end{proof}
    ${}$ \\ \thm{ch8P9}{Propriété}{ContSegBorn}{Soit $f$ une fonction continue sur un segment alors $f$ est bornée \\ sur ce segment et $f$ atteint ses bornes.}
    \vspace*{0.5cm} \\ \thm{ch8P9c}{Corollaire}{ImgSegFCont}{L'image d'un segment par une fonction continue est un segment.}
    \vspace*{0.5cm} \\ \thm{ch8P10}{Proposition}{8-P9}{Soit $f$ est continue sur $I$ à valeurs réelles, on suppose $f$ est injective sur $I$\\
    Alors $f$ est strictement monotone sur $I$}
    \begin{proof}
    Soit $(a,b) \in I^2$ avec $a<b$, on suppose sans perte de généralité que $f(a)<f(b)$ alors $f$ est strictement croissante sur $[a,b]$, en effet : \vspace*{0.2cm} \\
    Par l'absurde, soit $(x,y)\in ]a,b[^2$ tel que $a<x<y<b$ et $f(x)>f(y)$\\
    On considère alors $g~\appli{[0,1]}{t}{\R}{f\big((ta+(1-t)x\big) - f\big( tb+(1-t)y\big)}$  continue sur $[0,1]$\\
    Vu $g(0) = f(x)-f(y)>0$ et $g(1) = f(a)-f(b)<0$ par le TVI $g$ s'annule au moins une fois sur $]0,1[$ donc $f$ prend deux fois la même valeur en deux points distincts de $[a,b]$ ce qui est impossible d'où \textsc{cqfd}
    \end{proof}
    ${}$ \\ \thm{ch8th5}{Théorème de la bijection réciproque}{ThBijReciproque}{Toute fonction réelle définie et continue strictement monotone sur un intervalle \\admet une fonction réciproque de même monotonie sur l'intervalle image.}
    \begin{proof}
    Soit $f$ strictement croissante et continue sur $I$ alors $f$ réalise un bijection de $I$ sur $J=f_d(I)$. On considère alors $f^{-1}$\\
    D'après le théorème de la limite monotone $f^{-1}$ est continue à droite et à gauche en tout point de l'intervalle ouvert et par injectivité ces limites sont égales donc $f^{-1}$ est continue sur l'intervalle ouvert puis fermé donc strictement monotone avec les monotonie clairement identiques.
    \end{proof}
\subsection{Fonctions à valeurs complexes}
    $f : I \rightarrow \C$ et $x_0$ un point ou une extrémité de $I$. $f$ admet une limite $\ell\in \C$ en $x_0$ si \[ \forall \varepsilon >0 ,~\exists \delta>0 ~:~\forall x\in I,~\big( \mc{x-x_0} \leqslant\delta ~\Rightarrow~\mc{f(x)-\ell}\leqslant\varepsilon\big)\]
    Si $x_0\in I$ alors $f(x_0) = \ell$ et $f$ est \uline{continue en $x_0$}. On note $\ell = \limit{x}{x_0} f(x)$
    \vspace*{0.5cm} \\ \thm{ch8th6}{Théorème : caractérisation des limites par les parties réelles et imaginaires}{CarLimComplexe}{$f : I\rightarrow \C$ admet une limite $\ell\in \C$ en $x_0$ si et seulement si \\ $\Re(f)$ et $\Im(f)$ admettent des limites $(\ell_r,\ell_i) \in R^2$.\\
    On a alors $\ell=\ell_r+\imath \ell_i$}
    \begin{proof}
    Clair vu $\forall z\in \C ,~\mc{z} \leqslant \mc{\Re(z)} + \mc{\Im(z)}$ et $\max \big(\mc{\Re(z)} , \mc{\Im(z)} \big) \leqslant \mc{z}$
    \end{proof}
\section{Dérivabilité}
    \traitd \paragraph{Dérivabilité en un point}
        $f$ est dérivable en un point $a$ de $I$ si \[\tau_a(f)~\appli{I\setminus\{a\}}{x}{\R}{\frac{f(x)-f(a)}{x-a}}\] le taux d'accroissement de $f$ en $a$ admet une une limite finie $\ell\in \R$ quand $x$ tend vers $a$.\\
        On note $f'(a)$ cette limite. \trait
    \thm{ch8P11}{Proposition}{DerImplCont}{Si $f$ est dérivable en $a$ alors $f$ est continue en $a$.}
    \vspace*{0.5cm} \\ \thm{ch8P12}{Propriété}{CarDer}{$f$ est dérivable en $a$ si et seulement si il existe une fonction $\varepsilon$ \\ définie sur un voisinage de $0$ telle que \\
    \hspace*{2cm} $\cm{f(a+h) = f(a) + h\times \ell + h\varepsilon(h) }$ \\
    où $\ell\in \R$ et $\limit{h}{0} \varepsilon(h) = 0$. On a alors $\ell=f'(a)$} \\
    \traitd \paragraph{Dérivabilité sur un intervalle}
        On dit que $f$ est dérivable sur $I$ si elle est dérivable en tout point de $I$. On note alors $f'$ sa fonction dérivée qui à tout point $a$ de $I$ associe $f'(a)$ \trait
    \thm{ch8P13}{Propriétés}{OpeFDer}{Soit $f,g$ deux fonction dérivables en $a$ alors \\
    $\bullet$ $f+g$ est dérivable en $a$ et $(f+g)'(a) = f'(a)+g'(a)$\\
    $\bullet$ $fg$ est dérivable en $a$ et $(fg)'(a) = f'(a)g(a) + f(a)g'(a)$\\
    $\bullet$ Si $g'(a)\neq 0$ alors $\frac{f}{g}$ est dérivable en $a$ et $\left(\frac{f}{g}\right)'(a) = \frac{f'(a)g(a)-f(a)g'(a)}{\big(g(a)\big)^2}$ \\
    Soit $h$ dérivable en $f(a)$ \\ Alors $h\circ f$ est dérivable en $a$ et $(h\circ f)'(a) = f'(a)\times h'\big( f(a)\big)$ }
    \newpage ${}$ \\ \thm{ch8P14}{Proriété}{DerBij}{Si $f$ est bijective de $I$ sur $J$ dérivable en $a\in I$ \\
    Alors $f^{-1}$ est dérivable en $f(a)=b$ si et seulement si $f(a)\neq 0$. \\
    On a alors $\big( f^{-1}\big)'(b) = \dfrac{1}{f'(a)}$} \\
\subsection{Extremum local et point critique}
    \traitd
    \paragraph{Extremum local}
        \subparagraph{Maximum} On dit que \uline{$f$ présente un maximum local en $a\in I$} s'il existe $\delta>0$ tel que 
        \[\forall x\in [a-\delta,a+\delta] \cap I ,~f(x)\leqslant f(a) \]
        \subparagraph{Minimum} La définition est analogue \trait ${}$ \vspace*{-1.5cm} \\ \traitd 
    \paragraph{Point critique}
        Un \uline{point critique} est un zéro de la dérivée. \trait
    \thm{ch8P15}{Propriété}{ExtremumCritique}{Soit $a$ est un point intérieur à $I$ et $f$ dérivable en $a$.\\
     On suppose que $f$ présente un extremum local en $a$, alors $a$ est un point critique.} \\
     \uline{Rq} : Si $a$ un point intérieur à $I$ est un point critique et si $f$ ne présente pas d'extremum local en $a$, on dit que \uline{$a$ est un point d'inflexion de $f$}. \\
\subsection{Théorèmes de Michel \textsc{Rolle} et des accroissements finis}
       ${}$ \hspace*{-1cm} \fbox{ \begin{minipage}{15.7cm} 
    \thm{ch8th7}{\highlight{Théorème de Michel \textsc{Rolle}}}{ThRolle}{Soit $a,b \in \R$ avec $a<b$ et $f$ continue sur $[a,b]$ et \\dérivable sur $]a,b[$ à valeurs réelles. On suppose $f(a) = f(b)$ \\
    \hspace*{2cm} Alors il existe $c\in ]a,b[$ tel que $f'(c) = 0$ }
    \end{minipage}    }
    \begin{proof}
    Si $f$ est constante sur $[a,b]$ c'est vrai.\\
    Sinon l'image continue de $5a,b]$ par $f$ est un segment $[M,m]$ avec $M$ ou $m$ différent de $f(a) = f(b)$ atteint en $c\in ]a,b[$ qui est alors un point critique de $f$ d'où \textsc{cqfd}
    \end{proof}
    ${}$ \\ \thm{ch8th8}{Théorème des accroissements finis}{ThAccrFinis}{Soit $a,b \in \R$ avec $a<b$ et $f$ continue sur $[a,b]$ \\à valeurs réelles et dérivable sur $]a,b[$\\
    Alors $\exists c\in ]a,b[$ tel que \highlight{$f(b)-f(a) = f'(c)(b-a)$}}
    \begin{proof}
    On considère $h_{a,b}$ la corde à $\mathcal{C}_f$ joignant les points d'abscisse $b$ et $a$. Soit ensuite 
    \[ g : x\mapsto f(x) - h_{a,b}(x) = f(x) - \Big( f(a) + (x-a) \dfrac{f(b)-f(a)}{b-a} \Big) \]
    On a alors $g$ continue sur $[a,b]$ et dérivable sur $]a,b[$ avec $g(a) = 0 =g(b)$ soit donc d'après le théorème de Michel \textsc{Rolle} né à Ambert en 1652 $c\in ]a,b[$ tel que $g'(c) = 0$ \\or $g'(c) = f'(c) - \frac{f(b) - f(a)}{b-a}$ d'où \textsc{cqfd}
    \end{proof}
    ${}$ \\ \thm{ch8th8c}{Corollaire : Inégalité des accroissements finis}{InegAccrFinis}{Soit $f$ continue sur $[a,b]$ et dérivable sur $]a,b[$ à valeurs dans $\R$ ou $\C$, \\
    On suppose $\exists k\in R$ tel que $\forall x\in ]a,b[ ,~\mc{f'(x)} \leqslant k $\\
    \hspace*{2cm} Alors $\mc{f(b) - f(a) } \leqslant k\mc{b-a}$ } \\
    \traitd \paragraph{Fonction lipschitzienne}
        On dit que \uline{$f$ est $k$-lipschitzienne sur $I$} si \[ \forall x\in ]x,y[ \in I^2 , ~\mc{f(x)-f(y)} \leqslant k \mc{x-y} \]
        On dit que \uline{$f$ est lipschitzienne sur $I$} s'il existe $k\in \R$ tel que $f$ est $k$-lipschitzienne \trait
    \thm{ch8P16}{Propriété}{LipCont}{Si $f$ est lipschitzienne sur $I$ alors $f$ est continue sur $I$}
    \vspace*{0.5cm} \\ \thm{ch8P17}{Propriété}{InegAFLip}{Si $f$ est dérivable sur $I$ telle que $\forall x\in I ,~\mc{f'(x)} \leqslant k$ \\ Alors $f$ est $k$-lipschitzienne sur $I$}
    \vspace*{0.5cm} \\ \thm{ch8P18}{Propriété}{8-P18}{Soit $f$ dérivable sur $I$ à valeurs réelles \\
    \un $f$ est constante sur $I$ si et seulement si $f'$ est identiquement nulle sur $I$.\footnotemark[1] \\
    \deux $f$ est croissante sur $I$ si et seulement si $f'$ est positive sur $I$. \\
    \trois $f$ est strictement croissante sur $I$ si et seulement si \\
    \hspace*{0.5cm} $\left\{ \ard f'$ est positive sur $I \\ $Il n'existe pas $J\subset I$ contenant deux points distincts avec $f'$ nulle sur $J \arf \right.$ }
    \footnotetext[1]{Ceci reste vrai si $f$ est définie sur $I$ à valeurs complexes}
    \vspace*{0.5cm} \\ \thm{ch8th9}{Théorème de la limite de la dérivée}{ThLimDer}{Soit $a\in I$. Si $f$ est continue sur $I$ et dérivable sur $I\setminus\{a\}$\\
    On suppose $f'(x) \stox{a} \ell\in \R$ alors $f$ est dérivable en $a$ et $f'(a) = \ell$}
    \begin{proof}
    Pour tout $x\in I\setminus\{a\}$, il existe par le théorème des accroissements finis $c_x$ strictement compris entre $x$ et $a$ tel que $f(x)-f(a) = f'(c_x)\times (x-a)$ .\\
    Si $x$ tend vers $a$ alors par encadrement $c_x$ tend vers $a$ et par composition de limites $f(c_x)$ tend vers $\ell$.\\
    Ainsi $\tau_a(f)(x) \stox{a} \ell = f'(a)$
    \end{proof}
    ${}$ \\ \thm{ch8th9c}{Corollaire}{8-th9c}{Si $f$ est continue sur $I$ et dérivable sur $I\setminus\{a\}$ avec $\limit{x}{a} f'(x) = +\infty$ \\Alors $f$ n'est pas dérivable en $a$ et $\mathcal{C}_f$ admet une tangente verticale en $a$}
\subsection{Fonctions de classe $\Ck ,~(k\in\N\cup\{+\infty\})$}
    \traitd
    \paragraph{Définitions}
        Une fonction $f$ est dite de \uline{classe $\CO$ sur $I$} si elle est continue sur $I$.\\
        Elle est dite de \uline{classe $\Ck$ sur $I$} si elle est $k$ fois dérivable sur $I$ et si sa dérivée $k$-ième est continue sur $I$.
        Elle est dite de \uline{classe $\Cinf$ sur $I$} si elle est de classe $\Ck$ sur $I$ pour tout $k\in\N$ \trait
    \thm{ch8P19}{Propriété}{ExsFCinf}{\un Les fonction polynômiales sont de classe $\Cinf$ sur $\R$. \\
    \deux Les fonction rationnelles (quotient de fonctions polynômiales) \\sont de classe $\Cinf$ sur leur ensemble de définition.\\
    \trois Les fonctions $\sin$ et $\cos$ sont de classe $\Cinf$ sur $\R$\\
    \quatre Les fonction exponentielles sont de classe $\Cinf$ sur $\R$ \\
    {\scriptsize (5)} Les fonction logarithme et puissances sont de classe $\Cinf$ sur $\R^+$ } 
    \vspace*{0.5cm} \\ \thm{ch8P20}{Proposition}{ClasseP+Q}{Soit $f$ une fonction et $(p,q)\in \N^2$ \\
    Alors $f$ est de classe $\cont^{p+q}$ sur $I$ si et seulement si $f^{(p)}$ est de classe $\cont^q$ sur $I$\\
    \hspace*{2cm} On a alors $\big( f^{(p)} \big)^{^{(q)}} = f^{(p+q)}$ }
    \vspace*{0.5cm} \\ \thm{ch8P21}{Proposition}{SommeClassCk}{Soit $k\in \N$. Soit $f,g \in \Ck(I,\R)$ \\
    Alors $f+g \in \Ck(I,\R)$ et $(f+g)^{(k)} = f^{(k)} + g^{(k)}$}
    \vspace*{0.5cm} \\ \thm{ch8th10}{Théorème : Formule de \textsc{Leibniz}}{FormuleLeibniz}{Soit $n\in\N$ ; soit $f$ et $g$ deux fonctions de classe $\cont^n$ sur $I$ \\
    Alors $fg$ est de classe $\cont^n$ sur $I$ et \\
    \hspace*{2cm} $\cm{(fg)^{(n)} = \sk{0}{n}\binom{n}{k} f^{(n)}g^{(n-k)}}$ }
    \begin{proof}
    Clair par récurrence sur $n$.
    \end{proof}
    ${}$ \\ \thm{ch8P22}{Proposition : Formule de Faa \textsc{Di Bruno}}{CompClasseCn}{Soit $n\in\N^*$, si $f$ est de classe $\cont^n$ sur $I$ à valeur dans 
    \\ un intervalle $J$ non trivial, $g$ est de classe $\cont^n$ sur $J$ \\
    Alors $g\circ f$ est de classe $\cont^n$ }
    \vspace*{0.5cm} \\ \thm{ch8P22c}{Corollaire}{1/FCn}{Soit $n\in\N^*$, si $f$ et $g$ sont de classe $\cont^n$ sur $I$ avec $0\notin g_d(I)$ \\
    \hspace*{2cm} Alors $\dfrac{1}{g}$ et $\dfrac{f}{g}$ sont de classe $\cont^n$ sur $I$}
    \vspace*{0.5cm} \\ \thm{ch8P23}{Proposition}{BijRecCn}{Soit $n\in\N^*$, si $f$ est bijective de $I$ sur $J$ de classe $\cont^n$ sur $I$ \\
    et si $f$ ne s'annule pas sur $I$ alors $f^{-1}$ est de classe $\cont^n$ sur $J$. } \\
\section{Convexité}
\subsection{Généralités}
    \traitd
    \paragraph{Fonction convexe}
        Soit $f$ une fonction à valeurs réelles. On dit que \uline{$f$ est convexe sur $I$} si \[ \forall (x,y)\in I^2 ,~\forall \lambda\in [0,1],~ f\big( \lambda x+(1-\lambda)y) \leqslant\lambda f(x) + (1-\lambda- f(y) \] \trait ${}$ \vspace*{-1.5cm} \\ \traitd 
    \paragraph{Fonction concave}
     On dit que \uline{$f$ est concave sur $I$} si $-f$ est convexe sur $I$ \trait
    \thm{ch8th11}{Théorème : Inégalité de \textsc{Jensen}}{InegJensen}{Si $f$ est convexe sur $I$ alors $\forall n\in\N ,~n\geqslant 2$ \\
    $\forall (x_1,\dots ,x_n)\in I^n ,~\forall (\lambda_1, \dots ,\lambda_n )\in \R_+^n$ avec $\sk{1}{n} \lambda_k = 1$\\
    \hspace*{2cm} $\cm{ f\Big( \si{1}{n} \lambda_ix_i \Big) \leqslant \si{1}{n} \lambda_i f(x_i) } $ }
    \begin{proof}
    On a le résultat par récurrence en barycentrant en divisant par $1-\lambda_{n+1}$ (cas $\lambda_n+1 = 1$ trivial) puis en appliquant la propriété au rang $2$ (inégalité de convexité)
    \end{proof}
    ${}$ \\ \thm{ch8P24}{Propriété : Lemme des pentes}{LemmePentes}{Soit $f: I\rightarrow \R$ on a équivalence entre les propriétés suivantes :\\
    \un $f$ convexe sur $I$\\
    \deux $\forall (a,b,c)\in I^3 $ avec $a<b<c$, $\frac{f(b)-f(a)}{b-a}\leqslant\frac{f(c)-f(a)}{c-a} \leqslant \frac{f(c)-f(b)}{c-b}$\\
    \trois $\forall x_0 \in I,~ \tau_{x_0}(f)$ est croissant }
\subsection{Fonctions convexes dérivables et deux fois dérivables}
    ${}$ \\ \thm{ch8P25}{Proposition : Caractérisation des fonctions convexes dérivables}{CarFConvexeDer}{Soit $f$ un fonction dérivable sur $I$ alors \\
    $f$ est convexe sur $I$ si et seulement si $f'$ est croissante sur $I$.}
    \vspace*{0.5cm} \\ \thm{ch8P25c}{Corollaire}{CfConvexeSurTang}{Si $f$ est convexe sur $I$ alors $\mathcal{C}_f$ est située au-dessus de ses tangentes.}
    \vspace*{0.5cm} \\ \thm{ch8P26}{Proposition}{CarFConvexe2Der}{Soit $f$ une fonction deux fois dérivable sur $I$ alors \\
    $f$ est convexe sur $I$ si et seulement si $f''$ est positive sur $I$.}
    \vspace*{0.5cm} \\
    \begin{center}
    \fin
    \end{center}

\chapter{Arithmétique dans $\mathbf{Z}$}

    
% Chapitre 8 : Arithmétique dans Z

\minitoc
	\section{Relation de divisibilité dans $\Z$}
		\subsection{Principe de bon ordre}
		${}$ \\ \thm{ch9th1}{Théorème : Pincipe de bon ordre dans $\N$}{BonOrdre}{Toute partie \textbf{non vide} de $\N$ aadmet un plus petit 
		élément.}
		\vspace*{0.5cm} \\ \thm{ch9th1c}{Corollaire : Propriété archimédienne}{PropArchi}{Soit $a,b \in\N^*$ il existe $n\in\N$ tel que $a\times 
		n\geq b$.} \newpage
		\subsection{Multiples et partie $a\Z$}
		\traitd
		\paragraph{Notation}
			Si $a\in\Z$ alors on note $a\Z = \{ka ~|~k\in\Z \}$ \\ Si $(a,b)\in\Z^2$ alors $a\Z + b\Z = \{ka+lb ~|~(k,l)\in\Z^2 \}$. \trait
		\thm{ch9P1}{Propriété}{PartZstableaZ}{Toute partie de $\Z$ stable par somme est une partie de la forme $m\Z$ avec $m\in\N$.} \traitd
		\paragraph{Multiple} Soit $(a,b)\in\Z^2$ on dite que $b$ est un multiple de $a$ (ou $a$ divise $b$) et on note $a|b$ s'il existe $k\in\Z 
			~:~b=ka$. \trait
		\thm{ch9P2}{Propriété}{9P2}{Si $(a,b)\in\Z^2$ alors on a $a|b \Leftrightarrow b\Z \subset a\Z$ }
	\section{Algorithme de division euclidienne}
		${}$ \\ \thm{ch9th2}{Théorème}{DivEuclid}{Soit $a\in\Z , ~b\in\N^*$ alors il existe $(q,r)\in\Z^2$ unique tel que
		$a=bq+r$ et $0\leq r<b$ \\ On appelle $q$ et $r$ le quotient et le reste de la division euclidienne de $a$ par $b$.}
		\begin{proof} \underline{Unicité} : claire \\
		\underline{Existence} : On considère $S=\{a-bk ~|~k\in\Z \wedge a-bk\geq 0 \}$ on a alors $S\neq \varnothing$ puis on pose $r=\mathrm{min}(S)
		$ avec $r<b$ sinon $r-b = a-b(k_0+1) \geq 0$ donc $0\leq r <b$
		\end{proof}
	\section{\textsc{pgcd} et \textsc{ppcm}}
		\subsection{Egalité de Bézout}
		${}$ \\ \thm{ch9L1}{Lemme}{9-L1}{Si $a|b$ et $b\neq 0$ Alors $\mc{a} \leq \mc{b}$} \traitd
		\paragraph{\textsc{pgcd}} Pour tout $a,b\in\Z^*$, le plus grand commun diviseur de $a$ et $b$ est l'entier naturel $d$ vérifiant les 
			conditions suivantes : \begin{blockarray}[t]{\{l} {\tiny (1)} $d|a$ et $d|b$ \\ {\tiny (2)} $\forall c\in\Z ,~c|a$ et $c|b ~\Rightarrow 
			~c\leq d$ \end{blockarray} \trait
		\thm{ch9P3}{Propriété}{EgalBezout}{Soit $a,b \in \Z^*$ il existe $(u,v) \in \Z^2$ tel que $au+bv=a\wedge b$}
		\vspace*{0.5cm} \\ \thm{ch9P4}{Propriétés}{9-P4}{Soit $(a,b)\in (\Z^*)^{^2}$ et $m\in\Z$ Alors \\ 
		-> $a\wedge (b+ma) = a\wedge b = a\wedge (-b)$ \\ -> $ma\wedge mb = \mc{m} (a\wedge b)$ \\ -> si $d=a\wedge b$, $\frac{a}{d}\wedge 
		\frac{b}{d} = 1$ \\ -> si $g\in\Z^*$, $g|a$ et $g|b ~\Rightarrow ~\frac{a}{g} \wedge \frac{b}{g} = \frac{1}{\mc{g}} (a\wedge b)$}
		\subsection{Algorithme d'Euclide}
		${}$\\ \thm{ch9L2}{Lemme}{LemmeAlgoEuclide}{Soit $(q,r)$ le quotient et le reste de la division euclidienne de $a\in\Z$ par $b\in\N^*$ \\
		Alors $a\wedge b= b\wedge r$}
		\begin{proof}
		$a\wedge b = (a-bq) \wedge b = r\wedge b$
		\end{proof}
		\paragraph{Algorithme}\label{AlgoEuclid}
			Soit $a\in\Z,~b\in\N^*$ On pose $r_0=a$, $r_1=b$ et $r_2$ le reste de la division euclidienne de $r_0$ par $r_1$.\\
			-> Si $r_n=0$ alors \highlight{$a\wedge b=r_{n-1}$} sinon on considère $r_{n+1}$ le reste de la division euclidienne de $r_{n-1}$ par 
			$r_n$ avec $r_{n-1}\wedge r_n = r_{n-2}\wedge r_{n-1} = \cdots = a\wedge b$
		\subparagraph{Algorithme d'Euclide étendu} ${}$\\
		Si on souhaite obtenir les coefficients de \textsc{Bézout} en même temps que le \textsc{pgcd}, on détermine à chaque étape $(u_k,v_k) 
		\in\Z^2$ tels que $r_k=au_k+bv_k$ avec \[ \left\{ \ard u_{n+1} = u_{n-1}-q_nu_n \\ v_{n+1} = v_{n-1} - q_nv_n \arf \right. \]
		\traitd 
		\paragraph{\textsc{ppcm}} Soit $(a,b)\in (\Z^*)^{^2}$ le plus petit commun multiple de $a$ et $b$ est l'entier naturel $m$ vérifiant les 
			conditions suivantes : \begin{blockarray}[t]{\{l} {\tiny (1)} $a|m$ et $b|m$ \\ {\tiny (2)} $\forall c\in\Z ,~a|c$ et $b|c ~\Rightarrow 
			~m\leq c$ \end{blockarray} \trait
		\thm{ch9P5}{Propriété}{ppcmZ}{Soit $a,b\in\Z^*$ Alors $a\Z \cap b\Z = (a\vee b) \Z$}
		\vspace*{0.5cm} \\ \thm{ch9P6}{Propriété}{9-P6}{Soit $(a,b)\in (\Z^*)^{^2}$ Alors $(a\wedge b)(a \vee b)= \mc{ab}$}
	\section{Entiers premiers entre eux}
		\traitd
		\paragraph{Définition} Deux entiers $a,b \in\Z^*$ sont dits \underline{premiers entre eux} si $a\wedge b=1$ \trait
		\thm{ch9P7}{Proposition}{EgalBezout1}{Soient $a,b\in\Z^*$ deux entiers alors\\
		$a$ et $b$ sont premiers entre eux \underline{si et seulement si} il existe $(u,v)\in\Z^2$ tel que $au+bv=1$}
		\vspace*{0.5cm} \\ \thm{ch9L3}{Lemme de \textsc{Gauss}}{LemmGauss}{Soient $a,b,c\in\Z$ on a \\ Si $c$ divise $ab$ et $c$ est premier avec 
		$a$ alors $c$ divise $b$.}
		\vspace*{0.5cm} \\ \thm{ch9P8}{Propriété}{9-P10}{Soient $a_1,\dots ,a_n \in \Z$ on a \\ Si $\forall i\in\ent{1,n},~a_i$ est premier avec 
		$c$ alors $~~\pdi{1}{n}a_i$ est premier avec $c$}
		\vspace*{0.5cm} \\ \thm{ch9L4}{Lemme}{PGCDDistrib}{Soient $a,b,c\in\Z^*$ 
		Alors $(a\wedge b) \wedge c = a\wedge (b\wedge c) = a\wedge b \wedge c$} \traitd
		\paragraph{Entiers premiers entre eux dans leur ensemble} Soit $(a_1,\dots ,a_n)\in (\Z^*)^{^n}$ On dit que $(a_1,\dots ,a_n)$ sont 
		\underline{premiers entre eux dans leur ensemble} si $a_1\wedge \cdots \wedge a_n =1$,\\ Ceci équivaut à l'existence de $(u_1,\dots ,u_n) 
		\in\Z^n$ tel que $\sum_{i=1}^n u_ia_i=1$ \trait \vspace*{-1cm}
	\section{Nombres premiers}
		\traitd
		\paragraph{Définition} On dit qu'un entier naturel $p$ est (un nombre) premier si $p\geq 2$ et si les seuls diviseurs dans $\N$ de $p$ sont 
		$1$ et lui-même \trait \vspace*{-1.2cm} \\ Un nombre qui n'est pas premier est dit \underline{composé}. 
		\vspace*{0.5cm} \\ \thm{ch9L5}{Lemme}{DivisPremier}{Tout entier $n\geq 2$ admet un diviseur premier}
		\vspace*{0.5cm} \\ \thm{ch9L5c}{Corollaire}{NPinfini}{L'ensemble des nombres premiers est inifini.}
		\vspace*{0.5cm} \\ \thm{ch9L6}{Lemme}{9-L6}{Si $p$ est un nombre premier et $a\in\N$ \\ Alors $a\wedge p = 1 ~\Leftrightarrow ~
		p\hspace*{-4pt}\not\vert a$}
		\vspace*{0.5cm} \\ \thm{ch9L7}{Lemme d'\textsc{Euclide}}{LemmeEuclid}{Soit $p$ un nombre premier et $a,b\in\Z$ \\ Si $p|ab$ alors 
		$p|a$ ou $p|b$}
		\vspace*{0.5cm} \\ \thm{ch9th3}{\highlight{Théorème fondamental}}{THFondArithm}{Tout nombre entier supérieur à $2$ s'écrit comme produit de 
		facteurs premiers. \\Cette décomposition est unique à l'ordre des facteurs près.}
		\begin{proof}
		\underline{Existence} : Par récurrence forte avec l'existence d'un diviseur premier %(\ref{DivisPremier}) 
        \\
		\underline{Unicité} : Si $p=\pdi{1}{m} p_i^{\alpha_i} = \prod\limits_{j=1}^l q_j^{\beta_j}$ avec $p_i,q_j$ premiers distincts \\
		On pose $i_0\in\ent{1,m}$ et on a alors $p_{i_0} | \prod_{j=1}^l q_j^{\beta_j}$ donc il existe $j_0$ tel que $p_{i_0} | 
		q_{j_0}^{\beta_{j_0}}$ soit $p_{i_0}=q_{j_0}$ Ainsi $\{p_1,\dots ,p_m\} = \{q_1,\dots ,q_l\}$ et $m=l$\\
		On suppose $p_k=q_k$ et $\alpha_k<\beta_k$ avec $k\in\ent{1,m}$ alors $q_k^{\beta_k-\alpha_k} | \pdi{i\in\ent{1,m}\backslash \{k\} }{}
		p_i^{\alpha_k}$ soit $q_k | p_i ~;~q_k = p_i$ avec $k\neq i$ \textsc{impossible}
		\end{proof} \traitd
		\paragraph{Valuation $p$-adique}
		Soit $p$ un nombre premier et $n\in\N^*$, on appelle \underline{valuation $p$-adique de $n$} l'exposant de $p$ dans la décomposition de $n$ 
		en produits de facteurs premiers. \trait
		\thm{ch9L8}{Lemme}{DivisValu}{$\forall (m,n)\in (\N^*)^{^2}$ on a $m | n \Leftrightarrow \forall p\in\N$ premier, $v_p(m)\leq v_p(n)$}
		\vspace*{0.5cm} \\ \thm{ch9P9}{Proposition}{ValPGCDPPCM}{$\forall (a,b) \in (\N^*)^{^2}$, \\ $a\wedge b = \prod\limits_{p~\mathrm{premier}} 
		p^{\mathrm{min}\big(v_p(a),v_p(b)\big)} $ et $a\vee b = \prod\limits_{p~\mathrm{premier}} p^{\mathrm{min}\big(v_p(a),v_p(b)\big)}$}
		\vspace*{0.5cm} \\ \thm{ch9P10}{Propriété}{VpLog}{$\forall (a_1,\dots ,a_n)\in (\N^*)^{^n} ,~\forall p$ premier, $v_p\Big( \pdi{1}{n} 
		a_i\Big) = \si{1}{n} v_p(a_i)$} \\
	\section{Congruences}
		\traitd
		\paragraph{Définition}
			Soit $n\in\Z$ la relation de congruence modulo $n$ est définie par $a\equiv b[n] \Leftrightarrow n|a-b$\\ \textit{$a$ est congru à $b$ 
			modulo $n$.} \trait
		\thm{ch9P11}{Propriétés}{9-P11}{$\forall (a,b,c,d)\in\Z^4$, $n\in\N$ on a\\
		{\small 1)} $a\equiv b[n]$ et $c\equiv d [n] ~~\Rightarrow ~~ ac\equiv bd [n]$\\
		{\small 2)} $a\equiv b[n]$ et $c\equiv d [n] ~~\Rightarrow ~~ a+c\equiv b+d [n]$\\
		{\small 3)} $a\equiv b[n] ~~\Rightarrow ~~ \forall k\in\Z ,~ka\equiv kb[n]$ \\
		{\small 3)} $a\equiv b[n] ~~\Rightarrow ~~ \forall k\in\N ,~a^k \equiv b^k [n]$}
		\vspace*{0.5cm} \\ \thm{ch9L9}{Lemme}{PremierDivBinom}{Si $p\in\N$ est un nombre premier et $k\in\ent{1,p-1}$ Alors $p~| \binom{p}{k}$}
		\vspace*{0.5cm} \\ \thm{ch9th5}{Petit théorème de \textsc{Fermat}}{PThFermat}{Soit $p\in\N$ un nombre premier \\
		Alors $\ard $ {\small 1)} $\forall a\in\Z ,~ a^p \equiv a[p] \\ $ {\small 2)} Si $a\wedge p=1$ alors $a^{p-1} \equiv 1[p] \arf $}
		\begin{proof} {\small 1)} 
		\underline{si $p=2$} : $\forall a\in\Z$, $a^2$ et $a$ on la même parité d'où $a^2\equiv a[2]$ \\
		\underline{$p\geq 3$ (impair)} : Par récurrence sur $a\in\N$ vu $(a+1)^p\equiv a^p+1[p] $\\
		{\small 2)} $\forall a\in\Z ,~ p~|~a^p-a = a(a^{p-1}-1)$ donc si $a\wedge p=1$ on a $a^{p-1}\equiv 1 [p]$ (Lemme de \textsc{Gauss})
		\end{proof} \traitd
		\paragraph{Entier inversible}
		On dit que \underline{$a\in\Z^*$ est inversible modulo $n$} ($n\in\Z^*$) s'il existe $a'\in\Z$ tel que $a\times a'\equiv 1 [n]$ \trait
		\thm{ch9P12}{Propriété}{CNSInversible}{Soit $a\in\Z^*$ alors $a$ est inverible \underline{si et seulement si} $a\wedge n =1$}
		\vspace*{0.5cm} \\ 
		\begin{center}
		\fin
		\end{center}

\chapter{Structures algébriques usuelles}

    
% Chapitre 9 : Structures algébriques usuelles

\minitoc
	\section{Lois de composition interne}
		\traitd
		\paragraph{Définition}
			Une \underline{loi de composition interne sur un ensemble $E$} est une application \\
			\[ * ~~ \appli{E\times E}{(x,y)}{E}{x*y}\] \vspace*{-0.7cm} \trait \vspace*{-0.8cm} \\ 
		\hspace*{2cm} $\rightarrow ~*$ est \underline{associative} si $\forall (x,y,z)\in E^3 ,~ (x*y)*z = x*(y*z)$ \\
		\hspace*{2cm} $\rightarrow ~*$ est \underline{commutative} si $\forall (x,y) \in E^2 ,~x*y = y*x$ \\
		\hspace*{2cm} $\rightarrow ~*$ admet un \underline{élément neutre $e$} si $\forall x\in E ,~e*x=x*e=x$ \\
		\hspace*{2cm} $\rightarrow ~x\in E$ est dit \underline{inversible} s'il existe $e$ un élément neutre \\ \hspace*{3cm} et $x'\in E 
		~:~x*x'=x'*x=e$ 
		\\ \textit{\small Sous réserve d'existence, l'élément neutre et l'inverse sont uniques et on note $x'=x^{-1}$}
		\subparagraph{Exemple} 
		{\small $\times $ sur $\Z$ est une loi de composition interne associative et commutative de neutre $1$. \\ Les seuls éléments inversible 
		pour $\times$ sur $\Z$ sont $1$ et $-1$. }
		\vspace*{0.5cm} \\ \thm{ch10P1}{Propriété}{ProdInv}{Soit $E$ muni d'une loi de composition interne associative $*$. \\ Si $x$ et $y$ sont 
		deux éléments inversibles de $E$ \\ Alors $x*y$ est inversible et $(x*y)^{-1} = y^{-1}*x^{-1}$} \traitd
		\paragraph{Partie stable}
			Soit $E$ muni d'une loi de composition interne $*$. On dit que $F\in\mathcal{P}(E)$ est stable pour $*$ si \[ \forall (x,y)\in 
			E\times F ,~x*y \in F ~et ~y*x\in F \] \vspace*{-0.7cm}\trait \vspace*{-1.5cm}
	\section{Structure de groupe}
		\traitd 
		\paragraph{Groupe}
			Soit un ensemble $G$ muni d'un loi de composition interne $*$, on dit que \underline{$(G,*)$ est un groupe} si
			$*$ est associative ; $e \in G$ est un élément neutre pour $*$ et tout élément $x\in G$ est inversible.\trait ${}$ \vspace*{-1.2cm} \traitd
		\paragraph{Groupe abélien}
			On dit que $(G,*)$ est un \underline{groupe abélien (ou commutatif)} si $(G,*)$ est un groupe et $*$ est commutative sur $G$. \trait
		${}$ \vspace*{-1.4cm} \traitd
		\paragraph{Groupe produit}
			Soit $(G_1,*_1)$ et $(G_2,*_2)$ deux groupes on appelle \underline{groupe produit de $G_1$ et $G_2$} l'ensemble $G_1\times G_2$ muni 
			de la loi $*$ définie par \[\forall \big( (x_1,y_1),(x_2,y_2)\big) \in G_1^2\times G_2^2 ,~(x_1,x_2)*(y_1,y_2) = 
			(x_1*_1y_1 , x_2*_2y_2)\] \vspace*{-0.7cm} \trait
		\thm{ch10P2}{Propriété}{GrProdGr}{$G_1\times G_2 , *)$ est un groupe.} \traitd
		\paragraph{Sous-groupe}
			Soit $(G,*)$ un groupe. On dit que \underline{$F\subset G$ est un sous-groupe de $G$} \\ si $F\neq \varnothing ~;~F$ est stable par $*$ 
			et $\forall x\in F ,~x^{-1} \in F$. \trait
		\thm{ch10P3}{Propriété}{SGGr}{Si $F$ est un sous-groupe de $(G,*)$ alors $(F,*)$ est un groupe.}
		\vspace*{0.5cm} \\ \thm{ch10P4}{Propriété}{CarSG}{Soit $F\subset G$ alors \\ $F$ est un sous-groupe de $G$ $\Leftrightarrow ~F\neq 
		\varnothing$ et $\forall (x,y) \in F^2 ,~x*y^{-1} \in F$} \newpage\traitd
		\paragraph{Morphisme de groupe}
			Soit $(G_1,*-1)$ et $(G_2,*_2)$ deux groupe et $f: G_1 \rightarrow G_2$. On dit que \underline{$f$ est un morphisme de groupe} si 
			\[\forall (x,y) \in G_1^2 ,~f(x *_1 y) = f(x) *_2 f(y) \] \vspace*{-0.7cm} \trait
		\thm{ch10P5}{Propriété}{ImSGMorph}{Soit $(G_1,*_1)$ et $(G_2,*_2)$ deux groupes et $f: G_1 \rightarrow G_2$ un morphisme de groupe \\
		Alors $ \left\{ \ard \forall F_1\subset G_1$ sous-groupe de $G_1$, $f(F_1)$ est un sous-groupe de $G_2 \\ \forall F_2\subset G_2$ sous-
		groupe de $G_2$, $f^{-1} (F_2)$ est un sous-groupe de $G_1 \arf \right.$} \traitd
		\paragraph{Image et noyau}
			Soit $(G_1,*_1)$ et $(G_2,*_2)$ deux groupes et $f: G_1 \rightarrow G_2$ un morphisme de groupe, on défini l'image et le noyau de $f$ 
			et on note respectivement $Im(f)=f(G_1)$ et $Ker(f)=f^{-1} (\{e_2\})$ \hspace*{0.5cm}(\textit{\small $Im(f)$ et $Ker(f)$ sont des 
			sous-groupes respectifs de $G_2$ et $G_1$})\trait
		\thm{ch10P6}{Proposition}{MorphSurjInj}{Soit $(G_1,*_1)$ et $(G_2,*_2)$ deux groupes et $f: G_1 \rightarrow G_2$ un 
		morphisme de groupe \\ Alors $\left\{ \ard f$ est surjectif $\Leftrightarrow ~Im(f)=G_2 \\ f$ est injectif $\Leftrightarrow ~Ker(f) = 
		\{e_1\} \arf \right.$ } \\ \traitd
		\paragraph{Isomorphisme de groupe} 
			Soit $f:(G_1,*_1)\rightarrow (G_2,*_2)$ un morphisme de groupe. On suppose que $f$ est bijectif, alors $f^{-1} : (G_2,*_2)\rightarrow 
			(G_1,*_1)$ est un morphisme de groupe bien défini.\\ On appelle \underline{isomorphisme} un tel morphisme. \trait \vspace*{-1.3cm}
		\begin{proof}
		Soient $x',y' \in G_2$ ; soient $x,y\in G_1$ tels que $x=f^{-1}(x')$ et $y=f^{-1}(y')$, \\ on a $f(x*_1 y) = f(x)*_2f(y) = x' *_2 y'$ donc 
		$f^{-1}(x' *_2 y') = x *_1 y = f^{-1}(x') *_1 f^{-1}(y')$ \\ d'où \textsc{cqfd}
		\end{proof}
	\section{Structure d'anneau et de corps}
	\subsection{Structure d'anneau}
		\traitd
		\paragraph{Loi distributive}
			Soit $E$ un ensemble muni de deux lois de composition interne $\oplus$ et $\otimes$. \\On dit que $\otimes$ est distributive par rapport 
			à $\oplus$ si \[ \forall (x,y,z)\in E^3 ,~\left\{ \ard x\otimes (y\oplus z) = (x\otimes y)\oplus (x\otimes z) \\ (x\oplus y) \otimes z 
			=(x\otimes z) \oplus (y\otimes z) \arf \right. \] \vspace*{-0.7cm} \trait \newpage \traitd
		\paragraph{Anneau}
			Soit $A$ un ensemble muni de deux lois de composition internes $\oplus$ et $\otimes$. \\On dit que $(A,\oplus ,\otimes )$ est un anneau 
			si \\ \hspace*{2.5cm} $\ard \bullet ~(A,\oplus)$ est un groupe abélien $ \\ \bullet ~\otimes$ est associative et distributive par 
			rapport à $\oplus \\ \bullet ~$Il existe un élément neutre $1_A$ pour $\otimes \arf$ \trait
		\vspace*{-0.8cm}\\ \textit{\small On notera maintenant de manière équivalente $\otimes$, $\times$ et $.$ ainsi que $\oplus $ et $+$}
		\vspace*{0.5cm} \\ \thm{ch10P7}{Propriété}{A*Gr}{Soit $(A,+,\times )$ un anneau, si on note $A^*$ l'ensemble des éléments inversible \\
		de $A$ alors $(A,\times )$ est un groupe.} \traitd
		\paragraph{Anneau intègre}
			Soit $(A,+,.)$ un anneau, on dit que \underline{$(A,+,.)$ est intègre} si \[\forall (a,b)\in A^2 ,~ab=0 \Leftrightarrow a=0~ou~b=0 \] 
		\vspace*{-0.7cm} \trait ${}$ \vspace*{-1.4cm} \traitd
		\paragraph{Sous-anneau}
			Soit $A(+,\times)$ un anneau et $B\subset A$ alors $B$ est un sous-anneau de $A$ si $B$ est un sous-groupe de $A$ pour $+$, $1_A\in B$ 
			et $B$ est stable par $\times$. \trait 
		\thm{ch10P8}{Propriété}{SAAnn}{Un sous-anneau est un anneau pour les lois induites.} \\ \traitd
		\paragraph{Morphisme d'anneau}
			Soit $(A_1,+_1,\times_1)$ et $(A_2,+_2,\times_2)$ deux anneaux et $f:A_1\rightarrow A_2$, Alors $f$ est un morphisme d'anneaux si
			\[ \forall (x,y) \in A_1^2 ,~\ard f(x+_1y) = f(x) +_2 f(y) \\ f(x\times_1y) = f(x)\times_2 f(y) \\ f(1_{A_1}) = 1_{A_2} \arf \] 
		\vspace*{-0.7cm} \trait
		\thm{ch10P9}{Propriétés}{PropMorphAnn}{Soit $(A_1,+_1,\times_1)$ et $(A_2,+_2,\times._2)$ deux anneaux et $f:A_1\rightarrow A_2$ un 
		morphisme d'anneaux \\ Alors $\forall a\in A_1^*,~f(a) \in A_2^*$ avec $\big( f(a) \big)^{-1} = f\big( a^{-1} \big)$} \\
	\subsection{Structure de corps}
		\traitd
		\paragraph{Corps}
			Soit $K$ un ensemble muni de deux lois de compositions interne $+$ et $\times$ on dit que $(K,+,\times)$ est un corps si \\
			\hspace*{2.5cm} $\ard \bullet ~(K,+,\times)$ est un anneau commutatif $\\ \bullet ~ $Tout élément de $K$ différent de $0_K$ est 
			inversible ($K^* = K\backslash \{0\}$) $\arf$ \trait \newpage \traitd
		\paragraph{Sous-corps}
			Soit $(K,+,\times)$ un corps et $P\subset K$ alors $P$ est un sous-corps si $P$ est un sous-anneau de $K$, $P^*=P\backslash\{0\}$ et 
			$\forall x\in P^* ,~x^{-1} \in P$ \trait
		\thm{ch10P10}{Propriété}{10-P10}{Soit $(K,+,\times)$ un corps et $P\subset K$, les trois assertions suivantes sont équivalentes : \\
		{\small 1)} $P$ est un sous-corps de $K$ \hspace*{2cm} {\small 2)} $(P,+,\times)$ est un corps \\ {\small 3)} $P$ est un sous-groupe de $K$ 
		pour $+$ et $P\backslash \{0\} $ est un sous groupe de $K^*$ pour $\times$ }
		\vspace*{0.5cm} \\ \thm{ch10P11}{Propriété}{10-P11}{Soit $(K,+_1,\times_1)$ un corps et $(A,+_2,\times_2)$ un anneau \\ Soit 
		$f: K \rightarrow A$ un morphisme d'anneaux, alors $f$ est injectif}
		\vspace*{0.5cm} \\ 
		\begin{center}
		\fin
		\end{center}
    
\chapter{Calcul matriciel et systèmes linéaires}

    
% Chapitre 10 : Calcul matriciel et systèmes linéaires

\minitoc
	\section{Opérations sur les matrices}
		\paragraph{Définition d'une matrice}
			$Soit~(n,p)\in (\mathbb{N}^{*})^{2}$ \\
			Une matrice à $n$ lignes et $p$ colonnes est une application
			\[M\left( \begin{array}{l}
			\llbracket 1,n\rrbracket \times \llbracket 1,p\rrbracket ~\rightarrow\mathbb{K}\\ 
			\hspace*{15pt}(i,j)\mapsto M_{i,j} 
			\end{array} \right) \]
			\[On~note~ \hspace*{40pt}
			M~=~ \left( \begin{array}{ccc}
			m_{1,1} & \cdots & m_{1,p} \\
			m_{2,1} & \ddots & \vdots \\
			\ddots & m_{i,j} & \vdots \\
			m_{n,1} & \dots & m_{n,p}
			\end{array} \right) \]
	\subsection{Somme et Produit matriciel}
		\traitd
	 	\paragraph{Somme}
	 	 	$~~\forall (A,B) \in \mathcal{M}_{n,p}^{2} (\mathbb{K} ),$ \\ \hspace*{2.5cm} $ ~A + B = C ~\Longleftrightarrow  ~
	 	 	\forall (i,j) \in \llbracket 1,n\rrbracket \times \llbracket 1,p\rrbracket , 
	 	 	~~ a_{i,j} + b_{i,j} = c_{i,j} $ \trait
	 	\thm{ch11P1}{Structure}{StructMnpK}{
		$\forall (n,p)\in (\mathbb{N}^*)^{^2} ~~ (\mathcal{M}_{n,p} (\mathbb{K} ) , + )$
	 	est un groupe abélien et $0_{\mathcal{M}_{n,p} (\mathbb{K} )} = 0_{n,p}$} \vspace*{0.3cm}\traitd
		\paragraph{Combinaison linéaire}
	 		Si $l\in\mathbb{N}^* , ~
	 	 	(A_i)_{_{i \in \llbracket 1, l \rrbracket }} \in \mathcal{M}_{n,p}^l
	 	 	(\mathbb{K} ),~(\lambda _i)_{_{i \in \llbracket 1, l \rrbracket }} 
	 	 	\in\mathbb{K}^l$ \\
	 	 	\hspace*{2.5cm}
	 	 	$\sum\limits_{i=1}^l \lambda_i A_i$ est une combinaison linéaire de 
	 	 	$(A_i)_{_{i \in \llbracket 1, l \rrbracket }}$ \trait ${}$ \vspace*{-1.4cm} \traitd
	 	\paragraph{Produit}
	 	 	On peut effectuer le produit matriciel de A et B si A a autant de colonnes 
	 	 	que B a de lignes. Soit donc
	 	 	$ A \in \mathcal{M}_{n,p} (\mathbb{K} ), ~B \in \mathcal{M}_{p,q} (\mathbb{K} ), 
	 	 	~C = A.B \in \mathcal{M}_{n,q} (\mathbb{K} ) $\\ \hspace*{2.5cm}
	 	 	$\forall (i,j) \in \llbracket 1,n\rrbracket \times \llbracket 1,q\rrbracket ~~
	 	 	c_{i,j} = \sum\limits_{k=1}^{p} a_{i,k} b_{k,j} $ \trait
	 	\subparagraph{Note}
	 	 	 Le produit matriciel est bilinaire et associatif.
	 	 	 \[ \forall (A,A') \in\mathcal{M}_{n,p}^2 (\mathbb{K} ),~\forall 
	 	 	 B\in\mathcal{M}_{p,q} (\mathbb{K} ),~\forall (\lambda ,\lambda ')
	 	 	 \in\mathbb{K}^2 ~~~~ (\lambda A + \lambda 'A').B = \lambda AB + \lambda 'A'B\]
	 	 	 \[\forall (A,B,C) \in \mathcal{M}_{n,p} (\mathbb{K} ) \times 
	 	 	 \mathcal{M}_{p,q} (\mathbb{K} ) \times \mathcal{M}_{q,l} (\mathbb{K} )~~~~ 
	 	 	 ((A.B).C) = (A.(B.C)\]
	\subsection{Matrice élémentaire}
		Les matrices élémentaires de $\mathcal{M}_{n,p} (\mathbb{K} )$ sont les matrices 
		$(E_{i,j})_{_{(i,j)\in\llbracket 1,n\rrbracket \times \llbracket 1,p \rrbracket }}$ 
		dont tout le coefficients sont nuls celui en ligne $i$ et colonne $j$ qui vaut 1.
		 \[ \forall (k,l)\in \llbracket 1,n \rrbracket \times \llbracket 1,p \rrbracket ~~~~
		E_{i,j} (k,l) = \delta_{i,k} \delta_{j,l} \]
		\thm{ch11P2}{Propriété}{CombLinEij}{
		Toute matrice de $\mathcal{M}_{n,p} (\mathbb{K} )$ est une combinaison linéaire 
		de matrices élémentaires. \\ De plus, cette décomposition est unique.}
		\begin{proof}
		$A = (a_{i,j}))_{_{(i,j)\in\llbracket 1,n\rrbracket \times 
		\llbracket 1,p \rrbracket }} \in\mathcal{M}_{n,p} (\mathbb{K} )$
		donc $A = \sum\limits_{i=1}^n \sum\limits_{j=1}^p a_{i,j} E_{i,j}$ \\
		et $(\sum\limits_{i=1}^n \sum\limits_{j=1}^p m_{i,j} E_{i,j} )(k,l) = m_{k,l}$
		donc $\sum\limits_{i=1}^n \sum\limits_{j=1}^p a_{i,j} E_{i,j} = 
		\sum\limits_{i=1}^n \sum\limits_{j=1}^p a'_{i,j} E_{i,j}$
		si et seulement si $\forall (i,j)\in\llbracket 1,n\rrbracket \times 
		\llbracket 1,p \rrbracket $ on a $a_{i,j} = a'_{i,j}$
		\end{proof} \newpage \traitd
		\paragraph{Produit de matrices élémentaires}
			Si $(E_{i,j}^{n,p})_{_{(i,j)\in\llbracket 1,n\rrbracket \times \llbracket 1,p 
			\rrbracket }}$ sont les matrices élémentaires de 
			$\mathcal{M}_{n,p} (\mathbb{K} )$
			et $(E_{k,l}^{p,q})_{_{(k,l)\in\llbracket 1,p\rrbracket \times \llbracket 1,p 
			\rrbracket }}$ celles de $\mathcal{M}_{p,q} (\mathbb{K} )$.
			\[E_{i,j}^{n,p} \times E_{k,l}^{p,q} ~=~ \delta_{j,k} E_{i,l}^{n,q}\] \trait
	\subsection{Matrices colonnes}
		Une matrice colonne est $X\in\mathcal{M}_{n,1} (\mathbb{K} )$ soit
		\[X= \left( \begin{array}{l}
		x_1 \\ \hspace*{4pt} \vdots \\ x_n
		\end{array} \right) \]
		\thm{ch11P3}{Produit matriciel $AX$}{ProdMatCol}{
		Si $(A,X)\in\mathcal{M}_{n,p} (\mathbb{K} )\times\mathcal{M}_{p,1} (\mathbb{K} )$
		\\ Alors $AX$ est une combinaison linéaire des colonnes de $A$.}
		\begin{proof}
		\[ A = (C_1 ~~\cdots ~~C_2) = \left( \begin{array}{l}
		a_{1,1} \hspace*{40pt} \\ \hspace*{20pt}\ddots \\ \hspace*{40pt} a_{n,p} 
		\end{array} \right)\]
		\[ A \left( \begin{array}{l} x_1 \\\hspace*{4pt} \vdots \\ x_n 
		\end{array} \right) = (C_1 ~~\cdots ~~C_2) \left( \begin{array}{l} x_1 \\
		\hspace*{4pt} \vdots \\ x_n \end{array} \right) = \left( 
		\begin{array}{l} y_1 \\\hspace*{4pt} \vdots \\ y_n \end{array}\right) = Y\]
		avec $\forall j\in \llbracket 1,n\rrbracket ,~ y_j = \sum\limits_{k=1}^p 
		a_{j,k} x_k $ donc $Y = \sum\limits_{k=1}^p x_k C_k$
		\end{proof}
	\subsection{Matrice transposée}
		\traitd
		\paragraph{Définition}
			Le transposée de $A\in\mathcal{M}_{n,p} (\mathbb{K} )$ notée $A^T$ est la matrice 
			$B\in\mathcal{M}_{p,n} (\mathbb{K} )$ telle que
			\[\forall (i,j) \in \llbracket 1,n\rrbracket \times \llbracket 1,p \rrbracket ,~
			a_{i,j}=b_{j,i} \] \trait
		\thm{ch11P5}{Calculs}{CalcsTranspo}{
		$\rightarrow ~~ \forall (A,B)\in \big(\mathcal{M}_{n,p} (\mathbb{K} )\big)^2,~
		\forall (\lambda ,\mu)\in\mathbb{K}^2 ,~~ (\lambda A+\mu B)^T = \lambda A^T + \mu B^T$ \\
		$\rightarrow ~~ \forall A\in\mathcal{M}_{n,p} (\mathbb{K} ),~
		\forall B\in\mathcal{M}_{p,q} (\mathbb{K} ) ,~~ (AB)^T = B^T A^T $ }
		\begin{proof} ${}$\\
		\hspace*{30pt} $\rightarrow$ Si $\begin{array}{l} A = (a_{i,j}) 
		\hspace*{20pt} A^T = (a'_{i,j})\\ B = (b_{i,j}) \hspace*{20pt}
		B^T = (b'_{i,j}) \end{array} ~~~~
		\forall (i,j)\in \llbracket 1,n\rrbracket \times \llbracket 1,p \rrbracket $
		\[(\lambda A +\mu B)^T(i,j) = (\lambda A +\mu B)(j,i) 
		= \lambda A(j,i) + \mu B(j,i)\]  \[= \lambda (a'_{i,j}) 
		+\mu (b'_{i,j}) = \lambda A^T(i,j) + \mu B^T(i,j)\vspace*{20pt} \]
		\hspace*{30pt} $\rightarrow$ $A = (a_{i,j}),~B = (b_{i,j}),~C = AB = (c_{i,j})
		~~~~\forall (i,j)\in \llbracket 1,n\rrbracket \times \llbracket 1,p \rrbracket $
		\[C^T(i,j) = c_{j,i} = \sum\limits_{k=1}^p a_{k,i} b_{j,k}
		=\sum\limits_{k=1}^p b'_{i,k} a'_{i,j} = B^TA^T(i,j)\]
		\end{proof}
	\section{Opérations élémentaires}
		\traitd Pour $i\neq j$ des indices de lignes ou colonnes on considère les 4 opérations élémentaires suivantes :\\
		${} \hfill L_i \leftrightarrow L_j \hfill L_i \leftarrow L_i + L_j \hfill {} \\
		{} \hfill L_i \leftarrow \lambda L_i (\lambda \neq 0) \hfill L_i \leftarrow L_i + \lambda L_j \hfill {}$ \trait
		\thm{ch11P6}{Proposition}{EquivOpElemMat}{Chacunes des opérations ci-dessus sur les lignes (resp. les colonnes) d'une \\ matrice 
		$A\in \M_{n,p} (\K)$ se traduit par la multiplication à gauche (resp. à droite) \\ par la matrice obtenue en effectuant cette opération sur 
		$I_n$ (resp. $I_p$)}
		\\ \underline{ex} : \\ \hspace*{-0.9cm}
		{\scriptsize $L_i \leftrightarrow L_j ~:~ \begin{blockarray}{cccccccc}
		& C_1 & & C_i & & C_j & & C_p \\ 
		\begin{block}{c(ccccccc)}
		L_1 & 1 & 0 & & \cdots & & & 0 \\ & 0 & \ddots & \ddots & & & & \\ L_i & & \ddots & 0 & & 1 & & \\ & \vdots & & & \ddots & & & \vdots
		\\ L_j & & & 1 & & 0 & & \\ & & & & & & \ddots & 0 \\ L_n & 0 & & & \cdots & & 0 & 1  \\
		\end{block}
		\end{blockarray} ~~~~~~~~ L_i \leftarrow L_i+ \lambda L_j ~:~
		\begin{blockarray}{cccccccc}
		& C_1 & & C_i & & C_j & & C_p \\ 
		\begin{block}{c(ccccccc)}
		L_1 & 1 & 0 & & \cdots & & & 0 \\ & 0 & \ddots & \ddots & & & & \\ L_i & & \ddots & 1 & & \lambda & & \\ & \vdots & & & \ddots & & & \vdots
		\\ L_j & & & \ddots & & 1 & & \\ & & & & & & \ddots & 0 \\ L_n & 0 & & & \cdots & & 0 & 1  \\
		\end{block}
		\end{blockarray} $ } \vspace*{0.5cm} \\
		\thm{ch11L1}{Lemme}{ProdMatDistrib}{$\forall A\in\M_{n,p}(\K ) ,~\forall (B,C) \in \big( \M_{p,q} (\K ) \big)^2$ on a $A(B+C) = AB + AC$ }
	\section{Systèmes linéaires}
		On considère le système linéaire d'inconnues $(x_1 , \dots , x_p ) \in \K^p$ suivant : \\
		\hspace*{2.5cm} $ \big( S \big) ~=~ \left\{ \begin{array}{ccccccc}
		a_{1,1}x_1 & + & \cdots & + & a_{1,p} x_p & = & b_1 \\ \vdots & & & & \vdots & & \vdots \\ a_{n,1}x_1 & + & \cdots & + & a_{n,p}x_p &=&b_n
		\end{array} \right.$
		\\ En notant $A=\big(a_{i,j} \big)_{_{(i,j) \in\ent{1,n}\times\ent{1,p} }} \in \M_{n,p} (\K ) ~; \\ B = \big( b_i \big)_{_{i\in\ent{1,n}}} 
		\in\M_{n,1} (\K ) ~~~;~~~ X = \big( x_j \big)_{_{j\in\ent{1,p}}} \in\M_{p,1} (\K )$ on a \[ \big( S \big) \Leftrightarrow AX=B \] \newpage \traitd
		\paragraph{Système homogène}
			Le système \underline{homogène associé à $\big( S \big)$} est : \\ ${} \hfill \big( S_0 \big) =  \left\{ \begin{array}{ccccccc}
			a_{1,1}x_1 & + & \cdots & + & a_{1,p} x_p & = & 0 \\ \vdots & & & & \vdots & & \vdots \\ a_{n,1}x_1 & + & \cdots & + & a_{n,p}x_p &=&0
			\end{array} \right. ~~\Leftrightarrow ~~ AX=0 \in\M_{n,1} (\K ) \hfill {}$ \trait ${}$ \vspace*{-1.3cm } \traitd
		\paragraph{Système compatible} Un système est dit compatible s'il admet au moins une solution. \trait
		\thm{ch11P7}{Propriété}{CompBCombLinColA}{Si $\big( S \big)$ s'écrit matriciellement $AX=B$ \\ Alors $\big( S \big)$ est compatible si $B$ 
		est une combinaison linéaire des colonnes de $A$.}
		\vspace*{0.5cm} \\ \thm{ch11P8}{Structure le l'ensemble des solutions d'un système compatible}{StructSolSComp}{Les solutions du système 
		compatible $AX=B$ sont les matrices $X_0+Y$ \\ où $X_0$ est une solution particulière du système et $Y$ décrit l'ensemble \\ des solutions 
		du système associé.}
	\section{Anneau des matrices carrées}
		Soit $n\in\N^*$, $\M_n(\K )$ est l'ensemble des matrices carrée d'ordre $n$ \vspace*{0.5cm} \\
		\thm{ch11P9}{Propriété}{MnKAnneau}{$\Big( \M_n (\K ) , + , . \Big)$ est un anneau non commutatif si $n\geq 2$}
		\vspace*{-1.2cm} \\ \textit{Cet anneau n'est pas intègre.}
		\traitd 
		\paragraph{Matrice scalaire} On appelle \underline{matrice scalaire d'ordre $n$} toute matrice de la forme \\ \[ A=\lambda I_n = 
			\begin{blockarray}{(ccc)} \lambda & & (0) \\ & \ddots & \\ (0) & & \lambda \end{blockarray} ~~\lambda\in\K \] \trait ${}$ \vspace*{-1.2cm} \traitd
		\paragraph{Matrice symétrique} On appelle \underline{matrice symétrique d'ordre $n$} toute matrice $A \in\M_n (\K )$ telle que $A^T=A$ et 
			on note $\mathcal{S}_n (\K )$ l'ensemble des matrices symétriques d'ordre $n$ \trait ${}$ \vspace*{-1.3cm} \traitd
		\paragraph{Matrice antisymétrique} On appelle \underline{matrice antisymétrique d'ordre $n$} toute matrice $A\in\M_n(\K )$ telle que 
			$A^T=-A$ et on note $\mathcal{A}_n (\K )$ l'ensemble des matrices antisymétriques d'ordre $n$ \trait
		\thm{ch11P10}{Propriété}{A=U+V}{$\forall	 A\in\M_n(\K )$, $\exists (U,V)\in\mathcal{S}_n(\K) \times \mathcal{A}_n(\K)$ unique tel que 
		$A=U+V$}
		\newpage ${}$ \\ \thm{ch11P11}{Proposition}{IdRemarq}{$\forall (A,B) \in \M_n^2 (\K)$ tel que $A.B=B.A$ Alors on a \\
		{\footnotesize 1)} $\forall p\in\N ,~ A^p-B^p = (A-B) \sk{1}{p-1} A^kB^{p-k}$ \\{\footnotesize 2)} $\forall p\in\N ,~
		(A+B)^p=\sk{0}{p} \binom{p}{k} A^kB^{p-k}$ }
		\vspace*{0.5cm} \\ \thm{ch11L2}{Lemme}{MScalCommut}{Les matrices scalaires commuttent avec toutes les matrices.} \traitd
		\paragraph{Matrice diagonale} $A\in\M_n(\K)$ est dite diagonale si $\forall (i,j) \in \ent{1,n}^2 ,~i\neq j\Rightarrow A(i,j)=0$ \trait
		\thm{ch11P12}{Propriété}{DiagoStable}{Le produit de deux matrices diagonales d'ordre $n$ est une matrice diagonale \\ d'ordre $n$ ;  
		en particulier, \\si $A\in\M_n(\K) ,~A = \begin{blockarray}{(ccc)} d_1 & & (0) \\ & \ddots & \\ (0) & & d_n \end{blockarray} ~; ~\forall 
		p\in\N ,~A^p = \begin{blockarray}{(ccc)} d_1^{~p} & & (0) \\ & \ddots & \\ (0) & & d_n{~p} \end{blockarray}$ } \traitd
		\paragraph{Matrice triangulaire} On dit que $A\in\M_n(\K)$ est triangulaire supérieure (resp. inférieure) \\
			si $\forall (i,j)\in\ent{1,n}^2,~ i>j \Rightarrow A(i,j) = 0$ \big(resp. $i<j \Rightarrow A(i,j) =0 $\big) \\
			On note $\mathcal{T}_n^+(\K)$ \big(resp. $\mathcal{T}_n^-(\K)$\big) l'ensemble des matrices triangulaires supérieures 
			(resp. inférieures) \trait
		\thm{ch11P13}{Propriété}{MTriStable}{$\forall n\in\N^* ,~\mathcal{T}_n^+(\K)$ et $\mathcal{T}_n^-(\K)$ sont stable par produit matriciel.}
		\\ \traitd
		\paragraph{Matrice inversible} On dit que $A\in\M_n(\K)$ est inversible s'il existe $B\in\M_n(\K)$ telle que $A.B=B.A=I_n$ et on note 
			$\mathcal{GL}_n(\K)$ le groupe des matrices inversibles d'ordre $n$. \trait
		\thm{ch11P14}{Propriété}{TranspoInversible}{$\forall A\in\M_n(\K) ,~A\in\mathcal{GL}_n(\K) \Rightarrow A^T\in\mathcal{GL}_n(\K)$ et 
		$\big( A^T\big)^{-1} = \big( A^{-1} \big)^T$}
		\vspace*{0.5cm} \\ \thm{ch11P15}{Propriété}{MOpéElemInv}{Les matrices correspondantes aux opérations élémentaires sont inversibles. \\
		$\forall n\in\N^*,~\forall (i,j)\in\ent{1,n}^2 ~;~{\footnotesize \ard P_n(i,j) = I_n -E_{i,i}-E_{j,j} +E_{i,j} + E_{j,i} \\ 
		T_{i,j}(\lambda ) = I_n + \lambda E_{i,j} \\ D_n(\lambda ) = I_n + (\lambda -1)E_{i,i} \arf }~~ \in\mathcal{GL}_n(\K)$}
		\vspace*{0.5cm} \\ \thm{ch11P15c}{Corollaire}{OpéElemPresInv}{Les opérations élémentaires préservent l'inversibilité.}
		\vspace*{0.5cm} \\ \thm{ch11th1}{Théorème}{MTriInvCNSnd}{Une matrice triangulaire est inversible \underline{si et seulement si} tout ses 
		coefficients \\ diagonaux sont non nuls.}
		\begin{proof}
		$\rightarrow$ voir Chapitre 15 %(\ref{MatTriInvCNS})
		\end{proof}
		${}$ \\ \thm{ch11P16}{Propriété}{11-P16}{Soit $A\in\M_n(\K)$, alors \\
		$A\in\mathcal{GL}_n(\K) ~\Leftrightarrow ~\Big( \forall X\in\M_{n,1}(\K),~(AX=O \Rightarrow X=0 ) \Big)$}
		\vspace*{0.5cm} \\ \thm{ch11P16c}{Corollaire}{MDiagInvCNS}{Une matrice $A\in\M_n(\K)$ diagonale est inversible \underline{si et seulement 
		si} ses \\ coefficients diagonaux sont tous non nuls. Son inverse est alors la matrice \\ diagonale des inverses des coefficients 
		diagonnaux de $A$.}
		\vspace*{0.5cm} \\ \thm{ch11P17}{Propriété}{InvMatTriTri}{Si une matrice triangulaire supérieure (resp. inférieure) est inversible \\
		alors son inverse est une matrice triangulaire supérieure (resp. inférieure)}
		\vspace*{0.5cm} \\ 
		\begin{center}
		\fin
		\end{center}
        
\chapter{Polynômes et fractions rationnelles}

    
% Chapitre 11 : Polynômes et fractions rationnelles

\minitoc
	\section{Anneau des polynômes à 1 indéterminée}
		\paragraph{Anneau des polynômes ($\mathbb{K} [X], +, \times$)}
			Soit $\mathbb{K}^{(\mathbb{N} )}$ l'ensemble des suites à \\ valeurs dans 
			$\mathbb{K}$ sationnaires nulles.
			\[\forall u\in\mathbb{K}^{(\mathbb{N} )} , \exists n_{0} \in \mathbb{N} : 
			\forall n\in\mathbb{N} , (n\leq n_{0} \Rightarrow u_{n}=0)\]
			On définit une addition et une multiplication :
			\[ \forall n\in\mathbb{N} , (u+v)(n) = u_{n}+v_{n} \]
			\[\forall n\in\mathbb{N} , (uv)(n) = \sum\limits_{k=0}^{n} u(k)v(n-k) \]
			On considère la suite $X=(0,1,0,0,...)$ et on a alors $X^{n} =(0,...,0,1,0,...)$
			et $X^{0} = 1_{\mathbb{K}} =(1,0,0,...)$ \hspace*{20pt}
			On peut noter $\mathbb{K}^{(\mathbb{N} )})$ comme $\mathbb{K} [X]$.\\
			$P=(a_{0} ,~a_{1} ,...,~a_{n} ,~0,...) = \sum\limits_{k=0}^{n} a_{k} X^{k} 
			~~\in\mathbb{K} [X]$ \\
			($\mathbb{K}^{(\mathbb{N} )} [X] ,~+,~\times$) est l'anneau des polynômes.
			\subparagraph{Utilisation}
			 $(1+X)^{n+m} = (1+X^{n} (1+X)^{m} 
			 $\\
			 $\sum\limits_{l=0}^{n+m} \binom{n+m}{l} X^{l} = \sum\limits_{k=0}^{n} 
			 \binom{n}{k} X^{k} = \sum\limits_{j=0}^{m} \binom{m}{j} X^{j} \hspace*{10pt}$ 
			 Qui donne \hspace*{10pt} 
			 $$\binom{n+m}{l} = \sum\limits_{k=0}^{l} \binom{n}{k} \binom{m}{l-k} $$
		\subsection{Degré d'un polynôme}
			Si $P\in\mathbb{K} [X]~,~P\neq 0$ on appelle degré de P noté deg(P) ou d°(P) :
			\[ deg(P) = max\{ n\in\mathbb{N} \vert a_{n} \neq 0\} ~~
			(P=\sum\limits_{k=0}^{+\infty } a_{k} X^{k} )\]
			avec par convention le polynôme nul de degré $-\infty$.
			\paragraph{Degré de la somme et du produit}
			 Soit $(P,Q) \in (\mathbb{K} [X] )^{2}$\\
			 \[  deg (P+Q) \leq max\{ deg(P),deg(Q)\} \] égalité si $deg(P)\neq deg(Q)$\\
			 \[deg(PQ) = deg(P) + deg(Q)\]
			\paragraph{Ensemble}
			 Si $n\in\mathbb{N}$,
			 $\mathbb{K}_{n} [X]$ est l'ensemble des polynômes de degré au plus n.
			 \[ \mathbb{K}_{n} [X] = \{P\in\mathbb{K} [X] ~\vert ~ deg(P) \leq n\}\]
			 \subparagraph{Remarque}
			  $\mathbb{K}_{n} [X] $ est stable par combinaison linéaire
			  \[\forall (\lambda ,\mu )\in \mathbb{K}^{2} , ~\forall (P,Q)\in 
			  (\mathbb{K}_{n} [X])^{2} \hspace*{20pt} \lambda P+\mu Q \in\mathbb{K}_{n} [X]\]
			\paragraph{Intégrité de l'anneau ($\mathbb{K} [X],+,\times$)}
			 $\forall (P,Q) \in (\mathbb{K} [X] )^{2} $
			 \[ PQ = 0 ~\begin{array}{l}
			 ~\Leftrightarrow ~~deg(PQ) = -\infty \\
			 ~\Leftrightarrow ~~deg(P) + deg(Q) = -\infty \\
			 ~\Leftrightarrow ~~deg(P) = -\infty ~ou~ deg(Q) = -\infty \\
			 ~\Leftrightarrow ~~P=0 ~ou~ Q=0  
			 \end{array} \]	
			
		\subsection{Composition de polynômes}
			Si $(P,Q) \in (\mathbb{K} [X] )^{2}$ avec $P=\sum\limits_{k=0}^{+\infty} a_{k} X^{k}$
			\[P\circ Q ~=~ \sum\limits_{k=0}^{+\infty} a_{k} Q^{k} \]
			\paragraph{Degré du polynôme composé}
			 Si $(P,Q)\in \mathbb{K} [X] \times \mathbb{K} [X]\backslash \mathbb{K}_{0} [X]$
			 \[deg(P\circ Q) = deg(P) \times deg(Q)\]
			\paragraph{Coefficient dominant}
				Si $P=\sum\limits_{k=0}^{+\infty} a_k X^k \in \mathbb{K} [X] \backslash \{0\}$\\
				$a_{degP}$ s'appelle le coefficient dominant de $P$. 
				Si il vaut $1$ $P$ est dit unitaire.
	\section{Divisibilité et Division Euclidienne}
		\subsection{Divisibilité des polynômes}
			Si $(A,B) \in (\mathbb{K} [X])^{2}$\\
			On dit que A	 divise B si il existe $Q \in\mathbb{K} [X]$ tel que $B=AQ$\\
			On note alors $A\vert B$ (sinon $A\not\vert B$)
			\paragraph{Propriété}
				Soit $A\in\mathbb{K} [X] \backslash\{0\}$ et $B \in\mathbb{K} [X]$
				$A\vert B$ $\Rightarrow$ degA $\leq$ degB
				\subparagraph{Preuve}
				 $A = BQ \Rightarrow$ degA = degB + degQ $\geq$ degB
		\subsection{Polynômes associés}
			$(A,B) \in (\KX )^2$ est un couple de polynômes associés si
			\[A\vert B \hspace*{15pt} B\vert A\]
			\paragraph{$\rightarrow$}
				$(A,B)$ est un couple de polynômes associés si et seulement si
				\[\exists\lambda\in\mathbb{K}* ~:~A=\lambda B\]
		\subsection{Division euclidienne polynômiale}
			Si $B \neq 0$ , $B = \sum\limits_{k=0}^m b_k X^k$ avec $b_m\neq 0$ \vspace*{10pt}\\
			\[Si~ A\in\mathbb{K} [X] \backslash \mathbb{K}_{m-1} [X] ~~il~existe~(Q_0,R_0)
			\in (\mathbb{K} [X])^2~:~A=BQ_0+R_0 ~et~degR_0 < degA\]
			(Si $A=\sum\limits_{k=0}^{n} a_{k} X^{k}$ il suffit de considérer 
			$Q_0 =\frac{a_n}{b_m} X^{n-m}$)
			\newtheorem*{th10}{Théorème de la division euclidienne polynômiale}
			\begin{th10}\label{Th DEucl Pol} 
				${}$\\Si $B\in \mathbb{K} [X] \backslash \{0\}$ alors pou tout 
				$A\in\mathbb{K} [X]$
				\[\exists (Q,R) \in (\mathbb{K} [X])^2 ~:~ \left|\begin{array}{l}
				A=BQ+R\\ degR < deg B
				\end{array}\right. \] 
			De plus $Q$ et $R$ sont uniques appelés quotient et reste de la division 
			euclidienne de $A$ par $B$.
			\end{th10}
			\begin{proof}
				\underline{Existence} \medskip Récurence sur degA \\
				\hspace*{20pt} \textit{Initialisation} \hspace{20pt} Si degA < degB
				\[A=B\times 0 + A = BQ + R \]
	
				\hspace{20pt} \textit{Hérédité} \hspace{20pt} On suppose la propriété vraie 
				pour tout polynôme de degré $k<n$ avec $n\geq$ degB. 
				D'après la remarque préliminaire on a :
				\[\exists (Q_0,R_0) \in (\mathbb{K} [X])^2 ~:~A=BQ_0 +R_0 ~~~~ degR_0<degA=n\]
				\begin{center} d'après l'hyspothèse de récurrence $\exists (Q_1,R_1) 
				\in (\mathbb{K} [X])^2$ \\
				\hspace*{10pt} $R_0 =BQ_1+R_1$ avec $degR_1<degB$ soit 
				\[A=B(Q_0 + Q_1) + R_1\] \end{center}
				\hspace*{40pt} \underline{Unicité} \\
				\hspace*{20pt} Supposons $A=BQ_1 +R_1 = BQ_2 +R_2$ avec $degR_1,degR_2<degB$\\
				alors $B(Q_1 - Q_2) = R_1 - R_2$ donc
				\[degB + deg(Q_1 -Q_2) = deg(R_1 - R_2)~\leq~max\{degR_1,degR_2\} < degB\]
				d'où $deg(Q_1 -Q_2) = -\infty$ soit $Q_1=Q_2$ puis $R_1=R_2$
			\end{proof}
	\section{Fonctions polynômiales et racines}
		\subsection{Fonction polynômiale associée}
			À tout polynôme $P = \sum\limits_{k=0}^n a_k X^k \in \mathbb{K} [X]$ on peut associer 
			la fonction polynômiale 
			\[\widetilde{P} \left( \begin{array}{l}
			\hspace*{10pt}\mathbb{K} \longrightarrow \mathbb{K} \\
			x \mapsto \sum\limits_{k=0}^n a_k x^k \end{array} \right)\]
			\paragraph{Calculs}
				$\forall (P,Q)\in (\mathbb{K} [X])^2 ~~ \forall (\lambda ,\mu )\in\mathbb{K}^2$
				\[\widetilde{\lambda P+\mu Q} = \lambda\widetilde{P} +\mu\widetilde{Q}\]
				\[\widetilde{PQ} = \widetilde{P}\widetilde{Q} \hspace*{40pt} \widetilde{P\circ Q} 
				= \widetilde{P}\circ\widetilde{Q}\]
		\subsection{Racines du polynôme}
			$a\in\mathbb{K}$ est une racine de $P\in\mathbb{K} [X]$ si \[\widetilde{P}(a) = 0\]
			On notera ensuite $\mathcal{Z}(P)$ l'ensemble des racines (ou zéros) de $P$.
			\newtheorem{th11}{Divisibilité par $(X-a)$}
			\begin{th11}\label{Div X-a}
				$\forall P\in\mathbb{K}[X] ~~~~ \forall (a_1,~\cdots ~,a_n)\in\mathbb{K}^n$ 
				distincts \[\{a_1,~\cdots ~,a_n\} \subset \mathcal{Z}(P) ~\Leftrightarrow ~
				\prod\limits_{i=1}^n (X-a_i)\vert P\]
			\end{th11}
			\begin{proof}
			Récurrence sur n : P(n)(\ref{Div X-a})\\
			\underline{Initialisation} \hspace*{10pt}
			$(x-a)\vert P$ si et seulement si $\exists Q : P=(X-a)Q$ alors
			\[\widetilde{P}(a) = \widetilde{(X-a)}(a)\widetilde{Q}(a) = 0\]
			Si $a\in\mathcal{Z}(P)$ et $P=\sum\limits_{k=0}^n \alpha_k X^k$
			\[P = P-P(a) = \sum\limits_{k=0}^n \alpha_k X^k - \sum\limits_{k=0}^n \alpha_k a^k 
			= \sum\limits_{k=0}^n \alpha_k (X^k - a^k) \]
			\[= (X-a)\sum\limits_{k=0}^n \alpha_k \sum\limits_{l=0}^{k-1} a^{k-1-l} X^l = 
			(X-a)Q\vspace*{20pt}\]
			\underline{Hérédité} \hspace*{10pt} Supposons P(n) et considérons
			$\{a_1, ~\cdots ~,a_n,a_{n+1}\}\in\mathbb{K}^{n+1}$ distincts \\
			Par l'hypothèse de récurrence on a 
			\[\exists Q\in\mathbb{K}[X] ~:~ P=(\prod\limits_{i=1}^n (X-a_i))Q\]
			\[ a_{n+1} \in\mathcal{Z}(P) ~\Leftrightarrow ~\widetilde{P}(a_{n+1}) = 0
			\Leftrightarrow ~(\prod\limits_{i=1}^n (a_n+1-a_i))\widetilde{Q}(a_{n+1})=0\]
			\[ \Leftrightarrow ~a_{n+1} \in\mathcal{Z}(Q) ~\Leftrightarrow ~X-a_{n+1}\vert Q\]
			\end{proof}
			\paragraph{Nombre de racines}
				Le nombres de racines d'un polynôme \underline{non nul} est majoré par son degré.
				\subparagraph{dem.}
				 Par récurrence si $deg(\prod\limits_{i=1}^n (X-a_i)) = n$ et $P\neq O$
				 \[\prod\limits_{i=1}^n (X-a_i)\vert P ~\Longrightarrow ~n\leq P\]
			\paragraph{Corollaire : Caractérisation du polynôme nul}
				Le seul polynôme admettant une infinité de racines ou $n+1$ racines est le 
				polynôme nul.
				\subparagraph{appli.}
				 Soit $E=\{P\in\mathbb{K}[X]\vert\exists T\in\mathbb{K}* : \forall x\in\mathbb{K}
				 ,~\widetilde{P}(x+T) = \widetilde{P}(x)\}$, déterminons $E$.\\
				 $\mathbb{K}_0[X]\subset E$\\
				 Réciproquement si $\P\in E ~T-p\' eriodique~(T\neq 0)$ et $Q=P-\widetilde{P}(0)$ 
				 on a $T\mathbb{Z} \subset\mathcal{Z}(P)$ d'où $P-\widetilde{P}(0) = 0$ donc
				 $P= \widetilde{P}(0) \in\mathbb{K}_0[X] $\\
				 En conclusion on a $E = \mathbb{K}_0[X]$.
		\subsection{Ordre de multiplicité}
			$\forall P\in\KX $ \\
			Si $a\in\mathbb{K}$, $k\in\mathbb{N}$ et $(X-a)^n\vert P$ on dit que $a$ est une 
			racine de $P$ d'ordre de multiplicité au moins $n$.\\
			Si de plus $(X-a)\not\vert P$ alors $a$ est une racine de $P$ d'ordre de multiplicité 
			exactement $n$.
			\paragraph{$\rightarrow$}
			$a$ est une racine de $P$ d'ordre de multiplicité k si et seulement si 
			$\exists Q\in\KX$ tel que 
			\[P = (X-a)^k Q ~~et~~ a\not\in \mathcal{Z}(P)\]
		\subsection{Méthode de Horner pour l'évaluation polynômiale} 
			${}$\\ Soit $\Psn$ et $x_0\in\mathbb{K}$ on veut déterminer 
			$\widetilde{P}(x_0)$. \\On considère la suite
			$\left\{\ar* u_0 = a_n \\ u_{k+1} = u_kx_0+a_{n+1-k} \ar\right.$
			\[u_k = a_nx_0^k + ~\cdots ~+a_{n_k} ~~et~~u_n = \widetilde{P}x_0\]
		\subsection{Polynôme scindé}
			Un polynôme $P\in\KX$ est \underline{scindé} s'il peut s'écrire comme produit de 
			polynômes de degré 1.\\
			\thm{th12}{Formule de Viete}{fViete}{${}$\\ \underline{Relations entre les coefficients et les racines d'un polynôme scindé}
				\\Soit $P$ un polynôme scindé, $P = \Psn = \lambda\prod\limits_{k=1}^n (X-x_k)$
				\\ $\left.\ar* n\geq 2 \\ a_n \neq 0\ar\right| \Leftrightarrow \left\{\ar* 
				a_n = \lambda \\ a_{n-l} = \lambda\prod\limits_{1\leq i_1\leq 
				\cdots\leq i_l\leq n} \sum\limits_{r=1}^l (-x_{i_r}) \ar\right. $
				\\ $ \Leftrightarrow \left\{\ard a_n = \lambda \\ \prod\limits_{1\leq i_1\leq 
				\cdots\leq i_l\leq n} \sum\limits_{r=1}^l x_{i_r} = \frac{(-1)^l a_{n-l}}{a_n}
				\arf \right. $ }
			
			\begin{proof}
			Faire arbre
			\end{proof}
	\section{Dérivation}
		Si $P=\Ps \in\KX$ on appelle polynôme dérivé de P
		\[P' = \sum\limits_{k=1}^{+\infty}ka_kX^{k-1} = 
		\sum\limits_{k=0}^{+\infty}(k+1)a_{k+1}X^k\]
		puis par récurrence avec $\forall n\in\mathbb{N} ~~P^{(n++1)} = (P^{(n)})'$
		$P^{(n)} = \sum\limits_{k=n}^{+\infty} k(k-1)\cdots (k-n+1)a_kX^{k-n}
		= \sum\limits_{k=0}^{+\infty} (k+n)(k+n-1)\cdots (k+1) a_{k+n}X^k 
		 = \sum\limits_{k=n}^{+\infty}\frac{k!}{(k-n)!} a_kX^{k-n}
		= \sum\limits_{k=0}^{+\infty} \frac{(k+n)!}{k!} a_{k+n}X^k$
			\hspace*{40pt} \underline{\textbf{Calcul :}}
			$\forall (P,Q)\in (\KX )^2$ \\
			\[\rightarrow ~~ \forall (\lambda ,\mu )\in\mathbb{K}^2 ,~(\lambda P+\mu Q)'
			= \lambda P'+\mu Q'\]
			\[\rightarrow ~~ (PQ)' = P'Q + PQ' \hspace*{40pt} \rightarrow ~~(P\circ Q)'
			= P'\circ Q \times Q\]
		\begin{proof}
		$P = \Ps ~~Q =\sum\limits_{k=0}^{+\infty} b_kX^k$ \\
		$\rightarrow ~~ (\lambda P+\mu Q)' = (\sum\limits_{k=0}^{+\infty} 
		(\lambda a_k +\mu b_k)X^k)' = \sum\limits_{k=1}^{+\infty} 
		k(\lambda a_k +\mu b_k)X^{k-1} \\ = \lambda 
		(\sum\limits_{k=1}^{+\infty} ka_k X^{k-1} ) +\mu 
		(\sum\limits_{k=1}^{+\infty} kb_k X^{k-1} ) = \lambda P' +\mu Q'$\\
		$\rightarrow ~~ PQ = \sum\limits_{k=0}^{+\infty} c_k X^k ~~~~c_k = 
		\sum\limits_{l=0}^k a_l b_{k-l} \hspace*{30pt} donc~~~~
		(PQ)' = \sum\limits_{k=0}^{+\infty} (k+1)c_{k+1} X^k \\ avec \hspace*{40pt} P' = 
		\sum\limits_{k=0}^{+\infty}(k+1)a_{k+1} X^k ~~~~et~~~~
		Q' = \sum\limits_{k=0}^{+\infty} (k+1)b_{k+1} X^k \\
		PQ' = \sum\limits_{k=0}^{+\infty} d_k X^k ~~~~ d_k = \sum\limits_{l=0}^k
		a_l (k+1-l)b_{k+1-l} \\ P'Q = \sum\limits_{k=0}^{+\infty} \delta_k X^k  
		\delta_k = \sum\limits_{l=0}^k (l+1)a_{l+1} b_{k-l} \\
		d_k +\delta_k = \sum\limits_{l=0}^k a_l (k+1-l)b_{k+1-l} + \sum\limits_{l=0}^k 
		(l+1)a_{l+1}b_{k-l} \\ 
		= a_0 (k+1)b_{k+1} + (k+1)a_{k+1}b_0 + \sum\limits_{l=1}^k 
		a_l (k+1-l)b_{k+1-l} +\sum\limits_{l=1}^k la_l b_{k+1-l} \\
		= (k+1)\sum\limits_{l=0}^{k+1} a_lb_{k+1} = (k+1)c_{k+1}$
		\end{proof}

\chapter{Analyse asymptotique}

    
% Chapitre 12 : Analyse asymptotique

% TODO
        
\chapter{Espaces vectoriels et applications linéaires}

    
% Chapitre 13 : Espaces vectoriels et applications linéaires

% TODO

\chapter{Matrices II}

    
% Chapitre 14 : Matrices 2

\minitoc
	\section{Matrices et applications linéaires}
	\subsection{Matrice d'une application linéaire dans des bases}
		\traitd
		\paragraph{Matrice représenntative d'un vecteur}
			Soit $E$ un $\K$ espace vectoriel de dimension finie et $B=(e_1,\dots ,e_n)$ une base de $E$.\\
			On considère $x=\sum_{i=1}^n x_ie_i \in E$. La \underline{matrice représentative de $x$ dans la base $B$} \\ est la matrice colonne 
			$X=\left( \ard x_1 \\ \vdots \\ x_n \arf \right) =~ $\highlight{$Mat_B(x)$}$ ~\in\M_n(\K)$ \trait \newpage \traitd
		\paragraph{Matrice représentative d'une famille}
			Soit $E$ un $\K$ espace vectoriel de dimension finie et $B=(e_1,\dots ,e_n)$ une base de $E$.\\
			On considère $(x_1, \dots ,x_p)$ une famille de $p$ vecteurs de $E$. La \underline{matrice représentative de cette famille}\\ 
			\underline{dans cette base} est la matrice de $\M_{n,p}(\K)$ notée $Mat_B(x_1,\dots ,x_p)$ dont la $j^{\mathrm{e}}$ colonne est donnée 
			par $Mat_B(x_j)$, $\forall j\in\ent{1,p}$ \trait ${}$ \vspace*{-1.3cm} \traitd
		\paragraph{Matrice représentative d'une application linéaire}
			Soit $E$ et $F$ deux $\K$ espaces vectoriels de dimensions finies respectives $p$ et $n$ avec $\ard e=(e_1,\dots ,e_p)$ une base de $E 
			\\ f=(f_1,\dots ,f_n)$ une base de $F \arf $ \\
			On considère $u\in\lin(E,F)$. La \underline{matrice représentative de $u$ dans les bases $e$ et $f$} est la matrice de $\M_{n,p}(\K)$ 
			notée $Mat_{e,f}(u)$ définie par \highlight{ $Mat_{e,f}(u) = Mat_f\big( u(e_1),\dots ,u(e_p) \big)$ }\trait 
		\vspace*{-1.35cm} \\ \textit{Si $u\in\lin(E)$ on note $Mat_e(u) = Mat_{e,e}(u) = Mat_e\big( u(e_1,\dots ,u(e_n)\big)$}
		\vspace*{0.5cm} \\ \thm{ch15P1}{Proposition}{MatIsoEV}{Si $E$ et $F$ sont des $\K$-ev de dimensions $p$ et $n$ 
		rapportés à des bases $e$ et $f$, alors\\
		 $\Phi ~~ \begin{blockarray}[t]{(ccc)} \lin(E,F) & \longrightarrow & \M_{n,p}(\K) \\ u & \mapsto & Mat_{e,f}(u) \end{blockarray}$ est un 
		isomorphisme d'espace vectoriel.}
		\vspace*{0.5cm} \\ \thm{ch15P1c}{Corollaire}{IsoInduitBase}{Le choix d'une base $B$ sur $E$ induit un iomorphisme de $\lin(E)$ sur 
		$\M_n(\K)$ : \\ \hspace*{1.5cm} $\appli{\lin(E)}{u}{\M_n(\K)}{Mat_B(u)}$ }
		\vspace*{0.5cm} \\ \thm{ch15P2}{Proposition}{ApLinMat}{Soit $E,F$ deux $\K$-ev de dimensions $p$ et $n$ rapportés à des 
		bases $e$ et $f$ \\ Soit $u\in\lin(E,F) ~;~x\in E$, on considère $y=u(x) ~\in F$ et on note \\$X=Mat_e(x) ~; ~Y=Mat_f(y)~;~A=Mat_{e,f}(u)$ 
		Alors $Y=AX$}
		\vspace*{0.5cm} \\ \thm{ch15P3}{Proposition}{MatComposee}{$E$ de dimension $p$ et $e=(e_1,\dots ,e_p)$ une base de $E$. \\ 
		$F$ de dimension $q$ et $f=(f_1,\dots ,f_q)$ une base de $F$. \\ $G$ de dimension $n$ et $g=(g_1,\dots ,g_n)$ une base de $G$.\\
		Soit $u\in\lin(E,F) ,~v\in\lin(F,G) ~;~ A=Mat_{e,f}(u) ,~B=Mat_{f,g}(v)$ \\ Alors $C=Mat_{e,g}(v\circ u) = AB$}
		\vspace*{0.5cm} \\ \thm{ch15th1}{Théorème}{EndoInvMatInv}{Soit $E$ et $F$ deux $\K$-ev de dimension finie $n$ rapportés à des bases $e$ et 
		$f$ \\Soit $u\in\lin(E,F)$ on a $~~(u$ est un isomorphisme$)~\Leftrightarrow ~(Mat_{e,f}(u)$ est inversible$)$ \\ Dans ce cas on a 
		$\big( Mat_{e,f}(u) \big)^{-1} = Mat_{e,f} \big( u^{-1} \big)$} \newpage
	\subsection{Application linéaire canoniquement associée}
		\traitd
		\paragraph{Définition}
			Si $A\in\M_{n,p}(\K)$ on appelle \underline{Application linéaire canoniquement associée à $A$} l'unique application linéaire, notée 
			$u_A$ telle que \highlight{$Mat_{C(\K^p) , C(\K^n)}(u_A) = A$} \trait ${}$ \vspace*{-1.3cm} \traitd
		\paragraph{Noyau, image et rang}
			Si $A\in\M_{n,p}(\K)$ on appelle \\ \hspace*{2cm} $\bullet$ \underline{noyau de $A$} noté $Ker(A)$ défini par $Ker(A)=Ker(u_A)$ \\ 
			\hspace*{2cm} $\bullet$ \underline{image de $A$} notée $Im(A)$ définie par $Im(A)=Im(u_A)$ \\ \hspace*{2cm} $\bullet$ 
			\underline{rang de $A$} noté $rg(A)$ défini par $rg(A)=rg(u_A)$ \trait
		\thm{ch15P4}{Propriété}{ImKerColLign}{Les colonnes de $A$ engendre $Im(A)$ et ses lignes donnent un système \\d'équation de $Ker(A)$}
		\vspace*{0.5cm} \\ \thm{ch15P5}{Proposition}{AInvCNS}{Soit $A\in\M_n(\K)$ alors $~~~~~~A\in\GL_n(\K) ~\Leftrightarrow $\\$ Ker(A)= \{ 0 \} 
		~\Leftrightarrow ~ \K^n=Vect \big( C_1(A),\dots ,C_n(A) \big) ~\Leftrightarrow ~rg(A)=n $ }
		\vspace*{0.5cm} \\ \thm{ch15P5c}{Corollaire}{MatTriInvCNS}{Une matrice triangulaire est inversible \underline{si et seulement si} ses 
		coefficients \\ diagonnaux sont tous non nuls.}
		\begin{proof}
		Soit $A\in\mathcal{T}^+(\K)$ \\ \fbox{$\Leftarrow$} Si les coefficients $(a_jj)_{_{1\leq j\leq n}}$ sont tous non nuls alors 
		$(C_1,\dots ,C_n)$ est une famille libre donc une base de $\K^n$ d'où $A\in\GL_n\K$\\
		\fbox{$\Rightarrow$} Par contraposée si $\exists k_0 \in\ent{1,n}$ tel que $a_{k_0,k_0} = 0$ alors $\mathrm{dim}\big( 
		Vect(C_1,\dots ,C_{k_0} )\big) \leq k_0-1$ donc $\mathrm{dim}A \leq n-1$ d'où $A\notin \GL_n(\K)$
		\end{proof}
		${}$ \\ \thm{ch15P6}{Propriété}{15-P6}{Si $E$ est un $\K$-ev de dimension $n$ rapporté à une base $B$\\ Soit $(x_1,\dots ,x_p)$ une famille 
		de $p$ vecteurs de $E$ \\ Alors $rg(x_1,\dots ,x_p) = \mathrm{dim}\Big( Vect\big( Mat_B(x_1) ,\dots , Mat_B(x_p) \big)\Big) $\\$ = 
		\mathrm{dim}\Big(Im \big(X_1 \cdots X_p \big)\Big) = rg(u_A) ~$ Où $A=\big( X_1 ~X_2 \cdots X_p \big) ~\in\M_{n,p}(\K)$}
		\vspace*{0.5cm} \\ \thm{ch15P7}{Propriété}{InvDG}{Une matrice $A\in\M_n(\K)$ inversible à gauche ou à droite est inversible.}
	\subsection{Systèmes linéaires}
		$ \big( S \big) ~=~ \left\{ \begin{array}{ccccccc}
		a_{1,1}x_1 & + & \cdots & + & a_{1,p} x_p & = & b_1 \\ \vdots & & & & \vdots & & \vdots \\ a_{n,1}x_1 & + & \cdots & + & a_{n,p}x_p &=&b_n
		\end{array} \right. ~~\Leftrightarrow ~~ A \times ~\begin{blockarray}[t]{(c)} x_1 \\ \vdots \\ x_p \end{blockarray} = 
		\begin{blockarray}[t]{(c)} b_1 \\ \vdots \\ b_n \end{blockarray}$
		\newpage \textit{Résoudre le système homogène associé à $\big( S \big)$ c'est déterminer le noyau de $A$\\
		Par le théorème du rang, la dimension de l'espace des solutions du système homogène est donnée par $p-rg(A) ~(\geq p-n )$}
		\vspace*{0.2cm} \\ L'ensemble des solution de $\big( S \big)$ à une structure de sous-espace affine de $\K^p$ 
		\underline{si il est compatible}, soit si $X_0$ est une solution particulière \begin{center}
		\highlight{ $\mathcal{S} = X_0 + Ker(A) ~\subset \K^p$ } \end{center}
		\traitd
		\paragraph{Système de \textsc{Cramer}} Si $A\in\GL_n(\K)$ alors le systèle $\big( S\big)$ est compatible et admet une unique solution 
		$A^{-1} \times B$. \trait
	\section{Changement de bases}
		\traitd
		\paragraph{Matrice de passage}
			On appelle \underline{matrice de passage d'une base $e$ à un base $e'$} d'un même espace vectoriel $E$ et on note $P_e^{e'}$ la matrice 
			de $\M_n(\K)$ représentative des vecteurs de $e'$ dans la base $e$ \\
			${} \hfill P_e^{e'} = ~\begin{blockarray}[t]{(ccc)} a_{1,1} & \cdots & a_{1,n} \\ \vdots & \ddots & \\ a_{n,1} & & a_{n,n} 
			\end{blockarray} \hfill \forall j\in\ent{1,n} ,~e'_j = \si{1}{n} a_{i,j} e_i \hfill {}$ \trait
		\thm{ch15P8}{Propriété}{MatPassInv}{Si $P\in\M_n(\K)$ est la matrice de passage de $e$ à $e'$ alors $P$ est inversible \\et $P^{-1}$ est la 
		matrice de passage de $e'$ à $e$.}
		\vspace*{0.5cm} \\ \thm{ch15P9}{Propriété}{MatPassUtil}{Soit $E$ un $\K$-ev rapporté successivement à des bases $e$ et $e'$\\ On considère 
		$x\in E$ avec $X=Mat_e(x)$ ; $X'=Mat_{e'}(x)$ et $P=P_e^{e'}$ \\Alors $X=P\times X'$}
		\vspace*{0.5cm} \\ \thm{ch15th2}{Théorème}{MatPassApL}{Soit $E$ et $F$ deux $\K$-ev de dimensions finies $p$ et $n$ \\ rapporté 
		successivement à des bases $e,e'$ et $f,f'$. Soit $u\in\lin(E,F)$ \\ On note $A=Mat_{e,f}(u) ~;~A'=Mat_{e',f'}(u)$ et $~ Q=P_f^{f'} ~;~
		P=P_e^{e'}$ \\ Alors $A'=Q^{-1} \times A \times P$}
		\begin{proof}
		Soit $(x,y)\in E\times F$ tel que $y=u(x) $ alors on a $Y=AX \Leftrightarrow Y'=A'X' $ \\avec $Y=QY' ~;~ X=PX'$\\
		Ainsi $Y=AX \Leftrightarrow QY'=APX' \Leftrightarrow Y'=Q^{-1}APX' \Leftrightarrow A'=Q^{-1}AP$
		\end{proof}
		${}$ \\ \thm{ch15th2c}{Corollaire}{MatPassEndo}{Soit $E$ un $\K$-ev de dimension $n$ rapporté à deux bases $e$ et $e'$\\
		Soit $u\in\lin(E) $, on note $A=Mat_e(u) ~;~A'=Mat_{e'}(u)$ et $~P=P_e^{e'}$ \\Alors \highlight{$A'=P^{-1}\times A \times P$} }
	\section{Équivalence et similitude}
	\subsection{Matrices équivalentes et rang}
		${}$ \\ \thm{ch15P10}{Proposition}{EquiJr}{Soit $E$ et $F$ deux $\K$-ev de dimensions $p$ et $n$ et $u\in\lin(E,F)$ \\Soit $r\in\ent{1,n}$, 
		si $rg(u)=r$ alors il existe un couple de base $(e,f)$ \\ tel que $Mat_{e,f}(u) = J_r ~\in\M_{n,p}(\K)$}
		\begin{proof}
		D'après la forme géométrique du théorème du rang %(\ref{a completer}) 
        $u$ induit un isomorphisme de $S$ sur $Im(u)$ où $S$ est un 
		supplémentaire de $Ker(u)$\\ Soit $(e_1,\dots ,e_n)$ une base de $E$ adaptée à $Ker(u) \oplus S$ avec $(e_1,\dots ,e_r)$ base de $S$.
		On a alors $\big(f_1=u(e_1) ,\dots ,f_r=u(e_r) \big)$ une base de $Im(u)$ que l'on complète %(\ref{a completer}) 
        en une base de $F$
		\end{proof} ${}$ \traitd
		\paragraph{Équivalence}
			Deux matrice $A,B \in\M_{n,p}(\K)$ sont dites \underline{équivalentes} \\si il existe $Q\in\GL_n(\K)$ et $P\in\GL_p(\K)$ tels que 
			$B=Q^{-1} AP$. On note \highlight{$A\sim B$} \trait
		\thm{ch15P11}{Proposition}{CNSrgr}{Une matrice $A\in\M_{n,p}(\K)$ est de rang $r$ si et seulement $A\sim J_r$.}
		\vspace*{0.5cm} \\ \thm{ch15th3}{Théorème}{RgInvTranspo}{Le rang d'une matrice est invariant par transposition.} 
		\begin{proof}
		Soit $A\in\M_n(\K)$ ; ${^t\big(J_r^{n,p}\big)}=J_r^{p,n}$ \vspace*{0.2cm} \\ On a alors $rg(A)=r \Leftrightarrow \exists (Q,P)\in\GL_n(\K)
		\times\GL_p(\K) ~:~A=Q^{-1}J_r^{n,p}P \vspace*{0.2cm} \\ \Leftrightarrow {^tA} = \underbrace{^tP}_{\in\GL_p(\K)} \times {^t\big(J_r^{n,p}
		\big)}\times \underbrace{^t\big( Q^{-1}\big)}_{\in\GL_n(\K)} = Q'^{-1} J_r^{p,n} P' \Leftrightarrow rg({^tA} = r$
		\end{proof} \traitd
		\paragraph{Matrice extraite}
			Si $A\in\M_{n,p}(\K)$ on appelle matrice extraite de $A$ toute matrice obtenue à partir de $A$ par suppression de lignes et/ou 
			colonnes de $A$. \\$\Big( ~A'=\big(a_{i,j}\big)_{_{(i,j)\in I\times J}}$ où $I\subset\ent{1,n}$ et $J\subset\ent{1,p} ~\Big)$ \trait
		\thm{ch15P12}{Propriété}{RgExtrait}{Si $A'$ est extraite de $A$ alors on a $rg(A')\leq rg(A)$ }
		\vspace*{0.5cm} \\ \thm{ch15P13}{Proposition}{15-P14}{Si $A\in\M_{n,p}(\K)$ alors \\
		$rg(A) = \mathrm{max}\{k\in\N ~|~A'\in\GL_k(\K)$ et $A'$ extraite de $A \}$}
		\vspace*{0.5cm} \\ \thm{ch15P14}{Propriété}{OpeElemPresImKer}{Les opérations élémentaires sur les colonnes préservent l'image.\\
		Celles sur les lignes préservent le noyau.}
		\vspace*{0.5cm} \\ \thm{ch15P14c}{Corollaire}{OpeElemPresRg}{Les opérations élémentaires sur les lignes ou les colonnes \\
		de $A$ conservent le rang de $A$.} \traitd
		\paragraph{Matrice échelonnée}
			Une matrice \underline{échelonnée en ligne} est une matrice $A=\big(a_{i,j}\big)_{_{(i,j)\in\ent{1,n}\times\ent{1,p}}}$ telle que si on 
			note $l_i(A)=\mathrm{min}\{j\in\ent{1,p} ~|~a_{i,j}\neq 0\}~\forall i\in\ent{1,n}$ (par convention $\mathrm{min}\varnothing = +\infty$) 
			\\ Alors $\big(l_i(A)\big)_{_{1\leq i\leq n}}$ est une suite croissante. \trait
	\subsection{Matrices semblables et trace}
		\traitd
		\paragraph{Matrices semblables}
			Deux matrices $A,B\in\M_n(\K)$ sont dites semblables \\ s'il existe $P\in\GL_n(\K)$ telle que $B=P^{-1}AP$. \trait
		\vspace*{-1.3cm} \\ \textit{Deux matrices semblables sont équivalentes.}
		\vspace*{0.5cm} \\ \thm{ch15P15}{Propriété}{SemblCNS}{Deux matrices $A$ et $B$ sont semblables \underline{si et seulement si} elles
		représentent \\ un même endomorphisme d'un $\K$-ev de dimension finie dans deux bases différentes.} \traitd
		\paragraph{Trace}
			Si $A=\big( a_{i,j}\big)_{_{1\leq i,j\leq n}} ~\in\M_n(\K)$ on appelle \underline{trace de $A$} le scalaire $tr(A)=\si{1}{n} a_{i,i}$. 
		\trait
		\thm{ch15P16}{Propriété}{TrFLin}{$tr \in \big( \M_n(\K) \big)^*$ avec $\forall (A,B) \in \big(\M_n(\K) \big)^2 ,~tr(AB) = tr(BA)$}
		\vspace*{0.5cm} \\ \thm{ch15th4}{Théorème}{TrInvSim}{La trace est invariante par similitude.  $\Big( ~\forall (A,B)\in\big( 
		\M_n(\K) \big)^2 ,$ \\ $\big( \exists P\in\GL_n(\K)$ telle que $B=P^{-1}AP \big) \Rightarrow \big( tr(B) = tr(A) \big) ~\Big)$}
		\begin{proof}
		Soit un tel couple $(A,B)\in \big(\M_n(\K) \big)^2$ \\ Alors $tr(B)=tr(P^{-1}AP) = tr(APP^{-1})=tr(A)$
		\end{proof} \traitd
		\paragraph{Trace d'un endomorphisme}
			Si $u$ est un endomorphisme d'un $\K$-ev de dimension finie $E$, on appelle \underline{trace de $u$} le scalaire 
			$tr(u) = tr\big(Mat_e(u)\big)$ où $e$ est une base de $E$. \trait
		\thm{ch15P17}{Propriété}{TrFLinEndo}{$tr\in\big(\lin(E) \big)^*$ avec $\forall (u,v) \in\big(\lin(E)\big)^2 ,~tr(uv) = tr(vu)$}
		\vspace*{0.5cm} \\ \thm{ch15P18}{Proposition}{TrProjecteur}{Soit $E$ un $\K$-ev de dimension finie et $p$ un projecteur de $E$ \\ Alors 
		$tr(p)=rg(p)$}
		\vspace*{0.5cm} \\ 
		\begin{center}
		\fin
		\end{center}
    
\chapter{Groupe symétrique et déterminant}

    
% Chapitre 15 : Groupe symétrique et déterminant

% TODO
        
\chapter{Intégration}

    
% Chapitre 16 : Intégration

\minitoc
\textit{Dans tout le chapitre, $(a,b)\in\R^2$ avec $a<b$}
	\section{Continuité uniforme}
		\traitd
		\paragraph{Définition}
			Soit $I\subset\R$ intervalle et $f:I\rightarrow \R$, $f$ est dite \underline{uniformément continue sur $I$} si
			\[ \forall \varepsilon >0 ,~\exists \delta>0 ~tel~que~\forall (x,y)\in I^2,~\big( \mc{x-y}\leq \delta \Rightarrow \mc{f(x)-f(y)}\leq 
			\varepsilon \] \vspace*{-0.7cm} \trait
		\thm{ch17P1}{Propriété}{LipschImplUC0}{Soit $f: I\rightarrow \R$ on a \\ {\small 1)} Si $f$ est lipschitzienne sur $I$ alors $f$ est 
		uniformément continue sur $I$ \\ {\small 2)} Si $f$ est uniformément continue sur $I$ alors $f$ est continue sur $I$}
		\vspace*{0.5cm} \\ \thm{ch17th1}{\highlight{Théorème de \textsc{Heine}}}{ThHeine}{Soit $(a,b)\in\R^2 ,~a<b$ \\ Si $f$ est continue sur 
		$[a,b]$ alors $f$ est uniformément continue sur $[a,b]$.}
		\begin{proof}
		Par l'absurde : \\On suppose $\exists \varepsilon >0$ tel que $\forall n\in\N^* ,~\exists (x_n,y_n)\in [a,b]^2$ tq $\big( \mc{x_n-y_n} 
		\leq \frac{1}{n}$ et $\mc{f(x_n)-f(y_n)} > \varepsilon$ \\ D'après le théorème de Bolzano-Weierstrass %(\ref{a completer}) 
        $\exists 
		\varphi :\N \rightarrow\N$ et $\psi : \N\rightarrow\N$ extractrices tels que $x_{\varphi(n)} \ston l \in [a,b]$ et $y_{\varphi(\psi(n))} 
		\ston l' \in[a,b]$ donc $\mc{x_{\varphi(\psi(n))}-y_{\varphi(\psi(n))}}\leq \frac{1}{\varphi(\psi(n))} \leq \frac{1}{n}$ d'où $l=l'$ \\
		Ainsi par continuité de $f$ on a $f(x_{\varphi(\psi(n))}) - f(y_{\varphi(\psi(n))}) \ston f(l)-f(l') = 0 >\varepsilon >0$
		\end{proof} ${}$
	\section{Intégrations des fonctions en escalier}
	On note $\mathcal{Esc}([a,b],\R)$ l'ensemble des fonctions en escalier de $[a,b]$ dans $\R$.
	\subsection{Subdivision d'un segment}
		\traitd
		\paragraph{Définition}
			Une \underline{subdivision de $[a,b]$} est une suite finie strictement croissante $\sigma = (c_0=a<c_1<\cdots 
			<c_n=b)$. \\ On note $\delta(\sigma )$ le pas de $\sigma$ définit par $\delta(\sigma)= \underset{0\leq i\leq n-1}{\mathrm{max}} 
			(c_{i+1}-c_i)$. \\ On dit que $\sigma$ est à pas constant si la suite $(c_i)_{_{0\leq i \leq n}}$ est arithmétique. \trait
		Soit $\sigma'$ une subdivision de $[a,b]$, on dit que $\sigma'$ est \underline{plus finie} que $\sigma$ si tout point de $\sigma$ est un 
		point de $\sigma'$. On notera ici \highlight{$\sigma \subset \sigma'$}. \traitd
		\paragraph{Subdivision adaptée}
			Soit $f: [a,b] \rightarrow \R$ une fonction en escalier sur $[a,b]$, on considère $\sigma = (c_0 , \dots ,c_n)$ une subdivision de 
			$[a,b]$. On dit que \underline{$\sigma$ est adaptée à $f$} si \[ \forall i\in\ent{0,-1} ,~\exists \lambda_i \in \R ~:~ 
			f|_{]c_i,c_{i+1}[} = \widetilde{\lambda_i} \] \vspace*{-0.7cm} \trait
		\thm{ch17P2}{Proposition}{EscaStable}{$\Esc$ est stable par somme, produit et passage à la valeur absolue.}
		\traitd
		\paragraph{Intégrale d'une fonction en escalier}
			Soit $f \in \Esc$ ; soit $\sigma = (c_0,\dots ,c_n)$ une subdivision adaptée à $f$. \\ On appelle \underline{intégrale de $f$ sur 
			$[a,b]$} le scalaire \[ \int_{[a,b]}  f = \sum_{i=0}^{n-1} \lambda_i (c_{i+1}-c_i)\] \vspace*{-0.7cm} \trait
		\thm{ch17P3}{Propriétés}{17-P3}{Soit $f$ et $g$ des fonctions en escalier sur $[a,b]$ \\ {\small 1)} Si $f \geq 0$ sur $[a,b]$ alors 
		$\int_[a,b] f \geq 0$ \\ {\small 2)} Si pour tout $x\in [a,b] ,~f(x)\geq g(x)$ alors $\int_[a,b] f \geq \int_[a,b] g$ \\ 
		{\small 3)} $\mc{\int_[a,b] f} \ leq \int_[a,b] \mc{f} \leq (b-a) \underset{[a,b]}{\mathrm{sup}} \mc{f}$}
		\vspace*{0.5cm} \\ \thm{ch17P4}{Proposition}{ChaslesEsc}{Soit $f\in\Esc$ et $c\in[a,b]$ alors 
		$\int_{[a,b]} f = \int_{[a,c]} f|_{[a,c]} + \int_{[c,b]} f|_{[c,b]}$} \\
	\section{Fonctions continues par morceaux}
	\subsection{Généralités}
		\traitd
		\paragraph{Définition}
			Soit $f : [a,b] \rightarrow \R$, on dit que $f$ est continue par morceaux sur $[a,b]$ s'il existe $\sigma = (c_0,\dots ,c_n)$ une 
			subdivision de $[a,b]$ telle que $\forall i\in\ent{0,n-1} ,~f|_{]c_i,c_{i+1}[}$ est continue et prolongeable par continuité en $c_i$ et 
			$c_{i+1}$. \trait \vspace*{-1.1cm} \\
		On note $\cpm$ l'ensemble des fonctions continues par morceaux de $[a,b]$ dans $\R$.
		\vspace*{0.5cm} \\ \thm{ch17P5}{Propriété}{CpmBornee}{Si $f\in\cpm$ alors $f$ est bornée sur $[a,b]$}
		\vspace*{0.5cm} \\ \thm{ch17L1}{Lemme}{CpmEntreEsc}{Si $f\in\cpm$ alors $\forall \varepsilon >0,\exists (\varphi,\psi)\in\Big(\Esc\Big)^2$ 
		telles que \\ $\forall x\in[a,b] ,~\varphi(x) \leq f(x) \leq \psi(x)$ et $\psi(x) - \varphi(x) \ leq \varepsilon $}
		\vspace*{0.5cm} \\ \thm{ch17P6}{Propriété}{CpmStable}{$\cpm$ est stable par produit, combinaison linéraire et valeur absolue.}
	\subsection{Intégrale d'une fonction continue par morceaux}
		\traitd
		\paragraph{Définition}
			Soit $f\in\cpm$ \\On note $\mathcal{I}^+(f)=\Big\{ \int_{[a,b]} \psi ~|~\psi$ en escalier sur $[a,b]$ et $f\leq \psi \Big\} $ \\
			Alors $\mathrm{inf}\big(\mathcal{I}^+(f)\big)$ existe, on appelle intégrale de $f$ sur $[a,b]$ notée $\int_{[a,b]} f$ cette valeur.
			\trait\vspace*{-1.1cm} \\ 
		\underline{Rq} : $\int_{[a,b]} f = \mathrm{inf}\big(\mathcal{I}^+(f)\big) = \mathrm{sup}\big(\mathcal{I}^-(f)\big)$
		\vspace*{0.5cm} \\ \thm{ch17P7}{Propriété}{LineInt}{Soit $(f,g)\in\Big(\cpm\Big)^2 ~;~ (\alpha,\beta \in\R^2$ alors \\
		$\int_{[a,b]}\alpha f+\beta g = \alpha\int_{[a,b]} f + \beta\int_{[a,b]} g$}
		\vspace*{0.5cm} \\ \thm{ch17th2}{Théorèmes opératoires}{ThOperInt}{Soit $f,g \in\Big(\cpm\Big)^2$ on a \\ 
		{\small 1)} Si $f\geq 0$ sur $[a,b]$ alors $\int_{[a,b]} f \geq 0$ \\ {\small 2)} Si $\forall x\in[a,b],~g(x)\geq f(x)$ alors $\int_{[a,b]} 
		g \geq \int_{[a,b]} f$ \\ {\small 3)} $\mc{\int_{[a,b]} f} \leq \int_{[a,b]} \mc{f} \leq \underset{x\in [a,b]}{\mathrm{sup}}\mc{f(x)}$}
		\begin{proof}
		Clair d'après le lemme %(\ref{CpmEntreEsc})
        .
		\end{proof}
		${}$ \\ \thm{ch17th3}{Théorème : Inégalité de \textsc{Cauchy-Schwartz}}{InegCauchySchwartz}{Soient $f,g\in\cpm$ alors 
		$\Big(\int_{[a,b]} fg \Big)^2 \leq \int_{[a,b]} f^2 \times \int_{[a,b]} g^2$}
		\begin{proof}
		On pose $P(\lambda)=\int_{[a,b]} (\lambda f+g)^2 = \lambda^2 \int_{[a,b]} f^2 + 2\lambda \int_{[a,b]} fg + \int_{[a,b]} g^2 \geq 0$\\
		\underline{Si $\int_{[a,b]} f^2 =0$} alors $2\int_{[a,b]} fg=0$ et l'inégalité est vrai \\
		\underline{Sinon $\int_{[a,b]} f^2 >0$} et $\Delta = 4\Big( \big( \int_{[a,b]} fg \big)^2 - \int_{[a,b]} f^2 \times\int_{[a,b]} g^2 \leq 0$ 
		\end{proof} \traitd
		\paragraph{Valeur moyenne}
			Soit $f\in\cpm$ on appelle \underline{valeur moyenne de $f$ sur $[a,b]$} le scalaire \[ \frac{1}{b-a} \int_{[a,b]} f \]\vspace*{-0.7cm}
			\trait
		\thm{ch17P8}{Proposition}{IntNulle}{Soit $f$ une fonction \highlight{continue sur $[ab]$} à valeur dans $\R^+$\\ On suppose $\int_{[a,b]} 
		f=0 $ alors $f=0$}
		\vspace*{0.5cm} \\ \thm{ch17P9}{Propriété}{EgCSLiee}{Soit $f,g\in\cpm$ alors \\ $\big(\int_{[a,b]} fg \big)^2 = \int_{[a,b]} f^2 \times 
		\int_{[a,b]} g^2 ~\Leftrightarrow ~(f,g)$ sont liées.}
		\vspace*{0.5cm} \\ \thm{ch17P10}{Propriété}{ChgtVarInt}{ Soit $f\in\cpm$ et $\forall u\in\R$ on pose $f_u~\appli{[a+u,b+u]}{x}{\R}{f(x-u)}$
		\\ alors $\int_{[a+u,b+u]}f_u = \int_{[a,b]} f$}
		\vspace*{0.5cm} \\ \thm{ch17th4}{Théorème : Relation de \textsc{Chasles}}{RelChasles}{Soit $f$ continue par morceaux sur un segment $S$ de 
		$\R$ et $(a,b,c)\in S^3$ \\ alors $\int_a^b f(x)\dd x = \int_a^c f(x) \dd x + \int_c^b f(x) \dd x$}
		\vspace*{0.5cm} \\ \thm{ch17P11}{Propriété}{IntFcImpaire}{Soit $a\in\R ~;~f\in\cont^0_pm\big([-a,a],\R\big)$, on suppose que $f$ est paire 
		(resp. impaire) \\ alors $\int_{-a}^a f(x) \dd x = 2\int_0^a f(x) \dd x$ $\big($resp. $\int_{-a}^a f(x) \dd x=0\big) $}
		\vspace*{0.5cm} \\ \thm{ch17P12}{Propriété}{IntTperio}{Soit $f$ continue par morceaux sur $I\subset\R$, on suppose que $f$ est $T$-
		périodique \\ alors $\forall a\in I ,~\int_a^{a+T} f(x) \dd x = cte $ (ne dépend pas de $a$) } \\
	\section{Sommes de \textsc{Riemman}}
		\traitd
		\paragraph{Définition}
			Si $f$ est continue sur $[a,b]$ et $\sigma = (c_0,\dots ,c_n)$ est une subdivision de $[a,b]$, on appelle \underline{somme de 
			\textsc{Riemman} associée à $f$ sur $[a,b]$} l'expression \[\si{0}{n-1} (c_{i+1} -c_i )\times f(\xi_i ) ~avec ~\xi_i\in[c_i,c_{i+1}]\] 
			\vspace*{-0.6cm} \trait ${}$ \vspace*{-1.3cm} \traitd
		\paragraph{Somme de \textsc{Riemman} associée}
			Soit $f\in\cpm$ et $\sigma = (c_0,\dots c_n)$ une subdivision adaptée à $f$ sur $[a,b]$ on pose pour $i\in\ent{0,n-1} , ~\varphi_i = 
			f|_{]c_i,c_{i+1}[}$ que l'on prolonge par continuité sur $[c_i,c_{i+1}]$. \\ On appelle \underline{somme de Riemmann associée} 
			une somme de sommes de Riemman associées aux $\varphi_i$ \trait
		\thm{ch17P13}{Propriété}{LimSommRiemman}{Soit $f\in\cpm$ alors $\frac{b-a}{n}\sk{0}{n-1} f\big(a+k\frac{b-a}{n} \big) ~\ston ~\int_a^b f(t) 
		\dd t$ }
	\section{Lien entre intégrales et primitives}
		${}$\\ \thm{ch17th5}{\highlight{Théorème fondamental du calcul intégral}}{ThFondInt}{Soit $f$ un fonction continue sur un intervalle $I$ de 
		$\R$ et $a\in I$, \\ Alors $F : x\mapsto \int_a^x f(t) \dd t$ est l'unique primitive de $f$ qui s'annulle en $a$.}
		\begin{proof}
		$F$ est bien définiesur $I$, on considère alors $c\in I$ ; soit $x\in I\backslash\{c\}$ \\ Il existe alors $\xi_x$ compris entre $x$ et $c$ 
		tel que $\frac{F(x)-F(c)}{x-c} = f(\xi_x) \stox{c} f(c)$ par $\cont^0$ de $f$ en $c$ \\ Donc $F$ est dérivable en $c$ et $f'(c)=f(c)$
		\end{proof}
		${}$ \\ \thm{ch17th5c}{Corollaire}{IntDifPrim}{Pour toute primitive $F$ de $f$ sur $I$ on a $\int_a^b f(t)\dd t = F(b)-F(a) = [F(t)]_a^b$}
		\vspace*{0.5cm} \\ \thm{ch17th5c2}{Corollaire}{DL1f}{Soit $f$ continue sur $I$ et $a\in I$, on suppose $f\in\cont^1(I,\R)$ \\ alors $
		\forall x\in I ,~f(x) = f(a) + \int_a^x f'(t) \dd t$} \\
	\section{Formules de \textsc{Taylor} globales}
		${}$ \\ \thm{ch17th6}{Théorème : Formule de \textsc{Taylor} avec reste intégral}{FormTaylRstInt}{Soit $f\in\cont^{n+1}\big(I,\R\big),
		~a\in I$ \\ Alors $\forall x\in I ,~\displaystyle{ f(x) = \sk{0}{n} \frac{(x-a)^k}{k!}f^{(k)}(a) + \int_a^x 
		\frac{(x-t)^n}{n!} f^{(n+1)}(t) \dd t} $ }
		\begin{proof}
		Par récurrence sur $n$ : \\
		\underline{Initialisation} : $f(x) = f(a) + \int_a^x f'(t) \dd t$ d'après le corollaire précédant %(\ref{DL1f}) 
        \\
		\underline{Hérédité} : On suppose la propriété vraie au rang $n$ et on considère $f\in\cont^{n=2}(I,\R)$. \\ Comme $f\in\cont^{n+1}(I,\R)$ 
		on a $f(x) = \sk{0}{n} \frac{(x-a)^k}{k!}f^{(k)}(a) + \int_a^x \frac{(x-t)^n}{n!} f^{(n+1)}(t) \dd t $ \vspace*{0.2cm} \\ 
		$= \sk{0}{n} \frac{(x-a)^k}{k!}f^{(k)}(a) + \big[ -\frac{(x-t)^{n+1}}{(n+1)!}f^{(n+1)}(t) \big]_a^x + \int_a^x \frac{(x-t)^{n+1}}{(n+1)!}
		f^{(n+2)}(t) \dd t$ \vspace*{0.2cm} \\ $ = \sk{0}{n+1} \frac{(x-a)^k}{k!}f^{(k)}(a) + \int_a^x \frac{(x-t)^{n+1}}{(n+1)!} f^{(n+2)}(t) 
		\dd t $ par IPP
		\end{proof}
		${}$\\ \thm{ch17th6c}{Corollaire : Inégalité de \textsc{Taylor-Lagrange}}{InegTaylLagr}{Soit $f\in\cont^{n+1}(I,\R), ~a\in I$ et $M$ un 
		majorant de $\mc{f^{(n+1)}}$ sur $I$ \\ Alors $\forall x\in I ,~ \Big| f(x) - \sum\limits_{k=0}^n \frac{(x-a)^k}{k!} f^{(k)}(a)\Big| \leq 
		M\times \frac{\mc{x-a}^{n+1}}{(n+1)!}$}
		\vspace*{0.5cm} \\ 
		\begin{center}
		\fin
		\end{center}

\chapter{Dénombrement}

    
% Chapitre 17 : Dénombrement

\minitoc
	\section{Cardinal d'un ensemble}
	\subsection{Généralités}
		\traitd
		\paragraph{Équipotence}
			On dit que deux ensembles \uline{$E$ et $F$ sont équipotents} s'il existe une bijection de $E$ sur $F$. On note alors $E\sim F$ \trait ${}$ \vspace*{-1.2cm} \traitd
		\paragraph{Ensemble fini - Cardinal}
			Soit $E$ un ensemble, on dit que \uline{$E$ est fini} s'il est \textbf{vide} ou s'il existe $n\in\N^*$ tel que $E\sim \ent{1,n}$ \\
			On appelle alors $n$ le \uline{cardinal de $E$} noté $\abs{E}$ (ou $\mathrm{Card}(E)$) dont on admet l'unicité, sous réserve d'existence avec par convention $\abs{\varnothing } = 0$  \trait
		\thm{ch18L1}{Lemme}{SousEns1n}{Pour tout $n\in\N^*$ ; soit $F\subset \ent{1,n}$\\
		\hspace*{0.5cm} Alors $F$ est fini et $\abs{F} \leqslant n$ }
		\vspace*{0.5cm} \\ \thm{ch18L1c}{Corollaire}{18-L1}{Si $E$ et $F$ sont des ensembles avec $F$ fini et $E\subset F$ \\
		\hspace*{0.5cm} Alors $E$ est fini et $\abs{E} \leqslant \abs{F}$ \\
		avec égalité si et seulement si $E=F$ } \vspace*{0.5cm} \newpage
		\uline{Remarque} : Définition avec l'indicatrice
		Soit $A\in \mathcal{P}(E)$ \vspace*{0.2cm} \\
		$\mathbb{1}_A ~:~\appli{E}{x}{ \{0,1 \} }{ \left\{ \ard 1~si~x\in A \\ 0 ~si~ x\in \mathcal{C}_EA \arf\right.} ~$ et si $E$ est fini alors  $\abs{A} = \sum\limits_{x\in E} \mathbb{1}_A (x)$
		\vspace*{0.5cm} \\ \thm{ch18P1}{Proposition}{Appli&Card}{
		Si $E$ et $F$ sont deux ensembles finis et $f:E\rightarrow F$ alors \\
		\hspace*{15pt} {\small 1)} Si $f$ injective $\abs{f(E)} = \abs{E}$ et $ \abs{E} \leq \abs{F}$ \\
		\hspace*{15pt} {\small 2)} Si $f$ surjective $\abs{F} \leq \abs{E}$\\
		\hspace*{15pt} {\small 3)} Si $\abs{F} = \abs{E}$ alors $f$ est injective si et seulement si $f$ est surjective.}
		\vspace*{0.5cm} \\ \thm{ch18P2}{Propriété}{CardCompl}{Soit $E$ un ensemble fini et $A\in\mathcal{P}(E)$ alors $~\abs{\mathcal{C}_E A} = \abs{E} - \abs{A}$ }
		\vspace*{0.5cm} \\ \thm{ch18P2c}{Corollaire}{CardDiff}{Si $A$ et $B$ sont finis alors $A\setminus B$ est fini et $~\abs{A\setminus B} = \abs{A}-\abs{B}$}
		\vspace*{0.5cm} \\ \thm{ch18P3}{Proposition}{CardUnion}{ Si $A$ et $B$ sont finis alors $A\cup B$ est fini et $~\abs{A\cup B} ~=~ \abs{A} + \abs{B} - \abs{A\cap B}$}
	\subsection{Lemme des Bergers et principe des Tirroirs}
		${}$ \\ \thm{ch18P4}{Proposition}{CardPart}{Si $P$ est une partition de $E$ %(cf -> \ref{def partition} ) 
		alors $\abs{E} = \sum\limits_{X\in P} \abs{X}$} 
		\vspace*{0.5cm} \\ \thm{ch19th1}{Lemme des Bergers}{lemme bergers}{Soit $E,F$ deux ensembles finis et $f: E\rightarrow F$ telle que \\ $\exists p \in \mathbb{N}^* ~:~~\forall y\in F ~,~~ \abs{f_r^{-1} (\{y\} )} = p ~~$ alors $~~ \abs{E} ~=~ p\abs{F}$}
		\begin{proof}
		$~~\left( f_r^{-1} (\{y\} ) \right)_{_{y\in F}}$ est une partition de $E$ et on a alors\\
		$\abs{E} ~=~ \sum\limits_{y\in F} \abs{f_r^{-1} (\{y\} )} ~=~ \sum\limits_{y\in F} p ~=~ p\abs{F}$
		\end{proof}
		${}$ \\ \thm{ch19th2}{Principe des Tirroirs de Dirichlet}{princ.tirroirs}{Soit $E$ et $F$ deux ensemble finis de cardinaux respectifs $n$ et $p \in \mathbb{N}^*$\\$f:E\rightarrow F$ telle que s'il existe $k\in\mathbb{N} ~:~~n>kp ~$ alors $~\exists y\in F ~:~~ \abs{f_r^{-1} (\{y\} )} ~>~ k$}
		\begin{proof}
		On suppose que $\forall y\in F ~,~~ \abs{f_r^{-1} (\{y\} )} ~\leq ~k$ \\alors d'après le \underline{Lemme des Bergers} %(\ref{lemme bergers} ) 
        $~~\abs{E} ~\leq ~kp$ \end{proof} ${}$ 
	\subsection{Calcul sur les cardinaux}
		${}$ \\ \thm{ch18P5}{Proposition}{CalculCard}{Soit $E$ et $F$ deux ensembles finis alors \\
		\hspace*{0.5cm} $\rightarrow ~~ \abs{E\times F} ~=~ \abs{E} \times \abs{F}$ 
		\hspace*{51cm} $\rightarrow ~~ \abs{F^E} ~=~ \abs{F}^{\abs{E}}$} 
		\newpage
		${}$ \\ \thm{ch18P6}{Propriété}{CardBij}{ Soit $E$ et $F$ deux ensemble de même cardinal $n$ alors \\
		\hspace*{0.5cm} $\bullet$ $By(E,F)$ l'ensemble des bijections de $E$ sur $F$ est de cardinal $n!$ \\
		\hspace*{0.5cm} $\bullet$ $\mathcal{P}(E)$ est un ensemble fini de cardinal $2^n$}
		\vspace*{0.5cm} \\ \thm{ch18P6c}{Corollaire}{CardPermut}{Le cardinal de l'ensemble des permutations d'un ensemble à $n$ éléments es $n!$}
	\section{Listes et Combinaisons}
		\traitd
		\paragraph{Arrangement}
			On appelle \underline{arrangement de $k$ éléments parmi $n$} toute \textbf{application injective} de $\ent{1,k}$ dans $\ent{1,n}$ soit une \textbf{k-liste} d'éléments distincts de $\ent{1,n}$. \trait
			On note $A_{k,n}$ l'ensemble des arrangements de $k$ éléments parmis $n$.\trait
		\thm{ch18P7}{Propriété}{CardArrang}{Le nombre d'arrangement de $k$ éléments parmis $n$, noté $A_n^k$ vérifie \\ \hspace*{0.5cm}
		$A_n^k = \abs{A_{k,n}} = \left\{ \begin{array}{cl} 0 & si~k>n \\ \frac{n!}{(n-k)!} & si~0\leq k\leq n \end{array} \right.$ } \\ \traitd
		\paragraph{Combinaison}
			On appelle \underline{combinaison de $k$ objets parmis $n$} toute \textbf{partie} à $k$ éléments d'un ensemble à $n$ objets et on note $\mathcal{P}_k (E)$ l'ensemble des combinaisons à $k$ éléments de $E$. \trait
		\thm{ch18P8}{Propriété}{NbCombin}{Le nombre de combinaisons de $k$ éléments parmis $n$ est $\cm{\binom{n}{k}}$ }\\
		\paragraph{\highlight{Formule de Vandermonde}} ${}$ \\
		\hspace*{2cm} $\cm{(1+X)^n(1+X)^m = (1+X)^{n+m} ~\Rightarrow~ \binom{n+m}{k} = \sum_{i+j=k} \binom{n}{i}\binom{m}{j} } $ \vspace*{0.5cm} \\
		\begin{center}
			\fin
		\end{center}
        
\chapter{Probabilités}

    
% Chapitre 18 : Probabilités

\textsl{On désigne par expérience aléatoire toute expérience dont le résultat est soumis au hasard.}
\minitoc
	\section{Univers, évènements et variables aléatoires}
		\uline{Modéliser} une expérience aléatoire, c'est associer à cette expérience $\varepsilon$ trois objets mathématiques : $\Omega$ un univers fini (des possibles), $\mathscr{A} = \mathcal{P}(\Omega)$ l'ensemble des évènements associés à $\varepsilon$ et $P$ une probabilité.\\ \hspace*{0.5cm}
		$\big(\Omega ,\mathcal{P}(\Omega),P\big)$ est un \uline{espace probabilisé}.
		\traitd
		\paragraph{Ensemble des évènements}
			On appelle \uline{ensemble des évènements associés à $\varepsilon$} toute partie $\mathscr{A}$ de $\mathcal{P}(\Omega)$ vérifiant :\\
			\hspace*{2cm} \un $\Omega\in \A$ et $\varnothing\in\A$\\
			\hspace*{2cm} \deux $\forall A\in\A ,~ \overline{A}\in\A$\\
			\hspace*{2cm} \trois Soit $I$ un ensemble fini ou dénombrable et $(A_i)_{i\in I}$ une famille d'évènements alors 
			\[ \bigcup_{i\in I} A_i \in \A ~et~ \bigcap_{i\in I} A_i \in \A \]
			\trait \vspace*{-1.2cm} \\
			Dans le cas où $\abs{\Omega} < +\infty$ on prend $\A =\mathcal{P}(\Omega)$ \\ \traitd
		\paragraph{Système complet d'événements}
			${}$ \\ \hspace*{2cm} \un $\forall A,B\in \Part(\Omega)$, $A$ et $B$ sont dits incompatibles si $A\cap B = \varnothing$ \\
			\hspace*{2cm} \deux On appelle \uline{système complet d'événements} toute famille $(A_i)_{_{i\in I}}$ d'événements deux à deux incompatibles et dont la réunion est l'événement certain \trait ${}$ \vspace*{-1.3cm} \traitd
		\paragraph{Probabilité}
			Si $(\Omega,\A)$ est un espace probabilisable, on appelle \uline{probabilité sur $\A$} toute application telle que \\
			\hspace*{2cm} \un $\Part(\Omega) = 1$ \\ \hspace*{2cm} \deux $\forall (A,B)\in\A^2 ,~ A\cap B=\varnothing \Rightarrow P(A\cup B)=P(A)+P(B)$ \trait
		\vspace*{-1cm} \\ Dans le cas fini, $(\Omega,\A,P)$ est un \uline{espace probabilisé fini}.
		\vspace*{0.5cm} \\ \thm{ch19P1}{Propriétés}{PropProba}{\un $P(\varnothing)=0$ \\ \deux $\forall A\in\Part(\Omega) ,~P(\overline{A})=1-P(A)$ \\
		\trois $\forall (A,B)\in\Part^2(\A) ,~ P(A\setminus B) = P(A) - P(A\cup) B$\\
		\quatre $\forall (A,B)\in\Part^2(\Omega) ,~A\subset B \Rightarrow P(A)\leqslant P(B)$ \\
		{\scriptsize (5)} $\forall (A,B)\in\Part^2(\Omega) ,~ P(A\cup B) = P(A)+P(B) - P(A\cap B)$} \\ \traitd
		\paragraph{Variable aléatoire}
			On appelle \uline{variable aléatoire} toute application définie sur $\Omega$ à valeurs dans un ensemble $E$. Si $E\subset \R$ on dit que $X:\Omega \to E$ est une variable aléatoire réelle. \trait \vspace*{-1.7cm} \\
		\subparagraph{Notations}
			Si $X$ est une variable aléatoire, on note \\
			$\bullet$ Pour $A\in \Part(E) ,~X_r^{-1}(A) = (X\in A)$\\
			$\bullet$ Si $e\in E ,~X_r^{-1}(\{e\}) = (X=e)$\\
			$\bullet$ Si $E=\R,~ X_r^{-1}([a,b[) = (a\leqslant X <b)$ \\
	\section{Espaces probabilisés finis, probabilité uniforme}
	\subsection{Équiprobabilités}
		Si $\abs{\Omega} < +\infty$, une hypothèse classique est de considérer une probabilité $P$ telle que $\forall \omega\in\Omega,~P(\omega)=\dfrac{1}{\abs{\Omega}}$
		C'est bien une probabilité, dite \uline{équiprobabilité} car tout les événement réduits à une issue on la même probabilité
	\subsection{Probabilités conditionnelles}
		\traitd
		\paragraph{Définition}
			Si $A$ et $B$ sont deux événements de $(\Omega,\A,P)$ de probabilités non nulles, on défini \[P(A|B) = \dfrac{P(A\cap B)}{P(B)}\]\trait
		\thm{ch19P2}{Propriété}{PCondProba}{Si $B$ est un événement de probabilité non nulle dans un \\
		espace probabilisé fini $(\Omega,\A,P)$ \\
		\hspace*{0.5cm} Alors $\forall B\in\Part(\Omega) ,~ \appli{\Part(\Omega)}{A}{\R}{P(A|B)}$ \\
		est une probabilité}
		\vspace*{0.5cm} \\ \thm{ch19P3}{Proposition $\heartsuit$}{ProbaIntersec}{Si $A_1,\dots ,A_n$ sont des événements d'un espace \\
		probabilisé fini $(\Omega,\Part(\Omega),P)$\ alors \\
		\highlight{$P(A_1\cap\cdots\cap A_n) = P(A_1)\times P(A_2|A_1)\times\cdots\times P(A_n|A_1\cap\cdots\cap A_{n-1})$} }
		\vspace*{0.5cm} \\ \thm{ch19P4}{Propriété : Formule des probabilités totales}{ProbaTot}{Si $B$ est un événement et $(A_i)_{_{i\in\ent{1,n}}}$ est un système \\
		complets d'événements de $(\Omega,\Part(\Omega),P)$\\
		\hspace*{0.5cm} Alors $P(B) = \si{1}{n} P(B|A_i)\times P(A_i)$ }
		\vspace*{0.5cm} \\ \thm{ch19P5}{Proposition : Formule de \textsc{Bayes}}{FormuleBayes}{Soit $A$ un événement de $(\Omega,\Part(\Omega),P)$ tel que $P(A)\neq 0$
		\\ Si $(B_1,\dots ,B_n)$ est un système complet d'événements \\
		\hspace*{0.5cm} Alors $\forall i\in\ent{1,n} ,~P(B_i|A) = \dfrac{P(A|B_i)P(B_i)}{\sk{1}{n}P(A|B_k)P(B_k)}$}
	\section{Loi d'une variable aléatoire}
		\traitd
		\paragraph{Loi de probabilité}
			Si $X$ est une variable aléatoire définie sur un espace probabilisé fini $(\Omega,\Part(\Omega),P)$ à valeur dans $E$ on appelle \uline{loi de probabilité de la variable $X$} (ou \uline{distribution}) l'application 
			\[ P_X ~ \appli{\Part(E) }{A}{[0,1]}{P(X\in A)} \] \trait
		\thm{ch19P6}{Propriété}{DistribEstProba}{Si $X$ est une variable aléatoire sur un espace probabilisé fini $(\Omega,\Part(\Omega),P)$\\
		\hspace*{0.5cm} Alors $P_X$ est une probabilité sur $E$}
		\subparagraph{Notation}Si $X$ est $Y$ sont deux variables aléatoire définies sur un même espace probabilisé fini à valeurs dans $E$ on note $X\sim Y$ si $P_X=P_Y$
		\vspace*{0.5cm} \\ \thm{ch19P7}{Propriété}{FcVAR}{Soit $X:\Omega\to E$ est une variable aléatoire et $f:E\to F$\\
		\hspace*{0.5cm} Alors $f(X)=f\circ X :\Omega\to F$ est une variable aléatoire avec \\
		$\forall B\in\Part(F),~ P_{f(X)}(B) = P(f(X)\in B) = P(X\in f_r^{-1}(B)) = P_X(f_r^{-1}(B))$}
	\subsection{Variable uniforme sur une ensemble fini non vide}
		Soit $E$ un ensemble fini non vide.
		\traitd
		\paragraph{Loi uniforme}
			On dit que $X$ variable aléatoire \uline{suit une loi uniforme sur $E$} si 
			\[ \left\{ \ard X(\Omega) = E \\ \forall x\in E,~P(X=x)= \frac{1}{\abs{E}} \arf \right.\]
			On écrit alors $X\sim \mathcal{U}(E)$ \trait
		\vspace*{-1.1cm} \\ On prend un objet au hasard parmi $\abs{E}$ objets qui on tous la même probabilité d'être choisis et on note $X$ cet objet.
	\subsection{Variable de \textsc{Bernoulli}}
		On appelle expérience de \textsc{Bernoulli} une expérience aléatoire à deux issues. On appelle succès l'une des issues et échec l'autre. On peut donc lui associer une variable aléatoire réelle qui prend la valeur $1$ en cas de succès et la valeur $0$ en cas d'échec. \\ \traitd
		\paragraph{Loi de \textsc{Bernoulli}}
			On dit que $X$ \uline{suit une loi de \textsc{Bernoulli}} de paramètre $p\in[0,1]$ si 
			\[ \left\{ \ard X(\Omega) = \{0,1\} \\ P(X=1) = p \arf \right. \] \trait
		\vspace*{-1.1cm} \\ On note alors $X\sim \mathcal{B}(p)$
	\subsection{Loi binomiale}
		Si on répète $n$ fois une expérience de \textsc{Bernoulli}, la variable aléatoire associée au nombre de succès suit une loi de \textsc{Bernoulli} de paramètres $n$ et $p$.

\chapter{Espaces préhilbertiens réels}

    
% Chapitre 19 : Espaces préhilbertiens réels

\textsl{Dans ce chapitre, $E$ est un $\R$-espace vectoriel.}
\minitoc
	\section{Produit scalaire}
		\traitd
		\paragraph{Définition}
			Un \uline{produit scalaire $\scal{x,y}$} est une application $\varphi : E\times E ~\rightarrow ~\mathbb{R}$ telle que 
			\\ \hspace*{2cm} {\small 1) } $\varphi$ est \underline{bilinéaire} $\left\vert \ar* \varphi (\lambda x+\lambda 'x' , y) ~=~ \lambda\varphi (x,y) + \lambda '\varphi (x',y) \\ \varphi (x, \mu y + \mu 'y') ~=~ \mu\varphi (x,y) + \mu '\varphi (x,y') \ar\right.$\vspace*{3pt}\\
			\hspace*{2cm} {\small 2) } $\varphi$ est \underline{symétrique} $~\forall (x,y) \in E^2 ~,~~ \varphi (x,y) ~=~ \varphi (y,x)$ \vspace*{3pt}\\
			\hspace*{2cm} {\small 3) } $\varphi$ est \underline{définie positif} $~\varphi (x,x) \geqslant 0 ~\wedge ~ \varphi (x,x) = 0 ~\Leftrightarrow ~ x=0_E$\trait ${}$ \vspace*{-1.2cm} \traitd
		\paragraph{Espace euclidien}
			Soit $\left( E, \scal{.,.} \right)$ un espace préhilbertien réel, on dit que \uline{$(E,\scal{.,.})$ est un espace euclidien} si $E$ est de \textbf{dimension finie}. \trait
		\paragraph{Produit scalaires canoniques}
		\subparagraph{Sur $\mathbb{R}^n$}
		$\cm{~~\scal{x,y} ~=~ \scal{\si{1}{n} x_i , \si{1}{n} y_i } ~=~ \si{1}{n} x_i.y_i}$
		\subparagraph{Sur $\mathcal{M}_{np} \left(\mathbb{R}\right)$}
		$\cm{~~\scal{X,Y} ~=~ \mathrm{tr} \big(X\times{^TY}\big) ~=~ \si{1}{p}\sk{1}{n} a_{ki}.b_{ki}}$
		\subparagraph{Sur $\mathcal{C}^{0}\left([a,b],\mathbb{R}\right) ~~(a<b)$}
		$\cm{~~\scal{f,g} ~=~ \int_a^b f(t).g(t)\mathrm{d} t}$\\
		On a aussi sur $\mathbb{R}_n[X]$, $\cm{~~\varphi (P,Q) ~=~ \sk{0}{n} P(k).Q(k)~~}$ et $\cm{~~\psi (P,Q) ~=~ \sk{0}{n} P^{(k)}(0).Q^{(k)}(0)}$ \\
	\section{Norme associée à un produit scalaire}
		\traitd
		\paragraph{Norme}
			Si $E$ est un $\R$-espace vectoriel on dit que \uline{$N : E\to \R^+$ est une norme} si \\
			\hspace*{2cm} \un $\forall x\in E ,~\forall \lambda\in \R ,~N(\lambda x) = \mc{\lambda}.N(x)$\\
			\hspace*{2cm} \deux $\forall x\in E,~ N(x) = 0~\Leftrightarrow~ x=0_E$\\
			\hspace*{2cm} \trois $\forall (x,y) \in E^2 ,~ ~N(x+y)\leqslant N(x)+N(y)$
			\trait ${}$ \vspace*{-1.2cm} \traitd
		\paragraph{Distance}
			Si $E$ est un $\R$-espace vectoriel on dit que \uline{$d: E\times E\to \R^+$ est une distance} si \\
			\hspace*{2cm} \un $\forall (x,y)\in E^2 ,~d(x,y)=d(y,x)$\\
			\hspace*{2cm} \deux $\forall (x,y)\in E^2,~ d(x,y)=0 ~\Leftrightarrow~x=y$\\
			\hspace*{2cm} \trois $\forall (x,y,z)\in E^3 ,~d(x,z)\leqslant d(x,y)+d(y,z)$ \trait \vspace*{-1cm} \\
		\uline{Rq} : Si $(E,N)$ est un espace normé alors $d~\appli{E\times E}{(x,y)}{R}{N(x-y)}$ est une distance.
		\vspace*{0.5cm} \\ \thm{ch20th1}{Théorème : Inégalité de \textsc{Cauchy-Schwartz}}{InegCS}{Soit $\prehilb$ un espace préhilbertien réel alors \\
		\hspace*{2cm} $\cm{\forall (x,y)\in E^2 ,~\scal{x,y}^2 \leqslant \scal{x,x}\scal{y,y}}$\\
		Avec égalité si et seulement si $x$ et $y$ sont liés (égaux à un scalaire près)}
		\begin{proof}
		On pose pour tout $\lambda\in\R$, $P(\lambda) = \scal{x+\lambda y,x+\lambda y} \geqslant \in \R[X]$\\
		On a alors $P(\lambda) = \lambda^2 \scal{y,y} + 2\lambda \scal{x,y} + \scal{x,x}$\\
		\hspace*{2cm} $\bullet$ Si $\scal{y,y}=0$ alors $y=0$ et on a l'égalité.\\
		\hspace*{2cm} $\bullet$ Sinon vu $p(\lambda) \leqslant 0$ on a $\Delta =4\big(\scal{x,y}^2 - \scal{x,x}\scal{y,y} \big) \leqslant 0$ \\
		Si on a égalité alors il existe $\lambda_0\in\R$ tel que $P(\lambda_0) = 0 ~\Rightarrow~x+\lambda_0 y = 0$\\
		Réciproquement si $x=\lambda y$ alors $\scal{x,y}^2 = \lambda\scal{y,y}\times\lambda\scal{y,y} = \scal{\lambda y,\lambda y}\scal{x,x} = \scal{x,x}\scal{y,y}$ 
		\end{proof}
		${}$ \\ \thm{ch20P1}{Proposition}{NormEuclid}{Si $\prehilb$ est un espace préhilbertien réel\\
		Alors $x\mapsto \sqrt{\scal{x,x}}$ est une norme sur $E$ dite \uline{norme euclidienne} ($~\norm{.}~$) }
		\vspace*{0.5cm} \\ \thm{ch20P2}{Propriété}{InegTriAmelio}{$\forall (x,y)\in E^2$ si $N$ est la norme euclidienne associée à $\scal{.,.}$\\
		$N(x+y) \leqslant N(x) + N(y)$ avec égalité si et seulement si \\il existe $\lambda\in \R^+$ tel que $x=\lambda y$ ou $y = \lambda x$ }
		\vspace*{0.5cm} \\ \thm{ch20P3}{Propriété}{IdRemNorm}{Soit $\prehilb$ un espace préhilbertien réel et $\norm{.}$ la norme euclidienne associée.\\
		On a les identités remarquables suivantes :\\
		\hspace*{0.5cm} $\forall (x,y)\in E^2 ,~ \left\{ \ard \norm{x+y}^2 = \norm{x}^2 + 2\scal{x,y} + \norm{y}^2 \\ \norm{x-y}^2 = \norm{x}^2 -2\scal{x,y} + \norm{y}^2 \\ \norm{x+y}^2 + \norm{x-y}^2 = 2\big( \norm{x}^2 + \norm{y}^2\big) \arf \right.$ \\
		${}$ \\ On en déduit les formules de polarisation suivantes :\\
		\hspace*{0.5cm} $\forall (x,y)\in E^2,~ \left\{ \ard \scal{x,y} = \frac{1}{2} \big( \norm{x+y}^2 - \norm{x}^2 - \norm{y}^2\big) \\ \scal{x,y} = \frac{1}{2} \big( \norm{x}^2 + \norm{y}^2 - \norm{x-y}^2 \big) \\ \scal{x,y} = \frac{1}{4} \big( \norm{x+y}^2 - \norm{x-y}^2 \big) \arf \right. $ }
		\\ \uline{Rq} : $N : E\to \R^+$ est une norme euclidienne sur $E$ si et seulement si \\ \hspace*{2cm} $\varphi(x,y) = \frac{1}{4}\big(N^2(x+Y) - N^2(x-y)\big)$ est un produit scalaire.
	\section{Orthogonalité}
	\subsection{Résultats théoriques}
		\traitd
		\paragraph{Vecteur orthogonal}
			Si $\prehilb$ est un espace préhilbertien réel, \uline{$x$ et $y$ sont orthogonaux} si $\scal{x,y} = 0$. On note alors $x\perp y$ \trait ${}$ \vspace*{-1.2cm} \traitd
		\paragraph{Ensemble orthogonal}
			Si $\prehilb$ est un espace préhilbertien réel et $F\in \mathcal{P}(E)$ on appelle \uline{orthogonal de $F$} noté $F^\perp$ l'ensemble 
			\[ \{ y\in E ~|~\forall x\in F,~y\perp x \} \] \trait
		\thm{ch20P4}{Proposition}{OrthSousEspace}{$\forall F\in \mathcal{P}(E) ,~F^\perp$ est un sous-espace de $E$}
		\vspace*{0.5cm} \\ \thm{ch20P5}{Propriété}{20-P5}{Soit $\prehilb$ un espace préhilbertien réel, \\${}$ \vspace*{-0.4cm} \\
		\hspace*{0.5cm} \un $\forall (F,G) \in \mathcal{P}^2(E) ,~F\subset G ~\Rightarrow~G^\perp \subset F^\perp$ \\
		\hspace*{0.5cm} \deux $F^\perp = \big( \mathrm{Vect}(F)\big)^\perp$ \\
		\hspace*{0.5cm} \trois $F\subset\big( F^\perp \big)^{^{\perp}}$ avec égalité si et seulement si $F$ est une sous-espace vectoriel.} \newpage \traitd
		\paragraph{Famille orthogonale}
			Soit $I$ un ensemble, $\prehilb$ un espace préhilbertien réel et $\big(x_i\big)_{_{i\in I}}$ une famille de vecteurs de $E$.\\
			On dit que \uline{$\big(x_i\big)_{_{i\in I}}$ est orthogonale} si 
			\[ \forall (i,j)\in I^2,~i\neq j ~\Rightarrow~x_i\perp x_j \]
			On dit de plus que \uline{la famille est orthonormée} (ou orthonormale) si les vecteurs sont \textbf{normés} (ou unitaires), càd 
			\[ \forall i\in I ,~ \norm{x_i} = 1 ~~\Leftrightarrow ~~\forall (i,j)\in I^2 ,~\scal{x_i,x_j} = \delta_{i,j} \] \trait
		\thm{ch20P6}{Proposition}{FamOrthoNonNulleLibre}{Toute famille $\big(x_i\big)_{_{i\in I}}$ d'un espace préhilbertien réel \textbf{orthogonale} \\et ne contenant pas le vecteur nul est libre.\\
		Toute famille \textbf{orthonormée} est libre.}
		\vspace*{0.5cm} \\ \thm{ch20th2}{Théorème de \textsc{Pythagore}}{ThPythagore}{Soit $\prehilb$ un espace préhilbertien réel\\
		\hspace*{0.5cm} \un $x\perp y ~ \Leftrightarrow~ \norm{x+y}^2 = \norm{x}^2+\norm{y}^2$ \\ \hspace*{0.5cm} \deux Si $\big(x_i\big)_{_{i\in I}}$ est une famille orthogonale alors \\ \hspace*{2cm} \highlight{$\cm{\forall \big(\lambda_i\big)_{_{i\in I}} \in \R^{(I)} ,~ \norm{\sum_{i\in I} \lambda_i x_i }^2 = \sum_{i\in I} \lambda_i^2 \norm{x_i}^2}$}}
		\begin{proof}
		$\forall (x,y)\in E^2$\\
		\un $x\perp y ~\Leftrightarrow ~\scal{x,y} = 0 ~\Leftrightarrow~ \frac{1}{2}\big( \norm{x+y}^2 - \norm{x}^2 - \norm{y}^2 \big)=0 ~\Leftrightarrow~\norm{x+y}^2 = \norm{x}^2 + \norm{y}^2$\\
		\deux $\norm{\sum_{i\in I} \lambda_ix_i}^2 = \scal{\sum_{i\in I}\lambda_ix_i ~,~\sum_{j\in I} \lambda_jx_j} = \sum_{i\in I}\sum_{j\in I}\lambda_i\lambda_j \scal{x_i,x_j} = \sum_{i\in I} \lambda_i^2 \norm{x_i}^2$
		\end{proof}
	\subsection{Procédé d'orthonormalisation de \textsc{Gram-Schmidt}}
		\uline{Objectif} : Transformer par un algorithme une famille $\big(e_i)\big)_{_{i\in\ent{1,n}}}$ libre \\en une famille $\big( \varepsilon>0 \big)_{_{i\in\ent{1,n}}}$ orthonormée de telle sorte que \[ \forall k\in \ent{1,n} ,~F_k = \mathrm{Vect}\big(\{e_1,\dots ,e_k\} \big) = \mathrm{Vect}\big( \{\varepsilon_1,\dots ,\varepsilon_k\} \big ) = F_k'\]
		${}$ \\ Pour tout $k\in \ent{1,n}$ on pose $\cm{u_k = \si{1}{k-1} \scal{\varepsilon_i , e_k} \varepsilon_i }$
	\paragraph{Construction par récurrence} ${}$ \\
		\subparagraph{Initialisation}
			$\{e_1\}$ est libre car $e_1\neq 0$ on pose donc $\varepsilon_1 = \dfrac{e_1}{\norm{e_1}}$ qui convient.\\
		\subparagraph{Hérédité}
			Soit $k\in\ent{1,n}$ tel que $(\varepsilon_1 , \dots ,\varepsilon_{k-1})$ vérifie les contraintes et on considère $u_k' = e_k-u_k$. 
			On peut vérifier que $u_k' \in F_{k-1}'^\perp$ ; en effet : \\
			\[\forall l\in \ent{1,k-1}~,~~\scal{u_k',\varepsilon_k} = \scal{e_k-\si{1}{k-1} \scal{\varepsilon_i,e_k}\varepsilon_i ~,~\varepsilon_l } = \scal{e_k,\varepsilon_l} - \underbrace{\si{1}{k-1} \scal{\varepsilon_i,,e_l}\scal{\varepsilon_i,\varepsilon_l} }_{=\scal{\varepsilon_k,e_l}} = 0\]
			On a par contraposée $u'_k\neq 0$ (sinon $e_k\in F_{k-1}$), on peut donc considérer \fbox{$\varepsilon_k = \dfrac{u_k'}{\norm{u_k'}}$}. \\ Vérifions que $\varepsilon_k$ convient : \\
			On a déjà $(\varepsilon_1,\dots ,\varepsilon_k)$ est orthonormée vu $\varepsilon_k \in F_{k-1}'^\perp$. De plus $u_k'\in F_k$ d'où $\varepsilon_k\in F_k$ donc $F_k'\subset F_k$. Réciproquement $F_{k-1}'=F_{k-1}$ et $e_k\in \mathrm{Vect}(\varepsilon_1,\dots ,\varepsilon_{k-1}, u_k')$ d'où $F_k\subset F_k'$\\
			${}$ \\ On a ainsi une famille $\big( \varepsilon_i\big) _{_{i\in I}}$ orthonormée vérifiant les contraintes.
	\section{Bases orthonormées}
		${}$ \\ \thm{ch20th3}{Théorème}{BaseOrthEEuclid}{Tout espace euclidien admet une base orthonormée.}
		\begin{proof}
		Tout espace $E$ euclidien admet une base (dimension finie) donc on peut construire avec le procédé d'orthonormalisation de \textsc{Gram-Schmidt} une famille orthonormée génératrice de $E$ donc une base orthonormée de $E$
		\end{proof}
		${}$ \\ \thm{ch20th4}{Théorème de la base orthonormée incomplète}{ThBaseOrthoIncomplete}{Si $\prehilb$ est un espace euclidien de dimension $n$,\\
		pour tout $k\in\ent{1,n}$, soit $(e_1,\dots ,e_k)$ une famille orthonormée de vecteurs de $E$\\
		Alors cette famille peut être complétée en une base orthonormée de $E$.}
		\begin{proof}
		Comme $(e_1,\dots,e_k)$ est orthonormée elle est libre, on peut donc la compléter en une base de $E$ à laquelle on pourra appliquer le procédé d'orthonormalisation de \textsc{Gramm-Schmidt} pour obtenir une base orthonormée de $E$. \end{proof}
		${}$ \\ \thm{ch20P7}{Propriété}{CoordBase}{Si $\prehilb$ est un espace euclidien muni d'une base \\$(e_1,\dots ,e_n)$ orthonormée on a \\
		\hspace*{0.5cm} \un $\cm{\forall x\in E~,~~ x=\si{1}{n} \scal{x,e_i}.e_i} $\\
		\hspace*{0.5cm} \deux $\cm{\forall (x,y)\in E^2 ~,~~ \scal{x,y} = \si{1}{n}\scal{x,e_i}\scal{y,e_i} } $ }
		\vspace*{0.5cm} \\ \thm{ch20P7c}{Corollaire 1}{20-P7c}{Si $\prehilb$ est un espace euclidien rapporté à une base orthonormée $e$ \\
		\hspace*{0.5cm} Alors $~~\varphi~\appli{E}{x}{\M_{n,1}(\R)}{
		\left( \ard \scal{x,e_1} \\ \vdots \\ \scal{x,e_n} \arf\right)} ~~~~ $
		\begin{minipage}{5cm}
		 est un isomorphisme d'espaces vectoriel tel que \\ $(\scal{x,y}) = \varphi(x)\times {^T\varphi(y)} $
		\end{minipage}		}
		\newpage ${}$ \\ \thm{ch20P7c2}{Corollaire 2}{NormBaseE}{Si $x\in E$, $E$ euclidien rapporté à une base $(e_1,\dots , e_n)$ orthonormée\\
		\hspace*{2cm} Alors $\cm{ \norm{x}^2 = \si{1}{n}\scal{x_i,e_i}^2 } $ }
	\section{Projection orthogonale sur un sous-espace de dimension finie}
		${}$ \\ \thm{ch20P8}{Proposition}{SupplOrtho}{Si $F$ est un sous-espace de dimension finie de $\prehilb$ espace préhilbertien réel\\
		Alors $f^\perp $ est un supplémentaire de $F$ dans $E$ appelé \uline{supplémentaire orthogonal}\\
		de $F$ dans $E$. On note $F\overset{\perp}{\oplus} F^\perp$ }
		\vspace*{0.5cm} \\ \thm{ch20P8c}{Corollaire}{ThRangOrtho}{Si $F$ est un sous-espace vectoriel d'un espace $\prehilb $ euclidien \\
		\hspace*{2cm} Alors $\dim F^\perp = \dim E -\dim F$\\
		En particulier si $H$ est un hyperplan de $E$ tout vecteur \textbf{non nul} de $H^\perp$ \\ est dit \uline{vecteur normal à $H$} } \\
		\traitd
		\paragraph{Projection orthogonale}
			Si $F$ est un sous-espace de dimension finie d'une espace préhilbertien réel $\prehilb$ rapporté à une base orthonormée $(e_1,\dots ,e_p)$, alors
			$ \si{1}{p} \scal{x,e_i}e_i $
			est la projection de $x$ sur $F$ parallèlement  à $F^\perp$ autrement appelée \uline{projection orthogonale de $x$ sur $F$} parfois notée $p_F^{~\perp}(x)$
		\trait ${}$ \vspace*{-1.2cm} \traitd
		\paragraph{Distance à un ensemble}
			Si $F$ est un sous-espace de dimension finie d'un espace préhilbertien réel $\prehilb$, pour $x\in E$, on appelle \uline{distance de $x$ à $F$} et on note $d(x,F)$ le réel définit par \[ d(x,F) = \underset{y\in F}{\inf} \big\{ \norm{x-y} \big\} \] \trait
		\thm{ch20P9}{Propriété}{EcritureDistanceEnsemble}{Si $F$ est rapporté à une base orthonormée $(e_1,\dots ,e_p)$ \\
		\hspace*{0.5cm} Alors \fbox{$d(x,F) =\norm{x-p_F^{~\perp}(x)} = \norm{p_{F^\perp}^{~\perp}(x)}$} $= \norm{ x- \si{1}{p} \scal{x,e_i}e_i } $ }
		\vspace*{0.5cm} \\ \thm{ch20P10}{Proposition}{20-P10}{Si $u$ est un vecteur non nul d'un espace euclidien $\prehilb$ on a :\\
		\hspace*{2cm} $\cm{\forall x\in E ,~p_{(\mathrm{Vect(u))^\perp}}^{~\perp}(x) = x-\frac{\scal{x,u}u}{\norm{u}^2} } $ \\
		\hspace*{2cm} et $~~~~\cm{ d\big(x, (\mathrm{Vect}(u))^\perp \big) = \frac{\mc{\scal{x,u}}}{\norm{u}} } $ }
		\vspace*{0.5cm} \\ 
		\begin{center}
			\fin
		\end{center}

\chapter{Procédés sommatoires discrets}

    
% Chapitre 20 : Procédés sommatoires discrets

% TODO

\chapter{Fonctions de deux variables}

    
% Chapitre 21 : Fonctions de deux variables

\minitoc
	\section{Continuité}
	\subsection{Notion d'ouvert}
		\uline{Rappel} : \\
		Si la norme $\norm{.}$ dérive d'un produit scalaire on a :\\
		\un $\forall x\in \R^2 ,~\norm{x}\geqslant 0=(0,0)$ \hfill \deux $x\in \R^2 ,~\norm{x} = 0 \Leftrightarrow x=0$ \hfill ${}$ \\
		\trois $\forall x\in \R^2 ,~\forall \lambda\in\R,~ \norm{\lambda.x}=\mc{\lambda}\norm{x}$\\
		\quatre $\forall (x,y) \in (\R^2)^2,~ \norm{x+y}\leqslant\norm{x}+\norm{y}$ \\ \hspace*{0.5cm} avec égalité si et seulement si $\exists (\lambda,\mu)\in \R^2\setminus\{(0,0)\}$ tel que $\lambda x +\mu y = 0$\\
		{\scriptsize (5)} $\forall (x,y)\in (\R^2)^2 ,~\norm{x-y} \geqslant \norm{x} - \norm{y}$
		\vspace*{0.5cm} \\ \thm{ch22P1}{Propriété}{NormSupComposantes}{Si $\norm{x} = \sqrt{x_1^2 + x_2^2}$ avec $x=(x_1,x_2)\in\R^2$\\
		\hspace*{0.5cm} Alors $\norm{x} \geqslant\mc{x_1}$, $\norm{x} \geqslant\mc{x_2}$ et $\norm{x}\leqslant\mc{x_1}+\mc{x_2}$ } \newpage \traitd
		\paragraph{Boules}
			Soit $x_0\in\R^2$ et $r\in\R_+^*$ on appelle \\
			\hspace*{2cm} $\bullet$ \uline{Boule ouvert de centre $x_0$ et de rayon $R$} l'ensemble \[ B(x_0,r) = \{ x\in\R^2 ~|~\norm{x-x_0} < r\} \]
			\hspace*{2cm} $\bullet$ \uline{Boule fermée de centre $x_0$ et de rayon $r$} l'ensemble \[ \overline{B(x_0,r)} = \{ x\in\R^2 ~|~\norm{x-x_0} \leqslant r\}\]
			\trait ${}$ \vspace*{-1.2cm} \traitd
		\paragraph{Ouvert}
			Une \uline{partie $U$} de $\R^2$ est dit \uline{ouvert} lorsque  \[ \forall x\in U ,~ \exists r>0 ~:~ B(x,r) \subset U \] \trait
		\vspace*{-0.7cm} \\ \uline{Rq} : Un partie de $\R^2$ est dite \uline{fermée} si son complémentaire dans $/R^2$ est un ouvert.\\
		\vspace*{0.5cm} \\ \thm{ch22P2}{Propriétés}{22-P2}{\un $\varnothing$ et $\R^2$ sont des parties ouvertes de $\R^2$\\
		\deux Une union d'ouverts de $\R^2$ est un ouvert de $\R^2$\\
		\trois Une intersection \textbf{finie} d'ouverts de $\R^2$ est un ouvert de $\R^2$ } \\ 
	\subsection{Fonctions de deux variables}
		\traitd
		\paragraph{Définition}
			Si $U$ est un ouvert de $\R^2$ \uline{toute application $f : U\to \R$} est une fonction de deux variables réelles. \trait ${}$ \vspace*{-1.2cm} \traitd
		\paragraph{Continuité}
			Si $U$ est un ouvert de $\R^2$, $f : U\to \R$ et $x_0\in U$ on dit que \uline{$f$ est continue en $x_0$} si \[ \forall \varepsilon>0 ,~ \exists r>0 ~:~ \forall x\in U ,~ \big( x\in B(x_0,r) \Rightarrow \mc{f(x_0)-f(x)} \leqslant\varepsilon \big) \] \trait
		\thm{ch22P3}{Propriété}{PolynomXY}{Toute fonction polynômiale en $x$ et $y$ est continue sur $\R^2$}
		\vspace*{0.5cm} \\ \thm{ch22P4}{Proposition}{OpéF2Var}{Soit $f$ et $g$ définies sur un ouvert $U$ de $\R^2$ à valeurs réelles \\
		Soit $x_0\in U$, on suppose que $f$ et $g$ sont continues en $x_0$, alors \\
		\hspace*{0.5cm} \un $\forall (\lambda,\mu)\in \R^2 ,~\lambda f+\mu g$ est continue en $x_0$ \\
		\hspace*{0.5cm} \deux Si de plus $g(x_0) \neq 0$, il existe $r>0$ tel que $\forall x\in B(x_0,r) ,~g(x)\neq 0$ \\
		et $\frac{f}{g}$ est continue en $x_0$}\newpage \traitd 
		\paragraph{Applications partielles}
			Soit $f : U\to \R$ et $x=(x_1,x_2)\in U$ on définit les fonctions d'une variable réelle $f_1$ et $f_2$ 
			\[ f_1(t) = f(x_1,t) \hspace*{0.5cm} et \hspace*{0.5cm} f_2(t) = f(t,x_2)\] $f_1$ et $f_2$ sont dites \uline{applications partielles de $f$ au point $x=(x_1,x_2)$} \trait
		\thm{ch22P5}{Propriété}{ContApplPart}{Si $f:U\to \R$ est continue en $(x_1,x_2) \in U$ \\
		Alors $f_1$ et $f_2$ sont continues respectivement en $X_2$ et $x_1$}
	\section{Dérivation}
	\subsection{Dérivée partielles}
		\traitd
		\paragraph{Fonction différentiable}
			Soit $f:U \to \R$ avec $U$ un ouvert de $\R^2$ et $x=(x_1,x_2)\in U$\\
			On dit que \uline{$f$ est différentiable en $x$ par rapport à la première variable} si \[ t \mapsto \frac{f(x_1+t,x_2)-f(x_1,x_2)}{t}\] admet une limite en $0$, notée $\frac{\partial f}{\partial x_1} (x)$ sous réserve d'existence. \vspace*{0.2cm} \\ On considère une définition analogue en $x_2$
			\trait ${}$ \vspace*{-1.2cm} \traitd
		\paragraph{Dérivées partielles}
			Soit $f:U\to \R$ avec $U$ un ouvert de $\R^2$, on note $\mathscr{D}_f$ l'ensemble des points $x$ de $U$ tels que $f$ soit différentiable en $x$ selon la première variable.\\ On définie la \uline{dérivée partielle de $f$ selon la première variable} \[ \frac{\partial f}{\partial x_1} : \mathscr{D}_f \to \R \] qui à tout $x$ de $\mathscr{D}_f$ associe $\frac{\partial f}{\partial x_1}(x)$\vspace*{0.2cm} \\
			On définit de même la dérivée partielle de $f$ selon la deuxième variable. \trait \vspace*{-1.4cm} \\ 
		\begin{center} \begin{blockarray}{[c]}
		\textit{Si $f$ est différentiable en $x$ selon la première variable, on dit aussi que} \\ \textit{$f$ admet une dérivée partielle selon la première variable.}\\
		\textit{On a de même pour la deuxième variable.}
		\end{blockarray} \end{center}  ${}$\traitd
		\paragraph{Dérivabilité selon un scalaire}
			Si $f:U\to \R$, $h\in \R^2$ et $x\in U$ on dit que \uline{$f$ est dérivable en $x$ selon $h$} lorsque \[\frac{f(x+th)-f(x)}{t}\] 
			admet une limite finie quand $t\to 0$, notée $\dd_{f_x}(h)$ 
			\trait \newpage \traitd
		\paragraph{Classe $\Cun$}
			On dit que \uline{$f : U\to \R$ est de classe $\Cun$ sur $U$} si $\dfrac{\partial f}{\partial x_1}$ et $\dfrac{\partial f}{\partial x_2}$ sont définies et continues sur $U$ \trait
		\thm{ch22P7}{Propriétés}{CalcDer2Var}{Si $f$ et $g$ sont de classe $\Cun$ sur $U$ ouvert de $\R^2$ alors \\
		\hspace*{0.5cm} \un $\forall (\lambda,\mu)\in \R^2$, $\lambda f+\mu g$ est de classe $\Cun$ sur $U$ avec \\
		\hspace*{2cm} $\forall x\in U,~ \pfrac{(\lambda f + \mu g)}{x_i}(x) = \lambda \pfrac{f}{x_i}(x) + \mu\pfrac{g}{x_i}(x)$\\
		\hspace*{0.5cm} \deux $fg$ est de classe $\Cun$ sur $U$ avec \\
		\hspace*{2cm} $\forall x\in U,~\pfrac{(fg)}{x_i}(x) = f(x)\pfrac{g}{x_i}(x) + g(x) \pfrac{f}{x_i}(x)$\\
		\hspace*{0.5cm} \trois Si $g$ ne s'annule pas sur $U$ alors $\frac{f}{g}$ est de classe $\Cun$ sur $U$ avec \\
		\hspace*{2cm} $\forall x\in U ,~ \pfrac{(f/g)}{x_i}(x) = \dfrac{\pfrac{f}{x_i}(x) g(x) - f(x) \pfrac{g}{x_i}(x)}{g^2(x)}$ } \\
	\subsection{Différentielle}
		\traitd
		\paragraph{Fonction négligeable}
			Soit $f:U\to\R$ avec $(0,0)\in U$ on dit que \uline{$f(h)$ est négligeable devant $\norm{h}$} au voisinage de $(0,0)$ si 
			\[ \forall \varepsilon>0 ,~\exists \eta >0 ~:~ \forall h\in U,~\big( \| h\| \leqslant\eta \Rightarrow |f(h)| \leqslant\varepsilon \|h\|\big)\]
			On note alors $f(h) \underset{h \to (0,0)}{=} \circ\big(\|h\|\big)$ \trait
		\vspace*{-1.4cm} \\ \uline{Rq} : Si $f(h)$ est négligeable devant $\|h\|$ au voisinage de $(0,0)$ alors $f(h) \underset{h\to (0,0)}{\longrightarrow} 0$ et $f$ admet des dérivées selon tout vecteur en $(0,0)$ nulles.
		\vspace*{0.5cm} \\ \thm{ch22th1}{Théorème : Développement limité à l'ordre $1$ en $(x_0,y_0)$}{DL1x0y0}{Si $f:U\to \R$ est de classe $\Cun$ sur $U$ et $(x_0,y_0)\in U$ \\
		Alors pour tout $h=(h_1,h_2) \in\R^2$\\
		$\cm{ f(x_0+h_1,y_0+h_1) \underset{h\to (0,0)}{=} f(x_0,y_0) +  h_1\pfrac{f}{x}(x_0,y_0) + h_2\pfrac{f}{y}(x_0,y_0) + \circ \big( \|h\| \big) } $ }
		\begin{proof}
		Ce résultat est admis.
		\end{proof}
		${}$ \\ \thm{ch22th1c}{Corollaire}{C12VarImplCont}{Si $f$ est de classe $\Cun$ sur $U$ alors $f$ est continue sur $U$.}
		\vspace*{0.5cm} \\ \thm{ch22P8}{Proposition}{EcritureDer2Var}{Si $f:U\to R$ est de classe $\Cun$ sur $U$, $x\in U$\\
		Alors pour tout $h=(h_1,h_2)\in \R^2$, $f$ admet une dérivée en $x$ selon $h$ donnée par\\
		\hspace*{0.5cm} $\cm{ \dd_{f_x}(h) = h_1\pfrac{f}{x_1}(x) + h_2\pfrac{f}{x_2}(x) } $ }
		\vspace*{0.5cm} \\
		\begin{center}
			\fin
		\end{center}


%%%%%%%%%%%%%%%%%%%%%%%%%%%%%%%%%%%%%%%%%%%%%%%%%%%%%%%%%%%%%%%%%%%%
%                         Table of Contents
%%%%%%%%%%%%%%%%%%%%%%%%%%%%%%%%%%%%%%%%%%%%%%%%%%%%%%%%%%%%%%%%%%%%

%\setcounter{tocdepth}{1}

%\renewcommand{\contentsname}{Table des Matières - Première année}
%\tableofcontents

\tableofcontents

\anothertoc{../2_main_fr/mathematic_basics_fr_2}{Table des matières - Deuxième année}{2}
    

\end{document}